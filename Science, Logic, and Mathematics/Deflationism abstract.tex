\documentclass[]{birkjour}
\usepackage[backend=biber, giveninits=true]{biblatex}
\bibliography{jacob}

%opening
\title{Substantiality, attribution, and revenge}
\author{}

\begin{document}

\maketitle

\section{Abstract}
This paper develops a novel solution to various Liar paradoxes by bringing two seemingly unrelated claims about truth to bear on each other: the first, frequently debated, that truth is substantive; the second, advanced by Frege \cite{Frege1956} and hardly discussed since, is that the use of `true' as applied to sentences is essentially disparate from its non-sentential uses, - the use it has when predicated of friends, love, stories, geniuses, etc.

Let us collectively refer to those theses according to which truth is significant, important, etc. only if it is substantive as \textit{substantiality theses} [ST]. I intend ST to capture a wide variety of forms: one version might say that truth is a property only if it is substantive; another might take it to have a distinct domain only if it is substantive; others might take it to be meaningful, informative, metaphysically robust, etc. ST is not the categorical claim that truth is substantial. Rather, it is the hypothetical claim that truth is meaningful, etc. only if it is substantial.

To the degree that deflationist/inflationist debates are well-defined, ST is the common ground on which inflationists and deflationists agree. Kevin Sharp describes deflationism as holding that `truth is not a substantial notion and has no analysis. Instead, truth predicates play an important expressive role in our linguistic practice',  with the usual connotation that relegating the importance of truth predicates to the realm of language serves as a way of emptying them of any metaphysical baggage associated with substantiality \cite[13]{Sharp2013}. Asay [2014] characterizes deflationism in terms of two different kinds of substantiality claims: 1) that truth is not a (robust, metaphysically substantive, etc.) property; and 2) `that our concept of truth is explanatorily impotent' \cite[148]{Asay2014}.  On the inflationist side, various versions of the substantiality thesis have been adopted by Putnam, Wright, Lynch, and others \cite{Putnam1994,Wright2001,Lynch2009}.

If  we think of deflationism as roughly characterizable by the dictum that `there is no substantial property of truth, and so there is nothing - no domain of objects, or properties, or phenomena - which the theory of truth describes',\cite[13]{Glanzberg2003}  we can characterize the inflationist, conversely, as an adherent of the following thesis:
\begin{quote}
\textbf{Inflationism}: There is something - some domain of objects, properties, or phenomena - that the theory of truth describes, and so there is some substantial property of truth.
\end{quote}

Doing this allows us to see inflationism and deflationism as two sides of the same coin. Both concede the theory of truth is about something only if there is some substantial property of truth. From here, the primary difference is that of the deflationist's \textit{modus tollens} to the inflationist's \textit{modus ponens}.

Through revisiting Frege's thesis concerning the relation between the sentential and non-sentential uses of `true', I show that the substantiality thesis itself is mistaken. In the first section, I recount Frege's argument that `true', in its non-sentential and sentential uses, is equivocal, and show how the claim has been echoed down to today. 

In section two, I review a distinction, first drawn by Geach \cite{Geach1956} and more recently advocated by Thomson \cite{Thomson1997,Thomson2008}, between attributive and predicative uses of a term. In brief, in a sentence of the form `A is a BC' the term `B' is used predicatively if the sentence implies `A is B' - i.e. if B can be detached from C without change of meaning - and attributively when B is not detachable without a corresponding change in the way `B' is being applied to `A'. 

Much of the discussion of attributive and predicative uses of terms has surrounded the meaning of the term `good' in meta-ethics. Drawing on recent work by Almotahari and Hossein \cite{Almotahari2015}, section three shows that the same considerations leading one to regard `good' as lacking a predicative use also apply to the pre-sentential use of `true': terms used attributively serve to intensify the meaning of a predicate to which they attach, rather than adding substantial content, and thus depend on the explicit or implied presence of something they attach to for their meaning. 

Section four then shows how a sentential truth predicate can be syntactically defined in terms of its pre-sentential counterpart, on which the Liar and its strengthened cousins are syntactically ill-formed. I then provide a syntax and semantics for a logical language capable, on this assumption, of: i) expressing truth, falsity, and indeterminacy for sentences; ii) accommodating circularity for terms used predicatively, while providing a principled rejection of it for terms used attributively; and iii) expressing certain kinds of presupposition failure that themselves trigger indeterminacy. While this does not demonstrate that an equivocal sentential use of `true' does \textit{not} exist, consideration of Ockham's razor leaves the burden of proof on those who insist that one does. Barring such proof, logicians can safely assume that truth is a robust notion \textit{without} thereby taking it to signify a substantial property.
\printbibliography
\end{document}
