\documentclass[]{article}

%opening
\title{Special relativity for presentists}
\author{}

\begin{document}

\maketitle

\begin{abstract}
It has long been assumed that presentism is inconsistent with, or at best in tension with, the theory of special relativity. Hence, the strongest argument against presentism, and in favor of that of eternalism is often thought to be that from the truth of Einstein's theory. 

This paper argues that the argument from special relativity to eternalism suffers from two essential defects: first, that the presentist position is too narrow for the argument to work as an argument against presentism as such; second, even if it were sufficiently broad to address every variety of presentism, it is invalid - specifically, it relies on a quantifier shift fallacy. 

I begin by outlining the argument from special relativity to eternalism, focusing on the rejection of presentism in accordance with the relativity of simultaneity. 

I then move to a more detailed discussion of presentism, showing that there are at least four different theses thought to be characteristic of presentism: 1) that only the present exists; 2) that only present things exist; 3) that there is a non-indexical property of objective presentness, and 4) that to be is to be present. I argue that only the last of these should be accepted as an adequate characterization of presentism. 

Following this, I show that the presumption in favor of eternalism relies on a quantifier shift fallacy - specifically, from 'for every time, there is a reference frame in which it can be located' to `there is a reference frame' - i.e. the eternal one - `in which every time can be located'. If this is correct, then it is not merely that presentism and eternalism are on equal footing - rather, the lack of a single, universal reference frame in which all times can be located constitutes a decisive consideration against eternalism. 
\end{abstract}

\end{document}
