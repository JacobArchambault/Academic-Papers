\documentclass[]{amsart}
\usepackage[backend=bibtex]{biblatex}
\bibliography{jacob}
\usepackage{lplfitch}
%opening
\title{Monotonic and non-monotonic embeddings of Anselm's proof}
\author{Jacob Archambault}
\begin{document}

\maketitle

\begin{abstract}
A consequence relation $\vdash$ is monotonic iff for premise sets $\Gamma, \Delta$ and conclusion $\varphi$, if $\Gamma \vdash \varphi$, $\Gamma \subseteq \Delta$, then $\Delta \vdash \varphi$; and non-monotonic if this fails in some instance.  More plainly, a consequence relation is monotonic when whatever is entailed by a premise set remains entailed by any of its supersets. 

From the High Middle Ages through the Early Modern period, consequence in theology is assumed to be monotonic. Concomitantly, to the degree the argument formulated by Anselm at Proslogion 2-4 is taken up by later commentators, it is accepted or rejected in accordance with a monotonic notion of consequence.

Examining Anselm's use of parallelism in the Proslogion, I show Anselm embeds his famous argument in Proslogion 2-4 in a non-monotonic context. If this weren't so, Anselm would be contradicting himself when he adds God is greater than can be thought. The results found in this article challenge us to reconsider some deeply ingrained ideas governing the historiography of the long 12th century, particularly concerning how the theology of the later 11th through the 12th century relates to the scholasticism of the 13th.
\end{abstract}

\section{Introduction}
A consequence relation $\vdash$ is monotonic iff for premise sets $\Gamma, \Delta$ and conclusion $\varphi$, if $\Gamma \vdash \varphi$, $\Gamma \subseteq \Delta$, then $\Delta \vdash \varphi$; and non-monotonic if there is some instance where this relation fails. In English: any conclusion proven by a set of premises remains proven by any of its supersets. In any monotonic logic, the body of claims provable from a set of premises can only increase as new claims are added to it. Classical consequence is monotonic, as are the consequence relations for its best known competitors, e.g. intuitionistic logic and the many-valued logics K3 (i.e. Strong Kleene logic)) and Priest's LP (Logic of Paradox).

By analogy, we call a body of knowledge monotonic iff whatever is proven at any earlier stage remains proven at all later stages. Euclidean geometry is monotonic in this way: whatever is proven in the first book of the \textit{Elements} remains so in the last. Physics, by contrast, is not. The laws of Newtonian mechanics, for instance, had to be curved once general relativity was added to its stock of claims.\footnote{This does not preclude monotonicity from being present at a given stage in the development of physics taken by itself. It is rather that with the shift from Aristotelian to mechanistic physics, and from thence to Newtonian mechanics and then to general relativity, the progression of physics has often involved jettisoning old assumptions as new discoveries are made. By contrast, while the developments of Lobatchevskian and Riemannian geometries undermined the metageometrical identification of Euclidean geometry with geometry \textit{simpliciter}; and the discoveries of Einstein established Riemmanian geometry as better fitted to the description of actual space than Euclidean; these developments left the validity of Euclidean theory as a pure geometry intact in its own sphere. Thus, we take it as a mark of confusion, for instance, if someone calls Euclid's fifth postulate `false' because it is absent from Riemann.} 

From the High Middle Ages through the Early Modern period, theology behaves in the manner of a monotonic science. Concomitantly, to the degree the argument first formulated by Anselm at Proslogion 2-4 is taken up by later commentators, it is accepted (e.g. Alexander of Hales, Duns Scotus, Descartes, Leibniz) or rejected (e.g. Thomas Aquinas, Kant) in accordance with a monotonic ideal: either what it proves is eternally and indefeasibly proven, or the proof is no proof at all.

The following shows that in contrast with nearly the whole of its subsequent reception, Anselm himself embeds his famous argument in a non-monotonic context, i.e. there are connected subtexts of Anselm's \textit{Proslogion}, including chapters 2-4, entailing claims not entailed in the \textit{Proslogion} taken as a whole. Examining Anselm's use of parallelism in the Proslogion, and in particular the different ways the notion `that than which nothing greater can be thought' shows up in the work shows if this weren't so, Anselm would be contradicting himself when he adds in the middle of the treatise that God is greater than can be thought.  The results found in this case challenge us to reconsider some deeply ingrained ideas governing the historiography of the long 12th century, particularly concerning how the theology of the later 11th through the 12th century relates to the scholasticism of the 13th.

In what follows, I assume the following: the notions of monotonicity and non-monotonicity are applied to arguments as embedded in a body of claims, e.g. Euclidean geometry, the complete text of Anselm's \textit{Proslogion}, or some uninterrupted proper part of that text. $\Gamma \vdash \varphi$ means that within some body of claims $\Gamma$, there is an argument establishing conclusion $\varphi$. Needless to say, not all claims in $\Gamma$ need be used. Nor need all premises in an argument in $\Gamma$ be explicit; and in cases where non-monotonicity is in play, those that are not explicit may be represented by default rules. The elements of $\Gamma$ may be thought of as propositions, with two important caveats. First, they may change their truth value over time, e.g. `Socrates is sitting' should be thought of as a single proposition true at some times, false at others. Second, atomic propositions must not be thought of as altogether lacking composition, as is sometimes done in philosophical accounts of them. Rather, atomic propositions are composed of a subject and predicate, each of which signifies a concept, with more complex propositions built up from atomic propositions in the usual fashion. Concepts are not mental entities, but are to be identified with the objects or properties typically associated with them. Concepts may be coherent or incoherent, are said to be true of the objects they apply to, false of those they don't, and may apply possibly, necessarily, contingently, and so on for other modals. A proposition of subject-predicate form is true [false] when what is signified by its predicate is true [false] of its subject in the way the subject is taken (e.g. universally, particularly), as indicated by quantification. In this way, though propositions are true or false, what makes them so is determined at the conceptual level.
\section{A short long history of Anselm's unum argumentum}
\subsection{The argument in Anselm}
The earliest mention of the phrase `that than which nothing greater can be thought' is found in the \textit{quaestiones naturales} of the elder Seneca \cite{SenecaQN}. According to a twelfth century library list catalogued in \cite[266]{Bekker1885}, a copy of this work was at Bec already in the twelfth century, and it would be surprising if the work weren't present in Anselm's time. The work is listed immediately before a codex containing texts on grammar, logic, and rhetoric from Martianus Capella, Remigius of Auxerre, Priscian, Aristotle, and Boethius.

According to Anselm's biographer Eadmer, Anselm spent months searching for a single argument capable of doing the work of the several arguments of the \textit{Monologion}. When he was on the verge of giving up, the notion `that than which nothing greater can be thought' occurred to Anselm while praying in choir at the abbey.\cite[VA 1, 3, 26]{VA}.

Anselm's argument first appears in chapter two of his \textit{Proslogion}. It's basic structure is that of a simple syllogism:
\begin{enumerate}
\item[Maj.] That than which nothing greater can be thought exists in reality;
\item[Min.] God is that than which nothing greater can be thought;
\item[Con.] Therefore, God exists in reality.
\end{enumerate}

Anselm's argument is indebted to Boethius' synthesis of the Themistian and Ciceronian receptions of Aristotle's \textit{Topics}.\footnote{See \cite{Holopainen2007,Archambault2017}. Cf. \cite{Henry1974,Cerezo2015}.} Boethius, following Cicero, defines `topic' (from the Greek τόπος,  `place') as `the seat of an argument' (\textit{sedes argumenti}), and argument as `a notion making a doubtful matter sure' (\textit{ratio rei dubiae faciens fidem}) \cite[BDT 1174C, 1185A]{BDT} \cite[BTC 1048A]{BTC}.\footnote{Translations throughout, unless otherwise indicated, are my own.} The theory of topics treats not the formal or structural features of arguments, but rather the material, evidential sources for discovering and justifying new conclusions, e.g. definitions and descriptions; part-whole, genus-species, and cause-effect relations; correlativity, contrariety, etc.\footnote{See BDT L. II, \textit{passim}.} In the Boethian parlance, an `argument' (\textit{argumentum}) is not a set of premises which, when combined in the appropriate manner, lead to a conclusion; rather, it is an aspect of something whereby something further may be inferred about it or something else - typically, it is a concept signified by the middle term of a syllogism.\footnote{Thus, the modern meaning of an `argument' is arrived at from the Boethian one via an application of part-whole metonymy.} The premises of a topical syllogism need not be true, but may be merely reliable (\textit{probabilis}):\cite[BDT 1180C-82C]{BDT}. Topical argumentation leads one from accepted premises to conclusions entailed by them. 

While Anselm's argument is often thought of as a proof of God's existence, this isn't strictly correct. In the Boethian framework Anselm inherits, whatever can be thought in some way has being.\footnote{For contemporary work relevant to the concept `thinkable', see, for instance, the essays in \cite{Gendler2002}. Anselm himself is challenged on the meaning to further specify the meaning of the term by Gaunilo, who takes the proof to be circular on the only specification he thinks strong enough to ensure its validity \cite[Pro. ins. 2]{ProIns}. In reply, Anselm simply reaffirms a broader, common use of the term `thinkable' spanning its stronger and weaker specifications. See \cite[Resp. 1]{AnselmResp}.} Hence, God, being thinkable, is acknowledged to exist at least as an intentional object before the proof begins. Anselm's proof intends to establish that God exists not merely in thought, but in reality as well.\footnote{Cf. \cite{Klima2000}.}

Anselm's argument makes use of the Themistian topic \textit{from a description} (\textit{a descriptione}), the description in question being `that than which nothing greater can be thought'. Here, `description' is synonymous with what later medievals meant by `nominal definition', and close to what is meant today by `definite description'. A description in the Boethian sense suffices to pick out its referent uniquely, but only does so by way of certain non-essential features of the thing described. Hence, it need not apply to its referent necessarily. A description differs from a definition, which determines its bearer by way of its essential genera and differentiae, and is used in the Themistian topical argument \textit{from a definition}. 

The description `that than which a greater cannot be thought' serves as the \textit{argumentum}, or middle term of the argument concluding to God's existence: it is by unpacking the meaning of this description that a matter in doubt - whether God exists - is made sure, since, according to Anselm, this description includes within it the notion of existence. Thus, when Anselm claims the \textit{Proslogion} makes use of one argument (\textit{unum argumentum}) where the \textit{Monologion} used many, this does not preclude the presence of multiple deductions throughout the text leading to diverse conclusions: it means the one notion `that than which nothing greater can be thought' serves as the middle term in a series of arguments running through the text, whereby first, existence, then each of the divine attributes are shown to belong to God. In this way, the arguments of the rest of the text expand on that of \textit{Proslogion} 2, while remaining distinct arguments in the modern sense of the term.

In the argument of \cite[Pros. 2]{AnselmPros}, the minor premise is taken up as something worthy of belief (\textit{probabilis}). Boethius states a proposition is \textit{probabilis} if it `seems to be either to all, or to many, or to the wise; and among these [last] either to all, or many, or to those most renowned or distinguished; or to the specialist concerning his own province' \cite[BDT 1180CD]{BDT}. The claim that God is that than which a greater cannot be thought fits this in several ways: it was widely accepted in Anselm's time across religious divisions; it is found in Seneca the Elder, who, not always distinguished from his nephew Seneca the Younger, was a philosophical authority on par with Plato and Aristotle; and it was accepted on the authority of the Catholic faith, hence known to those wise concerning the things of God, while rejected by `the fool [who] says in his heart, there is no God.'[Ps. 14:1] \cite[Pros. 2, 4]{AnselmPros}. The last of these respects is the most prevalent in the \textit{Proslogion} and debate with Gaunilo. Thus, Anselm introduces his famous phrase with `And surely we \textit{believe} you to be something than which nothing greater can be thought' \cite[Pros. 2]{AnselmPros}, and makes a point of framing his response not to the fool, but to a Catholic on behalf of the fool.\cite[Resp. proem.]{AnselmResp}. 

In Anselm's response to Gaunilo, there is a faint attempt to justify the minor premise further, which proceeds via the Themistian topic \textit{from division} (\textit{ex divisione}). An argument under this topic begins by predicating two opposing properties disjunctively of a subject, and proceeds by either directly showing one to belong to it, or indirectly establishing the other doesn't. Anselm's justification is as follows:

\begin{enumerate}
\item[Maj.] `God is either that than which a greater cannot be thought, or is not understood or thought'\footnote{\cite[Resp. 1]{AnselmResp}. `Understood' and `thought' are here used synonymously. The critical text of the response has `God is \textit{not} that than which...' but the argumentative thread immediately following suggests this is a scribal error.}
\item[Min.] [God is thought]
\item[Con.] God is that than which nothing greater can be thought.
\end{enumerate}
Here again, however, the support for the minor premise bottoms out with an appeal to the interlocutor's faith: `I appeal to your faith and conscience as a most firm argument that this [i.e. the second disjunct of the major premise] is false' \cite[Resp. 1]{AnselmResp}.

The major premise of the main argument of Proslogion 2 is likewise supported by a topical argument from division, which is as follows:

\begin{itemize}
\item[Maj.] That than which nothing greater can be thought exists in understanding alone or in both understanding and reality
\item[Min.] That than which nothing greater can be thought does not exist in understanding alone
\item[Con.] That than which nothing greater can be thought exists in both understanding and reality
\end{itemize}
from which it follows straightforwardly that God exists in reality. 

The minor premise of the above syllogism, in turn, is supported by the following \textit{reductio}: suppose that than which nothing greater can be thought exists in thought alone. But it is better to exist both in reality and understanding than in thought alone. And, a being in all respects identical to that than which nothing greater can be thought, but existing as well, can be thought. But such a being is greater than one existing only in thought, and so can be thought greater than one existing only in thought, thus can be thought greater than that than which nothing greater can be thought - in which case, that than which nothing greater can be thought is something than which a greater can be thought, hence not that than which nothing greater can be thought. Therefore, that than which nothing greater can be thought does not exist in understanding alone.

\subsection{The later reception of Anselm's argument}
After the works of Anselm and Eadmer, the judgment that God is that than which nothing greater can be thought is mentioned in the \textit{Disputatio Iudei et Christiani} of Gilbert Crispin (c. 1045-1117), though neither it nor any argument based on it is made the subject of evaluation.\footnote{Crispin was a fellow monk of Anselm's at Bec. Though younger than Anselm, he received the monastic habit before him. Gilbert became abbot of Westminster around 1185, renewed contact with Anselm in 1092, and was a close companion of Anselm's until his death. See \cite[205-206]{Southern1963}.}

After this, the argument goes through a long period of relative silence. It seems to have been known in some form by Richard of St. Victor, but it is otherwise hardly mentioned throughout the 12th century.

In the thirteenth century, Anselm's argument experiences a renaissance, and becomes the subject of regular debate. It was first reintroduced by the Franciscan master Alexander of Hales in his \textit{Summa Theologiae}, and later adopted and modified by his student, St. Bonaventure in his \textit{Quaestiones Disputatae de Mysterio Trinitatis}.\footnote{See \cite[\textit{myst. trin.} q. 1, art. i]{BonaventureTrinity}, \cite{Seifert1992}.} During this same period, the argument was contested by Albertus Magnus and Thomas Aquinas.\footnote{\cite[ST Ia, q. 2, art. 2]{AquinasST}.} At the turn of the fourteenth century, the argument receives what is likely still its most nuanced formulation in the \textit{Tractatus de Primo Principio}  attributed to Duns Scotus \cite{Scotus1966}.\footnote{The work has extensive parallels to Scotus' proof of God's existence in his \textit{Ordinatio} \cite{ScotusOrd}, and if not by Scotus himself, it is a very early compilation from his disciples. Though more elaborate than Anselm's, Scotus calls his proof a `coloring of Anselm['s]' \textit{coloratio Anselmi}, thereby expressing his debt to the Benedictine.}

In the modern period, Descartes champions the argument in a modified form \cite{Meditations}. In his version, the notion `that than which a greater cannot be thought' is no longer present; and the key notion is instead that of infinity. Descartes' is the first version of the proof to explicitly characterize it as a deduction of God's existence from his concept. Descartes' formulation of the argument was criticized by Leibniz, who held it worked only if the concept of God was not contradictory. The argument then finds its most famous detractor in Kant \cite{KantCritique1}; and after Kant, is revived and put to a quite different use by Hegel \cite[par. 1530-1533]{HegelLogic}. In the 20th century, versions of the argument were accepted by Plantinga \cite{Plantinga1965}, Vallicella \cite{Vallicella2000}, and Russell \cite{RussellAutobio}; and rejected by Ryle \cite{Ryle2009}, Lewis \cite{Lewis1970}, and Russell \cite{RussellAutobio}, among others.

\section{Monotonicity in the post-13th century reception of Anselm's argument}
With the exception of Hegel, all major engagement with Anselm's argument after its 13th century resurgence locates it in a monotonic framework. This is partly a function of the historical circumstances underlying the argument's reappropriation, circumstances putting assumptions in place unchallenged until the 19th century, and still dominant today. By being received into texts like Alexander of Hales' and Thomas Aquinas' respective \textit{Summae Theologiae}, Anselm's argument was taken up into a wider debate over the status of theology as a science, governed by the recent reception of Aristotle's \textit{Posterior Analytics}.\footnote{Hereafter simply \textit{Analytics}.}  The \textit{Analytics} provides a determination of the principles of demonstration of a science, i.e. the principles whereby knowledge of necessary truths are brought into being in the mind of a knower. Aristotle holds the principles of scientific knowledge are necessary, which is \textit{prima facie} in conflict with the contingency of Christian salvation history. The aim of the great 13th century \textit{summae} was to establish that the truths of sacred doctrine are necessary and enduring, in much the same way as, e.g. whatever is proven in the first book of Euclid's elements remains so in the last.\footnote{See \cite[ST Ia, q. 1, art. 2]{AquinasST}, \cite[Ord. I, prol. 1a]{ScotusOrd}.

Many scholastics held that in scientific contexts one has a necessary proof precisely when one has proof of a necessity. The proof of this claim is based on the notion of science (\textit{scientia}) itself. What is known scientifically cannot come to have itself otherwise. But what cannot come to have itself otherwise is necessary. Hence, what is known scientifically is necessary. But a necessary conclusion is only from necessary principles. Proof: assume the contrary. Then there is some case where what the conclusion signifies is so without what is signified by the premises being so. But on the medieval account, the \textit{significata} of the premises are the causes on account of which what is signified by the conclusion is. And whatever can be without the being of its purported cause is not caused by that cause. Hence, the premises do not signify the causes of what the conclusion signifies, and so do not actually imply the conclusion, contrary to what was assumed. Necessary principles thus contain as the \textit{significata} of their terms the causes necessarily securing the conclusion, on account of which the conclusion itself is necessary. Because of this, every scientific proof becomes both a proof of a necessity and a necessary proof. See \cite[74b5-75a7]{Post.An}; \textit{in post. an.} \cite[In post. an. I, lec. 13, c. 6]{AquinasPA}.}

Theology is thus treated as a monotonic science throughout the scholastic and early modern period in the following sense: the only claims eligible for consideration as belonging to sacred doctrine (\textit{sacra doctrina}) are those serving to increase an accepted base of claims; each successive state of a science inherits everything given in the previous. It is this pre-technical assumption that, when formalized, is transformed into the sort of monotonicity found in different ways in the consequence relation of classical logic; in the valuations on Kripke-models of intuitionistic propositional logic; as a condition on augmented frames in first-order modal logic; or as a condition across different strong Kleene models in Kripke's theory of truth.\footnote{For the last of these, see \cite[95-96]{Gupta2001}, \cite{Kripke1975}. In intuitionistic semantics, each model $M = (W, R, v) $, is composed of a frame $F = (W, R)$, and valuation $v$ assigning truth values from the set $\{T, F\}$ to propositional variables $p$ at each element $w$ in $W$. Informally, $W$ is a set of possible worlds, and $R$ an accessibility relation between worlds. A valuation function $v$ is \textit{monotonic} with respect to a frame $<W, R>$, provided that for formulae $A$ and elements $w$, $w'$ in $W$: if $v_{w}(A) = T$ and $wRw'$, then $v_{w'}(A) = T$. This condition is sometimes also called the \textit{heredity condition}. It clearly parallels the monotonicity condition on variable domain augmented frames in first-order modal logic, expressed as follows: for each augmented frame $<W, R, D>$ with $D$ as a \textit{domain function} from each world $w$ in $W$ to its domain, if $wRw'$, then $D(w) \subseteq D(w')$. Cf. \cite[105, 422-424]{Priest2008}, \cite[110-112]{Fitting1998}.

The exact formal relations between these different kinds of monotonicity is left as a subject for further investigation. For formal introductions to  non-monotonic \textit{consequence}, see \cite{Horty2001}, \cite{Strasser2014}.} Formal theories whose models are monotonic in this way are what we should informally call ‘conservative'.

A negative corollary of this understanding of theology is that theological systems are presented as unrevisable. If any revision becomes necessary, this is invariably viewed as a defect on the part of the thinker, rather than as having something to do with the way the subject matter itself has to be approached.

\subsection{Anselm and Paradox}
Recent work on the ontological argument has shifted from questioning its validity to questioning the very coherence of its central notion. Thus, Jean-Luc Marion's work on Descartes has focused not on whether Descartes' ontological proof works, but on whether the notion of God found in his version of the proof is coherent. \cite{Viger2002} holds Anselm's argument succumbs to Russell's paradox. \cite{Roark2003}, arguing against \cite{Klima2000}'s reconstruction of the argument, holds any conceptual scheme strong enough to accommodate Anselm's reasoning would also be strong enough to generate paradox, thus inconsistent. \cite{Schlenker2009} argues Anselm's argument generates a version of Berry's paradox. \cite{Klima2003} has replied to Roark's argument, while \cite{Nowicki2006}, \cite{Neuhaus2007}, and \cite{Uckelman2010} have responded to that of Viger. To my knowledge, \cite{Schlenker2009} has not been contested.

The above attacks collectively presuppose that if the argument is valid, then its premises and conclusion must be consistent with the wider body of knowledge they are incorporated into. Moreover, its central notions must not admit of pathology, even under contingent circumstances. Defenses of the argument have likewise aimed to safeguard these very qualities.

Such strategies are in keeping with the post-13th century reception of the argument. Scotus, for instance, in rejecting that God's existence is known per se, writes:

\begin{quote}
If you insist the predicate is already posited in the subject in a proposition like “A necessary being exists” or “what is operating is in act” and consequently they are self-evident, I reply this does not follow, because it is not self-evidently known that the notions presumed to be present in the subject can actually go together. \cite[166]{Scotus1966}
\end{quote}

For Scotus, consistency of concepts thus serves as a constraining condition on what can be predicated of a subject. The same assumption is made by Leibniz in his criticism of Descartes' proof in Meditation III:

\begin{quote}
For I fully understand, for example, the nature of motion and speed and what it is to be greatest, but, for all that, I do not understand whether all those notions are compatible, and whether there is a way of joining them and making them into an idea of the greatest speed of which motion is capable. Similarly, although I know what being is, and what it is to be the greatest and most perfect, nevertheless I do not yet know, for all that, whether there isn't a hidden contradiction in joining all that together, as there is, in fact, in the previously stated examples. \cite[238]{AG}
\end{quote}

But if we actually examine the context within which the proof was originally offered – more plainly, if we were to read the Proslogion, rather than just chapters 2-4 – then we would see that treating the treatise as an increasing body of claims leads to inconsistency. This should force us to question the degree to which the aims of Anselm's treatise fall within the confines of post-Aristotelian science at all; and it should further force us to question the degree to which defenses that play into this context can really be defenses of Anselm.

\section{Diachrony, Religion, and Paradox}
\subsection{Parallelism and chiasm in the Proslogion}
To my knowledge, \cite{McMahon2004} is the only work to have drawn attention to the importance of Proslogion's use of chiasm and parallelism for understanding the text. McMahon explains chiasm as `a rhetorical pattern, a cross structure, and it may be rendered as A B B A. As a rhetorical scheme in a sentence, it can be contrasted with isocolon, or parallelism.' \cite[36]{McMahon2004} 
\begin{quote}
A Love and faithfulness\newline B \hspace{.3in} shall come together;\newline A Justice and peace \newline B \hspace{.3in} shall kiss [Ps. 85:11]
\end{quote}
is an example of parallelism; while 
\begin{quote}
A Faithfulness shall spring \newline B \hspace{.3in} from the earth; \newline B \hspace {.3in} and from the heavens \newline A righteousness shall look down [Ps. 85:12]
\end{quote}
is an example of chiasm. Chiasm is not limited to short verse, but can also be found in longer, more complicated structures.

The division of the Proslogion into chapters is found in the earliest manuscripts going back to Anselm's own time, though there, the chapter titles do not occur as headings for each chapter, but only at the beginning of the work, with the chapter numbers listed unobtrusively in the margins of an uninterrupted text.\footnote{McMahon (2004), 165.}  From this, it is likely that the chapter divisions came from Anselm himself.

\cite[166]{McMahon2004} offers the following chiastic breakdown of Anselm's Proslogion:

\bigskip

\begin{tabular}{l|l|l}
Chapters & Title & No. of chapters\\ \hline
1 & Prologue & 1\\ 
2-4 & The Famous Argument & 3\\ 
5-13 & Reviewing God's attributes & 9\\ 
14-22 & Reflection on and Renewal of Ascent & 9\\ 
23-25 & Diffusion of the Good: Climax & 3\\ 
26 & Epilogue & 1\\
\end{tabular}

\bigskip

Part of McMahon's justification for this analysis is the Trinitarian numerological symbolism he finds in the 1-3-9-9-3-1 structure.\footnote{McMahon (2004), 172-73.}  McMahon is surely correct to find a significant break between chapters 13 and 14, where what occurs after chapter 14 revisits themes broached in the earlier chapters. But the rhetorical pattern exhibited by the Proslogion as a whole is not chiasm, but \textit{parallelism}. The resulting outline of the work is as follows:

\bigskip

\begin{tabular}{l|l|l|l|l|l}
& Ch. & Contents & & Ch. & Content \\ \hline
A & 1 & Prologue & A' & 14 & Recapitulation \\
B & 2-4 & God exists & B' & 15-17 & God is greater \\
C & 5 & What God is generally & C' & 18 & What God is actually\\
D & 6-8 & God and time, 1 & D' & 19-21 & God and time, 2 \\
E & 9-11 & God is just & E' & 22-24 & God is good \\
F & 12-13 & God is holy & F' & 25-26 & God is with us \\
\end{tabular}

\bigskip

The \textit{Proslogion} splits into two parts. The first part, chapters 1-13, is primarily engaged in \textit{kataphatic} theology: in these chapters, the protagonist is concerned with attributes positively predicable of God. The second, chapters 14-26, primarily engages in \textit{apophatic} theology: there, the protagonist is not concerned so much with what God is as with what he is \textit{not}. The text exhibits these as two different stages in the spiritual life of Anselm the protagonist\footnote{i.e. the character who undergoes the dramatic movement of the \textit{Proslogion}, as distinguished from Anselm the author. For the importance of this distinction, see \cite{McMahon2004}.}, the apophatic being a later, higher stage than its kataphatic counterpart. Each section of the latter half of the work revisits some theme from its earlier counterpart and modifies it in some way. Where the first chapter begins by rousing the protagonist to seek God, the fourteenth reviews what he has found. Where the second through fourth chapters assume God's existence in understanding, emphasizing his presence even in the mind of the unbeliever, chapters 15-17 claim he is greater than can be thought, emphasizing his distance even from the understanding of the believer. Where the fifth chapter claims God is generally `whatever it is better to be than not to be'\footnote{\textit{Pros. 5}.}, the eighteenth gives this specific content. Where chapters 6-8 ascribe qualities to God traditionally associated with change, and hence with time, chapters 19-21 exhibit a God completely outside time.\footnote{In these earlier chapters, God is called sensible without being a body (6), omnipotent while being incapable of doing many things (7), and merciful yet impassible (8). The connection of these properties to temporality may not be obvious to us, but would have been straightforward in Anselm's time. In Boethius, for instance, the distinction between sense and understanding is grounded in one between sensibles and intelligibles: the former are subject to change and flux, while latter are eternal and unchanging \cite[BCP 82BC]{BCP}. By its very name, omnipotence implies potency, which in turn implies movement to actuality, hence change. See Aristotle, \cite[\textit{Metaph.} $\Theta$, 1045b 27-1046a 35]{Metaph}; \cite[\textit{in metaph.} 9, lec. 1]{AquinasMetaph}. Similarly mercy is thought of as involving passivity, in a way better captured in English phrases like `being touched' or `being moved to pity'.} Where chapters 9-11 examine God's justice, 22-24 assert his goodness. Where chapters 12-13 assert his unity and separateness, 23-25 examine his presence among his holy ones. To be sure, these are not the only places where earlier contents are revised or revisited in later chapters. Nor does every one of these later parallels require a rejection of contents tacitly or explicitly assumed in earlier cases. But some of them do. Among other examples, we may witness the following.

In chapter 6, Anselm explains how God is sensible though he is not a body. He writes:

\begin{quote}
How are you sensible, though you be not a body, but the highest spirit, which is better than a body? But if to sense is nothing other than to perceive, or to be toward what is to be perceived [...] it is not incongruent to say in a certain manner [something] senses whatever it perceives in a some way. Therefore, Lord, though you are not a body, yet truly you are supremely sensible, insofar as you supremely perceive all. \cite[Pros. 6]{AnselmPros}
\end{quote}
Compare this with the treatment of God's lack of sensibility in chapter 17:

\begin{quote}
[My soul] looks about and does not see thy beauty. It hearkens and does not hear your harmony [...] For you have these, Lord God, in your own ineffable manner, you who gave these [qualities] to things in their sensible manner; [...] but the senses of my soul are obstructed by the old languor of sin. \cite[Pros. 17.]{AnselmPros}
\end{quote}

In the movement from chapter 6-17, it is not as if the concept of sensibility were expanded to include God's sensibility, thereby creating a new concept. The base concept remains the same, and therefore the rarification of the earlier chapter is denied. The religious attitude of Anselm the protagonist requires the maintenance of both the affirmation and the denial at different moments, without giving up the synchronic incompatibility of these gestures.

A similar thing happens in \cite[Pros. 21]{AnselmPros}, where the age wherein God `abides'\footnote{Even though God abides in no place, since he is eternal and uncircumscribable. See \cite[Pros. 19]{AnselmPros}.}  is described in both singular and plural, `which is an age on account of its indivisible unity, ages on account of its interminable immensity.' 

\subsection{The inconsistency of the Proslogion as a whole}
To show this more clearly, we examine how the claims surrounding God's relation to thought interact throughout the Anselm's work. 

From God's being that than which nothing greater can be thought, Anselm infers the following in \cite[Pros. 15]{AnselmPros}:

\begin{quote}
Therefore, Lord, not only are you that than which a greater cannot be thought, but you are something greater than can be thought. For now something of this kind can be thought to be: if you are not this, then something greater than you can be thought, which cannot occur.
\end{quote}

`Can be thought greater than' has two different readings in Anselm, corresponding to what Geach called an \textit{attributive} and a \textit{predicative} interpretation.\footnote{\cite{Geach1956}. Cf. \cite{Thomson1997}, \cite{Almotahari2015}.} On the predicative reading, the phrase simply ascribes thinkability and the greater than relation separately to God, and so implies and is jointly implied by `God is thinkable' and `nothing is greater than God'. On the attributive reading, Anselm's phrase asserts no thought object satisfies the sentential function `x can be thought to be greater than God'. Anselm makes use of both readings throughout his Proslogion and the response to Gaunilo, the predicative reading somewhat more so. 

Let $g$ be a name for God, $T$ a monadic predicate meaning 'is thinkable'  $G$ a dyadic predicate `is greater than', and $\Theta$ an epistemic intensional operator, also meaning 'is thinkable'. Quantifiers are taken to range over intensions, as in \cite{CIFOL1}, \cite{CIFOL2}. Since the claim that `that than which nothing greater can be thought' applies uniquely to God plays no direct role in what follows, we leave it aside in our formalization. To capture Anselm's phrase in a single predicate, predicate abstracts are introduced as in \cite{Fitting1998}. Classical connective rules are assumed. A contradiction is obtained as follows:

\bigskip

\fitchprf{\pline[1.]{\uni{x}{(Tx \rightarrow Ggx)}}[(God is greater than can be thought)]\\
	\pline[2.]{\neg \exi{x}{(Gxx)}}[(Nothing is greater than itself)]\\
	\pline[3.]{Tg}[(God is thinkable)]}{
	\subproof{\pline[4.]{Ggg}}{
		\pline[5.]{\exi{x}{Gxx}}[\lexii{4}]}
	\pline[6.]{\neg Ggg}[\lnoti{4--5}]\\
	\pline[7.]{Tg \rightarrow Ggg}[\lalle{1}]\\
	\pline[8.]{Ggg}[\life{3}{7}]\\	
	\pline[9.]{\bot}[\lfalsei{6}{8}]
}

\bigskip

Here, the assumptions that God is both thinkable and greater than can be thought, combined with the irreflexivity of `greater than', lead to the contradiction that God both is and is not greater than himself.

To remedy this more fully with the intention of the text, let us introduce a default theory $\Delta = (W, D)$, where $W = \{(\lambda x. \neg \exists y (Ty \wedge Gyx) \wedge \neg\exists y(\Theta(Gyx)))(g), \neg \exists x (Gxx),\}$ is a set of premises, and $D = \{(\lambda x. \neg \exists y (Ty \wedge Gyx) \wedge \neg\exists y(\Theta(Gyx)))(g) : Tg/Tg\}$ is a default rule, instructing us that if it is accepted that God is that than which nothing greater can be thought, it can be assumed that God is thinkable barring evidence to the contrary. Here, $\Delta$ is meant to represent the situation of the protagonist at \textit{Proslogion} 2-4, who assumes even the fool who hears his phrase has God in his mind. In this situation, since $\neg Tg$ cannot be deduced, the default rule in $D$ can be initiated, and so $\Delta \vdash Tg$.

Next, let $\Delta' =(W', D)$ be a default theory where $D$ is as before, and $W' = W \cup \{\forall x (Tx \rightarrow Ggx)\}$. Intuitively, $\Delta'$ represents the situation of the \textit{Proslogion}'s protagonist at \textit{Pros.} 15. A deduction then only carries us as far as follows:

\bigskip

\fitchprf{\pline[1.]{(\lambda x. \neg \exists{y}{(Ty \wedge Gyx)} \wedge \neg \exists{y}{(\Theta(Gyx))})(g)}[(Pros. 2)]\\
	\pline[2.]{\uni{x}{(Tx \rightarrow Ggx)}}[(God is greater than can be thought)]\\
	\pline[3.]{\neg \exi{x}{(Gxx)}}[(Irreflexivity of `greater than')]}{
	\pline[4.]{\neg \exi{y}{(Ty \wedge Gyg)}}[From 1]\\
	\pline[5.]{\uni{y}{\neg (Ty \wedge Gyg)}}[From 4]\\
	\pline[6.]{\neg(Tg \wedge Ggg)}[\lalle{5}]\\
	\pline[7.]{\neg Tg \vee \neg Ggg}[De Morgan: 6]\\
	\pline[8.]{Tg \rightarrow Ggg}[\lalle{2}]\\
	\pline[9.]{\uni{x}{\neg Gxx}}[From 3]\\
	\pline[10.]{\neg Ggg}[\lalle{9}]\\
	\pline[11.]{\neg Tg}[Modus Tollens: 8, 10]
}

\bigskip

While the claim `God is that than which nothing greater can be thought' on its own implies \textit{either} God is not thinkable or that God is not greater than himself, and so is consistent with the former disjunct failing, the later claim entails both disjuncts altogether independently of the earlier one, assuming `greater than' is irreflexive. Here, because $\neg Tg$ is arrived at via application of the classical rules, the default rule in $D$ cannot be applied.

We see a similar thing happen again in chapter 12. In the chapters leading up to it, a variety of different qualities are predicated of God: he is called `sensible', `omnipotent', `merciful', `impassible', and `just'. For Anselm, following Aristotle, each of these terms refers to some being by means of an acquired concept with which the word is associated. But in each case, the concept the word attaches to is one taken from the realm of sensible things, and located in the Aristotelian category of quality. And every quality is, by definition, distinct from its bearer.\footnote{We find Anselm belaboring just this point in distinguishing between the reference, or \textit{appellation}, of the term `grammatical' (\textit{grammaticus}), namely a human being, and its \textit{signification}, namely grammar, in the \textit{De Grammatico}. See \cite[DG ch. 12]{Anselm1974}.} But in \cite[Pros. 12]{AnselmPros}, we find Anselm stating:
\begin{quote}
But surely, whatever you are, you are not so by another than by your very self. You are, then, the very life by which you life, and the wisdom by which you are wise: and so on in similar things.
\end{quote}.

Here, the manner in which \textit{any} quality predicated of God prior to the point is modified so as to drop a default assumption built into the very definition of quality: that qualities are distinct from their bearers. Let $\Delta$ be a default theory where $W = {Qg}$, and $D = {Qx: x \ne q/ x \ne q}$. Here, let $Qg$ be stand for any sentence predicating a quality of God anywhere up to chapter twelve in the \textit{Proslogion}, $q$ be a name for the quality predicated by $Q$, and $Qx: x \ne q/ x \ne q$ a normal default rule expressing for any $x$ that if it is established that $Qx$, then it is to be assumed that $x$ is not identical to the quality $q$ indicated by the predicated $Q$.\footnote{The convention of reading open formulas in defaults as generalizations, which we follow, goes back to \cite{Reiter1980}.} It is easy to see that $\Delta$ validates the claim $g \ne q$. However, if we let $\Delta' = (W', D)$ be the theory identical to $\Delta$, with respect to $D$, but where $W' = {Qg, g = q}$ the default rule in $D$ cannot be applied to $g$, thus representing the situation Anselm's protagonist finds himself in in \textit{Proslogion} 12.
\section{Conclusion}
The claims of the Proslogion should not be read as forming a systematic whole. Instead, in the movement of the work itself, the ascent of Anselm the protagonist sometimes involves a deepening of understanding that modifies or even jettisons claims advanced in earlier parts of the work.\footnote{This exegetical point also serves as the reason for rejecting the idea that Anselm's argument should be reconstructed in a paraconsistent logic with a monotonic consequence relation: the movement of the text clearly shows old concepts and the claims involving them being revised as the protagonist's understanding progresses.} The assumption that God is that than which nothing greater can be thought is one Anselm takes from faith. But in accordance with an analogous notion of faith, it is itself an object of faith, a medium of transcendence. It is from the dynamic instability of the notion that the Proslogion is a deeply unified work – \textit{unum argumentum} – yet without being linear or straightforward. By virtue of having this kind of unity, the Proslogion mirrors the unity of the religious life that Anselm the monk himself led – oriented toward God through Christ, the sign, or icon of the invisible God, one with the father without that unity extinguishing their distinction, and without making Christ the object of faith any less of a stumbling block. And in this, Anselm offers us a way out of \textit{our own} impasse between the theology of the geometer and the fideist flight from reason. Which is precisely why, given the dominance of these two standpoints in theology today, he himself has become a stumbling block in his own right.

\nocite{AlexanderST}
\nocite{RichardDT}
\nocite{Anselm1968}
\nocite{Meditations}
\nocite{CrispinDisp}


\printbibliography
\end{document}
