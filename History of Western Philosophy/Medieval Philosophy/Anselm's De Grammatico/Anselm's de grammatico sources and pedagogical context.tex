\documentclass[]{article}

%opening
\title{Anselm's \textit{De grammatico}: structure, sources, pedagogical context}
\author{}
%even
\date{}
\begin{document}

\maketitle

\section{Introduction}
One of the main impediments to understanding Anselm's \textit{De grammatico} [DG] is understanding what the text was \textit{for}. Prior to D. P. Henry's pioneering work, Anselm's text was at best regarded as a propaedeutic to theology: like Anselm's \textit{De Veritate}, \textit{De Libero Arbitrio} and \textit{De Casu Diaboli} in form, but without the intrinsic interest of the subjects discussed in these later dialogues. At worst, it was confused and not worthy of serious attention.\footnote{For an overview of the dialogue's 19th and early 20th century reception, see \cite{Steiger1969}.}


Though the work of Henry and others\footnote{See esp. \cite{Henry1960,Henry1974}; \cite{Galonnier1986a,Galonnier1987,Galonnier1996}; \cite{McCordAdams2000}; \cite{Boschung2006}; \cite{Cerezo2015}. Though not exclusively concerned with the DG, \cite{Mews1992}, \cite{Holopainen2007}, and \cite{Sharpe2009} are also helpful works.} has remedied this perception of the work's value, its context and aims remain obscure in certain respects. Basic questions, such as when the work was written, what circumstances gave rise to it, its sources, its relation to the later Trinitarian dispute with Roscelin of Compi\`{e}gne, and its essential contribution to medieval logic remain insufficiently discussed.\footnote{For instance, \cite{Sweeney2012} manages to discuss the work without ever mentioning its primary concern, the treatment of paronyms. See \cite{Holopainen2013}. And despite its being the most clearly philosophical of Anselm's works, the Stanford Encyclopedia of Philosophy article on Anselm, \cite{Williams2015}, contains no discussion of the DG. \cite{Uckelman2012} provides a useful overview of the various difficulties that have prevented deeper study of Anselm's text.}

This contribution surveys the basic issues surrounding the \textit{De grammatico}'s structure and origins, situating it in the monastic and pedagogical context of Anselm's time. I begin with an analysis of the structure of the \textit{DG}, given in the following section. Section 3 investigates the monastic circumstances surrounding the text's composition, as well as its place in instruction in the \textit{trivium}, i.e. the arts of grammar, logic, and rhetoric. Section 4 examines the main influences on the text's content. Library lists from Bec shortly after Anselm suggest the importance of two figures for instruction in the \textit{trivium}: Priscian, whose \textit{Institutiones Grammaticae} formed the core of instruction in grammar; and Boethius, whose commentaries accompanied the reading of Porphyry's \textit{Isagoge}, Aristotle's \textit{Categories} and \textit{De Interpretatione}, and Cicero's \textit{Topica}. \textit{Grammaticus} is a frequent example in both Priscian's \textit{Institutiones} and Aristotle's \textit{Categories}, recurring over a hundred times in Boethius commentary on the latter work. Section 5 concludes.

\section{The structure of Anselm's \textit{De grammatico}}
Structured as a dialogue between a master and a disciple, the opening question of the DG is whether \textit{grammaticus}, used both as a noun meaning `grammarian', and as an adjective meaning `grammatical', is a substance or a quality.\footnote{To preserve this ambiguity, Henry's translation replaces `\textit{grammaticus}' with `literate'. Accordingly, `\textit{grammatica}' is replaced not with `grammar', but `literacy'. This reading is grounded in an identification made at \cite[BC 257C]{BC}: 
\begin{quote}
For grammar itself, i.e. \textit{litteratura}, does not admit of better and worse, for no one says one grammar is better than another, but rather the one who participates in grammar itself.

Namque ipsa grammatica, id est litteratura, non suscipit magis et minus, nullus enim dicit alteram altera magis esse grammaticam, sed eum qui grammatica ipsa participat.
\end{quote}
I generally translate `\textit{grammatica}' as `grammar', and leave `\textit{grammaticus}' untranslated. But I employ Henry's translation when it is more helpful for understanding puzzles in the text.} Besides the ambiguity of the term, further difficulties arise for the modern understanding of this question: because Latin lacks both definite and indefinite articles; because quotation did not come into existence until several centuries after Anselm; and because unlike later medieval logicians, Anselm seldom made use of alternative techniques to indicate whether \textit{grammaticus} was being used or mentioned, such as prefacing a term or sentence with \textit{hoc} or \textit{ly} to indicate mention.\footnote{However, see \cite[DG ch. 6, 9]{Anselm1974}; cf. \cite[1.1.78]{ScotusMetaph}; \cite[pp. 56-57]{Ebbesen1979}; \cite[pp. 141-142]{Ebbesen1987}; \cite[pp. 149-50]{Ebbesen1993}; \cite[\textit{passim}]{Green-Pedersen1980a,Green-Pedersen1980b}; \cite[\textit{In post. an.} 1.33.9]{AquinasPA}.} Thus, to modern semantics, this opening question seems tangled in confusions: of sense and reference, of definite and indefinite, and above all of linguistic and metaphysical categories. This shows the distance between Anselm's linguistic framework and ours is quite wide. We'll nevertheless try to enter into Anselm's assumptions little by little. To start, we drop our use of quotations to mark use/mention from here on.

The immediate concern behind the dialogue's opening question is how to interpret \textit{paronyms}, or \textit{denominatives}. \textit{Denominativum} is the Latin translation of the Greek \textit{p\u{a}r\'{o}n\u{u}mon},\footnote{\textit{Denominativum} is derived from \textit{de}, meaning of or from, and \textit{nomen}, name. The etymology of the Greek term splits into the same two parts.} from which the English paronym is derived. The theory of paronyms holds a basic place in ancient grammatical theory.\footnote{See \cite[BC 167D-168D]{BC}; \cite[inst. gram. IV]{inst.gram.}.} According to the Boethian translation of Aristotle's \textit{Categories} Anselm was familiar with,`Whatever have their appellation from another only differing in case from a name, as literate from literacy, and courageous from courage, are called paronyms'.\footnote{`Denominativa vero dicuntur quaecunque ab aliquo solo differentia casu secundum nomen habent appelationem, ut a grammatica grammaticus, et a fortitudine fortis', \cite[BC 167D]{BC}.} It is clear from the examples that the theory does not concern case in the grammatical sense of what is signified by declension, but more broadly relations of priority and posteriority between terms having a common root. Boethius comments:

\begin{quote}
There is nothing obscure in this definition. For the ancients called certain transformations of names, as just from justice, courage from courageous, etc., cases. [...] And so whenever one thing partakes of another, this participation is inherited in the name just as in the thing. For instance, a certain man, because he partakes of justice, draws the thing near and hence the name, too: he is called just. Therefore, whatever differ from a principal name in case alone, i.e. only by a transformation, are called paronyms.\footnote{\cite[BC 167D-168A]{BC}:
\begin{quote}
Haec quoque diffinitio nihil habet obscurum. Casus enim antiqui nominabant aliquas nominum transfigurationes, ut a iustitia iustus, a fortitudine fortis, etc. Haec igitur nominis transfiguratio, casus ab antiquioribus vocabatur. Atque ideo quotiescunque aliqua res alia participat, ipsa participatione sicut rem ita quoque nomen adipiscitur, ut quidam homo, quia iustitia participat et rem quoque inde trahit et nomen, dicitur enim iustus. Ergo denominativa vocantur quaecunque a principali nomine solo casu, id est sola transfiguratione discrepant.
\end{quote}}\end{quote}

Here for Boethius, the seemingly straightforward syntactic-morphological question of what paronyms are leads to the semantic question of what they signify, and thereby to ontological concerns surrounding the categories of being. Anselm follows Boethius in this respect.

Anselm's \textit{De Grammatico} divides into a destructive part, aimed at warding off misunderstandings (ch. 1-11), and a constructive part, where the master provides a positive exposition of issues raised in the first part of the dialogue (ch. 12-21).\footnote{In this respect, the formal structure of the DG mimics that of Augustine's dialogues \textit{de libero arbitrio} and \textit{de magistro}, where the back-and-forth of the first half ultimately gives way to a more sustained explanation by the more authoritative of the dialogue's participants. For the influence of the \textit{de magistro} on the DG, see \cite[pp. 172-180]{Boschung2006}.} The first part is divided into three parts: the first proffers arguments for \textit{grammaticus} being a substance, then a quality (ch. 1); the second examines arguments against \textit{grammaticus} being a quality (ch. 2-5); the third, against it being a substance (ch. 6-11). The constructive part of the dialogue focuses first on the distinction between appellation and signification, and how things in the categories of substance and quality\footnote{That Anselm uses quality is accidental to his purpose: his main concern is to determine how things belonging to the various accidental categories are understood to do so, given that they apply to substances.} are distinguished from each other according to the \textit{signification} of their terms, rather than their appellation (ch. 12-18). The remaining chapters (ch. 19-21) discuss how different kinds of accidents are distinguished from each other, and specifically how things in the category of quality are distinguished from those of \textit{habere}, i.e. habit or having.
\section{The composition of the \textit{De Grammatico} in its monastic context}
\subsection{Why was the \textit{De Grammatico} written?}
%consider deleting this entire section if necessary
To understand exactly what circumstances gave rise to Anselm's dialogue, it is not enough to recognize the dialogue's immediate concern with paronyms; and though Anselm thought the dialogue useful for introducing students to dialectic \cite[DV, prol.]{AnselmDV}, it cannot be construed more specifically as an introduction to Aristotle's \textit{Categories}, as in \cite{McCordAdams2000}. This purpose was already amply satisfied by Porphyry's \textit{Isagoge}, Boethius' two commentaries on Porphyry, as well as Boethius commentary on the \textit{Categories} itself.\footnote{Anselm displays great familiarity with these works, copies of which are recorded at Bec in the 12th century. See \cite[p. 266]{Bekker1885}; also \cite{Lewry1981}, \cite{Holopainen2007}.} Rather, the disciple of the DG's references to the \textit{Categories} suggest an acquaintance with the work, albeit an incomplete one.\footnote{DG 17.}

A clue to the dialogue's purpose arises from an examination of the Bec library's 12th century holdings at \cite[p. 266]{Bekker1885}. After listing holdings from Augustine and other fathers, one comes to a section of the list headed `Libri Dionisi Areopagite', only the first few works of which are those attributed to Dionysius. The rest of the section lists a number of more philosophically rich theological texts,\footnote{E. g. Gregory of Nyssa's sermons (n. 134) and \textit{De opificio hominis} (n. 142 - mistakenly recorded as a work of Gregory of Tours), Origen's \textit{commentary on the Song of Songs} (n. 140).} as well as works belonging to classical antiquity.\footnote{Suetonius' \textit{de vita Caesarum} (n. 147), Cicero's \textit{de officiis} (n. 148), Seneca's epistles (n. 153)} Two of the final three works are clearly intended as manuals for studying the trivium (grammar, logic, and rhetoric) and quadrivium (arithmetic, geometry, music, and astronomy). The contents for the trivium text reads as follows: 
\begin{quote}
157. in alio Martianus Capella de nuptiis Mercurii et philologie lib. II et de VII artibus editis ab eo lib. VII et commentum Remigii super eumdem IX lib. Priscianus de VIII partibus et de constructionibus II. utraque rethorica II. dialectice III. utrumque commentum super Porphirium. primum super catheg. primum, secundum super periermenias. commentum super topica Ciceronis.
\end{quote}

The first works listed, Martianus' Capella's \textit{on the marriage of Mercury and Philology} along with the Carolingian Remigius of Auxerre's commentary on it, provide a general introduction to both the trivium and quadrivium. Then comes Priscian's \textit{Institutiones Grammaticae}, designated here by its contents, `on the eight parts [i.e. of speech] and 2 [books on] constructions'. Elsewhere, it has been argued the `dialectic in three [books]' should be identified with the first three books of Boethius' \textit{De differentiis topicis} [BDT], the fourth book of which lacks extant commentaries from prior to the 13th c. and circulates separately in some mss. of the text.\footnote{\cite{Archambault2017}; see also \cite[p. 124]{Green-Pedersen1984}.} The fourth book of the BDT is probably one of the rhetorics listed, the other likely being either Priscian's or the Pseudo-Ciceronian Herennian rhetoric, with which Anselm was probably familiar\footnote{\cite[p. 87]{Henry1974}.}. The remaining texts are Boethius' two commentaries on Porphyry's \textit{Isagoge}, on Aristotle's \textit{Categories} and \textit{De Interpretatione}, and on Cicero's \textit{Topics}.

This codex or one like provides the background course of study to Anselm's \textit{De grammatico}. We know from both Anselm's own testimony and elsewhere: that there was an increasing interest in dialectic in his time and province;\footnote{For similar library listings from Anselm's own lifetime, see the listings of a certain Bernardus, and the listings and ordering at Hamersleven, Chartres, and Toulouse \cite[pp. 138, 140, 144-145, 153]{Bekker1885}.} that Lanfranc had helped revive the study of the trivium at Bec;\footnote{\cite[pp. 14]{Southern1963}.} and that Anselm himself shows familiarity with a number of the works mentioned. The codex provides a course of study to be read from beginning to end: beginning with grammatical works and moving to logical ones.\footnote{That a codex could serve as the foundation for a course, under the guidance of a \textit{magister}, is confirmed in Anselm's advice to the monk Maurice `to try particularly to study very diligently the grammar of the books you have read, right from the beginning to the end' \cite[Ep. 64, p. 180]{Anselm1990}.} Given this, we can locate Anselm's DG within such a course of study, providing a comparative and clarificatory role in the transition from the study of grammar to that of logic.
\subsection{The dating of the \textit{De grammatico}}
Three different periods have been suggested for the composition of Anselm's text - one early, a second during Anselm's time as prior, and a third during his time as abbot of Bec. \cite[sec. 1]{Williams2015} dates the work to around 1059-60, immediately after Anselm's arrival at Bec in 1059.\footnote{Cf. \cite[p. 4]{Visser2009}.} Likewise \cite[p. 128]{Gasper2004}, following \cite[p. 65]{Southern1990}, suggests a date of 1060-63, prior to Lanfranc's departure from the abbey. A second position, followed by \cite{Schmidt1968}, \cite{Lewry1981}, \cite{Sharpe2009}, and in the introduction to \cite{Anselm1990}, holds Anselm wrote the work as abbot between 1080 and 1085. Another position, that of an anonymous 12th c. chronicle \cite[v. 158, 131-133]{PL}, states the work was written during Anselm's time as prior of Bec, hence after 1063, but prior to his election as abbot of the abbey in 1078. This position is also suggested by Eadmer, though Eadmer's exact wording does not force this interpretation.\footnote{`His temporibus scripsit tractatus tres, scilicet \textit{De veritate, De libertate arbitrii, et De casu diaboli} [...] Scripsit et quartum quem intitulavit De grammatico, in quo cum discipulo quem secum disputantem introducit, disputans cum multas quaestiones dialecticas proponit et solvit, tum qualitates et qualia, quomodo sint discrete accipienda exponit et instruit.' \cite[VA I, 25]{VA}.}

The positions of Southern, Williams, and Gasper garner their strength from the connection of the dialogue's subject matter to a time of Anselm's more secular preoccupations. But while Anselm may have been more occupied with secular learning at these times \cite[VA I, 4-6]{VA}, he is less likely to have been in a position to write dialogues then. The process of reading in medieval monasteries often involved an element of assimilation of the text to one's own surroundings, and both author and reader to its characters - an assimilation Anselm takes full advantage of in his \textit{Prayers and Meditations} and \textit{Proslogion}.\footnote{Cf. \cite{McMahon2004}, \cite{Jordan1992}.} It would have been unfitting for Anselm, and perceived as such, to have written a magisterial dialogue on dialectic at the time he was studying the discipline under Lanfranc.\footnote{Eadmer's description of the dialogue in \cite[VA I, 25]{VA} confirms the identification of Anselm himself with the \textit{magister} of the dialogue: Anselm introduces a disciple `disputing \textit{with him}'. Cf. The disciple and master's references to `your/my \textit{Monologion}' in \cite[DV ch. 1, 10]{AnselmDV}; also \cite[pp. 398-400]{Novikoff2011}.} And after Anselm was received as a monk at Bec in 1060 at 27 \cite[VA I, 8]{VA}, the humility of his position would likely not have permitted the leisure of writing for an audience \cite[RSB ch. 5-7]{RSB}. The Rule of St. Benedict forbids monks from writing letters (\textit{litterae}),\footnote{\textit{Litterae} can also be used to refer to writings generally, thus encompassing book writing.} to family members or anyone else without the express permission of the abbot \cite[RSB ch. 54]{RSB}, and the earliest of Anselm's letters are traditionally dated to around 1070. For the same reason, it is unlikely Anselm would have been writing works to be circulated outside the monastery prior to then.\footnote{The major 12th c. exception to this rule, Peter Abelard, confirms it. Abelard's early success and lack of deference towards his master led those around him to regard him as presumptious, as he himself admits. See \cite[HC, ch. 2-4]{AbelardHC}.}

The later, post-1080 date is supported by the traditional association of the \textit{De Grammatico} with Anselm's other three dialogues, the \textit{De Veritate}, \textit{De Libertate Arbitrii}, and \textit{De Casu Diaboli}, combined with evidence from the manuscript record as well the letters and Anselm's other works. 

The earliest extant manuscript of the \textit{De Grammatico}, Bodl. Rawl. A. 392, ff. 78-96, lacks chapter headings for the various chapters of the work and thus, given the headings appeared quite early, has the appearance of being an earlier `draft' copy of the work. Schmidt dates the ms. to c. 1085. 

Furthermore, the \textit{De Veritate} cites the \textit{Monologion} by name \cite[DV ch. 1; ch. 10]{AnselmDV}. But we know Anselm didn't name the \textit{Monologion} until 1083/84, as attested in a letter to Archbishop Hugh of Lyon where he asks the name of the work to be changed \cite[Ep. 109]{Anselm1990}. Hence, if the association of the DG with the other dialogues is trustworthy, it too should be dated to the mid to late 1080s. The DV's explicit references to the Monologion are also taken to undermine Eadmer's authority on the dating of the dialogues, who dates the dialogues before the \textit{Monologion} \cite[VA I, 25-26]{VA}.

Lastly, the content of the work itself appears rather advanced. Southern called the DG `the most difficult [of Anselm's works] to understand and therefore the easiest to misinterpret' \cite[p. 15]{Southern1963}. And though Southern himself preferred an early date, others have taken the advanced character of the dialogue as a reason for a later one. For Sharpe, Anselm's dialogue `is relevant to the nominalism affecting theology more than to basic teaching in logic' and this should support a later date in the early 1080s, `contemporary with the other dialogues' \cite[p. 22]{Sharpe2009}.\footnote{Cf. \cite{Mews1992}.}

To resolve the above disparity, we distinguish, as Sharpe does, between the time of a work's having been written, and that of its `publication', i.e. the beginning of its authorized circulation. We know such a distinction can be made in Anselm's case, given his well-known complaints about certain of his writings, e.g. the \textit{Cur Deus Homo}, having been circulated by overeager monks prior to being put into their final form \cite[p. 5]{Sharpe2009}. Doing so provides us with a better sense of the various stages of Anselm's dialogue, while also providing a clearer sense of what the various kinds of evidence can tell us about the stages of its composition and early circulation.

Of all Anselm's works, only certain of the letters and \textit{Prayers and Meditations} seem to have had much circulation prior to Anselm's election as abbot, and even here the circulation was probably limited. Anselm's election as abbot granted him not merely a position of authority within the monastery, but also one of importance outside it. For instance, prior to his election as abbot, Anselm circulated the \textit{Monologion} anonymously \cite[\textit{pros.} prol.]{AnselmPros}; wished to restrain its circulation for fear of being branded a heretic \cite[Ep. 83, pp. 217-218]{Anselm1990}, and was only prepared to circulate it more widely after its examination and correction by Lanfranc.\footnote{See \cite[Ep. 72, pp. 197-98; Ep. 74, pp, 201-202; Ep. 77, pp. 205-207]{Anselm1990}.} But later we see the text circulated under the title of abbot of Bec.\footnote{See \cite[Ep. 100, pp. 250-251]{Anselm1990} to Archbishop Hugh of Lyon.}  Anselm's election as abbot as well as the subsequent success of the \textit{Monologion} and \textit{Proslogion} thus afforded Anselm the opportunity to officially circulate works written years earlier under the circumstances of a prior or even a novice.\footnote{This is almost certainly the case with the \textit{Deploratio virginitatis malae amissae}, whose content and style suggest a date from the early 1260s, but which only received a much wider circulation when the \textit{Prayers and Meditations} began to be circulated as collections. See \cite[pp. 45-46]{Southern1963}, \cite[p. 105]{Southern1990}. Cf. \cite[p. 13]{Sharpe2009}.} This is likely the case with the DG, whose writing would have arisen most naturally out of Anselm's duties as prior, but whose publication benefits from the authority of an abbot.

In a Benedictine monastery of Anselm's time, the prior assisted the abbot in running the monastery. He made requests of the other monks under his authority \cite[RSB ch. 6, 7]{RSB}; chanted the Lord's Prayer during matins and vespers;\footnote{I.e. the daily morning and evening prayer services chanted at sunrise and sunset. \cite[RSB ch. 13]{RSB}.} preached on the reading during the daily meal \cite[RSB ch. 38]{RSB}; visited the sick of the monastery \cite[\textit{Decreta}, ch. 3]{LanfrancDecreta}; received guests, and determined whether a fast would be broken for their hospitable reception \cite[RSB ch. 53]{RSB}. Overseeing the brothers' instruction in disciplines vital to monastic education, yet subservient to scriptural study, would have been in keeping with this role.

That Anselm instructed students in the trivium while prior is confirmed in Anselm's letter 64, from about 1076, to the monk Maurice. There, Anselm expresses joy that Maurice is learning grammar from Arnulf of Saint Simphorien of Beauvais, a renowned teacher who became a monk at Christ Church Canterbury in 1073 \cite[See Ep. 38]{Anselm1990}. Anselm further confirms he himself had taught grammar to Maurice and the other monks, though he complains it had been a burden for him to teach the boys grammar, and thinks Maurice could have progressed much more with a better teacher than himself.\footnote{`Audivi quoque quod ipse [i.e. Arnulfus] multum valeat in declinatione; et tu scis quia molestum mihi semper fuerit pueris declinare. Unde valde minus quam tibi expediret scio te apud me in declinandi scientia profecisse' \cite[v. 158, 1124C]{PL} \textit{Fuerit} is subjunctive perfect, the perfect tense suggesting Anselm has not held these duties for some time.} Maurice left Bec for Canterbury by 1073 \cite[Ep. 33-36]{Anselm1990}, so we know Anselm instructed the other monks in grammar before then. And as abbot Herluin's health declined, Anselm would have increasingly taken on the administrative, pastoral, and theological concerns awaiting him as Herluin's successor upon his death. If, then, we assume that the \textit{De Grammatico} does not significantly antedate the earliest of Anselm's letters, and that the writing of the dialogue would have arisen naturally out of Anselm's responsibilities at the monastery at the time of writing, we can posit a likely date of composition between 1070 and 1076.\footnote{The earliest letters, however, may have been written before the traditional 1070 dating. See \cite[11-15]{Sharpe2009}.}

Given this, we can incorporate the evidence for a later date as follows. Evidence from Schmidt's dating of the earliest known ms. to ca. 1085 is circular, since the dating of the ms. itself is based on the assumed chronology of the works, and in particular on the association of the DG with the other dialogues. The record from this ms., given the quality of the text itself, witnesses to an early state of circulation, rather than composition - one prior to the addition of the chapter headings, which don't play a significant role in Anselm's DG.\footnote{The Rawlinson A. 392 ms. contains the DG as a booklet bound in a single quire with three other booklets, the DG being the third: the first contains the \textit{Monologion} and \textit{Proslogion}; the second, the \textit{Orationes sive Meditationes}; the fourth, the \textit{De veritate} and \textit{de libertate arbitrii}, these last being untitled, without chapter headings, with some significant textual variants, and unaccompanied by either the \textit{De Casu Diaboli} or Anselm's joint preface to these three dialogues. The Norman hand composing the DG is very similar, and may be the same as, that for the first booklet containing the \textit{Monologion} and \textit{Proslogion}.} The association of the DG with the other dialogues mentioned in the \textit{De Veritate} preface is always at arm's length: in that very preface, Anselm wishes the DG not to be numbered among the other dialogues on account of differences in their subject matter \cite[DV prol.]{AnselmDV}. The earliest manuscript evidence we have for the DG, the Bodleian's Rawlinson A. 392., supports the DG being earlier than the DV and \textit{De Libertate Arbitrii}, given that the DG text there, albeit lacking chapter headings, is not essentially different from its final version; while the DV and DLA recensions are clearly earlier.\footnote{That the DG was a difficult text to obtain is suggested by its placing in the Exeter Cathedral ms. 3520 (s. xii in.), where it is found immediately after a series of Anselm's other works from the \textit{De Veritate} to the \textit{Epistola de sacrificio azimi et fermentati}. Where these other works are found in chronological order, the positioning of the DG makes it appear appended to the list, sought not so much for the sake of its subject as the repute of its author. See \cite[p. 85]{Sharpe2009}.} Bec's own twelfth century holdings place the work in a codex where it is preceded by the \textit{monologion}, \textit{proslogion}, and \textit{orationes sive meditationes} - just as it is found in the Rawlinson ms. - rather than in a separate codex containing the other three dialogues.\footnote{Bekker (1885), 262.} Chronologically, this association seems more trustworthy to me than the later one with the other dialogues.

If we take seriously the idea that the DG was motivated by concerns that would erupt in the controversy with Roscelin, then the text should be dated not to the early 1080s, but rather closer to 1090. But such a late date is not plausible for the text. Furthermore, the tone of the text exhibits a high degree of respect and even deference to the dialecticians of Anselm's day, ending with the Master of the dialogue stating `I don't want you to stick to our findings to the extent of stubbornly hanging on to them should someone manage [...] to demolish our result and establish different ones'\cite[DG ch. 21, p. 80]{Anselm1974}. If Anselm had been close to a Trinitarian controversy rooted in a mistaken understanding of dialectic, he would not have been as cordial as he was in the DG with the dialecticians. Lastly, the testimony of Anselm himself on the theological character of the dialogue is negative: the DG is an `introduction to dialectic' precisely in contrast to the dialogues written for the study of sacred scripture \cite[DV prol.]{AnselmDV}.

We can thus sum up the remarks of this section as follows. The DG was likely \textit{written} during Anselm's time as prior of Bec, prior to his election as abbot in 1078, likely before 1076. It grew naturally out of his responsibilities teaching grammar at the abbey, and parts of the text may incorporate one or several conversations Anselm would have had with the monks under his care. The earliest \textit{publication} of the text, i.e. its deliberate circulation as a work of Anselm's in more or less final form, probably postdates Anselm's election as abbot, as well as his \textit{Monologion}, \textit{Proslogion}, and the earlier of his \textit{Orationes sive Meditationes}, on whose success the logic dialogue piggybacks. The association with the other dialogues, grounded in the literary form of the work, is not suggested by the ms. record, and is discouraged by Bec's own 12th c. listing for the work. The purpose of its explicit mention in the DV preface may have been to grant wider signal to the existence of, and thereby wider circulation to, the obscure dialogue, which would likely have been lost but for the fame of its author \cite[p. 16]{Southern1963}. Given the existence of this volume, the plan seems to have worked well enough.
\section{Priscian and Boethius on the grammarian}
In the above sections, we've concentrated on the DG's purpose and the circumstances surrounding its production. Now to better understand its content, we move to Priscian and Boethius' texts.
\subsection{Priscian on \textit{grammaticus}}
Priscian mentions paronyms in the dedicatory epistle, books 2, 3, 8, and 15 of the \textit{Institutiones Grammaticae}, and dedicates book 4 to their discussion. \textit{Grammaticus} appears in books 2, 8, 13, 17, and 18. Notably, \textit{grammaticus} does not appear in book four's discussion of paronyms. Rather, Priscian uses \textit{grammaticus} as an example of a noun or name (\textit{nomen}), specifically of a possessive name.

Priscian's divides names into principal names, also called names of first position, and derivative names. A derivative name is one derived from another name, while a principal name cannot be so derived.\footnote{\cite[\textit{inst. gram.} II, pp. 59-60]{inst.gram.}. The same terminology is found in Boethius at \cite[BC 168A]{BC}.} Though the distinction is a morphological one, it also serves as a semantic indicator, since some semantic functions are only found among derivative terms, while others are common to both principal and derivative names. Priscian then defines a name as follows:
\begin{quote}
The name is that part of speech which distributes a common or proper quality to each among bodies or things subjected. [...] And it shows either a certain common quality of bodies, as man; or a proper [quality], as Virgil; [or] a common [quality] of things, as discipline, art; [or] a proper [quality], as arithmetic of Nicomachus, grammar of Aristarchus.\footnote{\cite[\textit{inst. gram.}, II, pp. 56-57]{inst.gram.}:
\begin{quote}
`Nomen est pars orationis, quae uniquique subiectorum corporum seu rerum communem vel propriam qualitatem distribuit. [...] et communem quidem corporum qualitatem demonstrat, ut homo; propriam vero, ut Virgilius; rerum autem communem, ut disciplina, ars; propriam, ut arithmetica Nicomachi, grammatica Aristarchi'.
\end{quote}}
\end{quote}

Priscian divides names into common and proper names: proper names, e.g. Cicero, are what we would today take to be names in the Millian sense revived by Kripke;\footnote{See \cite{Mill1974}, \cite{Kripke1980}. There are two caveats to this claim: first, Priscian holds even proper names indicate some quality, and so do not lack connotation, as on Mill's analysis; second, Priscian includes the \textit{agnomen} among his proper names, a definite description of a person brought about by their association with some particular quality or happening, e.g. Publius  \textit{Africanus}, William \textit{the Conqueror}. See \textit{inst. gram.} II, p. 58.} common names, also called appellative names, are `naturally common to many'. Appellative names may be taken from substance or quantity or quality, and may be general or specific. Priscian's examples of general appellative names are `animal, body, virtue'; of specific appellative names, `human, stone, \textit{grammaticus}, white, black, big, short'. As this shows, adjectives are included under appellative names as those kinds of names `which it is common to add to other appellative or even proper names signifying substance to manifest a quality or quantity of theirs which can be augmented or diminished without the corruption of the substance, as good animal, great man, wise \textit{grammaticus}, great Homer.\footnote{Hoc autem interest inter proprium et appellatiuum, quod appellatiuum naturaliter commune est multorum, quos eadem substantia siue qualitas uel quantitas generalis specialisue iungit: generalis, ut animal, corpus, uirtus; specialis, ut homo, lapis, grammaticus, albus, niger, grandis, breuis. haec enim quoque, quae a qualitate uel quantitate sumuntur speciali, id est adiectiua, naturaliter communia sunt multorum: adiectiua autem ideo uocantur, quod aliis appellatiuis, quae substantiam significant, uel etiam propriis adici solent ad manifestandam eorum qualitatem uel quantitatem, quae augeri uel minui sine substantiae consumptione possunt, ut bonum animal, magnus homo, sapiens grammaticus, magnus Homerus.' \cite[\textit{inst. gram.} II, p. 58]{inst.gram.}; Cf. \cite[pp. 212-213]{Henry1974}.}

While the appearance of \textit{grammaticus} in Priscian's list of general appellative names leaves open whether it should be understood as signifying substance or quality, its use in the discussion of adjectives makes clear Priscian uses \textit{grammaticus} as a term signifying substance. Further complicating matters, both prior to\footnote{`Proprium est nominis substantiam et qualitatem significare' \cite[\textit{inst. gram.} II, p. 55]{inst.gram.}.} and immediately after these examples Priscian implies some names signify \textit{both} substance and quality.\footnote{`Proprium vero naturaliter uniuscuiusque privatam substantiam et qualitatem significat et in rebus est individuis, quas philosophi atomos vocant, ut Plato, Socrates'. \cite[\textit{inst. gram.} II, pp. 58-59]{inst.gram.}. 

Boschung fails to note this example in his discussion of Priscian, dissolving the conflict Henry and others have seen essentially on the grounds that logic and grammar are separate disciplines with distinct subjects. Thus `instead of opposing or even harmonizing grammar or grammatical ideas Anselm was disinterested in grammatical theory in \textit{De grammatico}. Such disinterest is not surprising \textit{given the respect for the boundaries of a discipline}, i.e. the respect for the proper context of technical terminology, which Anselm shares with his contemporaries' \cite[p. 205; see also p. 219]{Boschung2006}. But the situation is the contrary of what Boschung states: Priscian's taxonomy of different parts of speech is shot through with the same logical-metaphysical distinctions one finds in Aristotle's categories. Priscian distinguishes nouns and verbs based on the signification of time, just as in Aristotle, and specifically mentions the categories of action and passion. Cf. \cite[\textit{inst. gram.} VIII, p. 369]{inst.gram.} \cite[BDI 306B]{BDI1}; his distinction between \textit{ad aliquid} and \textit{quasi ad aliquid} terms builds on Aristotle's account of relation; and his taxonomy of species found among appellative names, as well as his description of them, is grounded in differences in what they signify \cite[\textit{inst. gram.} II, p. 60]{inst.gram.}. Boethius himself says in his dialogue on Porphyry that `we measure out the eight parts of speech via species, differences, and propria'  - presumably referring to the same eight parts found in Priscian - and hence that the science of logic is `of no little use' to grammar \cite[BDP 12C]{BDP}.}

In book 2 Priscian also uses \textit{grammaticus} as an example of a \textit{possessive}, which `signifies something out of those which are possessed with a genitive of principle, as Evandrian sword for Evandrius' sword, and kingly honor for the honor of a king'.\footnote{`possessivum est, quod cum genetivo principalis significat aliquid ex his quae possidentur, ut Evandrius ensis pro Evandri ensis et regius honos pro regis honor'. \cite[\textit{inst. gram.} II, p. 68]{inst.gram.}.  The `genitive' in question is not a genuine Latin genitive, but a Latin adjectival form derived from the form of a Greek genitive.} Priscian says possessives are found only among \textit{derivative} names \cite[\textit{inst. gram.} II, p. 60]{inst.gram.}, and his other examples of possessives include Caesarean, Aristotelian, extraneous, golden, and rhetorical \cite[\textit{inst. gram.} II, p. 69]{inst.gram.}. Possessives arise from various parts of speech, and from both the sublunary and celestial realm.\footnote{Fiunt igitur possesssiva vel a nominibus, ut Caesar Caesareus, vel a verbis, ut opto optativus, vel ab adverbiis, ut extra extraneus, et vel mobilia sunt, ut Martius Martia Martium, vel fixa, ut sacrarium donarium, armarium.' \cite[\textit{inst. gram.} II, p. 69]{inst.gram.}.} At the end of book 2 and again at the beginning of book 4, Priscian classifies possessives among denominatives in a broad sense, allowing for the term denominative to refer to both a broader class and a subclass of denominatives broadly construed \cite[\textit{inst. gram.} pp. 82, 117]{inst.gram.}. Thus, Priscian uses \textit{grammaticus} both as an example of a name signifying substance, and as an example of a denominative in the broader sense.
\subsection{Boethius on denominatives and the grammarian}
Anselm had a broad and deep familiarity with Boethius' work. \textit{Denominativum} and \textit{grammaticus} do not appear in Boethius' \textit{de differentiis topicis}, \textit{in Ciceronis Topica}, \textit{de divisione} or his works on hypothetical and categorical syllogisms. The shorter \textit{Peri Hermeneias} commentary mentions denominatives once. \textit{Grammaticus} does appear, however, in both Boethian commentaries on Porphyry, and most prominently in his \textit{Categories} commentary; and \textit{denominativum} and its variants appear 71 times in the \textit{Categories} commentary. Thus, we focus our discussion on the Boethian expositions of these.
\subsubsection{\textit{Grammaticus} in Boethius' \textit{Isagoge} commentaries}
Porphyry's \textit{Isagoge} discusses the five Porphyrian \textit{predicables} - genus (e.g. animal), species ( e.g. human), difference (e.g. rational), property (\textit{proprium}, e.g. risible), and accident (e.g. tall, black). Avoiding a more detailed exposition of Porphyry's doctrine for its own sake, these five predicables may be thought of as five ways what is signified by a predicate can belong to what it is predicated of. In the shorter dialogue on Porphyry, \textit{grammaticus} appears first in the discussion of difference, then in that of property (\textit{proprium}); in the longer commentary, the term is mentioned in the beginning of book 2 prior to its discussion of genus.\footnote{\cite[BDP 54D-55C]{BDP} \cite[BCP 87A-B]{BCP}. Cf. \cite[BC189D-190B]{BC}.} Of these, the second and third passages do not seem to have been used directly by Anselm in his dialogue. But the main point of the first of these passages was incorporated into Anselm's text.

In the \textit{in Porphyrium dialogi}, \textit{grammaticus} appears as an example of a difference, albeit not a specific difference:
\begin{quote}
Wherefore whatever differences are not profitable for being, nor part of the substance of any species do not have to be called specific differences, even if one species alone have it. For if man navigates, he can be called a navigating animal (\textit{animal navigabile}). But navigating is not [thereby] turned into the substance of man. For neither does man subsist from hence, that he navigates, though no other animal be able to have this, i.e. no [other] animal be able to navigate. it is the same with being a rhetor or grammarian. We do not posit these differences, then, which are not profitable for being, but only bring some art or science to mind, as being specific, even if [only] one given species of animal may have it.\footnote{\cite[BDP 54B-C]{BDP}:
\begin{quote}
Quare illae differentiae quaecunque non prosunt ad esse, neque partes substantiae cujuslibet speciei sunt, specificae differentiae dici non habent, quamvis sola hoc una species habeat. Nam si homo navigat, potest dici animal navigabile. Sed navigare in substantiam hominis non vertitur. Neque enim homo inde subsistit, quia navigat, quamvis hoc nullum aliud animal habere possit, id est, nullum possit animal navigare. Eodem modo et esse rhetorem vel grammaticum. Has igitur differentias quae ad esse non prosunt, sed tantum artem aliquam scientiamque commemorant, non ponimus specificas esse, quamvis una quaelibet animalis id species habeat. 
\end{quote}.}
\end{quote}
The sense of the text is clear enough: there are some differences between beings which are not thereby constitutive of the being of those beings differing from each other, even if the quality in question is only found among a single species. Though to ensure this point is not lost on the beginner, Anselm considers and rejects the same claim Boethius does here - that \textit{grammaticus} is a specific difference - in \cite[DG ch. 13]{Anselm1974}:
\begin{quote}
D.For just as \textit{man} comprises \textit{animal} along with rationality and mortality, so that \textit{man} signifies all three of these, so also since \textit{literate} comprises \textit{man} and literacy, the name \textit{literate} must signify both of these; after all, neither a man without literacy nor literacy apart from a man are ever asserted to be literate.

M. Then if you are correct, `A man displaying literacy', would define and state what is involved in being a literate. 

[...]

D. And so what?

M. It would follow, therefore, that literacy is not an accident, but a substantial difference, \textit{man} being the genus, and \textit{literate} the species. [...] But the treatment of the whole art shows this to be false. \cite[pp. 65-66, alt]{Henry1974}.\footnote{D: [...] Sicut enim homo constat ex animali, et rationalitate, et mortalitate, et idcirco homo significat haec tria, ita grammaticus constat ex homine, et grammatica; et ideo nomen hoc significat utrumque. Nunquam enim dicitur grammaticus aut homo sine grammatica, aut grammatica sine homine. M. Si ergo ita est, ut tu dicis, diffinitio et esse grammatici, est homo sciens grammaticam. [...] D. Quid inde? M. Non est igitur grammatica accidens, sed substantialis differentia; et homo est genus, et grammaticus species [...] quod falsum esse totius artis tractatus ostendit.}
\end{quote}
\subsubsection{\textit{Grammaticus} and \textit{appellatio} in the \textit{Categories} commentary}
In Boethius' translation of Aristotle's \textit{Categories}, \textit{grammaticus} appears first as an example of a paronym (167D; see also 252C-253A); then of a quality (180AB); then in the discussion of primary and secondary substance (189B). Besides these, Aristotle uses \textit{grammatica} to illustrate the relation between things in and said of a subject (169BC); as an example of things admitting of more or less (256CD); and again in discussing relatives and their pertinence to quantity (259CD).

Anselm does not use Aristotle's discussion of relatives in the DG, but manages to address each of the other topics Aristotle connects to the example. The discussions of paronyms and of whether \textit{grammaticus} is a quality appear right at the start of the dialogue; the claim that \textit{grammaticus} admits of more or less appears in an argument from chapter 2, which is revisited in chapter 4; Anselm's discussion of things said in a subject appears in chapter 9; his discussion of primary and secondary substance, in chapter 10. 

Besides this, Anselm's ultimate answer to whether \textit{grammaticus} is a substance or quality is based on a distinction between signification and \textit{appellation}. The term \textit{appellatio} appears nine times in Boethius' \textit{Categories} commentary, and Anselm's use of the term is clearly adapted from its Boethian use.\footnote{The term \textit{appellatio} appears only five times in Priscian: three times at \cite[\textit{inst. gram.} II, pp. 54-55]{inst.gram.}; once at \cite[VIII, p. 443]{inst.gram.}; once at \cite[XI, p. 548]{inst.gram.}. In its first appearance, it is listed as one of the Stoics' five parts of speech, alongside the name, verb, pronoun/article, and connective (\textit{coniunctio}). This list is contrasted with that of the `dialecticians' - presumably followers of Aristotle - who only admitted the noun and verb \cite[\textit{inst. gram.} II, p. 54]{inst.gram.}.

In contrast to Boethius, Priscian's discussion of appellation never treats it as a semantic function of a term, but rather as a purported part of speech, one Priscian himself prefers to subsume under the name. See \cite[\textit{inst. gram.} II. p. 55]{inst.gram.}.}

Boethius' most extensive discussion of \textit{grammaticus} and \textit{grammatica} occurs in his exposition of Aristotle's remarks on things in and said of a subject. According to Boethius, Aristotle's distinction between things that are and are not in a subject is one between substances and accidents; while that between things that are and are not said of a subject is one between universals and particulars.  Nothing prevents an object belonging to one half of one distinction from belonging to either part of the other. Hence four options are possible: universal substances, universal accidents, particular substances, and particular accidents. Examples of universal substances include man and animal. Particular substances are individuals, e.g. Plato and Socrates. Boethius gives science as an example of a universal accident, while the grammar of Aristarchus (\textit{grammatica Aristarchi}) serves as an example of a particular accident \cite[BC169C-170C; 171D]{BC}. In this connection, Boethius uses \textit{grammatica} in two different ways: first, as an example of something subjected to a universal accident, sc. \textit{scientia}; second, when ascribed to an individual, as itself an example of a particular accident. Boethius writes:

\begin{quote}
Therefore, whatever thing is itself in a subject, but is predicated of no subject, is a particular accident, as this grammar is, i.e. Aristarchus', or the individual grammar of another man: since it is of an individual man, it, too, becomes individual and particular; therefore, because this grammar is in the soul, it is an accident, and because it is predicated of no subject, it is particular; just as Aristarchus himself is said of no subject, so, too, his grammar is predicated of no subject. [Aristotle] does not say that grammar itself is particular, but that some grammar, i.e. the grammar of some individual man, namely what a particular man retains by his own knowledge.

And because he put forth an incorporeal accident that happens to the soul, i.e. grammar (which would be in the soul), he puts forth another bodily example; for he also says some white is in a subjected body, since every color is in a body.\footnote{\cite[BC 171D-172A]{BC}:
\begin{quote}
Ergo quaecunque res ipsa quidem in subjecto est, sed si de nullo subjecto praedicatur, accidens est particulare, ut est quaedam grammatica, id est Aristarchi, vel alicujus hominis individua grammatica: illa enim quoniam individui hominis, ipsa quoque facta est individua et particularis; ergo quoniam quaedam grammatica in anima est, accidens est, et quoniam de nullo subjecto praedicatur, particularis est; quemadmodum enim ipse Aristarchus de nullo subjecto dicitur, ita quoque ejus grammatica de nullo subjecto praedicatur. Non autem dicit quod ipsa grammatica particularis est, sed quod quaedam grammatica, id est alicujus hominis individui grammatica, quam scilicet homo particularis propria retinet cognitione. 

Et quoniam incorporale accidens posuit quod animae accideret, id est grammaticam, quae esset in anima, ponit quoque aliud exemplum corporale; ait enim et quoddam album in subjecto est corpore, omnis enim color est in corpore.
\end{quote}
Cf. \cite[BC 174A-175A; 246B-247B]{BC}.}
\end{quote}

In the above passage, we find the same pairing of \textit{grammatica} and white (\textit{albus}) we find throughout both Aristotle and Anselm's texts. In Boethius' exposition, both are used in examples of particular qualities: one pertaining to the soul, the other to the body.\footnote{\textit{Albus} first appears in ch. 9 of Anselm's text, and appears more prominently than the dialogue's titular quality in chapters 14-15 and 19-21.}

The distinctions between things in and said of a subject are found nowhere in Priscian, though Priscian likely would have known of them. But these Aristotelian distinctions and their Boethian discussion appear especially important in chapters. 9 and 10 of Anselm's DG. There, the disciple attempts to show that no \textit{grammaticus} is a man on the grounds that \textit{grammaticus} pertains to things in a subject, while man does not: 

\begin{quote}
D.Aristotle showed that a literate is one of those things which are in a subject, but no man is in a subject. Hence no literate is a man. 

M. Aristotle didn't want this consequence to be drawn from what he said, for the same Aristotle calls a certain man, both a man and a literate animal.\footnote{BC 189B; cf. BC 182CD; BCP 87A-B.} 

D. How then can this syllogism be disbanded?

M. Tell me know, when you speak to me about a literate, of what should I understand you to speak - of the name, or of the things that it signifies?

D. Of the things.

M. What things does it signify then? 

D. Man and literacy.

M. On hearing this name, then, I may understand man or literacy; and when I speak of a literate, I may be speaking of either man or literacy.

D. It should be so.

M. Tell me then whether man is a substance or in a subject?

D. Not in a subject, but a substance.

M. Literacy is a quality, and in a subject?

D. It is both.

M. What then, is so wondrous if someone says that literate is a substance and is not in a subject after man; and literate is a quality and in a subject after literacy? \cite[pp. 61-62, alt.]{Henry1974}\footnote{\cite[DG 9]{Anselm1974}:
\begin{quote}
D. Aristoteles ostendit grammaticum eorum esse quae sunt in subjecto: et nullus homo est in subjecto: quare nullus grammaticus homo. 

M. Noluit Aristoteles hoc consequi ex suis dictis: nam idem Aristoteles dicit quemdam hominem, et hominem et animal grammaticum. 

D. Quomodo ergo dissolvitur iste syllogismus? 

M. Responde mihi: cum loqueris mihi de grammatico, unde intelligam te loqui de hoc nomine, an de rebus quas significat? 

D. De rebus. 

M. Quas ergo res significat? 

D. Hominem, et grammaticam. 

M. Audito ergo hoc nomine, intelligam hominem aut grammaticam; et loquens de grammatico loquar de homine, aut de grammatica. 

D. Ita oportet. 

M. Dic ergo: Homo est substantia, an in subjecto? 

D. Non est in subjecto, sed est substantia. 

M. Grammatica est qualitas, et in subjecto. 

D. Utrumque est. 

M. Quid ergo mirum, si quis dicit quia grammaticus est substantia, et non est in subjecto, secundum hominem; et grammaticus est qualitas, et in subjecto, secundum grammaticam?
\end{quote}}
\end{quote}

Here, we can see both Anselm's debt to and improvement upon the Boethian reading of Aristotle. Neither Aristotle nor Boethius discuss whether \textit{grammaticus} is in a subject, but only \textit{grammatica}. The disciple of Anselm's dialogue assumes \textit{grammaticus} is in a subject on the grounds that \textit{grammaticus} signifies \textit{grammatica}. But in asking whether the disciple is speaking of the name or the thing, the master hints at a confusion of the in a subject/of a subject distinctions in the disciple's argument, grounded in one between words and things. The disciple assumes \textit{grammaticus} is in a subject because it signifies \textit{grammatica}. But because signification occurs in many ways, and because the signification of \textit{grammaticus} is more than merely \textit{grammatica}, the disciple is unable to infer that the word \textit{grammaticus}' signifying something in a subject precludes it from signifying something not in a subject in some other way. Thus, Anselm makes use of reduplication to provide an answer to what \textit{grammaticus} signifies in the context of the disciple's argument: it signifies something in a subject \textit{qua} grammar, (\textit{secundum grammatica}), but a substance \textit{qua} man (\textit{secundum hominem}).\footnote{Cf. \cite[pp. 110-116]{Back1996}.}

Both the same confusion on the disciple's part and the same solution on the master's part reoccur in DG 10: \textit{Grammaticus} is denied to be either primary or secondary substance because it is in a subject, hence not a substance; said of many, hence not primary substance; and is not a genus, hence not secondary substance. But the master replies it is only so \textit{secundum aliquid}, and not in every sense.

In short, neither Aristotle nor Boethius envision the in/not in a subject distinction as applying to any term whatsoever: rather, they apply the distinction exclusively to simple things in the various categories, which are referred to by non-derivative terms. Anselm himself makes this point more clearly than either Aristotle or Boethius themselves at DG 17: 
\begin{quote}
Aristotle's main intention in that book wasn't to show [that everything that is is substance or quality or quality, etc.], but rather to show how every noun or verb signifies one or other of them. He didn't intend to show what individual things are, nor yet of what things individual words can be appellative; but rather, what things they signify. But since words only signify things, he had, in order to indicate what it is that words signify, to indicate what those things could be. For without going into further detail, the division he undertook at the opening of his work on the \textit{Categories} sufficiently demonstrates what I am saying. He doesn't say `Of those that are, each is either substance or quantity...etc.'; nor does he say `of things expressed without combination, each is called (\textit{appellatur}) substance or quantity...', but rather `of things expressed without combination, each signifies substance or quantity.... \footnote{\cite[pp. 72-73, alt.]{Henry1974}:
\begin{quote}
non tamen fuit principalis intentio Aristotelis hoc in illo libro ostendere; sed quoniam omne nomen, vel verbum aliquid horum significat: non enim intendebat ostendere quid sint singulae res, nec quarum rerum sint appellativae singulae voces; sed quarum significativa sint. Sed quoniam voces non significant nisi res; dicendo quid sit quod voces significant, necesse fuit dicere quid sint res. Nam (ut alia taceam) sufficienter hoc quod dico, divisio, quam facit in principio tractatus Categoriarum, ostendit; non enim ait: Eorum quae sunt, singulum est substantia, aut quantitas, etc. Nec ait: Eorum quae secundum nullam complexionem dicuntur, singulo aut substantia appellatur, aut quantitas; sed ait: Eorum, quae secundum nullam complexionem dicuntur, singulum aut substantiam significat aut quantitatem.
\end{quote}}
\end{quote} 
A facile extension of the distinction to the \textit{significata} of all categorematic terms generally and paronymous terms specifically might expect a one-to-one mapping of each term to the primary significate from which its name is derived. But such a mapping would betray the polysemy of signification: not every term signifies only one thing as a unified whole (\textit{ut unum}). See \cite[DG ch. 12, 19]{Anselm1974}. Rather, Anselm's introduction of reduplication allows for the desired extension of the distinction, albeit without either `reading off' reality from language in too straightforward of a fashion, or wholly severing the connection between words and things.

The last major area of Boethian influence on Anselm's dialogue involves appellation and its distinction from signification. Boethius first uses \textit{appellatio} at the beginning of \textit{Categories} commentary, in introducing the distinction between first and second impositions of a name. Boethius writes:

\begin{quote}
That placing of a name was first, then, by which it would designate something subjected to understanding or the senses; second, that examination by which the particular properties of names and figures would be perceived, so that the first name may be itself a word for a thing. 

For instance, when something is called a man (\textit{homo}), that the word itself, i.e. man, is called a name does not refer to the signification of the name itself, but to the figure, such that it can be inflected in cases. 

Therefore, the first placing of a name is made following the signification of the word; the second, following the figure. And the first placing exists so that names might be placed on things; the second, so that names themselves might be designated by other names. For when man is a word for a substance placed under [it], that which is said, man, is a name of man, which is the appellation of that name. For we say, what kind of word is man? And one properly responds, a name.\footnote{\cite[BC 159BC]{BC}:
\begin{quote}
Prima igitur illa fuit nominum positio, per quam vel intellectui subjecta vel sensibus designaret. Secunda consideratio, qua singulas proprietates nominum figurasque perspicerent, ita ut primum nomen sit ipsum rei vocabulum. 

Ut, verbi gratia, cum quaelibet res homo dicatur, quod autem ipsum vocabulum, id est homo, nomen vocatur, non ad significationem nominis ipsius refertur, sed ad figuram, idcirco quod possit casibus inflecti. 

Ergo prima positio nominis secundum significationem vocabuli facta est, secunda vero secundum figuram: et est prima positio, ut nomina rebus imponerentur, secunda vero ut aliis nominibus ipsa nomina designarentur. Nam cum homo vocabulum sit subjectae substantiae, id quod dicitur homo, nomen est hominis, quod ipsius nominis appellatio est. Dicimus enim, quale vocabulum est homo? et proprie respondetur, nomen.
\end{quote}}
\end{quote}

In the above passage, we find a distinction between appellation and signification, albeit not one treated in the passage for its own sake. Only the appellation of a name according to its first placing is taken from its signification. Hence, signification and appellation may differ. 

On the Boethian account, there is no distinction between man and `man'. Rather, man, the same term used to refer to human beings, may also be used to refer to itself as a term. Hence, we have a single term capable of having different appellations, rather than, as in a Tarskian language, each term having a single referent, with the use of quote marks to indicate names of names.

Because the above passage uses man as its example, it is unclear whether the appellation of a term according to its first placing must be identical to its signification. Conveniently, Boethius' use of \textit{appellatio} in connection with \textit{grammaticus} shows this is not so in two different places in the \textit{Categories} commentary. The first explains why what is most properly called substance is particular substance, and is relevant to the point addressed in the above passage from DG 9 \cite[BC182CD]{BC}. Here is the second passage:

\begin{quote}
[Aristotle] says there are many qualities from whose being placed [under] and called by their proper name others are said paronymously [...]. For since whiteness is the name of a certain quality, from it [something] is called white. It is the same way with grammar: since it is the name of a thing, from this others are called such. For grammarians are named from grammar, and this is so even in the plural, so that the name being placed, if something is called so after those qualities the appellation will be derived from these same qualities.\footnote{\cite[BC 253A]{BC}:
\begin{quote}
Multae, inquit, sunt qualitates, quibus positis et proprio nomine nuncupatis, ab his alia denominative dicuntur [...]. Nam cum albedo cujusdam nomen sit qualitatis, ab eo dicitur albus. Eodem quoque modo et grammatica, cum rei sit nomen, ab ipso quales dicuntur. Grammatici enim a grammatica nominantur, atque hoc est in pluribus, ut posito nomine si quid secundum ipsas qualitates quale dicitur, ex his ipsis qualitatibus appellatio derivetur.
\end{quote}}
\end{quote}

Where Boethius' earlier example might have suggested the \textit{significatum} of a term is identical to its \textit{appellatum} on first imposition, this example shows that need not be so. The appellation of a paronymous term signifying quality is derived from the quality after which it is named; and may be distinct from the quality itself. Since \textit{grammaticus} refers not to grammar, but to things possessing it, we have an example where the appellation of a term, though derived from its signification, remains distinct from it.

Thus we find Anselm again making explicit a point only made by Boethius in passing. ``\textit{Grammaticus} does not signify man and grammar as one, but signifies grammar \textit{per se}, and man by another (\textit{per aliud}). And though this name is appellative of man, it is not properly called significative of it; and though it is significative of grammar, it is not properly appellative of it.''\footnote{\cite[DG ch. 12]{Anselm1974}: `Grammaticus vero non significat hominem, et grammaticam, ut unum; sed grammaticam per se, et hominem per aliud significat: et hoc nomen quamvis sit appellativum hominis, non tamen proprie dicitur ejus significativum: et licet sit significativum grammaticae, non tamen proprie est ejus appellativum.'}
\section{Conclusion}
Now let us take stock of the above. 

The immediate \textit{question} the \textit{De Grammatico} concerns itself with is whether \textit{grammaticus} is a substance or quality. The immediate concern behind the question is how to interpret paronyms. This concern, however, mediates a number of others, the most prominent of which being the nature of and intent of Aristotle's division of the categories of being.

The DG was likely written during Anselm's time as prior of Bec, at some point between the time of the earliest letters and his receiving the office of abbot. An earlier date of composition than this would be out of keeping with Anselm's monastic state at that time, as well as the absence of letters from this period, and so can be ruled out with a fairly high level of certainty. A date of composition in the 1280s, though more likely than a very early one, would be less in keeping both with Anselm's more theological prerogatives as abbot of the abbey, as well as with the cordial, deferential tone taken in the dialogue towards modern dialecticians as the Trinitarian controversy with Roscelin approaches. Even so, it seems likely the earliest authorized \textit{distribution} of the dialogue, and thus its final form, postdates Anselm's election as abbot, presupposing both the authority of the position of abbot as well as the prior success of the \textit{monologion}, \textit{proslogion} and \textit{prayers and meditations} - works the DG is found immediately after in both its earliest extant witness and Bec's own 12th c. copy of the work. Thus, while a date of \textit{composition} in the 1070s accords best with the responsibilities Anselm would have had as prior as witnessed in his letters, and adds sense to the dialogue's being coupled only at a distance with Anselm's other three theological dialogues, the manuscript and library records fit best with a later date of \textit{dissemination} in the early 1280s.

The \textit{De Grammatico}'s central problem arises out of a disparity between the interpretation of paronyms in Priscian, on the one hand, and Aristotle and Boethius, on the other. On a basic level, the pedagogical work of Anselm's text is to address and remedy the disparity between Aristotle and Priscian's uses of the dialogue's titular example, providing a comparative supplement to Aristotle's \textit{Categories}. Superficially, the text is thus a standard medieval exercise in reconciling conflicting authorities by positing distinctions. But on a deeper level, Anselm exalts the Boethian/Aristotelian treatment of the titular example while downplaying Priscian's.

Priscian uses \textit{grammaticus} as an example of a term signifying substance. Aristotle and Boethius, on the other hand, assume \textit{grammaticus} signifies a quality. The conflict between these positions would have arisen out of standard monastic instruction in the \textit{trivium} at Bec at the time: Priscian being a standard authority on grammar, the Boethian commentaries on Aristotle and Porphyry being canonical texts in dialectic, both being listed in a common codex for teaching the Trivium at Bec according to its 12th c. holdings. Though the dialogue form of the \textit{De Grammatico} owes much to Augustine's dialogues and the \textit{De Magistro} in particular, it owes its problematic to Priscian and Aristotle, while the resolution to that problem comes out of Anselm's own application and extension of an essentially Boethian framework. Where Priscian takes \textit{grammaticus} to be a common possessive name signifying substance (and likely quality as well), uses \textit{appellativum} as a synonym for `common' in his discussion of common names, and only mentions \textit{appellatio} as a purported separate part of speech which he dismisses, Anselm allows that \textit{grammaticus} is appellative, but not significative of substance, makes explicit the notion of appellation implied in Boethius as a semantic function from a name to its referent, and provides a way of extending the Aristotelian/Boethian account of signification, originally intended only as a taxonomy for the simple things referred to by principal names, to categorical terms generally and paronymous terms like \textit{grammaticus} specifically by means of reduplication. 

Focusing on the contrast between Priscian and Aristotle/Boethius thus highlights the central \textit{polemical} point of the dialogue: what Priscian's account mistakenly suggests, if not supplemented by Aristotle and Boethius, is a grammatical metaphysics (or metaphysical grammar), in which every term of a language is mapped to its object(s) in accordance with the different parts of speech, and without due regard for the distinction between appellation and signification. It is this sort of project that flourished in the speculative grammmar of the thirteenth centuries, prompting the more economical, reference-based approach of nominalist metaphysics as a reaction to it - thus leaving Priscian a father of realism and a grandfather of nominalism. 

But for Anselm, Aristotle `didn't intend to show what individual things are, nor yet of what things individual words can be appellative; but rather, what things they signify. But since words only signify things, he had, in order to indicate what it is that words signify, to indicate what those things could be' \cite[DG ch. 17, p. 72, alt.]{Anselm1974}. Against both the nominalism of an Ockham or Quine, and the realism of a Walter Burley or a David Lewis, we find the `realism' of Anselm's logic does not consist so much in its positing certain universals as the \textit{appellata} of terms, as in its rejecting the primacy of reference in the first place. These later understandings continue to govern nearly all aspects of metaphysics and semantics today. Perhaps in his remove from both nominalist and realist projects and their impasses, the monk of Bec points a way to a new path.

\nocite{Anselm1974}
\nocite{Anselm1968}
\nocite{Anselm1990}
\nocite{RSB}
\nocite{BDP}
\nocite{BCP}
\nocite{BC}
\nocite{BDI1}
\nocite{VA}
\nocite{Henry1974}
\nocite{ScotusMetaph}
\nocite{inst.gram.}
\nocite{LanfrancDecreta}
\nocite{PL}
\nocite{AbelardHC}
\nocite{AquinasPA}

%\begin{verse}
%Anselm of Canterbury (1968). \textit{Sancti Anselmi Opera Omnia} F. S. Schmitt (ed) (Stuttgart-Bad Canstatt: Fromman Verlag). [AOO]

%Anselm of Canterbury (1974) \textit{De grammatico}, in Henry (1974), 48-80. Cited by chapter. [DG]

%Anselm of Canterbury (1990) \textit{The Letters of Anselm of Canterbury}, Walter Fr\"{o}hlich (trans.) (Kalamazoo, MI: Cistercian Publications). Cited by letter number, then page number. [Ep.]

%Anselm of Canterbury (2007a) \textit{Basic Writings}, Thomas Williams (ed/trans) (Indianapolis: Hackett).

%Anselm of Canterbury (2007b) \textit{Proslogion}, in Anselm 2007a, 75-98. Cited by chapter. [Pros.] 

%Anselm of Canterbury (2007c) \textit{On Truth}, in Anselm 2007a, 117-144. Cited by chapter. [DV]

%B\"{a}ck, Allen (1996) \textit{On reduplication: logical theories of qualification} (Leiden: Brill).

%Benedict of Nursia (1955). \textit{Regula monachorum}. Textus critico-practicus sec. cod. Sangall. 914, adiuncta verborum concordantia cura D. Philberti Schmitz, addita Christianae Mohrmann enarratione in linguam S. Benedicti. Editio altera emendata (Maredsous). Cited by chapter. [RSB].

%Bekker, Gustavus (1885). \textit{Catalogi bibliothecarum antiqui} (Bonn: Max Cohen et filium).

%Boethius. \textit{In Porphyrium dialogi a Victorino translati}, in PL 64, 9A-70D. [BDP]

%Boethius. \textit{Commentaria in Porphyrium a se translatum}, in PL 64, 71A-158D. [BCP]

%Boethius. \textit{In categorias Aristotelis libri quatuor}, in PL 64, 159A-294C. [BC]

%Boethius. \textit{In librum Aristotelis de interpretatione libri duo: editio prima, seu minora commentaria}, in PL 64, 293D-392D.

%Boethius. \textit{In librum Aristotelis de interpretatione libri sex: editio secunda, seu majora commentaria}, in PL 64, 393A-640A.

%Boschung, Peter (2006) \textit{From a topical point of view: dialectic in Anselm of Canterbury's de grammatico} (Leiden: Brill).

%Cerezo, Maria (2015) `Anselm of Canterbury's theory of meaning: analysis of some semantic distinctions in \textit{de grammatico}', \textit{Vivarium} 53, 194-220.

%Eadmer of Canterbury. \textit{Vita Sancti Anselmi}. In PL 158, 49-118. [VA]

%Ebbesen, Sten (1979). `The Dead Man is Alive', \textit{Synthese} 40, 43-70.

%Ebbesen, Sten (1987) `Talking about what is no more. Texts by Peter of Cornwall (?), Richard of Clive, Simon of Faversham, and Radulphus Brito.' \textit{Cahiers de L'Institut du Moyen-\^{A}ge Grec et Latin} 55, 135-68.

%Ebbesen, Sten (1993) `Animal est omnis homo. Questions and sophismata by Peter of Auvergne, Radulphus Brito, William Bonkes, and Others', \textit{Cahiers de L'Institut du Moyen-\^{A}ge Grec et Latin} 63, 145-208.

%Galonnier, Alain (1986) `(Au sujet) de Grammarien', in A. Galonnier, M. Corbin, and R. de Ravinel (eds) \textit{Le Grammarien; De la v\'{e}rit\'{e}; la libert\'{e} du choix; la chute du diable: Anselm de Cantorb\'{e}ry} (Paris: Cerf), 25-49.

%Galonnier, Alain (1987) `Le \textit{de grammatico} et l'origine de la theorie des propriet\'{e}s des termes', in J. Jolivet and A. de Libera (eds) \textit{Gilbert de Poitiers et ses contemporains: aux origines de la logica modernorum} (Naples: Bibliopolis), 353-75.

%Galonnier, Alain (1996) `Sur quelques aspects annonciateurs de la litt\'{e}rature sophismatique dans le \textit{de grammatico}', in D. E. Luscombe and G. Evans (eds) \textit{Anselm: Aosta, Bec, and Canterbury: Papers in Commemoration of the Nine-Hundredth Anniversary of Anselm's Enthronement as Archbishop, 25 September 1093} (Sheffield: Academic Press), 209-28.

%Green-Pedersen, Niels J\o{}rgen (1980a) `Two early anonymous tracts on consequences' \textit{Cahiers de L'Institut du Moyen-\^{A}ge Grec et Latin} 35, 1-28.

%Green-Pedersen, Niels J\o{}rgen (1980b) `Walter Burley's \textit{de consequentiis}: an edition', \textit{Franciscan Studies} 40, 102-166.

%Green-Pedersen, Niels J\o{}rgen (1984) \textit{The tradition of the topics in the middle ages: commentaries on Aristotle's and Boethius' `topics'}, (Munich: Philosophia Verlag).

%Gasper, Giles (2004) \textit{Anselm of Canterbury and his Theological Inheritance} (Hampshire: Ashgate).

%Henry D. P. (1960) `Saint Anselm's de `Grammatico'' \textit{The Philosophical Quarterly} 10, 115-126.

%Henry D. P. (1974). \textit{Commentary on de grammatico: the historical-logical dimensions of a dialogue of St. Anselm's} (Dordrecht: D. Reidel).

%Holopainen, Toivo J. (2007) `Anselm's \textit{argumentum} and the early medieval theory of argument', \textit{Vivarium} 45, 1-29.

%Holopainen, Toivo J. (2013) `Review of \textit{Anselm of Canterbury and the desire for the word} by Eileen C. Sweeney', \textit{Journal of the History of Philosophy} 51, 314-15.

%John Duns Scotus (1997) \textit{Quaestiones super libros metaphysicorum Aristotelis Books 1-5} in G. Etzkorn et al. (eds) \textit{Opera Philosophica}, vol. 3 (St. Bonaventure, NY: Franciscan Institute). Cited by book, question, and paragraph.

%Jordan, Mark D, and Kent Emery Jr. (1992). \textit{Ad Litteram: Authoritative Texts and Their Medieval Readers} (Notre Dame, IN: University of Notre Dame Press).

%Keil, H. \textit{Prisciani Institutiones}. In \textit{Grammatici Latini} http://kaali.\newline linguist.jussieu.fr/CGL/text.jsp?id=T43. Cited by book then page number. [inst. gram.]

%Kripke, Saul (1980) \textit{Naming and Necessity}, (Cambridge, MA: Harvard University Press).

%Lanfranc of Canterbury, \textit{Decreta}. In PL 150, 443C-516B.

%Lewry, Osmund. (1981), `Boethian Logic in the Medieval West', in M. Gibson (ed), \textit{Boethius: His Life, Thought, and Influence}, (Oxford: Oxford University Press), 90-134.

%McCord Adams, Marilyn (2000) `Re-reading \textit{de grammatico}, or Anselm's introduction to Aristotle's \textit{categories}', \textit{Documenti e studi sulla tradizione filosofica medievale} 11, 83-112.

%McMahon, Robert (2004) \textit{Understanding the Medieval Meditative Ascent} (Washington, DC: Catholic University of America Press).

%Mews, Constant J. (1992) `Nominalism and theology before Abaelard: new light on Roscelin of Compi\`{e}gne' \textit{Vivarium} 30, 4-33.

%Migne, J. P. \textit{Patrologia Latina}

%Mill, J. S. (1974) \textit{A system of logic}, in \textit{The collected works of John Stuart Mill}, vols. 7-8. (Toronto: University of Toronto Press).

%Novikoff, Alex J. (2011) `Anselm, dialogue, and the rise of scholastic disputation' \textit{Speculum} 86, 387-418.

%Peter Abelard (1922) {Historia Calamitatum}, http://legacy.fordham.edu/\newline halsall/basis/abelard-histcal.asp. Cited by chapter. [HC]

%Sharpe, Richard (2009) `Anselm as author: publishing in the late eleventh century', \textit{Journal of Medieval Latin} 19, 1-87.

%Southern, R. W. (1963) \textit{Saint Anselm and his biographer: a study of monastic life and thought} (Cambridge: Cambridge University Press).

%Southern, R. W. (1990) \textit{Saint Anselm: a portrait in a landscape} (Cambridge: Cambridge University Press).

%Steiger, Lothar (1969) `Contexe Syllogismos. \"{U}ber Die Kunst Und Bedeutung Der Topik Bei Anselm', \textit{Analecta Anselmiana} 1, 107-143.

%Sweeney, Eileen (2012). \textit{Saint Anselm and the desire for the word} (Washington, DC: Catholic University of America Press).

%Thomas Aquinas. \textit{Expositio libri Posteriorum Analyticorum}. http://\newline www.corpusthomisticum.org/cpa1.html. Cited by book, lecture, and paragraph. [In post. an.]

%Uckelman, Sara L. (2012) `The reception of Saint Anselm's logic in the twentieth and twenty-first centuries', in G. E. M. Gasper and I. Logan (eds), \textit{Saint Anselm of Canterbury and His Legacy} (Durham: Institute of Medieval and Renaissance Studies), 405-26.

%Visser, Sandra and Thomas Williams (2009) \textit{Anselm} (Oxford: Oxford University Press).

%Williams, Thomas (2015). `Saint Anselm' \textit{Stanford Encyclopedia of Philosophy}. http://plato.stanford.edu/entries/anselm/. Cited by section.
%\end{verse}

\bibliography{jacob}
\bibliographystyle{plain}
\end{document}