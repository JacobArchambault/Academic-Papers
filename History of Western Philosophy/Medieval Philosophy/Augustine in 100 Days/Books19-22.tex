\chapter{Books 19-22}
\section{Book 19}
\subsection{Day 78: chapter 1}
\subsection{Day 79: chapters 2-3}
\subsection{Day 80: chapter 4}
In this chapter, Augustine first states the opinion of the church on man's ultimate end. He then shows in what the opinions of the philosophers are mistaken, namely, in their belief that happiness, whether bodily or psychic, can be achieved in this life. To show this, he first refutes the opinion of those who locate man's end in the bodily good which is satisfaction of his basic needs, then the opinion of those who locate it in the psychic good which is virtue. In doing the former, he first examines man's basic natural needs, beginning with lower needs and moving to higher ones. In doing the latter, he successively examines the four cardinal virtues, showing that the acquisition of none of them is sufficient for earthly happiness.

Throughout the following chapters, Augustine's main argument is as follows:
\begin{itemize}
	\item[1.] Happiness is not attainable in this life.
	\item[2.] Man's ultimate end is happiness.
	\item[Conclusion] Man's ultimate end is not attainable in this life.
\end{itemize}

To prove the major premise, he argues as follows:

\begin{itemize}
	\item [1.] If happiness is attainable in this life, it is found either in attendance to basic needs, or in virtue
	\item[2.] Happiness is not found in attending to basic needs
	\item[2.] Happiness is not found in virtue
	\item[Conclusion] Therefore, happiness is not attainable in this life.
\end{itemize}
\subsection{Day 81: chapters 5-6}
In chapters five through seven, Augustine begins by describing in what the philosophers were correct, namely in their belief in the social character of the good life. He then shows how this social character nevertheless fails to secure the good in natural societies, examining the various levels of society. First, he examines the natural family unit, then the city, then the world, showing in each case the ways in which their social character leads to misfortune and unhappiness.
\subsection{Day 82: chapters 7-11}
After having examined the three main levels of society, Augustine then discusses certain other societies: first, that of friendship; then that of the cosmos. He shows none of these to be entirely free from unhappiness. Friendship is accompanied by betrayal when false, and death when sweet. The cosmic companionship with the angels cannot guard against the possibility of demonic deception. 

In chapter ten, Augustine summarizes, contrasting the earthly prospects for deception with the certainty granted in heaven.
\subsection{Day 83: chapter 12}
In the middle of chapter ten, Augustine begins to support the claim that the end of man, heavenly beatitude, is peace. He continues expounding on this idea through chapter thirteen.
\subsection{Day 84: chapters 13-14}


In chapter thirteen, Augustine summarizes his remarks on peace, provides a definition, then draws some consequences from this definition. He first runs through the examples of peace for each kind of thing, moving successively from lower beings to higher beings, while simultaneously moving from the peace of the individual to that of the variuos forms of society. He begins with the peace of the body, of the soul, then of the creature composed of both body and soul; of man with God, of the secular city, then of the heavenly city, i.e. the Church. He concludes with a definition: for each thing peace is `the calm that comes of order (\textit{Pax omnium rerum tranquillitas ordinis}). Order, in turn, he defines as `an arrangement of like and unlike things whereby each of them is disposed in its proper place'.

Augustine then draws some consequences from this definition. Doing this, he begins with some general remarks, then provides two examples illustrating these remarks. In his general remarks, he shows that unhappiness is a consequence of lack of peace, and more generally that various forms of disorder, which go against peace, nevertheless presuppose peace for their intelligibility. Thus, the concept of war - i.e. the absence of peace in a polis - lacks sense without reference to peace itself; that of unhappiness - the absence of peace in a soul - without reference to happiness; that of pain - the absence of the peace of the body - without reference to life. 

He then applies the above to the specific cases of a) the unblessedness of the devil, and b) that of Adam upon his fall. 

\subsection{Day 85: chapters 14-17}
In chapter fourteen, Augustine delves further into the peace of the individual biological organism: first discussing the peace of beasts, then that of man. In the middle of the chapter (at `Meanwhile, God teaches') he begins to discuss the peace of the man in society. He opens the latter discussion by discussing the peace of the biological family, then that of the broader household, i.e. the family and any servants it may have. He then discusses the peace of the earthly city, then the heavenly city, and the conflict of this city with the earthly one. Augustine completes his discussion of peace with respect to the community at the end of chapter seventeen. 
\subsection{Day 86: chapters 18-22}
After discussing peace with respect to the various levels of community, Augustine treats it in connection with various other divisions which Varro makes regarding the good life. 

Augustine first discusses the customs of the Cynics, and regards the church as indifferent to these matters. Next, he discusses peace in connection with the three modes of life, i.e. active, contemplative, and mixed.
\subsection{Day 87: chapter 23t
\subsection{Day 88: chapters 24-28}
\section{Book 20}
\section{Book 21}
\section{Book 22}