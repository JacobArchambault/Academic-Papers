\documentclass[]{article}

%opening
\title{Treatise 1: On the prerequisites to logic (On Universals, Book 1, c. 1-5)}
\author{Albertus Magnus}
\begin{document}

\maketitle

\begin{abstract}

\end{abstract}

\section{Whether logic is a positive science}

Attending to logic,\footnote{A positive science (\textit{scientia specialis}) is a science distinct from others on account of its being directed to one unified \textit{species}. The question of whether logic is a \textit{scientia specialis}, then, has two components: unity, i.e. whether logic is a particular science distinct from the others; and objectivity, i.e. whether it is so by virtue of being directed toward a particular image or sort. Albert answers both components affirmatively. The question is not directly concerned with whether logic is special in the sense of being distinguished or more important than other sciences.} we must first consider what kind of science logic is and whether it is any part of philosophy, and what it is necessary and useful for; and from there, what it is about and how it is divided, so that having all its parts, we may know when it is perfectly or imperfectly taught or recorded.

The ancients seem to have said different things about all these matters. For some of the ancients contended that logic is not a science at all, saying that which is the method of every science or doctrine cannot be a science. But the method of every science is logic: since everyone who teaches or tries to establish something uses some kind of persuasion; but every persuasion by words is logic, whether the one who teaches or persuades uses syllogism or  enthymeme or induction or example. The method of every teaching, then, is logic. And so the method of every teaching is the science of logic. Therefore, there is no positive teaching which must be called `logic'.

On account of this, Aristotle in the 2nd book of his first philosophy, and Averroes likewise says every method of science is itself a science, which is and is called logic, and that a science and a method of science cannot be learned simultaneously; but one ought first to learn the method of science, and then attempt to learn the science throught the now completely-comprehended method. 

But these did not sufficiently consider that although there are many sciences and each has its special method which differs from the method of the others, there is still one common method of every science by something common which is in every science. And this is that by the investigation of reason it comes from the familiar `to the knowledge (\textit{scientiam}) of the unfamiliar'.\footnote{For consistency, I translate \textit{scientia} as `knowledge', and \textit{cognitio} as `familiarity'. Translating \textit{cognitio} as `knowledge' is both too strong and leads to unnecessary ambiguity; while translating it as `cognition', given discussions of cognition today, suggests the presence of a techical, peculiarly philosophical notion that isn't present in the Latin text. \textit{Cognitio} is just everyday knowledge, the kind of unthematized acquaintance with things that serves as the starting point for scientific knowledge.} For this happens in every science in whatever way it is called a science, whether it be demonstrative or non demonstrative. And thanks to this commonality which is in every science, there is one common method of every science. And this method is, by the act of reason which is called `ratiocination' or `argumentation', to proceed from familiarity with the familiar to the knowledge of that which was unfamiliar.

On account of this Isaac, defining reason in the book \textit{On Definitions}, says that `reason is' the power of an intellectual soul `making cause traverse to caused', generally and broadly calling a cause whatever - whether according to understanding simply or with respect to us - comes before (\textit{antecedit}) as familiar, through whose familiarity one comes by reason's lead to the acquaintance (\textit{notitiam}) of the unfamiliar. 

But this method, however common it is through the fact that it is present in any given science, considered according to itself and not as mixed with the sciences is still something distinct \textit{per se} from all the others. And considered in this way (\textit{modo})this method (\textit{modus}) can be the subject of a science, and from this it is a positive science. Still, in this science it is necessary that the same method be observed, since a science of this sort (\textit{modo}) can't be taught except in a logical way (\textit{logico modo}).

For as Avicenna says, this method is placed in all men by the fact that they are in a way intellectual by nature. But what is in nature is imperfect, yet perfected by holding to art. For that which is in nature is a seedbed, imperfect and existing as though in potency. For from the fact that a man is intellectual and `a straw bed for understanding', in whom intellectual forms are strewn in the act of the light of understanding; and can place one form with another by understanding: by composition and division a man comes to admire what he comprehends, whether receiving it by sense and understanding or by understanding alone. But by the fact that he admires, he hangs suspended towards inquiry; and by inquiry he compares one to another. By comparing one to something differing from it which is familiar (\textit{notum}), he is led to knowledge of the unknown (\textit{ad ignoti notitiam}). 

And so this method of logic at least begins from nature, but proceeds to art and receives completion in use and exercise, just as Victorinus says best that nature brings availability, art ease, and use power. Thus all arts and all sciences originated from nature. And from hence comes the expression `art imitates nature'. And so Homer says some things are complete and good in certain respects by nature alone; some not by nature, but because the teachers of such things place them in the soul, and those goods are thus effected in the soul placing 
\section{Whether logic is a part of philosophy}
\section{What the science of logic is necessary for and what it is useful for}
\section{What logic is about as its subject}
Now since logic is a contemplative science teaching how and through what one comes from the known to the knowledge of the unknown, 
\section{On the division of logic and its parts, and on the meanings of expressions}
\end{document}