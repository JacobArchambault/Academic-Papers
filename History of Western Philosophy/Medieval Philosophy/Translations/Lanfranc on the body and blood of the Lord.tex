\documentclass[]{article}

%opening
\title{On the body and blood of the Lord}
\author{Lanfranc of Bec}

\begin{document}

\maketitle

\begin{abstract}
Translator's Preface

The structure of Lanfranc's treatise is that of commentary on a work of Berengar of Tours. The first chapter introduces the work. Subsequent chapters comment on Berengar's text. In the opening chapter, Lanfranc tells us that the format he followed to carry out this plan was to display Berengar's text, then his own in text following Berengar's, distinguishing the author of a given section by a symbol for the author's name before the text to be ascribed to him.

Advances in editorial practice since the days of Lanfranc allow for a more advantageous way of separating out the authors, while still allowing the text of each to be read as a unified whole if desired. And so, after the first chapter, the text of which is wholly Lanfranc's, I have chosen to set Berengar's text as the main text, and to set Lanfranc's reply to a given section of text as a footnote to that text. In this way, Lanfranc's text - in form a series of notes on Berengar's - can be read continuously in the footnotes, while Berengar's text can be read continuously in the main body.

For ease of reference, I have retained the standard chapter divisions found in the text.
\end{abstract}

\section{Lanfranc's preface}
Lanfranc, by the grace of God a Catholic, to Berengar, the adversary of the Catholic Church.

If godly piety wouldst deign to breathe into your heart, inasmuch as you would will to speak in the presence of it and your soul; and you were to choose a fitting place in which it competently could become 
\section{Berengar's text with Lanfranc's commentary}

Blah blah blah
\end{document}