\documentclass[]{article}
\usepackage[backend=biber, style=authoryear-icomp]{biblatex}
\bibliography{jacob}

\title{The semantic account of formal consequence, from Alfred Tarski back to John Buridan}
\author{}
\begin{document}
	
\maketitle
\begin{abstract}
	The resemblance of the theory of formal consequence first offered by the 14th-century logician John Buridan to that later offered by Alfred Tarski has long been remarked upon. But it has not yet been subjected to sustained analysis. In this paper, I provide just such an analysis. I begin by reviewing today's classical understanding of formal consequence, then highlighting its differences from Tarski's 1936 account. Following this, I introduce Buridan's account, detailing its philosophical underpinnings, then its content. This then allows us to separate those aspects of Tarski's account representing genuine historical advances, unavailable to Buridan, from others merely differing from - and occasionally explicitly rejected by - Buridan's account.
\end{abstract}
Keywords: Formal consequence, John Buridan, Alfred Tarski, demarcation problem

\section{Introduction}
The resemblance between Buridan's and Tarski's theories of formal consequence has long been remarked upon.\footnote{See \autocite{Moody1952} \autocite{Kneale1962} \autocite{Dumitriu1974} \autocite{DutilhNovaes2012a} \autocite{DutilhNovaes2012c} \autocite{Parsons2014}, and the editor introductions to \autocite{BuridanTC}, \autocite{Kretzmann1982} and \autocite{King1985}.} But while long noticed, it has not yet been subjected to sustained analysis.

In this paper, I provide just such an analysis. I begin by highlighting the differences between classical consequence today and Tarski's 1936 account \autocite{Tarski2002}. Following this, I introduce Buridan's account, detailing its philosophical underpinnings, content, and distinction from the semantic accounts of Tarski and his successors. This in turn, provides the conditions requisite for a partial genealogy of the modern concept, i.e. an account of the conditions that had to arise between Buridan and Tarski's account before Tarski's account became a real possibility.
\section{From classical consequence back to Tarski}
On Tarski's account of formal consequence, $\Gamma \models \phi$ precisely when there is no uniform substitution on  non-logical constants in $\Gamma, \phi$ that models $\Gamma$ but not $\phi$. This agrees with the modern classical approach inasmuch as formality is determined by variation of models, though it differs with respect to what precisely is varied.

In model-theory today, the languages one works with are left uninterpreted until given an interpretation by the interpretation function. The languages Tarski worked with, by contrast, were fully interpreted artificial languages, like those for Riemannian geometry or Peano arithmetic. Thus for Tarski, there is no question of assigning an interpretation to the non-logical constant `0', say, or varying its interpretation across models. Rather, where contemporary model theory varies the interpretation of non-logical constants across models to determine what follows logically from what, Tarski left these constants interpreted as they were, but replaced them uniformly with variables, whose assignments are then varied accordingly.\footnote{According to \autocite[433]{Schiemer2013}, the contemporary practice originates in the work of John Kemeny. See \autocite{Kemeny1956} \autocite{Kemeny1956b}.} The same point also holds for predicate and relation symbols: where modern practice varies their interpretation, Tarski's replaces them with second-order variables, and then varies their assignments.\footnote{See e.g. \autocite[122-23]{Tarski1941}. Cf. \autocite[69]{Etchemendy1988} \autocite[448]{Schiemer2013}.} While there is a conceptual difference between the two approaches, both lead to the same material results.


For Tarski, in contrast with modern practice, a model does not include an interpretation, but is simply a sequence of objects. As Tarski puts it:
\begin{quote}
	One of the concepts which can be defined with the help of the concept of satisfaction is the concept of \textit{model}. Let us assume that, in the language which we are considering, to each extra-logical constant correspond certain variable symbols, and this in such a way that, by replacing in an arbitrary sentence a constant by a corresponding variable, we transform this sentence into a sentential function. Let us further consider an arbitrary class of sentences $L$, and let us replace all extra-logical constants occurring in the sentences of class $L$ by corresponding variables (equiform constants by equiform variables, non-equiform by non-equiform); we shall obtain a class of sentential functions $L'$. An arbitrary sequence of objects which satisfies each sentential function of the class $L'$ we shall call a \textit{model of the class} $L$ (in just this sense one usually speaks about a model of the system of axioms of a deductive theory); if in particular the class $L$ consists of only one sentence $X$, we will simply speak about a \textit{model of the sentence} $X$ \autocite[185-186]{Tarski2002}.
\end{quote}

Though both Tarski's and the modern approach call a consequence $\Gamma \models \phi$ formal when every model of the set $\Gamma$ is at the same time a model of $\phi$, they differ to the degree that their underlying concepts of `model' differ. Tarski interprets `model of $\Gamma$' to mean a sequence of objects satisfying the sentential functions obtained from $\Gamma$ - i.e. constant-free versions of the sentences of $\Gamma$, where like constants are replaced by like variables. The modern approach takes a model of $\Gamma$ to be an interpretation of the (uninterpreted) strings in $\Gamma$, mapping them to a domain of objects arbitrary both in content and in number. Tarski's approach corresponds more fully to the intuitive \textit{meaning} of `model'; while the modern approach, if idiosyncratic in its choice of terms, corresponds more fully to the intuitive \textit{extension} of following formally. 

Let us then see how these compare to Buridan's account.
\section{Buridan's theory of formal consequence}
Today, logicians distinguish consequence from hypothetical propositions, usually identifying the latter with conditionals. One then must provide a deduction theorem to establish that $A \models B$ iff $\models A \rightarrow B$. Buridan, by contrast, identifies these: `A consequence is a hypothetical proposition; for it is constituted from several propositions conjoined by the expression `if' or the expression `therefore' or something equivalent' \autocite[I. 3, 66, alt.]{Buridan2015}. 

Just as modern practice reserves the term `consequence' for formally valid consequence, Buridan reserves it for true hypotheticals, excluding false ones\autocite[I. 3, 66]{Buridan2015}. But where the classical emphasis on formal validity arises already in removing non-logical content from the models of a sentence, Buridanian consequences are not as such formal in this way. \footnote{This difference in definition hints at a much deeper one. For Buridan, consequences are always individual sentence tokens, i.e. actually written or spoken hypothetical expressions, which are evaluated as true or false by virtue of determining whether the connections they express hold in all possible situations (including those where the expressions themselves do not exist, and hence are neither true nor false). For Tarski and the modern approach, by contrast, consequences are \textit{never} actual sentences, both because of the aforementioned abstraction at the level of the models of a sentential function, and because the antecedent $\Gamma$ of a classical consequence $\Gamma \models \phi$ is always at least denumerably infinite, since it is closed under entailment.}



After considering definitions of consequence in terms of 1) the impossibility of the antecedent being true and the consequent not so, 2) the impossibility of the antecedent being true and the consequent false \textit{when both are formed}, and 3) the impossibility of things being as the antecedent signifies without being as the consequent signifies, Buridan ultimately settles on the following definition of consequence: 

\begin{quote}
	A consequence is a hypothetical proposition composed of an antecedent and consequent, indicating the antecedent to be antecedent and the consequent to be consequent; this designation occurs by the word `if' or by the word `therefore' or an equivalent \autocite[I. 3, 67]{Buridan2015}.
\end{quote}

Unlike the model-theoretic definition, Buridan's is deflationary in spirit: Buridan takes `antecedent' and `consequent' to apply to the respective parts of a hypothetical proposition precisely when `consequence' applies to the whole. Thus, whether the definition is materially adequate or not,\footnote{It is not. See [reference omitted].} it does nothing to further determine either the extension or the intension of `consequence' beyond what is already given in the name itself.



Buridan divides consequences into formal and material, dividing the latter into simple and as-of-now consequences. A simple consequence is one where things cannot be as the antecedent signifies and not as the consequent signifies; an as-of-now consequence, one where things cannot \textit{now} be as the antecedent signifies, without also being as the consequent signifies. Thus, as-of-now consequence limits those cases considered in simple consequence to the present situation alone. Thus, every simple material consequence is an as-of-now material consequence, but not conversely. 

Buridan explains the division between formal and material consequence as follows: 
\begin{quote}
	A consequence is called `formal' if it is valid in all terms retaining a similar form. Or, if you want to put it explicitly, a formal consequence is one where every proposition similar in form that might be formed would be a good consequence [...]. A material consequence, however, is one where not every proposition similar in form would be a good consequence, or, as it is commonly put, which does not hold in all terms retaining the same form \autocite[I. 4, 68]{Buridan2015}.
\end{quote}

Buridan continues: 
\begin{quote}
	It seems to me that no material consequence is evident in inferring except by its reduction to a formal one. Now it is reduced to a formal one by the addition of some necessary proposition or propositions whose addition to the given antecedent produces a formal consequence \autocite[I. 4, 68]{Buridan2015}.
\end{quote}

Where the former passage shows formal and material consequence on equal footing with respect to their validity - both are true hypothetical propositions - the latter shows they are not so with respect to their evidential status: a material consequence is only evident in inference if it can be transformed into a formal one by appropriate additions to the antecedent. One might get the impression from this that for Buridan, material consequences are all enthymemes.\footnote{Cf. \autocite[66]{BurleyDPAL}.} But Buridan lists enthymemes as only one kind of material consequence, alongside examples and inductions \autocite[III. 1, 113]{Buridan2015}; and in Buridan's treatment of dialectical topics, inductions may be proven not `by virtue of being a formal consequence or by being reduced to a formal consequence, but by the natural inclination of the understanding towards truth' \autocite[6.1.5]{BuridanLoci}. In this way, the peculiar importance of formal consequence lies not in its preserving \textit{truth}, but \textit{evidence}.

But to know which consequences exactly are formal, one must know what the form of a proposition is. And just as in both Tarski's earlier and later work, this is determined by a partition. Buridan writes:

\begin{quote}
	I say that when we speak of matter and form, by the matter of a proposition or consequence we mean the purely categorematic terms, namely the subject and predicate, setting aside the syncategoremes attached to them by which they are [1] conjoined [2] or denied [3] or distributed [4] or given a certain kind of supposition; we say all the rest pertains to form \autocite[I. 7, 74]{Buridan2015}.
\end{quote}

There are two central differences between Buridan's division and its Tarskian counterpart. First, where Tarski's is a fixed division of terms in a \textit{language} into two different \textit{kinds}, Buridan's is one of terms in a \textit{sentence} into two different \textit{roles}. Thus, for instance, in the sentence `if is a syncategoreme, therefore if is a word', Buridan's analysis correctly characterizes `if' as a categorematic term, while Tarski's requires it be treated as a logical constant. Second, where Tarski's approach prioritizes determining the set of logical constants of a language, Buridan's begins by defining the \textit{categorematic} terms of a sentence, i.e. the analogue of classical non-logical constants, and then defines the syncategoremata of a sentence as those not belonging among the categorical terms of the sentence. The categorematic terms are simply those operating as the subject and predicate of the sentence.\footnote{A note on the language of `syncategoremata': the phrase `syncategorematic terms' does not occur in Buridan, nor to my knowledge in other medieval discussions of consequence. Terms are those words in which every sentence `bottoms out' (hence the name `term', i.e. end or limit), and so are just those words against which syncategoremes are divided.} Buridan then lists four types of words as pertaining to form: 1) those conjoining the subject and predicate (e.g. `is'); 2) those separating the subject and predicate (e.g. `not'); 3) those giving the terms a certain distribution (e.g. quantifiers); 4) those giving terms a certain kind of supposition (e.g. modal, tense, and other intensional operators). These last two are the subjects to which Buridan's discussions of supposition and ampliation in chapters five and six pertain. Notably absent from this list are propositional connectives, which Buridan doesn't consider in the \textit{Treatise on Consequences}.

Though Buridanian formal consequences may be represented schematically, these consequences are never themselves schematic,\footnote{This is also true on Tarski's account, though it is not so on the received classical analysis. The basic reason for the latter is the decision to regard the constant symbols as uninterpreted.} but rather remain individual hypothetical propositions. As such, there is no problem on Buridan's account about whether the \textit{ordering} and choice of schematic variables in a schematic consequence belongs to its matter or form.\footnote{I thank Milo Crimi for bringing this problem to my attention.} For Buridan, while different good consequences may be representable by the same schema (e.g. different instances of \textit{modus ponens}); while these same consequences may be representable by other schemata (i.e. by uniformly replacing the schematic variables used in the first schema with others); and while schemata for good and bad formal consequences may have the same ordering of their syncategorematic parts (e.g. \textit{modus ponens} and affirming the consequent): because schematic consequences are not properly consequences at all for Buridan, these problems disappear. Buridan's formal consequences are hypothetical propositions of a natural language; it is not because they belong to the same schema that they are formal consequences; rather, these formal consequences evidently belong to an equivalence class, and because of this can be represented schematically under the same form.
\section{Formal consequence from Tarski back to Buridan}
Having set Buridan's and Tarski's accounts of formal consequence in order, we can now contrast them summarily. 

Both modern and Tarskian approaches begin with a partition of all terms of a \textit{language} into logical and non-logical terms; Buridan's partition of terms into categorical and syncategorematic occurs not at the level of a language, but at that of the sentence. 

Tarski's project prioritizes determining the logical terms of a language, the determination of the set of non-logical terms falling out of this. Buridan's partition begins by determining those terms pertaining to the matter of the sentence, fixing the set of formal terms in a sentence as the complement of those pertaining to the matter. Where Tarski's partition is one of different types of terms in a language, Buridan's is of different roles had by terms in a sentence. For Buridan, the terms pertaining to the matter of a sentence are its subjects and predicates.

Tarskian consequence was designed for recursively defined artificial languages, particularly those being developed in mathematics. Buridanian consequence was designed to capture Buridan's stilted fourteenth-century scholastic Latin. \textit{Pace} humanist objections, the latter remained a natural language, albeit one making use of mild regimentation when doing so aided discussion. Buridan counts the copula, negation, modalities, tenses, quantifiers, intentional operators, as well as disjunction, conjunction and negation for terms among the formal parts of a sentence. But because he explains the formal parts of a sentence as those affecting the supposition of \textit{terms}, he does not mention sentential connectives as pertaining to form. While admitting sentential connectives, Tarski discounted modalities, tenses, and intensional operators. While initially silent on the status of identity, Tarski later explicitly admitted it as a logical notion. Since then, modalities and other intensional operators have become standard in extensions of classical logic.

For Tarski, consequences are distinguished from hypothetical propositions, not least because the set of premises from which a consequent is derived is closed under entailment, hence countably infinite. Buridan, by contrast, explicitly identifies consequences with hypothetical propositions.


Where Tarski provides an informative definition of consequence in terms of models, Buridan provides a deflationary one in terms of the correlative notions of antecedent and consequent. However, the work done by the notion of a model on Tarski's account is mimicked by that done by the notion of a cause of truth on Buridan's. 

On the received classical account, a model of a sentence $\phi$ in a language $L$ consists of a domain $D$ and an interpretation $I$, i.e. a mapping of the sentences of $L$ to truth or falsity, recursively determined by a mapping of terms to elements in $D$ and n-ary predicates to sets of $n$-tuples in $D^{n}$.

On Tarski's account, a model of a set of sentences $\Gamma$ is a sequence of objects in a fixed domain satisfying the sentential functions obtained by uniformly replacing each non-logical constant in the sentences of $\Gamma$ with variables of the appropriate order and arity. 

The differences between the received classical and Tarskian understandings of a model thus lead to differences in their understanding of both the intension and extension of the concept of formal consequence. Both the Tarskian and received classical accounts of the models of a sentence, however, are general from the beginning: for instance, the classical models of an atomic sentence will not be the objects making it true on the intended interpretation of its non-logical constants, but those making it true on \textit{any} interpretation; and the Tarskian models will be sequences satisfying any sentential function of the same form as the initial sentence.

Buridan's account of causes of truth, by contrast, maps hypothetical propositions to states sufficient to make them true on their \textit{intended} interpretation. The determination of the relative number of causes of truth a sentence has is given by the supposition of its terms, i.e. the manner in which one is permitted to descend from a general term modified by a determiner to a new sentence or sentences replacing the determined general term with a name (names) for an individual(s) falling under it. On Buridan's account, the causes of truth of a proposition are relative to a time of utterance; the models of a Tarskian sentence are not; and while not fixed, classical models are relativized in a way not based on external circumstances, but arbitrarily. On Buridan's account, formality is then achieved by the determination of the equivalence class of a hypothetical proposition: a Buridanian consequence is formal iff for every proposition equivalent in form to it that could be formed, it is impossible for things to be as the antecedent signifies without being as the consequent signifies (leaving aside problems Buridan recognizes with talk about things being as propositions signify).

\section{Conclusion}
The concept of formal consequence in classical logic today, in perfect verbal agreement with Tarski's 1930s definition, holds $\Gamma \models \phi$ iff every model of $\Gamma$ is at the same time a model of $\phi$. But behind this verbal agreement lies a substantive disagreement, grounded in different concepts of a model. Today's classical models interpret uninterpreted linguistic strings by mapping them to a domain of arbitrarily many objects. Tarskian models, by contrast, are sequences of objects, within the fixed domain of all objects in the world, satisfying sentential functions obtained from interpreted sentences.

Tarski's account represents a genuine development from the Buridanian account to the degree that it employs the concepts of model, sentential function, and recursion, which were unavailable to Buridan. Other differences, however, represent more substantive disagreements. Buridan's acceptance of modality, tense, and variable domains; his prioritization of the determination of the material parts of the sentence over the formal; his adoption of a token-based semantics grounded in natural languages, all were taken up \textit{against} analogues of the contrary positions, found in Tarski, in Buridan's own time. In other ways, the difference between Buridan and Tarski's approach to consequence is not so wide as their chronological distance from each other would suggest. In contrast with modern practice, neither construes the \textit{relata} of formal consequence schematically; Buridanian causes of truth form analogues to the Tarskian concept of models of a sentential function; and both Buridanian and Tarskian accounts of following formally are given in terms of substitution - a Buridanian formal consequence is good if all sentences that could be formed by uniform substitutions on its categorematic terms are good, a Tarskian one if it is invariant under satisfaction of sentential functions obtained from it by substituting its non-logical constants with variables. Given this closeness, it is perhaps unsurprising that may of the genuine developments in formal logic over the past sixty years have involved a reappropriation of the Buridanian standpoint on just those topics where he disagrees with Tarski. In this reappropriation of the best elements of Buridan's account into the context brought about by genuine developments since it, one might hope to find progress toward ... well, if not truth, at least how things are signified to be.
\printbibliography
\end{document}
