\documentclass[]{article}

%opening
\title{Dialectically situating medieval accounts of the subject matter of logic}
\author{Jacob}

\begin{document}

\maketitle

\begin{abstract}

\end{abstract}

\section{Introduction}
\section{Early remarks on the subject matter of logic}
\subsection{Boethius}
Nihil quippe dicimus impedire ut eadem logica partis vice simul instrumentique fungatur officio. Quoniam enim ipsa suum retinet finem, isque finis a sola philosophia [Col.0074D] consideratur, pars quidem philosophiae esse ponenda est; quoniam vero finis ille logicae quem sola speculatur philosophia, ad alias ejus partes suam operam pollicetur, instrumentum esse philosophiae non negamus: est autem finis logicae inventio judiciumque rationum; quod quidem non esse mirum videbitur, quod eadem pars eadem quoddam ponitur instrumentum, si ad partes corporis animum reducamus, quibus et fit aliquid ut his quasi quibusdam instrumentis utamur, et in toto tamen corpore partium obtinent locum. Manus enim ad tractandum, oculi ad videndum, caeteraeque corporis partes proprium quoddam videntur officium habere. Quod tamen si ad totius utilitatem corporis referatur, instrumenta [Col.0075A] quaedam corporis esse deprehenduntur quae etiam partes esse nullus abnuerit. Ita quoque logica disciplina pars quaedam philosophiae est, quoniam ejus philosophia sola magistra est. Supellex vero est, quod per eam inquisita veritas philosophiae vestigatur. Sed quoniam, quantum mihi quidem brevitas succincta largita est, ortum logicae et quid ipsa logica esset explicavi, nunc de eo nobis libro pauca dicenda sunt quem in praesens sumpsimus exponendum.
\subsection{The Arabic tradition}
\subsection{Some 12th century Latin thinkers}
Science is divided into rational, natural and moral philosophy.... Rational science is divided into three parts: grammar, rhetoric, and logic. Grammar teaches the proper arrangement of 
\section{Earliest scholastic discussions: logic as a \textit{scientia sermocinalis}}
\section{Two families of positions, two senses of `subject'}
\subsection{Logic as about beings of reason}
\subsubsection{Beings of reason}
\subsubsection{Second intentions}
\subsection{Logic as about argument}
\subsubsection{The subject matter of logic is the syllogism}
\subsubsection{The subject matter of logic is the argument}
\section{Summary and table}
\section{Synthetic observations}
\section{Conclusion}
\section{Appendix 1: Albertus Magnus' \textit{On Universals}, Bk. 1, ch. 1-5}
\section{Appendix 2: Duns Scotus, \textit{Questions on Porphyry's Isagoge}, qq. 1-3}
\section{Appendix 3: Walter Burley, \textit{Literal commentary on Porphyry's Isagoge}}
\end{document}