\documentclass[]{article}
\usepackage[backend=biber, style=authoryear-icomp]{biblatex}
\bibliography{../../../jacob}

%opening
\title{13th-14th century theories of inference}
\author{Jacob Archambault}

\begin{document}

\maketitle

\section{Why study consequence}
Through the notion of consequence, 
logicians aim to give precision to our everyday notion of something following from something else. 
Consequence is arguably the core notion studied in formal logic today, 
where it has stood since Tarski's groundbreaking work on the topic in the 1930s, 
and also plays a major role in adjacent fields including computing and the philosophy of science. 

When compared to the intuitive notion of consequence they aim to capture, 
however, 
the most widely-used theories of consequence suffer from several known deficiencies: 
\begin{itemize}
	\item Classical theories of consequence validate inference rules that are highly unintuitive. 
	The best-known of these is \emph{explosion}, 
	which allows anything to be inferred from a contradiction.
	\item Within natural language, 
	it is possible for statements to refer, 
	directly or indirectly, 
	to themselves. 
	Because of this, 
	natural language inference is said to be \emph{semantically closed}. 
	The artificial languages studied in formal logic today, 
	by contrast, 
	tend to employ various devices to prevent semantic closure, 
	leaving them less expressive than their natural counterparts.
	\item The inference schemata studied in modern logical systems tend, 
	by design, 
	to be indifferent to whatever content might be expressed in actual natural language inferences whose formalizations they capture, 
	and commonly accepted regimentations of natural language sentences in formal logic are not at all straightforward. 
	Particularly for novices, this can leave the use of these systems opaque. 
\end{itemize}

\subsection{Why study medieval consequences}
Given these difficulties, 
the study of consequence at the height of its cultivation in the high medieval period can be profitable in a number of ways. 

\subsubsection{To solve problems in modern logic}
More immediately, 
work on inference in this period reaches a sophistication that it would not approach again until the beginning of the current tradition of work on the topic in the late 19th and early 20th century, 
and therefore provides a repository that can be drawn upon to improve modern theories of consequence. 
For example, 
where modern definitions of consequence are typically insensitive to tense and modality, 
medieval theories like that of John Buridan can vary their criterion for valid inference relative to the tense and modality of the parts of the inference under evaluation \autocite[63]{Read2015}.

\subsubsection{To understand how we got here}
Less obviously but arguably more importantly, 
several essential aspects of the theory of consequence as we understand it today first arose during this period: 
the earliest known treatises directly devoted to consequence were written at the turn of the 14th century \autocite{Archambault2017d}, 
and the notion of \emph{formal} consequence became a primary locus of attention shortly thereafter \autocite{DutilhNovaes2012a}. 
This period therefore provides the backdrop for understanding, 
prior to the more mathematical aspects provided at the turn of the 20th century, 
some of the more general aims of research on consequence as it continues to be carried out today. 
From it, we can also catch a glimpse into the research program it replaced. 

\section{Their origin}
According to seminal research carried out in the late 70's and early 80's,
theories of consequence appear to have arisen out of two groups of sources. 

\subsection{Topics}
The first was the body of work devoted to the theory of topics the medievals inherited from Aristotle, 
Cicero, 
Themistius, 
and Boethius \autocite{Stump1982}, 
with the most conspicuous support for this being the placement of William of Ockham's treatise on consequences in his \emph{Summa Logicae}, 
which sets out its subjects in an order corresponding to that of the books of Aristotle's logic, 
in the place normally reserved for topics \autocite{OckhamSL}.

\subsection{Sophismata}
The second, both more amorphous and influential source, 
came out of the \emph{logica modernorum} 
- a series of logical works devoted to topics not addressed directly in the scope of Aristotle's organon 
- particularly treatises on 
\emph{syncategoremata} (i.e. terms roughly similar to today's logical constants) 
and fallacies
%and \emph{sophismata}, 
%which treated particularly difficult-to-handle inferences, 
%including liar paradoxes and several forerunners to Curry's paradox 
\autocite{Green-Pedersen1984,Spruyt2018}.
13th century authors working on consequence in this tradition include 
Nicholas of Paris, 
William of Sherwood, 
Lambert of Lagny 
and Peter of Spain.

\section{Their nature}
In medieval logic, 
\emph{consequence} refers to a relation between two parts of a hypothetical proposition, 
respectively called the \emph{antecedent} and \emph{consequent}, 
according to which what is stated in the consequent follows from what is stated in the antecedent, 
e.g. `If Socrates is running, then he is moving'. 
\section{Their Criteria}

\subsection{Impossible for the antecedent...}

\subsection{Containment}

\section{Their division}


\section{Their evolution}

\section{open questions}

\printbibliography

\end{document}
