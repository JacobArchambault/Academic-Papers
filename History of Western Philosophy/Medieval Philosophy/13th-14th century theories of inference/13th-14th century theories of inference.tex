\documentclass[]{article}

%opening
\title{13th-14th century theories of inference}
\author{Jacob Archambault}

\begin{document}

\maketitle

\section{Why study consequences}
Through the notion of consequence, logicians aim to give precision to our everyday notion of something following from something else. Consequence is arguably the core notion studied in formal logic today, where it has remained since Tarski's groundbreaking work on the topic in the 1930s, and also plays a major role in adjacent fields including the philosophy of science and computing. 

When compared to the intuitive notion of consequence it aims to capture, however, the most widely-used theories of consequence suffer from several known deficiencies: 
\begin{itemize}
	\item Classical theories of consequence validate inference rules that are highly unintuitive. The best-known of these is \emph{explosion}, which allows anything to be inferred from a formal contradiction.
	\item Within natural language, it is possible for statements to refer, directly or indirectly, to themselves. Because of this, Natural language inference is said to be \emph{semantically closed}. The artificial languages studied in formal logic today, by contrast, tend to employ various devices to prevent semantic closure, leaving them less expressive than their natural counterparts.
	\emph The inference schema studied in modern logical systems tend, by design, to be indifferent to whatever content might be expressed in actual natural language inferences whose formalizations they capture, and commonly accepted regimentations of natural language sentences in formal logic are not at all straightforward. Particularly for novices, this can leave the use of these systems opaque. 
\end{itemize}

\subsection{Why study medieval consequences}
%Given these difficulties, the study of medieval logic can provide two 
\section{Their nature}
%In medieval logic, \emph{consequence} refers to a relation between two parts of a complex sentence (or a hypothetical proposition, in medieval parlance), respectively called the \emph{antecedent} and \emph{consequent}, according to which what is stated in the consequent can be said to follow from what is stated in the antecedent, e.g. `If Socrates is running, then he is moving'. 
\subsection{To solve problems in modern logic}
%Semantic closure

%Temporal logic
%Modal logic
\section{Their Criteria}
\subsection{Impossible for the antecedent...}
\subsection{Containment}
\section{Their division}
\section{Their origin}
\subsection{Topics}
\subsection{Sophismata}
\section{Their evolution}
\section{open questions}

\end{document}
