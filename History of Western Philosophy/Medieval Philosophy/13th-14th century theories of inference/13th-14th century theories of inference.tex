\documentclass[]{article}
\usepackage[backend=biber, style=authoryear-icomp]{biblatex}
\bibliography{../../../jacob}

%opening
\title{13th-14th century theories of inference}
\author{Jacob Archambault}

\begin{document}

\maketitle

In medieval logic, 
\emph{consequence} refers to a relation between two parts of a hypothetical proposition, 
respectively called the \emph{antecedent} and \emph{consequent}, 
according to which what is stated in the consequent follows from what is stated in the antecedent, 
e.g. `If Socrates is running, then he is moving'. 

Consequence is arguably the core notion studied in formal logic today, 
where it has stood since Alfred Tarski's groundbreaking work on the topic in the 1930s, 
and also plays a major role in adjacent fields including computing and the philosophy of science. 
When compared to the intuitive notion of consequence that they aim to capture, 
however, 
the most widely-known theories of consequence today suffer from several deficiencies: 
\begin{itemize}
	\item Classical theories of consequence validate inference rules that are highly unintuitive. 
	The best-known of these is \emph{explosion}, 
	which allows anything to be inferred from a contradiction.
	\item Natural language inference is \emph{semantically closed}. 
	making it possible for statements to refer, 
	directly or indirectly, 
	to themselves. 
	The artificial languages studied in formal logic today, 
	by contrast, 
	tend to employ various devices to prevent semantic closure, 
	leaving them less expressive than their natural counterparts.
	\item The inference schemata studied in modern logical systems tend, 
	by design, 
	to be indifferent to whatever content might be expressed in actual natural language inferences whose formalizations they capture.
%	and commonly accepted regimentations of natural language sentences in formal logic are not at all straightforward. 
	Particularly for novices, this can leave the use of these systems opaque. 
\end{itemize}

Several essential aspects of the theory of consequence as we understand it today first arose during the later medieval period: 
the earliest extant treatises directly devoted to consequence, 
translated in \autocite{Archambault2017d}, 
were written at the turn of the 14th century, 
and the notion of \emph{formal} consequence became a primary locus of attention shortly thereafter \autocite{sep-consequence-medieval}. 
This period therefore provides the backdrop for understanding, 
prior to the more mathematical aspects provided at the turn of the 20th century, 
some of the more general aims of research on consequence as it continues to be carried out today. 
%From it, we can also catch a glimpse into the research program it replaced. 

\section{The sources from which the theory of consequence arose}
According to seminal research carried out through the 1980's,
theories of consequence appear to have arisen out of two groups of sources. 

The first group, 
included treatises 
from the \emph{logica modernorum} 
- 
works written on 
\emph{syncategoremata} (i.e. terms roughly similar to today's logical constants) 
fallacies,
and other subjects not fully addressed in Aristotle's organon. 
%and \emph{sophismata}, 
%which treated particularly difficult-to-handle inferences, 
%including liar paradoxes and several forerunners to Curry's paradox 
\autocite{Green-Pedersen1984,Spruyt2018}.
13th century authors working on consequence in this manner include 
Nicholas of Paris, 
William of Sherwood, 
Lambert of Lagny 
and Peter of Spain.

The second was the body of work devoted to the theory of topics the medievals inherited from Aristotle, 
Cicero, 
Themistius, 
and Boethius, 
with the most conspicuous support for this being the placement of William of Ockham's treatise on consequences in his \emph{Summa Logicae}, 
which sets out its subjects in an order corresponding to that of the books of Aristotle's logic, 
in the place normally reserved for topics \autocite{Stump1982,OckhamSL}. 
Traditional work on topics aimed to classify various properties or relations, 
e.g. between species and genera, 
parts and wholes, 
or causes and effects, 
which could ground the discovery of new conclusions about the entities to which those properties and relations applied. 
Early treatises on consequences, 
however, 
departed from their topical forebearers in 
grounding inferences not in a richer variety of arguably metaphysically robust relations, but almost entirely on thinner extensional relations pertaining to the distribution of terms as outlined in medieval theories of \emph{supposition} \autocite{HodgesBurley,sep-medieval-terms,Archambault2022}. 

Compared with modern theories of consequence, 
the medieval theory's roots 
in the second set of influences leave it with a more concrete focus than its modern counterpart;
its roots 
in the first group mentioned 
make it more interested in 
- and arguably more resilent to - 
natural language paradoxes, 
with solutions tending to be more likely to come from analyses of the meaning of particularly problematic terms than global rules applied to a language \autocite{Klima2004,Klima2016}; 

\section{Their division}
Through the 13th and early 14th century, 
consequences are commonly divided into \emph{simple consequences}, 
which always hold, 
and \emph{as-of-now} (\emph{ut nunc}) consequences, 
which only hold at a given time. 
Simple consequences, 
in turn, 
are divided into natural and accidental, 
and natural consequences hold by virtue of the antecedent in some way `containing' the consequent 
- a stronger relationship than the always-good, 
but non-intrinsic relation of holding accidentally. 
One of the more striking aspects of this division is that it lays bare not merely a different way of doing what modern logic does, 
but a different goal: 
medieval logic in this period was interested not primarily in the classification of valid forms of inference, 
but in the classification and discovery of \emph{sound} ones. 
With Walter Burley being an especially important transitional figure \autocite{Archambault2018b}, 
the division between natural and accidental consequence gives way to one between formal and material consequence. 
%with John Buridan and others subordinating the earlier division between simple and as-of-now consequence to the latter. 
Particularly in the British Isles and later in Italy, 
the containment criterion mentioned above continues to be appealed to in discussions of formal consequence by figures like Thomas Bradwardine and Paul of Venice; 
while figures including Buridan, 
Albert of Saxony 
and Marsilius of Inghen on the European continent prefer to discuss formal consequences in terms of the impossibility of the antecedent holding with the consequent not holding. 
It is within this latter tradition especially that we see increased attention to formal consequence 
%not in the sense of one holding by virtue of the form or essence of a thing 
%(leaving aside whether that be construed along realist or conceptualist lines), 
%but 
in the sense of one holding schematically, 
providing the prototype for the more general aims of logical research from Tarski to today. 
\printbibliography

\end{document}
