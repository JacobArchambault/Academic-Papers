% $Header$

\documentclass{beamer}
\usepackage[backend=biber]{biblatex}
\bibliography{jacob}
\usepackage{amssymb}
\usepackage[all]{xy}

% This file is a solution template for:

% - Giving a talk on some subject.
% - The talk is between 15min and 45min long.
% - Style is ornate.



% Copyright 2004 by Till Tantau <tantau@users.sourceforge.net>.
%
% In principle, this file can be redistributed and/or modified under
% the terms of the GNU Public License, version 2.
%
% However, this file is supposed to be a template to be modified
% for your own needs. For this reason, if you use this file as a
% template and not specifically distribute it as part of a another
% package/program, I grant the extra permission to freely copy and
% modify this file as you see fit and even to delete this copyright
% notice. 


\mode<presentation>
{
	\usetheme{Warsaw}
	% or ...
	
	\setbeamercovered{transparent}
	% or whatever (possibly just delete it)
}


\usepackage[english]{babel}
% or whatever

\usepackage[latin1]{inputenc}
% or whatever

\usepackage{times}
\usepackage[T1]{fontenc}
% Or whatever. Note that the encoding and the font should match. If T1
% does not look nice, try deleting the line with the fontenc.


\title[] % (optional, use only with long paper titles)
{Divided modality for Ockhamists}

\subtitle
{} % (optional)

\author[] % (optional, use only with lots of authors)
{Jacob Archambault}
% - Use the \inst{?} command only if the authors have different
%   affiliation.

\institute[] % (optional, but mostly needed)
{Unilog 2018, Vichy}
% - Use the \inst command only if there are several affiliations.
% - Keep it simple, no one is interested in your street address.

\date[] % (optional)
{25 June 2018}

\subject{Talks}
% This is only inserted into the PDF information catalog. Can be left
% out. 



% If you have a file called "university-logo-filename.xxx", where xxx
% is a graphic format that can be processed by latex or pdflatex,
% resp., then you can add a logo as follows:

% \pgfdeclareimage[height=0.5cm]{university-logo}{university-logo-filename}
% \logo{\pgfuseimage{university-logo}}



% Delete this, if you do not want the table of contents to pop up at
% the beginning of each subsection:
\AtBeginSection{\begin{frame}
		<beamer>{Outline}
		\tableofcontents[currentsection]
	\end{frame}}
	
\AtBeginSubsection[]
{
	\begin{frame}<beamer>{Outline}
		\tableofcontents[currentsection, currentsubsection]
	\end{frame}
}


% If you wish to uncover everything in a step-wise fashion, uncomment
% the following command: 

%\beamerdefaultoverlayspecification{<+->}


\begin{document}
	
	\begin{frame}
		\titlepage
	\end{frame}
	
	\begin{frame}{Outline}
		\tableofcontents[pausesections]
		% You might wish to add the option [pausesections]
	\end{frame}
	
	
	% Since this a solution template for a generic talk, very little can
	% be said about how it should be structured. However, the talk length
	% of between 15min and 45min and the theme suggest that you stick to
	% the following rules:  
	
	% - Exactly two or three sections (other than the summary).
	% - At *most* three subsections per section.
	% - Talk about 30s to 2min per frame. So there should be between about
	%   15 and 30 frames, all told.
	
\section{Introduction}
\begin{frame}{Introduction}
	\begin{itemize}
		\item Standard discussions of the composite-divided distinction in Ockham's logic assimilate the difference between the one and the other to a contemporary scope distinction. \pause
		\item In a composite modal, the modality takes wide scope; in a divided modal, narrow scope. \pause 
		\item (COMPOSITE$_{1}$) $\square$(Some A is B)
		\item (COMPOSITE$_{2}$) $\square(\exists x)(Ax \wedge Bx)$
		\item (DIVIDED$_{1}$) (Some A is $\square$B)
		\item (DIVIDED$_{2}$) $(\exists x)(Ax \wedge \square Bx)$
	\end{itemize}
\end{frame}	
\section{The assimilation of divided to narrow-scope modality}
\begin{frame}{The assimilation of divided to narrow-scope modality}
	\begin{enumerate}
		\item[(1)] Nullum impossibile esse verum est necessarium, igitur nullum verum esse impossibile est necessarium. \cite[p. 298]{OckhamSL2} \pause 
		\item[(1')]	`Nothing impossible is necessarily true' therefore `Nothing true is necessarily impossible.' \cite[p. 166]{OckhamSLEng}.
	\end{enumerate}
\end{frame}
\begin{frame}{The assimilation of divided to narrow-scope modality}
	\begin{quote}
		`For the antecedent is true, and the consequent is false. For nothing impossible is possibly true, and yet something true is possibly impossible. [...] For if I go to Rome, then it will be impossible afterwards [that I have not been to Rome]'. \cite[p. 166]{OckhamSLEng}
	\end{quote}
\end{frame}
\begin{frame}{The assimilation of divided to narrow-scope modality}
	\begin{quote}
		`All that follows is that `Something true is possibly impossible', but this is not what Ockham needs to show. He needs to show that `Something true is necessarily impossible,' which appears to be a stronger proposition'.  \cite[p. 243]{Johnston2015}
	\end{quote}
\end{frame}
\begin{frame}{The assimilation of divided to narrow-scope modality}
	\begin{itemize}
		\item The choice to read Ockhamist divided modals according to the canonical reading invalidates some of the examples Ockham uses to illustrate his theory. \pause
		\item While there is nothing inherently wrong with this, it has not sufficiently been tested whether there might be a better reading of Ockham in view.
	\end{itemize}
\end{frame}
\begin{frame}{The reasons for this assimilation: conversion}
	\begin{itemize}
		\item An important piece of evidence for the analysis of divided modals into their narrow scope formal counterparts is Ockham's account of their conversion.
	\end{itemize}
\end{frame}
\begin{frame}{Fallacious conversions}
	\begin{itemize}
		\item[(1)] `That no impossibility be true is necessary, therefore that no truth be impossible is necessary'
		\item[(2)]	`A being which creates is necessarily God' therefore `God is necessarily a being who creates.' 
		\item[(3)]	`A man is necessarily understood by God' therefore `Something understood by God is necessarily a man.' \cite[II. 24, p. 298]{OckhamSL2}
	\end{itemize}
\end{frame}
\begin{frame}{Their correct counterparts}
	\begin{itemize}
		\item[(4)]	Every impossibility of necessity is not true, therefore something, which of necessity is not true, is impossible.
		\item[(5)]	The Creator of necessity is God, therefore something, which of necessity is God, is creating.
		\item[(6)]	Man of necessity is understood by God, therefore something, which of necessity is understood by God, is man. \cite[II. 24, p. 298]{OckhamSL2}
	\end{itemize}
\end{frame}
\begin{frame}{The reasons for the assimilation: conversion}
	Given the consequents of the above all conform to the canonical formalization of \textit{de re} modals, it is assumed that the antecedents should as well.
\end{frame}
\section{Divided modality in William of Ockham: a formal reconstruction}
\begin{frame}{Divided modality in William of Ockham}
\begin{quote}
`It should be known that for the truth of such a proposition [i.e. a modal proposition in the divided sense] we require that the predicate, under its proper form, belong to that for which the subject supposits, or to the pronoun referring to that for which the subject supposits; sc. such that the mode expressed in such a proposition may truly be predicated of an assertoric proposition in which the very same predicate is predicated of a pronoun referring to that for which the subject supposits, in a manner proportionate to that stated regarding propositions about the past and about the future.'
\end{quote}	
\end{frame}		
\begin{frame}{Divided modality in William of Ockham}
\begin{quote}
	`For example, for the truth of this: `Every truth of necessity is true', we require that any given proposition be necessary in which this predicate `true' is predicated of anything for which the subject [term] `truth' supposits? that is, that any such [proposition] be necessary:  `This is true', `that is true', indicating anything for which the subject supposits. And since not every such [proposition] is true, it follows that this is false simpliciter: `Every truth of necessity is true' \cite[II. 10, p. 249]{OckhamSL2}. 
\end{quote}
\end{frame}
\begin{frame}{Divided modality in William of Ockham}
	We can think of this passage as giving the following procedure for determining the truth of a modal proposition. \pause
	\begin{itemize}
		\item[1.] Introduce a collection of terms we can treat as rigid; for Ockham, both demonstrative pronouns as well as proper names have this feature.\pause
		\item[2.] Replace the subject term in the sentence with the variable. \pause 
		\item[3.] Check the truth-value of the sentence(s) predicating a mode, tense, etc. of the proposition indicated by the replacement sentence. 
	\end{itemize}
\end{frame}
\begin{frame}{Divided Modality in William of Ockham}
	\begin{itemize}
		\item[3.] Check the truth-value of the sentence(s) predicating a mode, tense, etc. of the proposition indicated by the replacement sentence. \pause 
			\begin{itemize}
				\item When the subject term is a name, definite description, or quantified by a particular quantifier, then the truth of one such sentence suffices for the truth of the divided modal claim; if the subject term is universally quantified, then the number of true sentences must exhaust the variables designating objects in the subject class. \pause
				\item EXAMPLE: if the only human beings are Socrates and Plato, then for the truth of `Every human of necessity is an animal', the sentences `that x is an animal is necessary' and `that y is an animal is necessary' must be true, where x and y designate Socrates and Plato, respectively.		
			\end{itemize}
		\end{itemize}
\end{frame}
\begin{frame}{Divided Modality in William of Ockham}
	\begin{itemize}
		\item To make this more conspicuous, let us introduce two-place modal and tense operators instead of their standard unary operators, where the first place is filled by a sentence, and the second by a term designating an object or collection of objects. \pause 
		\item For tense and possibility operators, we append subscripts to the operators to indicate whether they require their subject terms to supposit for present objects or past/future/possible objects.
\end{itemize}
\end{frame}
\begin{frame}{Divided modality in William of Ockham}
\begin{itemize}
	\item[($W_{1}$)] $W_{1}$(`$x$ is a playwright', The pope) \pause 
	\begin{itemize} 
	\item This requires that something presently designated by the definite description `the pope', and rigidly designated by $x$, was a playwright. \pause
	\end{itemize}
	\item[($W_{2}$)] $W_{2}$(`$x$ is a playwright', The pope) \pause 
	\begin{itemize}
		\item This is true in the case where x rigidly designates something that once answered to the definite description `the pope', but perhaps no longer does.  \pause 
		\item In neither case is it required that the sentence `the pope is a playwright' have been true.
	\end{itemize}
\end{itemize}
\end{frame}


\begin{frame}
	In brief, using $M_{d}$ as a placeholder for a binary divided modal operator; $M_{c}$, for a unary composite modal operator analogous to $M_{d}$; $p_{x}$, for a sentential function, with free $x$, denoting some state of affairs; $Q$, for a quantifier; and $t$, for a term denoting a class of objects, we can generalize Ockham's account as follows:
	\begin{quote}
		$M_{d}$($p_{x}$, $Qt$) $\Leftrightarrow$ for $Q$ formula(s) $p'$, where  $p'$ is exactly like $p_{x}$ except that each free occurrence of $x$ in $p_{x}$ is replaced by $x'$, where $x'$ rigidly designates some member of $t$: $v$($M_{c}$($p'$)) = True.
	\end{quote}
\end{frame}

\subsection{Scope, negation, and equipollence in Ockham's modal theory}
\begin{frame}{Scope, negation, and equipollence in Ockham's modal theory}
	\begin{itemize}
		\item Ockham's says sentences of the form `S is possibly P' are equipollent to sentences of the form `that S is P is possible' \cite[II. 9, p. 246; 10, p. 248]{OckhamSL2}. \pause 
		\item But Ockham's theory is complicated by the interaction of divided modals with negation. \pause
		\item According to Ockham, divided modals with unresolved negations fall into three types: \pause
		\begin{enumerate}
			\item Those where the quantity alone is negated; \pause 
			\item Those where both the quantity and the mode are negated; \pause 
			\item those where the mode alone is negated.
		\end{enumerate} 
	\end{itemize}
\end{frame}
\begin{frame} \small 
	\begin{tabular}{|p{2cm}|p{4cm}|p{4cm}|}
		\hline & \textbf{Example} & \textbf{Resolution}  \\
		\hline \textbf{Quantity negated}  & It is necessary that not every animal be a man. & It is necessary that some animal not be a man. \\
		& It is possible that not every man be an animal. & It is possible that some man not be an animal. \\ \hline 
		\textbf{Mode and quantity} & It is not necessary that every animal be a man. & It is possible that some animal not be a man. \\
		& It is not possible that every animal be a man. & It is necessary that some animal not be a man. \\ \hline 
		\textbf{Mode alone negated} & No man of necessity is an animal. & Every man can not be an animal. \\
		& No man can be an animal. & Every man of necessity is not an animal.\footnote{SL III-3. 14.} \\ \hline
	\end{tabular}
\end{frame}
\subsection{Relations between Ockhamist divided modal propositions}
\begin{frame} \tiny
	\begin{tabular}{|l|l|l|l|l|}
		\hline Equipollences & Contraries & Subcontraries	& Contradictories & Subalterns \\ \hline
		$\square(AaB)$ & $\square(AeB)$ & None & $\diamondsuit_{1}(AoB)$ & $\square(AiB)$ \\
		
		$I(AoB)$ & $\square(AoB)$ &  &  & $\diamondsuit_{1}(AaB)$ \\
		
		& $\diamondsuit_{1}(AeB)$ & & & \\ \hline
		
		$\square(AeB)$ & $\square(AaB)$ & None & $\diamondsuit_{1}(AiB)$ & $\square(AiB)$ \\
		
		$I(AiB)$ & $\square(AiB)$ & & & $\diamondsuit_{1}(AeB)$ \\
		
		& $\diamondsuit_{1}(AaB)$ &  & & \\ \hline
		
		$\square(AiB)$ & $\square(AeB)$ & $\diamondsuit_{1}(AoB)$ & $\diamondsuit_{1}(AeB)$ & $\diamondsuit_{1}(AiB)$ \\
		
		$I(AeB)$ &  & & & \\ \hline
		
		$\square(AoB)$ & $\square(AaB)$ & $\diamondsuit_{1}(AiB)$ & $\diamondsuit_{1}(AaB)$ & $\diamondsuit_{1}(AoB)$ \\
		
		$I(AaB)$ &  & & & \\ \hline
		
		$\diamondsuit_{1}(AaB)$ & $\square(AeB)$ & $\diamondsuit_{1}(AoB)$ & $\square(AoB)$ & $\diamondsuit_{1}(AiB)$ \\ \hline
		
		$\diamondsuit_{1}(AeB)$ & $\square(AaB)$ & $\diamondsuit_{1}(AiB)$ & $\square(AiB)$ & $\square_{1}(AoB)$ \\ \hline
		
		$\diamondsuit_{1}(AiB)$ & None & $\square(AoB)$ & $\square(AeB)$ & None \\
		
		& & $\diamondsuit_{1}(AeB)$ & & \\
		
		& & $\diamondsuit_{1}(AoB)$ & & \\ \hline
		
		$\diamondsuit_{1}(AoB)$ & None & $\square(AiB)$ & $\square(AaB)$ & None \\
		
		& & $\diamondsuit_{1}(AaB)$ & & \\
		
		& & $\diamondsuit_{1}(AiB)$ & & \\ \hline
	\end{tabular}
\end{frame}
\begin{frame}
Ockham's necessity pairs with $\diamondsuit_{1}$. But $\diamondsuit_{2}$ has no dual. Given this, we can subscript Ockham's necessity operator as $\square_{1}$, and add a second, $\square_{2}$, to complement $\diamondsuit_{2}$. From here, we note the following entailments:
	\begin{enumerate}
		\item $\square_{2}(AaB) \rightarrow \square_{1}(AaB)$, $\square_{2}(AeB) \rightarrow \square_{1}(AeB)$
		\item $\diamondsuit_{2}(AaB) \rightarrow \diamondsuit_{1}(AaB)$, $\diamondsuit_{2}(AeB) \rightarrow \diamondsuit_{1}(AeB)$
		\item $\square_{1}(AiB) \rightarrow \square_{2}(AiB)$, $\square_{1}(AoB) \rightarrow \square_{2}(AoB)$
		\item $\diamondsuit_{1}(AiB) \rightarrow \diamondsuit_{2}(AiB)$, $\diamondsuit_{1}(AoB) \rightarrow \diamondsuit_{2}AoB$ 
		\item $\square_{n}\phi \rightarrow \diamondsuit_{n}\phi$
		\item The subalternations of the assertoric square continue to hold when embedded under any mode. 
	\end{enumerate}
\end{frame}
\begin{frame} \tiny
	\begin{displaymath} 
	\xymatrix@C=.1em{
		& \square_{2}(AaB) \ar[dddd] \ar[drr] \ar [dl] \ar@{-}[rrrrrr] \ar@{-}[drrrrr] &  &  &  &  &  & \square_{2}(AeB) \ar[dddd] \ar[dl] \ar[dll]\\
		\square_{2}(AiB) \ar[dddd] \ar@{-}[urrrrrrr] &  &  & \square_{1}(AaB) \ar[dd] \ar[dl] \ar@{-}[rr] \ar@{-}[dr]
		&  & \square_{1}(AeB) \ar[dd] \ar[dl] \ar@{-}[dlll] & \square_{2}(AoB) \ar[dddd] &  \\
		&  & \square_{1}(AiB) \ar[dd] \ar[ull] &  & \square_{1}(AoB) \ar[dd] \ar[urr] &  &  &  \\
		&  &  & \diamondsuit_{1}(AaB) \ar[dl] \ar@{.}[dr] &  & \diamondsuit_{1}(AeB) \ar[dl] &  &  \\ 
		1 & \diamondsuit_{2}(AaB) \ar[dl] \ar[urr] \ar@{.}[drrrrr] & \diamondsuit_{1}(AiB) \ar[dll] \ar@{.}[rr] \ar@{.}[rrru]& & \diamondsuit_{1}(AoB) \ar[drr] &  &  & \diamondsuit_{2}(AeB) \ar[ull] \ar[dl] \\
		\diamondsuit_{2}(AiB) \ar@{.}[rrrrrr] \ar@{.}[rrrrrrru] &  &  &  &  &  & \diamondsuit_{2}(AoB) & }
	\end{displaymath}
\end{frame}

	\section{Conclusion}
	\begin{frame}{Conclusion}
\begin{itemize}
	\item The question of the syntactic scope of the quantifier is orthogonal to that of whether a modal sentence is to be construed as composite or divided \pause 
	\item Rather, there is a sense in which even divided modals can take `wide scope', though this sense will always be equivalent to a different sentence where the modal takes `narrow' scope. \pause 
	\item The real difficulties in representing Ockhamist divided modality do not so much concern the mode itself, but rather its relations to negation and quantification.
\end{itemize}		
	\end{frame}
	\begin{frame}
	\begin{itemize}
		\item For negation, we must represent the quality of the formula as belonging to the \textit{predicate}, rather than the subject, if we wish to obtain the correct truth conditions for a wide-scope divided modal. \pause  
		\item Ockhamist quantifiers do not ampliate. Ockham requires that the range of quantification be specified independently of its place in the sentence: 
		\item Every sentence containing an intensional operator will thus be semantically ambiguous between different readings in accordance with the different possible readings of the scope of quantification.
	\end{itemize}
	\end{frame}
\end{document}


