\section{Appendix}
%and that consequently, 
%stuff about Aristotle on being and the same, 
%the weakness and derived character of the analytic debate
%better explains the intentional unity of metaphysics itself
%better conforms to the understanding of the subject of a science in the medieval context
%the graph is more important than the points

%Function argument distinction is itself likely, by way of a long historical chain, indebted to the Via Antiqua analysis that Klima mentions
%Distinction between names  and predicates only makes sense as a corollary to an ontological distinction between things and properties.
%Frege gets the linguistic distinction wrong by construing it globally at the level of the language, where Buridan gets it right by construing it at the sentential level. The correct account is also in Aquinas' De Ente.

troublemakers cosplaying as scientists by making the meanings of common terms unduly obscure

1. Frameworks may differ in their expressive power.
2. This is true not only of artificial, but also of constructed languages.
3. 

%Likewise, the stance expressed in this critique is much closer to the stance of Klima than that of his late 90's Notre Dame colleagues, 
%while also expressing the claim that there is a respect in which they were right in spite of themselves. 
%Meaning of \emph{being} [ontology] vs \emph{meaning} of being [semantics] - Aristotle's remark that logic and metaphysics concern the same. 

%There is a respect in which Klima's Notre Dame colleagues were right in spite of themselves
%diminishing effect of $\diamond$ on quantifiers in its scope.
%And yet both the possibilist and actualist will find themselves in agreement against the classical logician.
%The denial of a metaphysical thesis remains a metaphysical thesis. 

In any case, it expresses a sense in which there are at least two senses of being. 
Likewise with the acceptance or rejection of the axiom of reducibility, 
the implicit exclusion of contradictory objects from the domain, 

%Such a semantics should already be unacceptable to the Megarian, the Monist, the Spinozist, etc. 
- The grounding of the unique relation of semantics to metaphysics in the convertibility of being and truth.

%How does this allay Klima's worries? It shows that the problem was never about metaphysical presuppositions, and always about tribalism.

Indeed, this helps form the foundation for the possibility of recognizing that an ontology can be inadequate, 
%as, for instance, Klima himself recognizes that restricted quantification is a better model of x-that-is-a-F than of `F'
%(I think the reference is to ) 

%Not everything is decided: e.g. domain can be interpreted as tokens rather than types

%Metaphysics is not prejudiced at the level of judgment \emph{about} the model, but it is at the level of the model itself. 

%the type of x in the universal quantifier is confused.
%Translation from classical into free logic. 
%Even in propositional calculus, the notion of being is retained confusedly. 
%higher order entities aren't quantified over in first-order logic
%Everything is in one of the ten categories. The Russel set is something, therefore, etc.
%Type theory. Problems of naive set theory, superenumerable domains, the axiom of reducibility, etc. 
%1. Conflicts with the Aristotelian notion of science
%2. The failure to mark off a domain.
%2. Its fruitlessness. (i.e. the Quinean program in metaphysics has been exhausted. )
%3. The Quinean notion may be subsumed by the Aristotelian one.
%The fact that claims of cardinality are derivable in Tarski's original concept of semantics is a problem, and prima facie evidence that it's mistaken.
%Everything is or isn't my eldest daughter arguably isn't a thesis of metaphysics. 
%Primacy of first syllogistic form over the third.
%The semantics, in itself apart from our judgments about it, will not be free from ontological assumptions.
%Why this isn't an ignoratio elenchi
To begin, it's worth taking a moment to recognize just how odd this is. 
It's not that other disciplines lack statements of this sort - even purely mathematical disciplines have famous theorems proving the existence or non-existence of things like superenumerable domains or greatest prime numbers - 
but it is that even all these years after Aristotle, the principal statements even in newer bodies of knowledge, including semantics itself, tend to take the form of affirmations or denials of intrinsic or extrinsic properties of that science's principal subject and/or its specific and integral parts (Cf. Post. An. 71a 11-16).

rather than, 
as is the case with every other known science in the ancient world,
a body of statements affirming and denying properties of a fundamental notion and those that are integral to it. 
%Consequently, if metaphysics is taken in the sense in which it has been for most of the history of analytic metaphysics, Klima's claims may hold, 
%but not in 

%The fragment of classical propositional calculus containing only $\neg$ and $\wedge$ as logical constants equally interpretable as both a logic for provable entailments interpreting $\wedge$ as English `and' 
%and as one for refutable entailments interpreting \wedge as English `or'

%e.g. there is a notion of being implicit in that of \emph{being} entailed by 
%Incompleteness theorem is a particularly bad case of this: A system can state its own consistency only if it's inconsistent.

Domain variation

a language may be more expressive merely in the sense of having symbols with derived meanings 

Every model available in classical logic is also a model available in Kripke possible world semantics.
Expressive = has more theorems? or has more models?
Still, the independence of semantics from 

%1. Some frameworks are more expressive than others
%2. The claim that semantics is independent of metaphysics is only true in a less expressive frameworks, i.e. that of modern analytic philosophy
%3. If we accept the classical notion of metaphysics, the claim is false. 

%Even analytic philosophy today recognizes that the Quinean program in metaphysics has been exhausted, as evidenced by the proliferation of new fundamental concepts of metaphysics (e.g. Schaeffer).
%Objections:
%Is metaphysics deferred indefinitely? 
% can anything be proven in it, or is everything debatable so long as we are willing to make adjustments elsewhere?
% 


The case of one framework being fundamentally more expressive than another.
\footnote{
	Indeed, we have a close analogue to this in medieval semantics concerning signification, 
	where the signification of a term is taken by the majority tradition to mean the meaning of the term 
	considered in itself apart from any propositional context, 
	while a minority including William of Sherwood and Roger Bacon identifies a term's signification with its reference to actual presently existing things. 
	See \autocite[170-171]{DeRijk1982}. %Don't like this. 
} 
\subsection{Semantics and expressiveness}
Let us begin, in good scholastic fashion, with a distinction.
`semantics' is said in many ways: 
Minimally, it can refer to any number purely mathematical disciplines providing rules for recursively mapping symbols of an artificial language via interpretation functions to domain elements, sets, types, truth functions, etc. - essentially what today we mean by model theory and other closely related disciplines. 
In only a slightly more robust sense, 
it can refer to the same discipline, 
albeit along with its intended informal interpretation such that, 
e.g. that fragment of the classical propositional calculus containing $\wedge$ as its only a logical constant be understood as providing a logic for entailments with that symbol intuitively given the meaning of `and', 
and not one for refutations where it is intuitively given the meaning of `or' .
In a third sense, 
it can refer to a discipline for providing rules for the use of terms in a language, as exemplified in the tradition of proof-theoretic semantics.\footnote{See \autocite{Francez2016b,Read2010,Schroeder-Heister2006}.} 

%Cognitive semantics
\subsection{Objections}
1. Frege point - semantics can have presuppositions without the acceptance and judgment. 
2. Object-language metalanguage distinction is unnecessary. Therefore metalanguages aren't a good source for ontological commitment. 
\section{Notes}
%Now, as a relatively trivial example, the claim that \emph{something} exists, i.e. $(\exists x)x = x$ is a theorem in nearly all semantic systems of predicate logic with identity, 
%given the exclusion of non-empty domains from the class of models.

%The problem: the task of positing different senses of being to avoid full ontological commitment isn't a way of avoiding metaphysics: it \emph{is} metaphysics.
Modern metaphysics confuses the question of what being is with the question of what beings there are. (this isn't a new insight: once one gets past the mysticism of some of his interpreters, this is the fundamental point behind the talk of ontological difference in the philosophy of Martin Heidegger)
\subsection{Positions taken}
In \autocite{Klima2008a}, Klima recognizes that representing \emph{via antiqua} semantics would require substantial modifications to modern quantification theory, while representing \emph{via moderna} semantics requires fewer modifications
1) Via antiqua semantics requires a different account of predication, and multiple copula to be introduced to represent the different senses of being
2) Via moderna semantics requires the introduction of restricted quantifiers.
3) both require the rejection of the object-metalanguage distinction.

Klima agrees with Buridan that the notion of truth is not strictly needed for a semantics concerned with formal validity, but it is needed to explain the semantics of sentences that themselves predicate that notion.
Analogy: nobody complains that we don't have a formal definition of the term `red' in our logic, even though a basic grasp of the semantics of that term is needed for using the term in sentences about red things.

Two uses for semantics of truth: 1) as part of a theory of validity, 2) for its own sake.
In `Logic without truth', Klima rejects Buridan's solution to the liar paradox.

%The Rises and Falls of Analysis and Metaphysics in Metaphysical Themes, Medieval and Modern - highly autobiographical. Helps to explain his aversion to the idea of metaphysics bleeding into semantics

%Note. Klima can't get away from the notion of semantics as functions for mapping language to reality. This is part of why he can't accept the idea of metaphysics impinging on logic itself. But this is exactly what the theory of the different main types of supposition, grounded in a theory of analogy, should entail.

(1) Natural languages are semantically closed (2) Natural language inference has to be token-based.
Both Klima/Buridan and Tarski come to the conclusion that defining consequence in terms of truth and falsity doesn't work from similar considerations: Tarski's consideration is related to superenumable domains and the possibility that a language may simply an appropriate selection of denoting terms; Klima/Buridan's considerations come from the possibility that a claim may be not exist to even be true or false, or it may be self-falsifying while nevertheless describing a possible state of affairs \autocite[96]{Klima2004}.

The primary impetus behind Klima's work is one of charity.
Examples: 
Positive:
1. His analysis of parasitic reference in his work on Anselm
2. His attempts to translate between via antiqua and via moderna semantics
3. The entirety of his body of work on John Buridan
Negative:
1. The infrequency with which Klima actually reveals his own philosophical positions in his work (exceptions: 
Aquinas' hylomorphism and proof of immortality, 
Anselm's proof, 
Per Buridan, the semantic closure and token-based character of natural language inference)
2. His adopting semantics that build on classical logic while rejecting non-classical semantics.

%Metaphysical questions are undecidable by semantics alone. In the same sense in which the body of truths of revealed theology are undecidable by natural philosophy, or the continuum hypothesis is undecidable by set theory.
%The rises and falls of analysis and metaphysics
%Being, unity, and identity in the Fregean and Aristotelian traditions

%metaphysics \textit{does} determine semantics. Consequently, the attempt to infer metaphysics from semantics deductively, as with all similar cases of causality, would logically have to be a case of affirming the consequent. 
%demonstratio quia

\subsection{Why semantics?}
Semantics: 
1) a theory of meaning broadly construed
2) Tarskian/Montaguean mathy stuff
a semantics is almost never actually this, given that most semantics have a canonical interpretation and a domain to which they are expected to apply (e.g. Model theory handles solids better than liquids or gases).
3) e.g. a dictionary
4) e.g. proof-theoretic semantics
5) a philosophy of language
6) a theory of language, thought, and reality
%Two contradicting claims: 1) The modern turn in metaphysics was conditioned by a prior one in semantics; 2) neither via antiqua nor via moderna semantics strictly entail the metaphysical positions they are usually associated with.

%Two phenomena: 1) the breakdown of communication within analytic philosophy 2) its lack of breadth

%Buridan's logic and the ontology of modes - neither via antiqua nor via moderna semantics entailed realism or reductionist metaphysics, respectively.

\subsection{Buridan}
Discussing the value of studying Buridan's philosophy in particular and that of the history of philosophy generally, he writes: 

\begin{quote}
	this study can put our own philosophical problems in an entirely different light, providing us with
	such theoretical perspectives that otherwise might entirely escape us as we are working in our
	set ways determined by the intellectual habits of our philosophical period, which in modern
	times tends to stretch to a mere couple of decades. \autocite[17]{Klima2005}
\end{quote}
Besides the frank exhortation to study the history of philosophy as a way to expand one's intellectual horizons, 
one finds an indictment of the historical shallowness of much philosophy in our own period. 

%I personally remember feeling the bait-and-switch that seemed to occur when I moved from a historically rich undergraduate program to a heavily analytic department.
%The rank materialism of the vast majority of professors didn't threaten me: instead, I found it completely stupid..

an expansion of one's horizons that goes beyond merely getting the right answers.

- This is philosophy, and its difference from sophistry.

An atheist who's one \emph{modus tollens} away from sincere devotion.
%What is the criterion for provability of a metaphysical claim: provability across all frameworks?

%\section{Philosophy of mind and epistemology}
%\subsection{\textit{Via antiqua} and \textit{via moderna} cognizers}
%\section{Metaphysics}
%\subsection{Hylomorphism, personal identity and immortality}
%\subsection{Causation}
%\section{Overview of the articles}

\section{quotes}
%The efficient causality model fixes the relation of natural signification on the basis of natural laws systematically connecting causes with their effects. But all that these natural laws guarantee is the systematic correspondence between causes and their effects, supposing the normal course of nature. The effects, however, may be essentially different from their causes, and may, therefore, be produced also by other essentially different causes too, which means that it is clearly possible that an absolute concept be caused in our intellects by a cause which is totally alien from what this concept is supposed to represent. - Ontological Alternatives.
%IMPORTANT!, the immediately above quote is self-defeating for Klima, since his remarks on the relation between semantics and metaphysics are themselves conceived on the efficient, rather than formal, causality model mentioned above.

`The primary purpose of a logical semantic theory is to define logical consequence in terms of the truth values of propositions in different interpretations' \autocite[79]{Klima1991b}. 

However, again, if we need to represent the finer details of the via moderna
conception, we need to depart considerably from the standard construction of the
semantics of quantification theory. In the first place, although using restricted
variables to represent common terms in personal supposition yields “the right re-
sults” concerning the square of opposition and syllogistic, nevertheless, it does so
at the expense of representing simple common terms (say, F ) as complex vari-
ables with an intrinsic propositional structure embedded in their matrix (‘x.F x’,
amounting to something expressible as ‘thing that is an F ’). But according to
our via moderna authors it is the simple term ‘man’, for example, that has this
referring function, and not a complex term like ‘thing that is a man’. In fact Buri-
dan would pointedly distinguish the two in various contexts. So, to represent this
feature of via moderna semantics, we would need to devise “term-logics” along the
lines proposed by Lesniewski, Lejewski, Henry, Sommers and Englebretsen.\autocite{Klima2008a}

Buridan has only one language to talk about the world as well as about that
language and its semantic relations to the world. And in that one language
we cannot truly say that there are mere possibilia, or that something that is
merely possible exists. Accordingly, from this Buridanian perspective, the
issue of ontological commitment in terms of a meta-linguistic description of
the relationship between language and the world is radically ill-conceived.\autocite[172]{Klima2009}

A merely possible being or a fi ctitious entity is not just
a special kind of entity; indeed, no more than a fake diamond is a special
kind of diamond or forged money is a special kind of money. Just as a fake
diamond is not something that is both a diamond and is fake and forged
money is not something that is both money and forged, so a fi ctitious entity
is not something that is both an entity and is fi ctitious. And just as a fake
diamond is no diamond at all, and forged money is no money at all, so a
fictitious entity is not an entity at all.\autocite[173]{Klima2009}
%Good quote for illustrating Klima's reliance on Geach here. 

`Buridan’s nominalism is obtainable by the adverbialization of Peter
of Spain’s semantics.'
`Nominalism is obtainable by the adverbialization of realist
semantics.'


A project of eliminating unwanted
ontological commitment is not at all about finding out anything about anything;
rather it is a project to show just how much one can get away with in one’s
semantics on the basis of how little in one’s ontology \autocite[412]{Klima2012}.

In this paper I will attempt to dig further to the roots of
their disagreements, trying to establish those primary logical-semantic differences that may have
motivated their conflicting intuitions concerning these metaphysical principles. - Thomas of Sutton v. Henry of Ghent

Therefore, it should also be clear that the laws of logic in this framework
are supposed to be fundamentally different from the laws of psychology.
For while the former are the laws of the logical relations among objective
concepts, the latter are the laws of the causal relations among formal
concepts. - The problem of universals and the subject matter of logic, Klima 2014, p. 173.

These different theories can be arranged on a `theoretical scale', ranging from extreme realism to extreme nominalism, meaning maximal semantic uniformity along with maximal ontological diversity on the realist end [...], and maximal ontological uniformity with maximal semantic diversity on the nominalist end.- The problem of universals and the subject matter of logic, Klima 2014, p. 176.

Well, conceptual diversity is obviously
a great hindrance to understanding: if we don’t have the same concepts, we can-
not have the same thoughts, which means we are doomed to talking past each
other all the time\autocite[36]{Klima2021}

So, what should be our guiding light, in this rational discourse? In one word:
rationality, which is love or goodwill on its active side, on the part of the will, and
understanding on its receptive, theoretical side, on the part of the intellect \autocite[41]{Klima2021}.
\autocite{Parsons2014,Read2015b}

