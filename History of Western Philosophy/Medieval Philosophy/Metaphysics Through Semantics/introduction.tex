\documentclass[]{article}
\usepackage[backend=biber, style=authoryear-icomp]{biblatex}
\bibliography{../../../jacob}

%opening
\title{Introduction}
\author{Jacob Archambault}

\begin{document}

\maketitle

\section{Klima's contributions in the history of semantics}
The breadth of Klima's scholarship stretches 
historically from some of philosophy's best-known figures in Anselm, Aquinas, Ockham and Descartes 
to lesser-known figures including Thomas of Sutton and Henry of Ghent, 
to Frege, Geach, Kenny, and others who have produced some of the most impactful scholarship in the analytic tradition; 
and 
thematically from debates 
on identity, categories, and causation in metaphysics, 
on skepticism in epistemology and theories of mental content in philosophy of mind, 
to others too numerous to mention. 

From his earliest work in semantics, 
Klima recognized that classical logic, 
being primarily interested in developing an account of the semantics of propositions as a precondition for the development of a theory of consequence, 
affords much less attention to its account of the components of propositions themselves.
%namely names and n-ary predicate relations. 
Klima fills this lacuna by providing some of the earliest and most ambitious applications of restricted quantification in the history and philosophy of logic, 
using it both to formalize the medieval theory of supposition 
and to provide a general account of quantitatively ambiguous natural language sentences \autocite{Klima1988,Klima1990,KlimaSandu1990,Klima1991b}. Cf. \autocite{Parsons2014}. 
Expansions on the same theme - 
namely, formalizations of supposition theory specifically 
and medieval semantics more broadly 
as a means to resolve apparently intractable interpretative problems in historical scholarship 
and debates in contemporary philosophy - 
provide us with an account of the semantics of intensional verbs \autocite{Klima1991}, 
a semantic foundation for Aquinas' theory of the analogy of being in his theory of the copula \autocite{Klima1996,Klima2002}, 
and a clean resolution of the problem of existential import in the Aristotelian square of opposition \autocite{Klima2001}. 

Elsewhere, Klima decouples \emph{via antiqua} and \emph{via moderna} semantics 
from the realist and anti-realist metaphysics with which they are most commonly paired,
contending that neither semantics by itself strictly entails its associated metaphysics \autocite{Klima1999,Klima2011}. 
Rather, archtypical realists were required to adopt non-straightforward semantic accounts of the meanings of terms in at least some cases by their antecedent metaphysical commitments (e.g. to divine simplicity) \autocite{Klima2002b},
while some of the best known nominalist logicians incorporated what today would be regarded as `realist' elements in their logic \autocite{Klima2005}. 
For Klima, the \emph{via antiqua} and \emph{via moderna} traditions of medieval logic 
differ not in their \emph{quantity} of ontological commitments, 
but in the tools they provide for \emph{handling} ontological commitments, 
which in turn differ from those of the model-theoretic framework dominant today. 
\emph{Via antiqua} semantics takes an affirmative statement to be true when what is signified by its predicate inheres in what is signified by its subject - 
sometimes called the \emph{inherence theory of predication}. 
Within this framework, 
terms predicating common natures or accidental features of a subject are taken to ultimately refer to exactly the categorical entities one might expect. 
But the framework 
avoids full, immediate, fundamental commitment to entities today's nominalists might find objectionable by providing a rich theory according to which being is predicated in different degrees.\footnote{See \autocite{Klima2002}.}
Conversely, the \emph{via moderna} framework that became ascendant after Ockham 
takes an affirmative statement to be true when its subject and predicate term refer to the same object 
 - sometimes called the \emph{identity theory of predication}. 
Within this framework, 
terms predicating common natures or accidental features of a subject need not be taken to ultimately refer to different types of objects such as abstract genera or relations, 
but instead refer to familiar objects \emph{differently}. 
For example, the truth of `Socrates is a father' does not require commitment to a distinct entity that is Socrates' fatherhood.
Instead, the sentence's predicate may (non-rigidly) refer to Socrates himself, 
albeit connoting his being a father, 
and hence refer to the same object as that rigidly referred to by the proper name `Socrates'. 
Granting some license for intensional contexts,\footnote{See \autocite{Klima2005}.} 
the verb `is' or `exists' in \emph{via moderna} semantics is equally ontologically committing in its various uses, 
but \emph{what} one is committed to by its uses need not be immediately apparent  \autocite[437-430]{Klima2008a}. 
Both medieval frameworks would reject the object-language metalanguage distinction taken for granted since Tarski in their theory of truth, 
and 
in their use of ampliation for tensed, modal, and intensional contexts,
both provide ample tools for rejecting a na\"{i}ve application of Quine's account of ontological commitment in terms of existential quantification.\footnote{Cf. \autocite{Klima2004}, 
\autocite[171-174]{Klima2009}}

\subsection{Ontological neutrality and independence}
None of this means that there is \emph{no} relationship between an author's positions in metaphysics and his semantics: 
rather, 
the semantic framework an author adopts conditions what options that author has in metaphysics without fully determining them. 
For example, extreme realism in metaphysics doesn't follow strictly from the \emph{via antiqua}'s inherence theory of predication, 
but it is the most natural fit for that theory 
if one accepts the view that terms signifying accidental being denote their referents rigidly 
while rejecting that framework's insistence on multiple, analogically related senses of `being' \autocite[125]{Klima1999}. 
Conversely, the broad outlines of Ockham's account of the relation between language, thought, and reality 
serve not only as a foundation for Ockham's own metaphysical reductionism, 
but also for the realism of a Descartes, Malebranche, Putnam or a Leibniz \autocite{Klima1991}. 
In the most extreme case, the choice of a mistaken semantic framework may inhibit the speaker from constitutively referring to, 
and thus believing in, 
an actually existing God whose existence is only adequately assertable in an alternate framework \autocite[74]{Klima2008b}. %This page reference is an estimate. P. 22 in Klima's online copy.

What is clear, however, is that there is no relationship of \emph{entailment} from purely semantic principles to metaphyscal truths. Klima writes: 
\begin{quote}
To be sure, this is not to say that metaphysical principles are to be derived from, or somehow justified 
in a weaker sense on the basis of, semantic principles. Metaphysical principles, being first principles 
using the most general terms, such as the transcendentals and the categories, cannot be derived from 
prior principles, and their terms cannot be defined on the basis of more general terms. What 
semantics can do, however, is that it can provide the principles of interpretation of metaphysical 
principles. On the basis of these principles of interpretation the implications of metaphysical 
principles are more clearly delineated, which then can be used in their evaluation in dialectical 
disputations concerning their acceptability in the interpretations thus clarified. Furthermore, if the 
semantic principles of interpretation are made explicit, they can also be subject to further evaluation, 
in a disputation on a different level, the sort appropriate to the comparison of different logical 
theories  \autocite[49]{Klima2011b}.
\end{quote}

Modern mathematics calls this relation of non-entailment \emph{independence}, 
though as the name implies the fundamental notion itself is by no means a recent one. 
Just as Cantor's continuum hypothesis is neither provable nor refutable from the principles of Zermelo-Fraenkel set theory alone, 
or - to provide a medieval example - 
truths of revealed theology are neither provable nor refutable from the principles of natural philosophy, 
neither on Klima's account are metaphysical principles provable or refutable from those of semantics alone. 

\subsection{Pluralism, linguistic imperialism, and the problem of cross-cultural communication}
Two complications distinguish the semantic case from those mentioned. 
The first is that while both the set-theoretic and theological case mentioned above concern provability and refutability in a single system, 
the sheer multiplicity of semantic frameworks itself may provide a barrier to a broadly acceptable account of provability across those frameworks. 
The second generalizes a problem nearly the opposite of that established by G\"{o}del in his first incompleteness theorem \autocite{Godel1931}:
where that theorem established the expressibility of unprovable claims of number theory in any sufficiently robust system, 
the semantic problem we face here is that a claim of metaphysics may be taken to be established or refuted merely on account of the \emph{lack} of expressibility of the particular semantic framework one is working in. 

Klima's response to these problems is anti-pluralist 
without thereby being dogmatically classical. 
While it would be easy enough to, for instance, 
construct a metalogical account of validity by quantifying over distinct logical systems on the model of possible world semantics and regarding as valid all and only those theorems valid in every system, 
Klima instead recognizes the known limitations of classical semantics 
while also taking the provable equivalence of systems containing distinct logical primitives
as \emph{prima facie} evidence for the possibility of a fundamental diversity at the \emph{conceptual} level 
that nevertheless doesn't entail a despairing or indifferent anti-realism at the \emph{metaphysical} level \autocite{Klima2012}.
Instead, 
Klima's response, 
on both a technical and philosophical level, 
is to \emph{extend the framework}. 
Meeting the tradition where it is, he extends classical semantics to allow for treatment of 
donkey sentences \autocite{Klima1988,Klima2010}, 
non-existent entities \autocite{Klima2001}
and quantificational phenomena \autocite{KlimaSandu1990},
while more broadly appealing (in a rare quote of a `continental' philosopher that shows up in multiple places throughout his \emph{oevre}) 
to the possibility of a `fusion of horizons' mentioned by Gadamer 
as a solution to the impasse of communication across distinct semantic frameworks, cultures, or philosophical traditions 
and the attitude of metaphysical anti-realism it encourages \autocite{Klima2000,Klima2009a}.

\subsection{Why medieval semantics?}
Despite the depth and breadth of his work, 
the amount of space Klima devotes to advancing positions that are unambiguously his own, 
rather than to steel-manning positions of historical or contemporary figures 
he may or may not agree with,
is comparatively little.\footnote{Exceptions include his acceptance of both Anselm's proof of God's existence and Aquinas' proof of the immateriality of the intellect as sound \autocite{Klima2000,Klima2009a} 
	and his advancing, 
	based on an examination of Buridan's treatment of reciprocal liar paradoxes, 
	that any adequate semantics for natural language must be semantically closed and token-based \autocite{Klima2004}.} 
Still, 
Klima's solution here provides a window into the answer to a more personal question that his scholarship solicits: 
namely, 
of all the areas of philosophy to devote oneself to, 
why study medieval philosophy, 
and specifically medieval semantics?

In one uncommonly autobiographical passage, Klima writes: 
\begin{quote}
I remember that when I was at Notre Dame (so this happened in the second
half of the nineties), I asked several of my colleagues, and even the then
visiting David Armstrong, to provide metaphysically non-committal
clarifications of the semantics of the language they were using in
describing their metaphysical theories. In response, I was given puzzled
looks and declarations strongly reminiscent of the way medieval
nominalists characterized the attitude of their realist opponents: we don’t
care about names; we go right to the things themselves!—Well, just look 
at the history of late-scholasticism and early modern philosophy to see
what good that attitude did for them.

So, what can we do to avoid the late-scholastic scenario, going on another
cycle of endless and more and more meaningless metaphysical debates
until the arrival of another Kant declaring the whole enterprise ill-founded
and another Carnap declaring it to be meaningless, to launch another anti-
metaphysical cycle of meaningless search for meaning to be abandoned
yet again for metaphysics, etc., etc.? Why don’t we try both in tandem,
i.e., analysis and metaphysics at the same time, as the very designation
“analytic metaphysics” would seem to demand? For then we could start by
laying down our clearly defined semantic principles (instead of making
them up and twisting them around as we go) and engage each other in our
metaphysical debates according to the same principles, instead of talking
past each other, making clear that whoever is talking according to different
semantic principles is just playing a different game \autocite[86-87]{Klima2014}.
\end{quote}

Here, the difficulty that Klima's apology for semantics aims to alleviate remains - 
namely, that in much debate in the core disciplines of analytic philosophy  
and in metaphysics in particular, 
rival participants are often unable or unwilling to state their positions in a linguistic context their opponents would be able to agree to, 
leaving such debates unfruitful from the start. 

Against this, 
the independence of metaphysical claims from the semantics in which they are expressed takes on the character not only of a metaphilosophical thesis, 
but also of a moral one: 
without the opportunity for common ground that semantics provides, 
not only shared understanding, 
but even proof and refutation, 
intellectual conversion and disagreement itself become unattainable.

With this problem in mind, the study of medieval semantics, 
as a study of frameworks of meaning remarkably foreign to that of our own time, 
provides an example \emph{par excellence} of the kind of interpretive charity needed to surmount our own crises of meaning and communication. 
\section{John Buridan}
Nowhere has this effort been more sustained than in Klima's scholarship on John Buridan', 
which has helped elevate the 14th century Arts Master from a lesser-known figure to one whose stature is closer to that of an Ockham, 
arguably surpassing the Franciscan in his logic.

\section{Semantic expressiveness and the meaning of metaphysics}
In this way, 
Klima's seminal contributions on Buridan and in medieval philosophy more broadly provide an excellent example of how one can solve apparently intractable philosophical and communicational problems 
simply by \emph{expanding the framework}
 - whether that framework be classical semantics expanded to include an existence predicate and restricted quantifiers
 or the broader historical consciousness of Anglophone philosophy of the past seventy or so years. 
Still, one wonders whether one can take the fundamental motivation further than even Klima himself does. 

\subsection{The concept of expressive power in formal systems}
We know that some semantic frameworks are fundamentally more expressive than others. 
This can occur in a trivial sense when a language introduces new derivative syntactic elements, 
as occurs e.g. when we introduce a symbol $\rightarrow$ into classical propositional logic 
as a shorthand for $\neg(A \wedge \neg B)$. 
In other cases, e.g. by augmenting classical first-order logic with an identity predicate, 
we can add a new logical constant to a language that thereby allows for the expression of previously inexpressible theorems and entailments. 
In still other cases, 
as occurs, for instance when extending the relatively minimal normal modal logic $K$ to a more robust one like $S4$ or $S5$
we can provide a more robust semantic interpretation of a term or symbol by further restricting the class of its previously permissible interpretive models, 
and thereby establish theorems and entailments for which countermodels previously existed.
In still others, we can provide a translation of the claims of a logic $L$ to those of another $L'$ 
such that, 
given a translation function $f$ from $L$ into $L'$, 
a claim $c$ is provable in $L$ exactly when  $f(c)$ is in $L'$.
Perhaps the best-known case of this last mentioned is G\"{o}del's proof that the claims of classical arithmetic can all be expressed in intuitionistic arithmetic, 
with the base case being that a claim $A$ holds classically if and only if $\neg\neg A$ holds intuitionistically. 
This particular example also shows that a language may be fundamentally more expressive than another even when we have the same set of lexical elements defined across both systems, 
and even when the weaker system appears at first glance to be the stronger one. 

\subsection{On what ontology is}
%Semantics: 
%1. Presupposes non-empty domain
%2. that its domain doesn't include impossible objects 
%3. Tarskian semantics includes a fixed domain. 
%4. Liar paradoxes assume that our domain may contain both first-order elements (e.g. dogs) and higher-order elements (e.g. sentences about dogs).
%	cf. Axiom of reducibility
%- In short, there are tacit assumptions about the extension of `being' that are built into our various modal logics. 
%2. Counterexample again: no one assumes that natural language is actually functional, given the possibility of equivocal and analogical meanings - or even of material or simple supposition. 
% An axiomatization of Euclidean geometry does not thereby deny the existence of non-euclidean spaces, nor (more broadly) of non-geometrical objects. 
%In short, there are questions that every semantics, considered in itself, must implicitly or explicitly decide. Among these is the question of what there is. 
Expanding this point about technical frameworks to natural ones,
we can say that 
philosophy of the past seventy or so years has primarily interpreted ontology 
as a discipline whose theorems are, 
in their basic syntax, 
positive or negated existential, non-copulative uses of the verb `is' or `exists' joined to a common or proper noun, 
e.g. `baseballs exist', `bats don't exist', etc. 
Given a suitably robust concept of induction, it then becomes possible to make universal statements about what is, 
usually expressed formally via a universal quantifier ranging over a disjunction, 
with each disjunct
attributing a monadic predicate to the value of a variable bound by the universal
quantifier 
such that 
the intersection of the value any two predicates is the empty set, 
the empty set is the value of no predicate, 
and the union of all of them
yields the domain of quantification - 
e.g. Everything is God or a creature; 
a mind or a body; 
a person or a mere thing; 
a substance or quantity or quality, etc., 
to provide some of philosophy's better-known examples. 

This understanding of ontology, 
however, 
is arguably 
%not the sense in which that part of metaphysics was understood by the ancient and earlier medieval interpreters of Aristotle, 
%but 
an attentuation of a broader notion,
one concerned not primarily with the question of what things there are, but of what \emph{being is}. 
On this understanding, 
the principal syntactic form that a statement in ontology will take will be one predicating some monadic or disjunctive predicate of being itself as a subject. 
and claims like `everything is a substance or a quantity...', etc., 
if true, 
will be derivative on that that \emph{to be} is to be a substance or a quantity, etc.\footnote{
One simple yet powerful validation of the primacy of this understanding of ontology is that the converse entailment schema intuitively fails: 
the fact that everything is or isn't my eldest daughter, for instance, 
does not entail that \emph{to be} is to be or not be my eldest daughter. 
Furthermore, the reason for this failure is itself intuitively clear, 
namely, 
that it doesn't belong to the \emph{meaning} of being to contain any information about my daughter one way or another. Cf. \autocite{Fine1994}.}
%Essence : modality :: being : quantification. Cf. Kit Fine, Essence and Modality 
%What is said essentially of a subject is different from what follows necessarily from what is essentially predicated to that subject. 
Consequently, we find the classical expanse of ontology embracing matters not typically covered in its more recent Anglophone counterpart: 
the convertibility of the transcendental predicates with being itself;  
the different senses of being pertaining 
to past present, and future being;
to real and rational;
to substance and accident; 
to potential and the actual, 
along with the relations of priority, posteriority and relative perfection displayed therein
that prevent these distinctions from being understood as differences in kind. 
%Life is beautiful, it is raining, what's going on? 
%attitudes of love, happiness, boredom, angst.  
%Analytic metaphysics is an effect of metaphysics, but is not itself metaphysics.

But on this understanding, the task of positing different senses of being to avoid full ontological commitment - 
or, for that matter, refusing to make such distinctions - 
isn't a way of avoiding metaphysics: it \emph{is} metaphysics.

This doesn't entail that a semantics is ontologically committing considered purely as an interpretation mapping lexical elements to a suitable mathematical model: 
nothing, for instance, so much as entails $\wedge$ be understood to mean `and'.\footnote{For example, 
	that fragment of the classical propositional calculus containing $\wedge$ as its only a logical constant may be understood as providing a logic for preserving falsehood rather than truth 
	where $wedge$ is intuitively given the meaning of `or' . Cf. \autocite[222]{Kripke2015}.}
Nor does it entail any commitment at the level of our judgment concerning a particular system: 
to consider some of its less plausible claims, 
one may, for instance, adopt a standard classical semantics for any number of reasons 
without thereby committing oneself to the claims 
that relations are sets of tuples, 
that natural language reference is functional in the mathematical sense, 
that there has to be at least one thing (though not always \emph{the same} one thing), 
or that modality doesn't exist.
It does, however, mean
that a semantics containing structures meant to model being as a whole, 
considered in itself and on its intended interpretation, 
must decide ontological questions on the matters it pertains to directly. 

Let us take, for example, the case of the convertibility of actualist and possibilist quantification. 
On possibilist semantics, a model $(W, R, D)$ consists of a frame $(W, R)$ 
itself consisting of a set of worlds $W$ and a binary relation $R$ on worlds, 
and a set $D$ containing all elements existing at any element in $W$. 
Quantifiers range over $D$ in its entirety, 
and to represent actual existence at a given world $w$, 
possibilist semantics employs a monadic predicate $E$ 
whose interpretation intuitively maps to the set of actual existents at $w$. 
Conversely, on actualist semantics, 
we take a model $(W, R, D)$ to be as before, 
but with $D$ being not a set, but a function from each element $w$ in $W$ to its actual existents. 
The interpretation of a quantifier at $w$ is restricted to $D(w)$, 
and no first-order existence predicate $E$ is admitted.
We've long known that actualist and possibilist semantics are expressively equivalent, 
given the existence of a one-to-one translation procedure between the two on which, e.g. 
$\forall x Fx $ and $\exists x Fx$ hold in actualist semantics 
exactly when $\forall x (Ex \rightarrow Fx)$ and $\exists x(Ex \wedge Fx)$ hold, respectively, in possibilist semantics.\footnote{
	Cf. \autocite{Lejewski1954}. 
	The full translation requires recursion, 
	and the explication of the semantics here is necessarily cursory. 
	For a full explication, see chapter 4 of \autocite{Fitting1998}.}

Now, 
both possibilist and actualist semantics alike rule out a number of metaphysical theses: 
for instance, while there is ample room for interpretation in certain respects - 
the members of $W$ could be interpreted as cases rather than worlds\footnote{Cf. \autocite{CIFOL1,CIFOL2}.} - 
neither semantics is suited to Monism or fatalism. 
Furthermore, 
on each's shared canonical assumption that the primary sense of being is that expressed by being the value of a bound variable, 
a genuine metaphysical disagreement will be expressed in these semantics themselves, 
on which the actualist accepts, 
and the possibilist rejects, 
the ontological thesis that to be is to be actual, 
while, the possibilist accepts, 
and the actualist rejects, 
the thesis that to be is to be possible. 
Despite their disagreement, the two will have no \emph{linguistic} barrier to communicating with each other. 
An employer of possibilist semantics may even reject the semi-Quinean assumption about quantification above, 
and instead opt for the claim 
that the primary sense of being is that expressed by being in the extension of the first-order existence predicate $E$ at some priviledged world $w$, 
in which case possibilist semantics itself will be used to express the content of actualism. 
Consequently, the presence of distinctly ontological theses in a semantics' canonical interpretation does not 
\emph{eo ipso} prevent understanding across distinct linguistic paradigms.  

Lastly, 
each of these frameworks is fundamentally more expressive than classical semantics taken on its own.
and the ontological viewpoint actually expressed in each is closer to that of the other than either is to that of the classical exclusivist who rejects modality altogether, 
in spite of the superficial similarity between the actualist and Quinean positions. 
Consequently, if we were to imagine the Quinean classicist attempting to grasp the meaning of either the actualist or the possibilist thesis in ontology 
from his own semantic vantage point
\emph{he could only misconstrue it}.
Instead, grasping either ontological position would require genuine learning to take place, 
even if only to reject the positions advanced. 

Similarly, 
Klima's body of work evinces a depth and breadth of expression rarely found in the discipline, 
and 
the position expressed here is ultimately closer to Klima's own than that of his late 90s Notre Dame colleagues.
Indeed, to translate it into the language of Klima's own position, 
we could say that theses such as the distinction between the various analogically related senses of being 
concerned as they are with the \emph{meaning} of being, 
are regarded by Klima as providing a semantic foundation for metaphysics distinct from metaphysics itself;\footnote{Cf. \autocite[88]{Klima1996}, \autocite[49]{Klima2011b}.}
where the position expressed here takes such theses, 
being concerned as they are with the meaning of \emph{being},
to be properly part of ontology. 
And yet, 
as with the Quinean actualist's superficial agreement with the modal actualist, 
there is a respect in which Klima's former colleagues were correct in their stated position, 
albeit not on their own terms. 

Does this make the disagreement itself merely semantic? 
Arguably, no. 
At best, we have a rare instance of a claim that belongs to distinct disciplines in different ways, 
much as with Aristotle's claim that the subject matter of the logic in the broadest sense - 
a discipline whose scope is strikingly close to that of modern semantic theory - 
is the same as that the metaphysician works on.\footnote{\autocite[1004b 22-23]{Metaph}. Cf. \autocite[q. 3]{ScotusIsagoge}.}
In its favor, 
this older understanding of metaphysics is independently recommended 
by its greater intentional unity, 
by its increased expressive power, 
by its closer proximity to the understanding one finds in Aristotle's own treatment of the subject, 
and even by the relative fruitlessness of the Quinean understanding over the past seventy years.
Conversely, 
relegating claims about the meaning of being to semantics 
rather than to metaphysics itself 
and thereby construing them as linguistic or conceptual \emph{rather than} ontological, 
arguably motivates one of the bolder misconceptions in Klima's work, 
namely that 
`Medieval realism and nominalism are just different versions of conceptualism, differing especially in how they handle the problems of 
describing and identifying mental content' \autocite[110]{Klima2011} 
- a claim which, 
tacitly assuming an ontological distinction between mental and real spheres that is somehow `crossed' or `bridged' by language, 
itself relegates claims about meaning to a purely mental sphere via its own question-begging metaphysical post-Kantianism.

\printbibliography

\end{document}
