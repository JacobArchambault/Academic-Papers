\documentclass[]{article}
\usepackage[backend=biber, style=authoryear-icomp]{biblatex}
\bibliography{../../../jacob}

%opening
\title{Introduction}
\author{Jacob Archambault}

\begin{document}

\maketitle

\begin{abstract}

\end{abstract}

\section{Logic and semantics}
The breadth of Klima's scholarship stretches thematically from debates on identity, categories, and causation in metaphysics, 
on skepticism in epistemology and theories of mental content in philosophy of mind, to others too numerous to mention; historically, from some of philosophy's best-known figures in Anselm, Aquinas, Ockham and Descartes 
to lesser-known figures including Thomas of Sutton and Henry of Ghent, 
to Frege, Geach, Kenny, and others who have produced some of the most impactful scholarship in the analytic tradition. 

\subsection{Klima's contributions in the history of semantics}
Arguably, Klima's most notable contributions come in his research on John Buridan 
- which has helped elevate Buridan from a lesser-known figure to one whose stature is closer to that of an Ockham, arguably surpassing the Franciscan in his logic - 
and in the field of semantics. 

From his earliest work in semantics, 
Klima recognized that classical logic, 
being primarily interested in developing an account of the semantics of propositions as a precondition for the development of a theory of consequence, 
affords much less attention to its account of the components of propositions themselves, 
namely names and n-ary predicate relations. 
Klima fills this lacuna by providing a theory not only of simple, 
but also of complex terms.
Klima provides some of the earliest and most ambitious applications of restricted quantification in the history and philosophy of logic, 
using it both to formalize the medieval theory of supposition 
and to provide a general account of quantitively ambiguous natural language sentences \autocite{Klima1988,Klima1990,KlimaSandu1990,Klima1991b}. Cf. \autocite{Parsons2014}. 
Expansions on the same theme - 
namely, formalizations of supposition theory specifically 
and medieval semantics more broadly 
as a means to resolve apparently intractable interpretative problems in historical scholarship 
and debates in contemporary philosophy - 
provide us with an account of the semantics of intensional verbs \autocite{Klima1991}, 
a semantic foundation for Aquinas' theory of the analogy of being in his theory of the copula \autocite{Klima1996,Klima2002}, 
and a clean resolution of the problem of existential import in the Aristotelian square of opposition\autocite{Klima2001}. 

Elsewhere, Klima's work decouples \emph{via antiqua} and \emph{via moderna} semantics 
from the realist and anti-realist metaphysics with which they are most commonly paired,
contending that neither semantics by itself strictly entails its associated metaphysics \autocite{Klima1999,Klima2011}. 
Rather, archtypical realists were required to adopt non-straightforward semantic accounts of the meanings of terms in at least some cases by their antecedent metaphysical commitments (e.g. to divine simplicity) \autocite{Klima2002b}.
Conversely, some of the best known nominalist logicians incorporated what today would be regarded as `realist' elements in their logic \autocite{Klima2005}. 
For Klima, the \emph{via antiqua} and \emph{via moderna} traditions of medieval logic aren't semantic frameworks that differ in their quantity of ontological commitments, 
but distinct frameworks differing in the kind of tools they provide for handling ontological commitments, 
which in turn differ from the model-theoretic framework dominant today. 
In particular, the \emph{via antiqua} semantic framework takes an affirmative statement to be true when what is signified by its predicate inheres in what is signified by its subject - sometimes called the \emph{inherence theory of predication}.
Within this framework, terms predicating common natures or accidental features of a subject are taken to ultimately refer to exactly the types of entities one might expect, 
but the framework also provides a rich theory according to which being is predicated in different degrees - 
which may, for instance, be represented formally by the use of different subscripted uses of the verb `is' - 
thus avoiding the full, immediate, fundamental commitment to possibilia, abstract metaphysical nominalists today might find objectionable.\footnote{See \autocite{Klima2002}.}
Conversely, the dominant semantic framework post-Ockham takes an affirmative statement to be true when its subject and predicate term refer to the same object 
- sometimes called the \emph{identity theory of predication}. 
Within this framework, terms predicating common natures or accidental features of a subject need not be taken to ultimately refer to different types of objects such as abstract genera or relations, 
but may instead be taken to refer to \emph{familiar} objects \emph{differently}. 
For example, the truth of `Socrates is a father' does not require commitment to a distinct entity that is Socrates' fatherhood.
Instead, the sentence's predicate may be taken to (non-rigidly) refer to Socrates himself, albeit connoting his being a father, 
and hence to refer to the \emph{same} object as that rigidly referred to by the proper name `Socrates', 
albeit in a different way. 
Granting some license for intensional contexts,\footnote{See \autocite{Klima2005}.} 
the verb `is' or `exists' in \emph{via moderna} semantics is equally ontologically committing in its various uses, 
but \emph{what} one is committed to by its uses need not be immediately apparent \autocite[437-430]{Klima2008a}.
Both frameworks would reject the object-language metalanguage distinction taken for granted since Tarski in their theory of truth,
and both provide ample tools to reject a naive application of the Quinean criterion for ontological commitment in terms of quantification in their use of ampliation to for tensed, modal, and intensional contexts.

\subsubsection{The independence of metaphysics from semantics}
None of this means that there is \emph{no} relationship between an author's positions in metaphysics and his semantics: 
rather, the semantic framework an author adopts conditions what options that author has in metaphysics without fully determining them. 
For example, extreme realism in metaphysics doesn't follow strictly from the \emph{via antiqua}'s inherence theory of predication, 
but it is the most natural fit for that theory 
if one accepts the view that terms signifying accidental being denote their referents rigidly 
while rejecting that framework's insistence on multiple, analogically related senses of `being' \autocite{Klima1999}. 
Conversely, the broad outlines of Ockham's account of the relation between language, thought, and reality 
serves not only as a foundation for Ockham's own metaphysical reductionism, 
but also for the realism of a Descartes, Malebranche, Putnam or a Leibniz \autocite{Klima1991}. 
In a particularly drastic example, the choice of a mistaken semantic framework may inhibit the speaker from constitutively referring to, 
and thus believing in, 
an actually existing God whose existence is only adequately assertable in an alternate framework \autocite[74]{Klima2008b}. %This page reference is an estimate. P. 22 in Klima's online copy.

What is clear, however, is that there is no is relationship of \emph{entailment} from purely semantic principles to metaphyscal truths. Klima writes: 
\begin{quote}
To be sure, this is not to say that metaphysical principles are to be derived from, or somehow justified 
in a weaker sense on the basis of, semantic principles. Metaphysical principles, being first principles 
using the most general terms, such as the transcendentals and the categories, cannot be derived from 
prior principles, and their terms cannot be defined on the basis of more general terms. What 
semantics can do, however, is that it can provide the principles of interpretation of metaphysical 
principles. On the basis of these principles of interpretation the implications of metaphysical 
principles are more clearly delineated, which then can be used in their evaluation in dialectical 
disputations concerning their acceptability in the interpretations thus clarified. Furthermore, if the 
semantic principles of interpretation are made explicit, they can also be subject to further evaluation, 
in a disputation on a different level, the sort appropriate to the comparison of different logical 
theories.  \autocite[49]{Klima2011b}
\end{quote}

Modern mathematics calls this notion of \emph{independence}
, though as the name implies the fundamental notion itself is by no means a recent one. 
Just as Cantor's continuum hypothesis is neither provable nor refutable in Zermelo-Frankel set theory, 
or - to provide a more medieval example - 
truths of revealed theology are neither provable nor refutable from the principles of natural philosophy, 
neither are metaphysical principles provable or refutable from those of semantics alone. 

\subsubsection{Charity and interpretation}
Two complications distinguish the semantic case from those mentioned. 
first, that as implied above, a metaphysics may be limited by the semantics it inhabits;
second, that while both the set-theoretic and theological case mentioned above are concerned with provability and refutability in a single system, 
the sheer multiplicity of semantic frameworks itself may provide a barrier to a broadly acceptable account of provability across those frameworks.
The first problem generalizes one nearly the opposite of that established by G\"{o}del first incompleteness theorem:
where that theorem established the expressibility of unprovable claims of number theory in any sufficiently robust system, 
the semantic problem we face here is that a claim of metaphysics may be taken to be proven or refuted merely on account of the \emph{lack} of expressibility of the particular semantic framework one is working in. 
The second provides us with the difficulty of elaborating what provability of a claim even means across a variety of semantic frameworks.

The first problem provides us a window into the answer to a more personal question that Klima's scholarship 
(and, if we're honest, that of most of the contributors to this volume) solicits: 
namely, of all the possible areas of philosophy to devote oneself to, 
or more broadly of all the things to do professionally in one's life, 
why choose to study medieval philosophy, 
and specifically medieval semantics?
Despite the depth and breadth of his work, 
the amount of space Klima devotes to advancing positions that are unambiguously his own, 
rather than to steel-manning the positions of historical or contemporary figures 
which he may or may not agree with,
is comparatively little.
\footnote{Exceptions include his acceptance of both Anselm's proof of God's existence and Aquinas' proof of the immateriality of the intellect are sound \autocite{Klima2000,Klima2009a} 
	and his advancing, 
	based on an examination of Buridan's treatment of reciprocal liar paradoxes, 
	that any adequate semantics for natural language must be semantically closed and token-based \autocite{Klima2004}.} 
%Exceptions: anselm's proof, aquinas on immortality, Buridan's Token-based semantics
Still, there are several places that touch on this question indirectly. 
In one uncommonly autobiographical passage, Klima writes: 

\begin{quote}
I remember that when I was at Notre Dame (so this happened in the second
half of the nineties), I asked several of my colleagues, and even the then
visiting David Armstrong, to provide metaphysically non-committal
clarifications of the semantics of the language they were using in
describing their metaphysical theories. In response, I was given puzzled
looks and declarations strongly reminiscent of the way medieval
nominalists characterized the attitude of their realist opponents: we don’t
care about names; we go right to the things themselves!—Well, just look 
at the history of late-scholasticism and early modern philosophy to see
what good that attitude did for them.

So, what can we do to avoid the late-scholastic scenario, going on another
cycle of endless and more and more meaningless metaphysical debates
until the arrival of another Kant declaring the whole enterprise ill-founded
and another Carnap declaring it to be meaningless, to launch another anti-
metaphysical cycle of meaningless search for meaning to be abandoned
yet again for metaphysics, etc., etc.? Why don’t we try both in tandem,
i.e., analysis and metaphysics at the same time, as the very designation
“analytic metaphysics” would seem to demand? For then we could start by
laying down our clearly defined semantic principles (instead of making
them up and twisting them around as we go) and engage each other in our
metaphysical debates according to the same principles, instead of talking
past each other, making clear that whoever is talking according to different
semantic principles is just playing a different game \autocite[86-87]{Klima2014}.
\end{quote}

Here, the difficuly that the study of semantics generally is meant to aid is one that remains palpable even now, 
namely, that much debate in the core disciplines of analytic philosophy, 
and in metaphysics in particular, 
remains as provincial as it is intractable. 
Rival participants are often unable to state their positions in a linguistic context their opponents would be able to agree to, 
leaving such debates unfruitful from the start. 
Against this, the assertion that metaphysical claims are independent from the semantics in which they are expressed takes on the character not only of a metaphilosophical thesis, 
but also of a moral demand: 
without the opportunity for common ground that semantics provides, 
not only shared understanding, 
but also proof and refutation, 
intellectual conversion and even disagreement itself become unattainable.

With this problem in mind, the study of medieval semantics, 
as a study of a framework of meaning which is itself remarkably foreign to that of our own time, 
provides an example \emph{par excellence} of the kind of interpretive charity needed to surmount our own crises of meaning and communication. 

The same motivation provides a direction for addressing the sheer mutliplicity of semantic systems present today. 
While it would be easy enough to, for instance, 
construct an account of metalogical account of validity quantifying over distinct logical systems on the model of possible world semantics and regarding as valid all and only those theorems of valid in every systems, 
this isn't the path Klima himself takes. 
Klima's sympathies in logic are resolutely anti-pluralist 
without thereby being dogmatically classical. 

grow the framework.

interpretive charity.
Case study in charity. 

tribalism vs imperialism.

Given the 

In this account, much of what 
Here Klima speaks frankly about both the provincialism and intractability of what, 

%Problem: provincialism and intra
Discussing the value of studying Buridan's philosophy in particular and that of the history of philosophy generally, he writes: 

\begin{quote}
	this study can put our own philosophical problems in an entirely different light, providing us with
	such theoretical perspectives that otherwise might entirely escape us as we are working in our
	set ways determined by the intellectual habits of our philosophical period, which in modern
	times tends to stretch to a mere couple of decades. \autocite[17]{Klima2005}
\end{quote}
Besides the frank exhortation to study the history of philosophy as a way to expand one's intellectual horizons, 
one finds an indictment of the historical shallowness of much philosophy in our own period. 
The ultimate problem here isn't 

%I personally remember feeling the bait-and-switch that seemed to occur when I moved from a historically rich undergraduate program to a heavily analytic department.
%The rank materialism of the vast majority of professors didn't threaten me: instead, I found it completely stupid..
% Provincial intractable

an expansion of one's horizons that goes beyond merely getting the right answers.

- This is philosophy, and its difference from sophistry.

An atheist who's one \emph{modus tollens} away from sincere devotion.
%Problems as provincial as they are intractable
%Answer to both problems: expansion of horizons. expressibility is important here, because it provides inductive evidence for correctness.
%Pluralism vs imperialism
%What is the criterion for provability of a metaphysical claim: provability across all frameworks?
\subsubsection{Why semantics?}
Semantics: 
1) a theory of meaning broadly construed
2) Tarskian/Montaguean mathy stuff
	a semantics is almost never actually this, given that most semantics have a canonical interpretation and a domain to which they are expected to apply (e.g. Model theory handles solids better than liquids or gases).
3) e.g. a dictionary
4) e.g. proof-theoretic semantics
5) a philosophy of language
6) a theory of language, thought, and reality
%Two contradicting claims: 1) The modern turn in metaphysics was conditioned by a prior one in semantics; 2) neither via antiqua nor via moderna semantics strictly entail the metaphysical positions they are usually associated with.

%Two phenomena: 1) the breakdown of communication within analytic philosophy 2) its lack of breadth

%Buridan's logic and the ontology of modes - neither via antiqua nor via moderna semantics entailed realism or reductionist metaphysics, respectively.

\subsection{Buridan}

%\section{Philosophy of mind and epistemology}
%\subsection{\textit{Via antiqua} and \textit{via moderna} cognizers}
%\section{Metaphysics}
%\subsection{Hylomorphism, personal identity and immortality}
%\subsection{Causation}
%\section{Overview of the articles}
\subsection{The semantics of `metaphysics'}
\subsubsection{Independence revisited}
\section{Notes}
%The problem: the task of positing different senses of being to avoid full ontological commitment isn't a way of avoiding metaphysics: it \emph{is} metaphysics.
Modern metaphysics confuses the question of what being is with the question of what beings there are. (this isn't a new insight: once one gets past the mysticism of some of his interpreters, this is the fundamental point behind the talk of ontological difference in the philosophy of Martin Heidegger)
\subsection{Positions taken}
In \autocite{Klima2008a}, Klima recognizes that representing \emph{via antiqua} semantics would require substantial modifications to modern quantification theory, while representing \emph{via moderna} semantics requires fewer modifications
1) Via antiqua semantics requires a different account of predication, and multiple copula to be introduced to represent the different senses of being
2) Via moderna semantics requires the introduction of restricted quantifiers.
3) both require the rejection of the object-metalanguage distinction.

Klima agrees with Buridan that the notion of truth is not strictly needed for a semantics concerned with formal validity, but it is needed to explain the semantics of sentences that themselves predicate that notion.
Analogy: nobody complains that we don't have a formal definition of the term `red' in our logic, even though a basic grasp of the semantics of that term is needed for using the term in sentences about red things.

Two uses for semantics of truth: 1) as part of a theory of validity, 2) for its own sake.
In `Logic without truth', Klima rejects Buridan's solution to the liar paradox.

%The Rises and Falls of Analysis and Metaphysics in Metaphysical Themes, Medieval and Modern - highly autobiographical. Helps to explain his aversion to the idea of metaphysics bleeding into semantics



%Note. Klima can't get away from the notion of semantics as functions for mapping language to reality. This is part of why he can't accept the idea of metaphysics impinging on logic itself. But this is exactly what the theory of the different main types of supposition, grounded in a theory of analogy, should entail.


(1) Natural languages are semantically closed (2) Natural language inference has to be token-based.
Both Klima/Buridan and Tarski come to the conclusion that defining consequence in terms of truth and falsity doesn't work from similar considerations: Tarski's consideration is related to superenumable domains and the possibility that a language may simply an appropriate selection of denoting terms; Klima/Buridan's considerations come from the possibility that a claim may be not exist to even be true or false, or it may be self-falsifying while nevertheless describing a possible state of affairs \autocite[96]{Klima2004}.

The primary impetus behind Klima's work is one of charity.
Examples: 
Positive:
1. His analysis of parasitic reference in his work on Anselm
2. His attempts to translate between via antiqua and via moderna semantics
3. The entirety of his body of work on John Buridan
Negative:
1. The infrequency with which Klima actually reveals his own philosophical positions in his work (exceptions: 
Aquinas' hylomorphism and proof of immortality, 
Anselm's proof, 
Per Buridan, the semantic closure and token-based character of natural language inference)
2. His adopting semantics that build on classical logic while rejecting non-classical semantics.


%Metaphysical questions are undecidable by semantics alone. In the same sense in which the body of truths of revealed theology are undecidable by natural philosophy, or the continuum hypothesis is undecidable by set theory.
%The rises and falls of analysis and metaphysics
%Being, unity, and identity in the Fregean and Aristotelian traditions

%metaphysics \textit{does} determine semantics. Consequently, the attempt to infer metaphysics from semantics deductively, as with all similar cases of causality, would logically have to be a case of affirming the consequent. 
%demonstratio quia

\section{quotes}
%The efficient causality model fixes the relation of natural signification on the basis of natural laws systematically connecting causes with their effects. But all that these natural laws guarantee is the systematic correspondence between causes and their effects, supposing the normal course of nature. The effects, however, may be essentially different from their causes, and may, therefore, be produced also by other essentially different causes too, which means that it is clearly possible that an absolute concept be caused in our intellects by a cause which is totally alien from what this concept is supposed to represent. - Ontological Alternatives.
%IMPORTANT!, the immediately above quote is self-defeating for Klima, since his remarks on the relation between semantics and metaphysics are themselves conceived on the efficient, rather than formal, causality model mentioned above.

`Buridan’s nominalism is obtainable by the adverbialization of Peter
of Spain’s semantics.'
`Nominalism is obtainable by the adverbialization of realist
semantics.'
`Medieval realism and nominalism are just different versions of conceptualism, differing especially in how they handle the problems of
describing and identifying mental content.'\autocite[110]{Klima2011}

In this paper I will attempt to dig further to the roots of
their disagreements, trying to establish those primary logical-semantic differences that may have
motivated their conflicting intuitions concerning these metaphysical principles. - Thomas of Sutton v. Henry of Ghent

%Therefore, it should also be clear that the laws of logic in this framework
are supposed to be fundamentally different from the laws of psychology.
For while the former are the laws of the logical relations among objective
concepts, the latter are the laws of the causal relations among formal
concepts. - The problem of universals and the subject matter of logic, Klima 2014, p. 173.

These different theories can be arranged on a `theoretical scale', ranging from extreme realism to extreme nominalism, meaning maximal semantic uniformity along with maximal ontological diversity on the realist end [...], and maximal ontological uniformity with maximal semantic diversity on the nominalist end.- The problem of universals and the subject matter of logic, Klima 2014, p. 176.

Well, conceptual diversity is obviously
a great hindrance to understanding: if we don’t have the same concepts, we can-
not have the same thoughts, which means we are doomed to talking past each
other all the time\autocite[36]{Klima2021}

So, what should be our guiding light, in this rational discourse? In one word:
rationality, which is love or goodwill on its active side, on the part of the will, and
understanding on its receptive, theoretical side, on the part of the intellect. -Klima 2021, p. 41
\autocite{Parsons2014,Read2015b}

`The primary purpose of a logical semantic theory is to define logical consequence in terms of the truth values of propositions in different interpretations' \autocite[79]{Klima1991b}. 

\printbibliography

\end{document}
