\documentclass[]{article}
\usepackage[backend=biber, style=authoryear-icomp]{biblatex}
\bibliography{../../../jacob}

%opening
\title{Introduction}
\author{Jacob Archambault}

\begin{document}

\maketitle

\begin{abstract}

\end{abstract}

\section{Klima's contributions in the history of semantics}
The breadth of Klima's scholarship stretches 
historically, from some of philosophy's best-known figures in Anselm, Aquinas, Ockham and Descartes 
to lesser-known figures including Thomas of Sutton and Henry of Ghent, 
to Frege, Geach, Kenny, and others who have produced some of the most impactful scholarship in the analytic tradition; 
and 
thematically from debates 
on identity, categories, and causation in metaphysics, 
on skepticism in epistemology and theories of mental content in philosophy of mind, 
to others too numerous to mention. 
From his earliest work in semantics, 
Klima recognized that classical logic, 
being primarily interested in developing an account of the semantics of propositions as a precondition for the development of a theory of consequence, 
affords much less attention to its account of the components of propositions themselves, 
namely names and n-ary predicate relations. 
Klima fills this lacuna by providing a theory not only of simple, 
but also of complex terms.
Klima provides some of the earliest and most ambitious applications of restricted quantification in the history and philosophy of logic, 
using it both to formalize the medieval theory of supposition 
and to provide a general account of quantitively ambiguous natural language sentences \autocite{Klima1988,Klima1990,KlimaSandu1990,Klima1991b}. Cf. \autocite{Parsons2014}. 
Expansions on the same theme - 
namely, formalizations of supposition theory specifically 
and medieval semantics more broadly 
as a means to resolve apparently intractable interpretative problems in historical scholarship 
and debates in contemporary philosophy - 
provide us with an account of the semantics of intensional verbs \autocite{Klima1991}, 
a semantic foundation for Aquinas' theory of the analogy of being in his theory of the copula \autocite{Klima1996,Klima2002}, 
and a clean resolution of the problem of existential import in the Aristotelian square of opposition\autocite{Klima2001}. 

Elsewhere, Klima's work decouples \emph{via antiqua} and \emph{via moderna} semantics 
from the realist and anti-realist metaphysics with which they are most commonly paired,
contending that neither semantics by itself strictly entails its associated metaphysics \autocite{Klima1999,Klima2011}. 
Rather, archtypical realists were required to adopt non-straightforward semantic accounts of the meanings of terms in at least some cases by their antecedent metaphysical commitments (e.g. to divine simplicity) \autocite{Klima2002b}.
Conversely, some of the best known nominalist logicians incorporated what today would be regarded as `realist' elements in their logic \autocite{Klima2005}. 
For Klima, the \emph{via antiqua} and \emph{via moderna} traditions of medieval logic aren't semantic frameworks that differ in their quantity of ontological commitments, 
but distinct frameworks differing in the kind of tools they provide for handling ontological commitments, 
which in turn differ from the model-theoretic framework dominant today. 
In particular, the \emph{via antiqua} semantic framework takes an affirmative statement to be true when what is signified by its predicate inheres in what is signified by its subject - sometimes called the \emph{inherence theory of predication}.
Within this framework, terms predicating common natures or accidental features of a subject are taken to ultimately refer to exactly the types of entities one might expect, 
but the framework also provides a rich theory according to which being is predicated in different degrees - 
which may, for instance, be represented formally by the use of different subscripted uses of the verb `is' - 
thus avoiding the full, immediate, fundamental commitment to possibilia, abstract metaphysical nominalists today might find objectionable.\footnote{See \autocite{Klima2002}.}
Conversely, the dominant semantic framework post-Ockham takes an affirmative statement to be true when its subject and predicate term refer to the same object 
- sometimes called the \emph{identity theory of predication}. 
Within this framework, terms predicating common natures or accidental features of a subject need not be taken to ultimately refer to different types of objects such as abstract genera or relations, 
but may instead be taken to refer to \emph{familiar} objects \emph{differently}. 
For example, the truth of `Socrates is a father' does not require commitment to a distinct entity that is Socrates' fatherhood.
Instead, the sentence's predicate may be taken to (non-rigidly) refer to Socrates himself, albeit connoting his being a father, 
and hence to refer to the \emph{same} object as that rigidly referred to by the proper name `Socrates', 
albeit in a different way. 
Granting some license for intensional contexts,\footnote{See \autocite{Klima2005}.} 
the verb `is' or `exists' in \emph{via moderna} semantics is equally ontologically committing in its various uses, 
but \emph{what} one is committed to by its uses need not be immediately apparent \autocite[437-430]{Klima2008a}.
Both frameworks would reject the object-language metalanguage distinction taken for granted since Tarski in their theory of truth,
and both provide ample tools to reject a naive application of the Quinean criterion for ontological commitment in terms of quantification in their use of ampliation to for tensed, modal, and intensional contexts.

\subsection{metaphysics, semantics, and independence}
None of this means that there is \emph{no} relationship between an author's positions in metaphysics and his semantics: 
rather, the semantic framework an author adopts conditions what options that author has in metaphysics without fully determining them. 
For example, extreme realism in metaphysics doesn't follow strictly from the \emph{via antiqua}'s inherence theory of predication, 
but it is the most natural fit for that theory 
if one accepts the view that terms signifying accidental being denote their referents rigidly 
while rejecting that framework's insistence on multiple, analogically related senses of `being' \autocite{Klima1999}. 
Conversely, the broad outlines of Ockham's account of the relation between language, thought, and reality 
serves not only as a foundation for Ockham's own metaphysical reductionism, 
but also for the realism of a Descartes, Malebranche, Putnam or a Leibniz \autocite{Klima1991}. 
In a particularly drastic example, the choice of a mistaken semantic framework may inhibit the speaker from constitutively referring to, 
and thus believing in, 
an actually existing God whose existence is only adequately assertable in an alternate framework \autocite[74]{Klima2008b}. %This page reference is an estimate. P. 22 in Klima's online copy.

What is clear, however, is that there is no is relationship of \emph{entailment} from purely semantic principles to metaphyscal truths. Klima writes: 
\begin{quote}
To be sure, this is not to say that metaphysical principles are to be derived from, or somehow justified 
in a weaker sense on the basis of, semantic principles. Metaphysical principles, being first principles 
using the most general terms, such as the transcendentals and the categories, cannot be derived from 
prior principles, and their terms cannot be defined on the basis of more general terms. What 
semantics can do, however, is that it can provide the principles of interpretation of metaphysical 
principles. On the basis of these principles of interpretation the implications of metaphysical 
principles are more clearly delineated, which then can be used in their evaluation in dialectical 
disputations concerning their acceptability in the interpretations thus clarified. Furthermore, if the 
semantic principles of interpretation are made explicit, they can also be subject to further evaluation, 
in a disputation on a different level, the sort appropriate to the comparison of different logical 
theories.  \autocite[49]{Klima2011b}
\end{quote}

Modern mathematics calls this notion of \emph{independence}, 
though as the name implies the fundamental notion itself is by no means a recent one. 
Just as Cantor's continuum hypothesis is neither provable nor refutable in Zermelo-Frankel set theory, 
or - to provide a more medieval example - 
truths of revealed theology are neither provable nor refutable from the principles of natural philosophy, 
neither are metaphysical principles provable or refutable from those of semantics alone on Klima's account. 

\subsection{Charity and interpretation}
Two complications distinguish the semantic case from those mentioned. 
The first is that while both the set-theoretic and theological case mentioned above are concerned with provability and refutability in a single system, 
the sheer multiplicity of semantic frameworks itself may provide a barrier to a broadly acceptable account of provability across those frameworks. 
The second generalizes a problem nearly the opposite of that established by G\"{o}del in his first incompleteness theorem \autocite{Godel1931}:
where that theorem established the expressibility of unprovable claims of number theory in any sufficiently robust system, 
the semantic problem we face here is that a claim of metaphysics may be taken to be established or refuted merely on account of the \emph{lack} of expressibility of the particular semantic framework one is working in. 

Klima's response to these problems is anti-pluralist 
without thereby being dogmatically classical. 
While it would be easy enough to, for instance, 
construct an account of metalogical account of validity quantifying over distinct logical systems on the model of possible world semantics and regarding as valid all and only those theorems valid in every system, 
Klima instead recognizes the known limitations of classical semantics 
while also taking the provable equivalence of systems containing distinct logical primitives
as \emph{prima facie} evidence for the possibility of fundamental diversity at the \emph{conceptual} level 
that nevertheless doesn't entail a despairing or indifferent anti-realism at the \emph{metaphysical} level \autocite{Klima2012}.
Instead, Klima's response, on both a technical and philosophical level, 
is to \emph{extend the framework} -
Meeting the tradition where it is, he extends classical semantics to allow for treatment of 
donkey sentences \autocite{Klima1988,Klima2010}, 
non-existent entities \autocite{Klima2001}
and quantificational phenomena \autocite{KlimaSandu1990},
while more broadly appealing (in a rare quote of a `continental' philosopher that shows up in multiple places throughout his \emph{oevre}) 
to the possibility of a `fusion of horizons' mentioned by Gadamer 
as a solution to the impasse of communication across distinct semantic frameworks, cultures, or philosophical traditions and the attitude of metaphysical anti-realism it encourages \autocite{Klima2000,Klima2009a}.
%Quote Gadamer

Klima's solution here provides a window into the answer to a more personal question that Klima's scholarship 
(and, if we're honest, that of many of us as contributors to this volume) solicits: 
namely, of all the possible areas of philosophy to devote oneself to, 
or more broadly of all the things to do professionally in one's life, 
why choose to study medieval philosophy, 
and specifically medieval semantics?
Despite the depth and breadth of his work, 
the amount of space Klima devotes to advancing positions that are unambiguously his own, 
rather than to steel-manning the positions of historical or contemporary figures 
which he may or may not agree with,
is comparatively little.
\footnote{Exceptions include his acceptance of both Anselm's proof of God's existence and Aquinas' proof of the immateriality of the intellect are sound \autocite{Klima2000,Klima2009a} 
	and his advancing, 
	based on an examination of Buridan's treatment of reciprocal liar paradoxes, 
	that any adequate semantics for natural language must be semantically closed and token-based \autocite{Klima2004}.} 
Still, there are several places that touch on this question indirectly. 
In one uncommonly autobiographical passage, Klima writes: 

\begin{quote}
I remember that when I was at Notre Dame (so this happened in the second
half of the nineties), I asked several of my colleagues, and even the then
visiting David Armstrong, to provide metaphysically non-committal
clarifications of the semantics of the language they were using in
describing their metaphysical theories. In response, I was given puzzled
looks and declarations strongly reminiscent of the way medieval
nominalists characterized the attitude of their realist opponents: we don’t
care about names; we go right to the things themselves!—Well, just look 
at the history of late-scholasticism and early modern philosophy to see
what good that attitude did for them.

So, what can we do to avoid the late-scholastic scenario, going on another
cycle of endless and more and more meaningless metaphysical debates
until the arrival of another Kant declaring the whole enterprise ill-founded
and another Carnap declaring it to be meaningless, to launch another anti-
metaphysical cycle of meaningless search for meaning to be abandoned
yet again for metaphysics, etc., etc.? Why don’t we try both in tandem,
i.e., analysis and metaphysics at the same time, as the very designation
“analytic metaphysics” would seem to demand? For then we could start by
laying down our clearly defined semantic principles (instead of making
them up and twisting them around as we go) and engage each other in our
metaphysical debates according to the same principles, instead of talking
past each other, making clear that whoever is talking according to different
semantic principles is just playing a different game \autocite[86-87]{Klima2014}.
\end{quote}

Here, the difficuly that the study of semantics generally is meant to aid is one that remains palpable even now, 
namely, that much debate in the core disciplines of analytic philosophy, 
and in metaphysics in particular, 
remains as provincial as it is intractable. 
Rival participants are often unable to state their positions in a linguistic context their opponents would be able to agree to, 
leaving such debates unfruitful from the start. 
Against this, the assertion that metaphysical claims are independent from the semantics in which they are expressed takes on the character not only of a metaphilosophical thesis, 
but also of a moral demand: 
without the opportunity for common ground that semantics provides, 
not only shared understanding, 
but also proof and refutation, 
intellectual conversion and even disagreement itself become unattainable.

With this problem in mind, the study of medieval semantics, 
as a study of a framework of meaning which is itself remarkably foreign to that of our own time, 
provides an example \emph{par excellence} of the kind of interpretive charity needed to surmount our own crises of meaning and communication. 
\section{John Buridan}
Nowhere has this effort been more sustained than in Klima's scholarship on John Buridan', 
which has helped elevate the 14th century Arts Master from a lesser-known figure to one whose stature is closer to that of an Ockham, 
arguably surpassing the Franciscan in his logic.

\section{Semantic expressiveness and the meaning of `metaphysics'}
In this way, 
Klima's seminal contributions on Buridan and in medieval philosophy more broadly provide an excellent example of how one can solve apparently intractable philosophical and communicational problems 
but simply by \emph{expanding the framework}.
% - whether that framework be classical semantics expanded to include an existence predicate and restricted quantifiers
% or the mindset of Anglophone philosophy of the past seventy or so years to include an account of analogy.
Still, one wonders whether one can take the fundamental motivation further than even Klima himself does. 

We know that some semantic frameworks are fundamentally more expressive than others. 
This can occur in a trivial sense when a language introduces new derivative syntactic elements, 
as occurs e.g. when we introduce a symbol $\rightarrow$ into that fragment of classical propositional logic including only $\wedge $ and $\neg$ as connectives, 
as a shorthand for $\neg(A \wedge \neg B)$. 
In other cases, e.g. by augmenting classical first-order logic with an identity predicate, 
we can add a new logical constant to a language that thereby allows for the expression of previously inexpressible theorems and entailments. 
In still other cases, 
as occurs, for instance when extending the relatively minimal normal modal logic $K$ to a more robust one like $S4$ or $S5$
we can provide a more robust semantic interpretation of a term or symbol by further restricting the class its previously permissible interpretive models, 
and thereby establish theorems and entailments for which countermodels previously existed.
In still others, we can provide a translation of the claims of one semantic system $S$ to those of another $S'$ 
such that, 
given a translation function $f$ from $S$ into $S'$, 
a claim $c$ is provable in $S$ exactly when  $f(c)$ is in $S'$.
Perhaps the best-known case of this last mentioned is G\"{o}del's proof that the claims of classical arithmetic can all be expressed in intuitionistic arithmetic, 
with the base case being that a claim $A$ holds classically if and only if $\neg\neg A$ holds intuitionistically. 
This particular example also shows that a language may be fundamentally more expressive than another even when we have the same set of lexical elements defined across both systems, 
and even when the weaker system appears at first glance to be the stronger one. 
%As G\"{o}del's incompleteness theorems show, it is possible for a framework to be \emph{too} expressive, 
%but generally speaking, both in technical and non-technical contexts, 
%a more expressive framework is preferable to a weaker one.

%Semantics: 
%1. Presupposes non-empty domain
%2. that its domain doesn't include impossible objects 
%3. Tarskian semantics includes a fixed domain. 
%4. Liar paradoxes assume that our domain may contain both first-order elements (e.g. dogs) and higher-order elements (e.g. sentences about dogs).
%	cf. Axiom of reducibility
%- In short, there are tacit assumptions about the extension of `being' that are built into our various modal logics. 
%2. Counterexample again: no one assumes that natural language is actually functional, given the possibility of equivocal and analogical meanings - or even of material or simple supposition. 
% An axiomatization of Euclidean geometry does not thereby deny the existence of non-euclidean spaces, nor (more broadly) of non-geometrical objects. 
%In short, there are questions that every semantics, considered in itself, must implicitly or explicitly decide. Among these is the question of what there is. 
Expanding this point about technical frameworks to natural ones,
we can say that 
philosophy of the past seventy or so years has primarily interpreted ontology 
as a discipline whose theorems are, 
in their basic syntax, 
positive or negated existential, non-copulative uses of the verb `is' or `exists' joined to a common or proper noun, 
e.g. `baseballs exist', `bats don't exist', etc.
%Now, as a relatively trivial example, the claim that \emph{something} exists, i.e. $(\exists x)x = x$ is a theorem in nearly all semantic systems of predicate logic with identity, 
%given the exclusion of non-empty domains from the class of models.
%Consequently, 
Given a suitably robust concept of induction, it then becomes possible to make universal statements about what is, 
usually expressed formally via a universal quantifier ranging over a disjunction, 
with each disjunct
attributing a monadic predicate to the value of a variable bound by the universal
quantifier, 
such that the empty set is the value of no predicate, 
the intersection of the range of any two predicates is the empty set, 
and the union of all of them
yields the domain of quantification - 
e.g. `Everything is God or a creature; a mind or a body; a divine idea; a monad; a person or a mere thing; a substance or quantity or quality, etc., to provide some of philosophy's better-known examples. 
This understanding of ontology, 
however, 
is arguably 
%not the sense in which that part of metaphysics was understood by the ancient and earlier medieval interpreters of Aristotle, 
%but 
an attentuation of a broader notion,
one concerned not primarily with the question of what things there are, but of what \emph{being is}. 
One might answer this question in a limited way by providing a universally quantified disjunction of the aforementioned kind. 
But on this broader understanding, 
the principal syntactic form that a statement in ontology will take will be one predicating some monadic or disjunctive predicate of being itself as a subject. 
And so, for instance, the claim that everything is a substance or a quantity, etc., 
if true, 
will be derivative on that that \emph{to be} is to be a substance or a quantity, etc.
And yet the converse entailment schema intuitively fails: 
the fact that everything is or isn't my eldest daughter, for instance, 
does not entail that to be is to be or not be my eldest daughter. 
Furthermore, the reason this entailment holds is intuitively clear, 
namely, 
that it doesn't belong to the \emph{meaning} of being to contain any information about my daughter one way or another. 
%Essence : modality :: being : quantification. Cf. Kit Fine, Essence and Modality 
Consequently, the classical expanse of ontology embraces matters not typically covered in its more recent Anglophone counterpart: 
the convertibility of the transcendental predicates with being itself;  
the different senses of being pertaining 
to past present, and future being;
to real and rational;
to substance and accident, 
to potential and the actual, 
along with the relations of priority, posteriority and relative perfection displayed therein
that prevent these distinctions from being understood as differences in kind. 
%Life is beautiful, it is raining, what's going on? 
%attitudes of love, happiness, boredom, angst.  
%Just as its possible to derive statements about what is necessity from those concerning what is essential, so too we can derive universal quantifications from statements about being without those statements thereby being about being. 
%Analytic metaphysics is an effect of metaphysics, but is not itself metaphysics.

But on this understanding, the task of positing different senses of being to avoid full ontological commitment - 
- or, for that matter, refusing to make such distinctions - 
isn't a way of avoiding metaphysics: it \emph{is} metaphysics.

This doesn't entail that a semantics is ontologically committing considered purely as an interpretation mapping lexical elements to a suitable mathematical model: 
nothing, for instance, so much as entails $\wedge$ be understood to mean `and'.\footnote{
	e.g. that fragment of the classical propositional calculus containing $\wedge$ as its only a logical constant may be understood as providing a logic for entailment with that symbol intuitively given the meaning of English `and', 
	or as one for rejection where it is intuitively given the meaning of `or' . Cf. \autocite[222]{Kripke2015}.}
Nor does it entail any commitment at the level of our judgment concerning a particular system: 
to consider some of its less plausible claims, 
one may, for instance, adopt a standard classical semantics for any number of reasons or occasions without thereby committing oneself to the claims 
that relations are sets of tuples, 
that natural language reference is functional in the mathematical sense, 
that there has to be at least one thing, 
or that modality doesn't exist.
It does, however, mean
that a semantics containing structures meant to model being as a whole, 
%such as quantifiers or a domain, 
considered in itself and on its intended interpretation, 
must decide ontological questions on the matters it directly pertains to or in which it is indirectly implicated. 

Let us take, for example, the case of the convertibility of actualist and possibilist quantification. 
On possibilist semantics, a model $(W, R, D)$ a frame $(W, R)$ 
consisting of a set of worlds $W$ and a binary relation $R$ on worlds, 
and a set $D$ containing all elements existing at any element in $W$. 
Quantifiers range over $D$ in its entirety, and to represent actual existence at a given world $w$, possibilist semantics employs a monadic predicate $E$ 
whose interpretation intuitively maps to the set of actual existents at $w$. 
Conversely, on actualist semantics, 
we take a model $(W, R, D)$ to be as before, 
but with $D$ being not a set, but a function from each element $w$ in $W$ to its actual existents. 
The interpretation of a quantifier at $w$ is restricted to $D(w)$, 
and no first-order existence predicate $E$ is admitted as a logical constant.
We've known that actualist and possibilist semantics are expressively equivalent, 
given the existence of a one-to-one translation procedure between the two on which, e.g. 
$\forall x Fx $ and $\exists x Fx$ hold in actualist semantics 
exactly when $\forall x (Ex \rightarrow Fx)$ and $\exists x(Ex \wedge Fx)$ hold, respectively, in possibilist semantics.\footnote{
	Cf. \autocite{Lejewski1954}. 
	The full translation requires recursion, 
	and the explication of the semantics here is necessarily cursory. 
	For a full explication, see chapter 4 of \autocite{Fitting1998}.}

Now, on the common assumption that the primary sense of being is that expressed by being the value of a bound variable, 
a genuine metaphysical disagreement will be expressed in these semantics themselves, 
on which the actualist accepts, 
and the possibilist rejects, 
the ontological thesis that to be is to be actual, 
while, the possibilist accepts, 
and the actualist rejects, 
the thesis that to be is to be possible. 
Despite this, the two will have no \emph{linguistic} barrier to communicating with each other, 
given the existence of the above alluded to translation procedure. 
Consequently, the presence of distinctly ontological theses in a semantics' canonical interpretation does not 
\emph{eo ipso} prevent understanding across distinct linguistic paradigms.  
An employer of possibilist semantics may even reject the semi-Quinean assumption about quantification above, 
and instead opt for the claim 
that the primary sense of being is that expressed by being in the extension of the first-order existence predicate $E$ at some priviledged world $w$, 
in which case possibilist semantics itself will be used to express the content of actualism. 
But while there is ample room for interpretation in certain respects - 
the members of $W$ could be interpreted as cases rather than worlds, for instance\footnote{Cf. \autocite{CIFOL1,CIFOL2}} - 
both the possibilist and the actualist semantics rule out a number of metaphysical theses 
(e.g. neither semantics is suited to Monism or fatalism), 
and the ontological viewpoint actually expressed in each of them is closer to that of the other than either is to that of the classical exclusivist who rejects modality altogether, 
in spite of the superficial similarity between the actualist and Quinean positions on the role of quantification in expressing ontological commitment. 

Lastly, both of these frameworks are fundamentally more expressive than classical semantics taken on its own.
Consequently, if we were to imagine the Quinean classicist attempting to grasp the meaning of either the actualist or the possibilist theses in ontology 
from within his own semantic vantage point
\emph{he could only misconstrue it}, 
despite his superficial agreement with the actualist. 
Instead, grasping either ontological position would require genuine learning to take place, 
even if only to reject the positions advanced. 

Thus, 
according to the position taken here, 
as with the Quinean actualist's superficial agreement with the modal actualist, 
there is a respect in which Klima's Notre Dame colleagues in the late 90s were correct in their stated position, 
albeit not on their own terms. 
And yet, as in with modal actualist's disagreement with the possibilist, 
the position expressed here is ultimately closer to Klima's own than the one it superficially ressembles.
Indeed, to translate it into the language of Klima's own position, 
we could say that theses such as the distinction between the various analogically related senses of being 
concerned as they are with the \emph{meaning} of being, 
are regarded by Klima as providing a semantic foundation for metaphysics distinct from metaphysics itself;\footnote{Cf. \autocite[88]{Klima1996}, \autocite[49]{Klima2011b}.}
where the position expressed here takes such theses, 
being concerned as they are with the meaning of \emph{being},
to be properly part of ontology itself. 

Does this make the disagreement itself merely semantic? 
Arguably, no. 
At best, we have a rare instance of a claim that belongs in distinct disciplines albeit in different ways, 
much as in Aristotle's claim that the subject matter of the logic in the broadest sense - 
a discipline whose scope is strikingly close to that of modern semantic theory in many respects - 
is the same that the metaphysician works on.\footnote{\autocite[1004b 22-23]{Metaph}. Cf. \autocite[q. 3]{ScotusIsagoge}.}
This older understanding of metaphysics is independently recommended 
by its greater intentional unity, 
by its increased expressive power, 
by its closer proximity to the understanding one finds in Aristotle's own treatment of the subject, 
and even by the relative fruitlessness of the Quinean understanding over the past seventy years.
%`Thought' is not a noun, but a passive participle
%Semantics in the modern sense is quite close to what is meant by logic in the widest Aristotelian sense. 
On the other hand, 
relegating claims about the meaning of being to semantics 
rather than to metaphysics itself 
and thereby construing them as linguistic or conceptual \emph{rather than} ontological, 
arguably motivates one of the bolder misconceptions in Klima's work, 
namely that 
`Medieval realism and nominalism are just different versions of conceptualism, differing especially in how they handle the problems of
describing and identifying mental content.'\autocite[110]{Klima2011} 
- a claim which, 
tacitly assuming an ontological distinction between mental and real spheres that is somehow `crossed' or `bridged' by language, 
itself relegates claims about meaning to a purely mental sphere via its own question-begging metaphysical post-Kantianism.
%and instantiates a post-Ockhamist error that Klima himself identifies, 
%viz. that of construing the relationship between thought and thing on the model of efficient rather than formal causality, 
%that makes both skepticism and realism possible. 

Klima's work in medieval semantics, 
both directly and by way of its example, 
provides a way out of a genuine problem of communication especially present in analytic metaphysics,
to which the predominant responses have been 
a reactionary linguistic imperialism that fails to address the genuine grievances against the previously dominant classical logic paradigm 
and a live-and-let-live tribalism that only exacerbates the problem. 
In his own work, Klima's \emph{tertia via} has been that of expanding the existing semantic framework, 
both with technical apparatus where expressibility was previously lacking 
and with historical understanding that situates fundamental concepts 
and opens up unique avenues for responding to difficulties in contemporary metaphysics. 
At the same time, 
the concept of ontology that Klima takes up in his debates with various interlocutors is subject to this same expansion: 
the more recent concept of metaphysics is 
one that conflates the question of what being is with that of what things there are;
one derivative upon a more intentionally focused and semantically expressive conception concerned with what it is to be, 
one that even contemporary philosophy itself increasingly recognizes as exhausted,\footnote{Cf. \autocite{Schaffer2009}.}
and indeed, one that `tends to stretch to a mere couple of decades' \autocite[17]{Klima2005}; 

%and that consequently, 
%stuff about Aristotle on being and the same, 
%the weakness and derived character of the analytic debate
%better explains the intentional unity of metaphysics itself
%better conforms to the understanding of the subject of a science in the medieval context
%the graph is more important than the points

%Function argument distinction is itself likely, by way of a long historical chain, indebted to the Via Antiqua analysis that Klima mentions
%Distinction between names  and predicates only makes sense as a corollary to an ontological distinction between things and properties.
%Frege gets the linguistic distinction wrong by construing it globally at the level of the language, where Buridan gets it right by construing it at the sentential level. The correct account is also in Aquinas' De Ente.

troublemakers cosplaying as scientists by making the meanings of common terms unduly obscure

1. Frameworks may differ in their expressive power.
2. This is true not only of artificial, but also of constructed languages.
3. 

%Likewise, the stance expressed in this critique is much closer to the stance of Klima than that of his late 90's Notre Dame colleagues, 
%while also expressing the claim that there is a respect in which they were right in spite of themselves. 
%Meaning of \emph{being} [ontology] vs \emph{meaning} of being [semantics] - Aristotle's remark that logic and metaphysics concern the same. 

%There is a respect in which Klima's Notre Dame colleagues were right in spite of themselves
%diminishing effect of $\diamond$ on quantifiers in its scope.
%And yet both the possibilist and actualist will find themselves in agreement against the classical logician.
%The denial of a metaphysical thesis remains a metaphysical thesis. 

In any case, it expresses a sense in which there are at least two senses of being. 
Likewise with the acceptance or rejection of the axiom of reducibility, 
the implicit exclusion of contradictory objects from the domain, 

%Such a semantics should already be unacceptable to the Megarian, the Monist, the Spinozist, etc. 
- The grounding of the unique relation of semantics to metaphysics in the convertibility of being and truth.

%How does this allay Klima's worries? It shows that the problem was never about metaphysical presuppositions, and always about tribalism.

Indeed, this helps form the foundation for the possibility of recognizing that an ontology can be inadequate, 
%as, for instance, Klima himself recognizes that restricted quantification is a better model of x-that-is-a-F than of `F'
%(I think the reference is to ) 

%Not everything is decided: e.g. domain can be interpreted as tokens rather than types

%Metaphysics is not prejudiced at the level of judgment \emph{about} the model, but it is at the level of the model itself. 

%the type of x in the universal quantifier is confused.
%Translation from classical into free logic. 
%Even in propositional calculus, the notion of being is retained confusedly. 
%higher order entities aren't quantified over in first-order logic
%Everything is in one of the ten categories. The Russel set is something, therefore, etc.
%Type theory. Problems of naive set theory, superenumerable domains, the axiom of reducibility, etc. 
%1. Conflicts with the Aristotelian notion of science
%2. The failure to mark off a domain.
%2. Its fruitlessness. (i.e. the Quinean program in metaphysics has been exhausted. )
%3. The Quinean notion may be subsumed by the Aristotelian one.
%The fact that claims of cardinality are derivable in Tarski's original concept of semantics is a problem, and prima facie evidence that it's mistaken.
%Everything is or isn't my eldest daughter arguably isn't a thesis of metaphysics. 
%Primacy of first syllogistic form over the third.
%The semantics, in itself apart from our judgments about it, will not be free from ontological assumptions.
%Why this isn't an ignoratio elenchi
To begin, it's worth taking a moment to recognize just how odd this is. 
It's not that other disciplines lack statements of this sort - even purely mathematical disciplines have famous theorems proving the existence or non-existence of things like superenumerable domains or greatest prime numbers - 
but it is that even all these years after Aristotle, the principal statements even in newer bodies of knowledge, including semantics itself, tend to take the form of affirmations or denials of intrinsic or extrinsic properties of that science's principal subject and/or its specific and integral parts (Cf. Post. An. 71a 11-16).

rather than, 
as is the case with every other known science in the ancient world,
a body of statements affirming and denying properties of a fundamental notion and those that are integral to it. 
%Consequently, if metaphysics is taken in the sense in which it has been for most of the history of analytic metaphysics, Klima's claims may hold, 
%but not in 

%The fragment of classical propositional calculus containing only $\neg$ and $\wedge$ as logical constants equally interpretable as both a logic for provable entailments interpreting $\wedge$ as English `and' 
%and as one for refutable entailments interpreting \wedge as English `or'

%e.g. there is a notion of being implicit in that of \emph{being} entailed by 
%Incompleteness theorem is a particularly bad case of this: A system can state its own consistency only if it's inconsistent.

Domain variation

a language may be more expressive merely in the sense of having symbols with derived meanings 

Every model available in classical logic is also a model available in Kripke possible world semantics.
Expressive = has more theorems? or has more models?
Still, the independence of semantics from 

%1. Some frameworks are more expressive than others
%2. The claim that semantics is independent of metaphysics is only true in a less expressive frameworks, i.e. that of modern analytic philosophy
%3. If we accept the classical notion of metaphysics, the claim is false. 

%Even analytic philosophy today recognizes that the Quinean program in metaphysics has been exhausted, as evidenced by the proliferation of new fundamental concepts of metaphysics (e.g. Schaeffer).
%Objections:
	%Is metaphysics deferred indefinitely? 
	% can anything be proven in it, or is everything debatable so long as we are willing to make adjustments elsewhere?
	% 


The case of one framework being fundamentally more expressive than another.
\section{Appendix}
\footnote{
	Indeed, we have a close analogue to this in medieval semantics concerning signification, 
	where the signification of a term is taken by the majority tradition to mean the meaning of the term 
	considered in itself apart from any propositional context, 
	while a minority including William of Sherwood and Roger Bacon identifies a term's signification with its reference to actual presently existing things. 
	See \autocite[170-171]{DeRijk1982}. %Don't like this. 
} 
\subsection{Semantics and expressiveness}
Let us begin, in good scholastic fashion, with a distinction.
`semantics' is said in many ways: 
Minimally, it can refer to any number purely mathematical disciplines providing rules for recursively mapping symbols of an artificial language via interpretation functions to domain elements, sets, types, truth functions, etc. - essentially what today we mean by model theory and other closely related disciplines. 
In only a slightly more robust sense, 
it can refer to the same discipline, 
albeit along with its intended informal interpretation such that, 
e.g. that fragment of the classical propositional calculus containing $\wedge$ as its only a logical constant be understood as providing a logic for entailments with that symbol intuitively given the meaning of `and', 
and not one for refutations where it is intuitively given the meaning of `or' .
In a third sense, 
it can refer to a discipline for providing rules for the use of terms in a language, as exemplified in the tradition of proof-theoretic semantics.\footnote{See \autocite{Francez2016b,Read2010,Schroeder-Heister2006}.} 

%Cognitive semantics
\subsection{Objections}
1. Frege point - semantics can have presuppositions without the acceptance and judgment. 
2. Object-language metalanguage distinction is unnecessary. Therefore metalanguages aren't a good source for ontological commitment. 
\section{Notes}
%The problem: the task of positing different senses of being to avoid full ontological commitment isn't a way of avoiding metaphysics: it \emph{is} metaphysics.
Modern metaphysics confuses the question of what being is with the question of what beings there are. (this isn't a new insight: once one gets past the mysticism of some of his interpreters, this is the fundamental point behind the talk of ontological difference in the philosophy of Martin Heidegger)
\subsection{Positions taken}
In \autocite{Klima2008a}, Klima recognizes that representing \emph{via antiqua} semantics would require substantial modifications to modern quantification theory, while representing \emph{via moderna} semantics requires fewer modifications
1) Via antiqua semantics requires a different account of predication, and multiple copula to be introduced to represent the different senses of being
2) Via moderna semantics requires the introduction of restricted quantifiers.
3) both require the rejection of the object-metalanguage distinction.

Klima agrees with Buridan that the notion of truth is not strictly needed for a semantics concerned with formal validity, but it is needed to explain the semantics of sentences that themselves predicate that notion.
Analogy: nobody complains that we don't have a formal definition of the term `red' in our logic, even though a basic grasp of the semantics of that term is needed for using the term in sentences about red things.

Two uses for semantics of truth: 1) as part of a theory of validity, 2) for its own sake.
In `Logic without truth', Klima rejects Buridan's solution to the liar paradox.

%The Rises and Falls of Analysis and Metaphysics in Metaphysical Themes, Medieval and Modern - highly autobiographical. Helps to explain his aversion to the idea of metaphysics bleeding into semantics

%Note. Klima can't get away from the notion of semantics as functions for mapping language to reality. This is part of why he can't accept the idea of metaphysics impinging on logic itself. But this is exactly what the theory of the different main types of supposition, grounded in a theory of analogy, should entail.

(1) Natural languages are semantically closed (2) Natural language inference has to be token-based.
Both Klima/Buridan and Tarski come to the conclusion that defining consequence in terms of truth and falsity doesn't work from similar considerations: Tarski's consideration is related to superenumable domains and the possibility that a language may simply an appropriate selection of denoting terms; Klima/Buridan's considerations come from the possibility that a claim may be not exist to even be true or false, or it may be self-falsifying while nevertheless describing a possible state of affairs \autocite[96]{Klima2004}.

The primary impetus behind Klima's work is one of charity.
Examples: 
Positive:
1. His analysis of parasitic reference in his work on Anselm
2. His attempts to translate between via antiqua and via moderna semantics
3. The entirety of his body of work on John Buridan
Negative:
1. The infrequency with which Klima actually reveals his own philosophical positions in his work (exceptions: 
Aquinas' hylomorphism and proof of immortality, 
Anselm's proof, 
Per Buridan, the semantic closure and token-based character of natural language inference)
2. His adopting semantics that build on classical logic while rejecting non-classical semantics.

%Metaphysical questions are undecidable by semantics alone. In the same sense in which the body of truths of revealed theology are undecidable by natural philosophy, or the continuum hypothesis is undecidable by set theory.
%The rises and falls of analysis and metaphysics
%Being, unity, and identity in the Fregean and Aristotelian traditions

%metaphysics \textit{does} determine semantics. Consequently, the attempt to infer metaphysics from semantics deductively, as with all similar cases of causality, would logically have to be a case of affirming the consequent. 
%demonstratio quia

\subsection{Why semantics?}
Semantics: 
1) a theory of meaning broadly construed
2) Tarskian/Montaguean mathy stuff
a semantics is almost never actually this, given that most semantics have a canonical interpretation and a domain to which they are expected to apply (e.g. Model theory handles solids better than liquids or gases).
3) e.g. a dictionary
4) e.g. proof-theoretic semantics
5) a philosophy of language
6) a theory of language, thought, and reality
%Two contradicting claims: 1) The modern turn in metaphysics was conditioned by a prior one in semantics; 2) neither via antiqua nor via moderna semantics strictly entail the metaphysical positions they are usually associated with.

%Two phenomena: 1) the breakdown of communication within analytic philosophy 2) its lack of breadth

%Buridan's logic and the ontology of modes - neither via antiqua nor via moderna semantics entailed realism or reductionist metaphysics, respectively.

\subsection{Buridan}
Discussing the value of studying Buridan's philosophy in particular and that of the history of philosophy generally, he writes: 

\begin{quote}
	this study can put our own philosophical problems in an entirely different light, providing us with
	such theoretical perspectives that otherwise might entirely escape us as we are working in our
	set ways determined by the intellectual habits of our philosophical period, which in modern
	times tends to stretch to a mere couple of decades. \autocite[17]{Klima2005}
\end{quote}
Besides the frank exhortation to study the history of philosophy as a way to expand one's intellectual horizons, 
one finds an indictment of the historical shallowness of much philosophy in our own period. 

%I personally remember feeling the bait-and-switch that seemed to occur when I moved from a historically rich undergraduate program to a heavily analytic department.
%The rank materialism of the vast majority of professors didn't threaten me: instead, I found it completely stupid..

an expansion of one's horizons that goes beyond merely getting the right answers.

- This is philosophy, and its difference from sophistry.

An atheist who's one \emph{modus tollens} away from sincere devotion.
%What is the criterion for provability of a metaphysical claim: provability across all frameworks?

%\section{Philosophy of mind and epistemology}
%\subsection{\textit{Via antiqua} and \textit{via moderna} cognizers}
%\section{Metaphysics}
%\subsection{Hylomorphism, personal identity and immortality}
%\subsection{Causation}
%\section{Overview of the articles}

\section{quotes}
%The efficient causality model fixes the relation of natural signification on the basis of natural laws systematically connecting causes with their effects. But all that these natural laws guarantee is the systematic correspondence between causes and their effects, supposing the normal course of nature. The effects, however, may be essentially different from their causes, and may, therefore, be produced also by other essentially different causes too, which means that it is clearly possible that an absolute concept be caused in our intellects by a cause which is totally alien from what this concept is supposed to represent. - Ontological Alternatives.
%IMPORTANT!, the immediately above quote is self-defeating for Klima, since his remarks on the relation between semantics and metaphysics are themselves conceived on the efficient, rather than formal, causality model mentioned above.

`The primary purpose of a logical semantic theory is to define logical consequence in terms of the truth values of propositions in different interpretations' \autocite[79]{Klima1991b}. 

However, again, if we need to represent the finer details of the via moderna
conception, we need to depart considerably from the standard construction of the
semantics of quantification theory. In the first place, although using restricted
variables to represent common terms in personal supposition yields “the right re-
sults” concerning the square of opposition and syllogistic, nevertheless, it does so
at the expense of representing simple common terms (say, F ) as complex vari-
ables with an intrinsic propositional structure embedded in their matrix (‘x.F x’,
amounting to something expressible as ‘thing that is an F ’). But according to
our via moderna authors it is the simple term ‘man’, for example, that has this
referring function, and not a complex term like ‘thing that is a man’. In fact Buri-
dan would pointedly distinguish the two in various contexts. So, to represent this
feature of via moderna semantics, we would need to devise “term-logics” along the
lines proposed by Lesniewski, Lejewski, Henry, Sommers and Englebretsen.\autocite{Klima2008a}

Buridan has only one language to talk about the world as well as about that
language and its semantic relations to the world. And in that one language
we cannot truly say that there are mere possibilia, or that something that is
merely possible exists. Accordingly, from this Buridanian perspective, the
issue of ontological commitment in terms of a meta-linguistic description of
the relationship between language and the world is radically ill-conceived.\autocite[172]{Klima2009}

A merely possible being or a fi ctitious entity is not just
a special kind of entity; indeed, no more than a fake diamond is a special
kind of diamond or forged money is a special kind of money. Just as a fake
diamond is not something that is both a diamond and is fake and forged
money is not something that is both money and forged, so a fi ctitious entity
is not something that is both an entity and is fi ctitious. And just as a fake
diamond is no diamond at all, and forged money is no money at all, so a
fictitious entity is not an entity at all.\autocite[173]{Klima2009}
%Good quote for illustrating Klima's reliance on Geach here. 

`Buridan’s nominalism is obtainable by the adverbialization of Peter
of Spain’s semantics.'
`Nominalism is obtainable by the adverbialization of realist
semantics.'


A project of eliminating unwanted
ontological commitment is not at all about finding out anything about anything;
rather it is a project to show just how much one can get away with in one’s
semantics on the basis of how little in one’s ontology \autocite[412]{Klima2012}.

In this paper I will attempt to dig further to the roots of
their disagreements, trying to establish those primary logical-semantic differences that may have
motivated their conflicting intuitions concerning these metaphysical principles. - Thomas of Sutton v. Henry of Ghent

Therefore, it should also be clear that the laws of logic in this framework
are supposed to be fundamentally different from the laws of psychology.
For while the former are the laws of the logical relations among objective
concepts, the latter are the laws of the causal relations among formal
concepts. - The problem of universals and the subject matter of logic, Klima 2014, p. 173.

These different theories can be arranged on a `theoretical scale', ranging from extreme realism to extreme nominalism, meaning maximal semantic uniformity along with maximal ontological diversity on the realist end [...], and maximal ontological uniformity with maximal semantic diversity on the nominalist end.- The problem of universals and the subject matter of logic, Klima 2014, p. 176.

Well, conceptual diversity is obviously
a great hindrance to understanding: if we don’t have the same concepts, we can-
not have the same thoughts, which means we are doomed to talking past each
other all the time\autocite[36]{Klima2021}

So, what should be our guiding light, in this rational discourse? In one word:
rationality, which is love or goodwill on its active side, on the part of the will, and
understanding on its receptive, theoretical side, on the part of the intellect \autocite[41]{Klima2021}.
\autocite{Parsons2014,Read2015b}

\printbibliography

\end{document}
