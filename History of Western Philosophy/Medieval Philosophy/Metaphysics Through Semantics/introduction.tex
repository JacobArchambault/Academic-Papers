\documentclass[]{article}
\usepackage[backend=biber, style=authoryear-icomp]{biblatex}
\bibliography{../../../jacob}

%opening
\title{Introduction}
\author{Jacob Archambault}

\begin{document}

\maketitle

\begin{abstract}

\end{abstract}

\section{Introduction}
\section{Biographical introduction}
\section{Logic and semantics}
The breadth of Klima's scholarship stretches from some of philosophy's best-known figures in Anselm, Aquinas, Ockham and Descartes 
to lesser-known figures including Thomas of Sutton and Henry of Ghent, 
to Frege, Geach, Kenny, and others who have produced some of the most impactful scholarship in the analytic tradition; 
from debates on identity, categories, and causation in metaphysics, 
on skepticism in epistemology and theories of mental content in philosophy of mind, to others too numerous to mention.

Klima's most widely recognized contributions come in his research on John Buridan - which has helped elevate Buridan from a lesser-known figure to one whose stature is closer to that of an Ockham, arguably surpassing the Franciscan in his logic - and in the field of semantics. 
\subsection{Buridan}

\subsection{Semantics}
In some of his earliest work in semantics, 
Klima recognized that classical logic, 
being primarily interested in developing an account of the semantics of propositions as a precondition for the development of a theory of consequence, 
affords much less attention to its account of the components of propositions themselves, 
namely names and n-ary predicate relations. 
Klima fills this lacuna by providing a theory not only of simple, but also of complex terms. 

As part of this emphasis, 
his work provides some of the earliest and most ambitious applications of restricted quantification in the history and philosophy of logic, 
using it both to formalize the medieval theory of supposition and to provide a general account of quantitively ambiguous natural language sentences \autocite{Klima1988,Klima1990,KlimaSandu1990,Klima1991b}. 
Expansions on the same theme - 
namely, formalizations of supposition theory specifically and medieval semantics more broadly as a means to resolve apparently intractable interpretative problems in historical scholarship and debates in contemporary philosophy - 
provide us with an account of the semantics of intensional verbs \autocite{Klima1991}, 
a semantic foundation for Aquinas' theory of the analogy of being in his theory of the copula \autocite{Klima1996,Klima2002}, 
and a clean resolution of the problem of existential import in the Aristotelian square of opposition\autocite{Klima2001}. 

Elsewhere, Klima's work decouples \emph{via antiqua} and \emph{via moderna} semantics from the realist and anti-realist metaphysics with which they are most commonly paired with by showing 
that neither semantics by itself strictly entails its associated metaphysics; 
that archtypical realists were required to adopt non-straightforward semantic accounts of the meanings of terms in at least some cases by their antecedent metaphysical commitments (e.g. to divine simplicity),
and that some of the best known nominalist logicians adopted what today would be regarded as `realist' metaphysical positions \autocite{Klima1999,Klima2005,Klima2011}. 

None of this means that there is \emph{no} relationship between an author's positions in metaphysics and his semantics: 
rather, the semantic framework an author adopts conditions what options that author has in metaphysics without fully determining them: 
for example, extreme realism in metaphysics doesn't follow strictly from the \emph{via antiqua}'s inherence theory of predication, 
but it is easily the most natural fit for that theory if one also adopts the view that terms connoting accidental being denote their referents rigidly \autocite{Klima1999}; 
conversely, the broad outlines of Ockham's account of the relation between language, thought, and reality serves as a foundation for the realism of a Descartes, Malebranche, Putnam or a Leibniz \autocite{Klima1991}.
%But there is not thereby \emph{no} relation between an author's metaphysics and his semantics. In the case of Ockham and his opponents, for instance, certain metaphysical commitments arise naturally out of the assumption that names of accidental qualtiies are or aren't rigid designators.



\autocite{Parsons2014,Read2015b}

`The primary purpose of a logical semantic theory is to define logical consequence in terms of the truth values of propositions in different interpretations' \autocite[79]{Klima1991b}. 

The primary impetus behind Klima's work is one of charity.
%Metaphysical questions are undecidable by semantics alone. In the same sense in which the body of truths of revealed theology are undecidable by natural philosophy, or the continuum hypothesis is undecidable by set theory.
%The rises and falls of analysis and metaphysics
%Being, unity, and identity in the Fregean and Aristotelian traditions

%metaphysics \textit{does} determine semantics. Consequently, the attempt to infer metaphysics from semantics deductively, as with all similar cases of causality, would logically have to be a case of affirming the consequent. 
%demonstratio quia
\subsubsection{Why semantics?}
%Two contradicting claims: 1) The modern turn in metaphysics was conditioned by a prior one in semantics; 2) neither via antiqua nor via moderna semantics strictly entail the metaphysical positions they are usually associated with.

%Two phenomena: 1) the breakdown of communication within analytic philosophy 2) its lack of breadth

%Ontological alternatives vs. alternative semantics - recognition that categories like realism, nominalism, and conceptualism are a poor fit for medieval philosophy generally - The broad outlines of Ockham's account of the relation between language, thought, and reality serves as a foundation for the realism of a Descartes, Malebranche, or a Leibniz, the skepticism of a Hume, or the idealism of a Kant or Hegel.
%Buridan's logic and the ontology of modes - neither via antiqua nor via moderna semantics entailed realism or reductionist metaphysics, respectively.
\section{Philosophy of mind and epistemology}
\subsection{\textit{Via antiqua} and \textit{via moderna} cognizers}
\section{Metaphysics}
\subsection{Hylomorphism, personal identity and immortality}
\subsection{Causation}
\section{Overview of the articles}

\printbibliography

\end{document}
