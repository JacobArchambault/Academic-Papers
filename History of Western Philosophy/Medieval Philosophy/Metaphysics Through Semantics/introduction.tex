\documentclass[]{article}
\usepackage[backend=biber, style=authoryear-icomp]{biblatex}
\bibliography{../../../jacob}

%opening
\title{Introduction}
\author{Jacob Archambault}

\begin{document}

\maketitle

\begin{abstract}

\end{abstract}

\section{Introduction}
\section{Biographical introduction}
\section{Logic and semantics}
The breadth of Klima's scholarship is vast: 
historically, it stretches from some of philosophy's best-known figures in Anselm, Aquinas, Ockham and Descartes 
to lesser-known figures including Thomas of Sutton and Henry of Ghent, 
to Frege, Geach, Kenny, and others who have produced some of the most impactful scholarship in the analytic tradition; 
systematically, he has contributed to debates on identity, categories, and causation in metaphysics, 
on skepticism in epistemology and theories of mental content in philosophy of mind, on others too numerous to mention.

Klima's most widely recognized contributions come in his research on John Buridan - which has helped elevate Buridan from a lesser-known figure to one whose stature is closer to that of an Ockham, arguably surpassing the Franciscan in his logic - and in the field of semantics. 
\subsection{Buridan}

\subsection{Semantics}
In some of his earliest work in semantics, 
Klima recognized that classical logic, 
being primarily interested in developing an account of the semantics of propositions as a precondition for the development of a theory of consequence, 
spends much less attention on its account of the components of propositions themselves, 
namely names and n-ary predicate relations. 
Klima fills this lacuna by providing a theory not only of simple, but also of complex terms. 
As part of this emphasis, 
his work provides some of the earliest and most ambitious applications of restricted quantification in the history and philosophy of logic, 
using it both to formalize the medieval theory of supposition and to provide a general account of quantitively ambiguous sentences in natural language \autocite{Klima1988,Klima1990,KlimaSandu1990,Klima1991b}. 



`The primary purpose of a logical semantic theory is to define logical consequence in terms of the truth values of propositions in different interpretations' \autocite[79]{Klima1991b}. 
%Ars Artium, Approaching Natural Language via Mediaeval Logic, Numerical Quantifiers in Game-Theoretical Semantics

%The rises and falls of analysis and metaphysics
%Being, unity, and identity in the Fregean and Aristotelian traditions
\subsection{Why semantics?}
\section{Philosophy of mind and epistemology}
\subsection{\textit{Via antiqua} and \textit{via moderna} cognizers}
\section{Metaphysics}
\subsection{Hylomorphism, personal identity and immortality}
\subsection{Causation}
\section{Overview of the articles}

\printbibliography

\end{document}
