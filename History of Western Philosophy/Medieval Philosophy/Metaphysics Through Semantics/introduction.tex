\documentclass[]{article}
\usepackage[backend=biber, style=authoryear-icomp]{biblatex}
\bibliography{../../../jacob}

%opening
\title{Gyula Klima as medievalist}
\author{Jacob Archambault}

\begin{document}

\maketitle
\begin{abstract}
This essay provides a broad introduction to Gyula Klima's contributions in the field of medieval philosophy, 
with special attention to 
his pioneering writings in semantics 
and on 
the 14th century arts master and rector at the University of Paris 
John Buridan. 
Klima's scholarship 
provides one of the best examples available among philosophers living today not only 
of how to read thinkers in the medieval tradition, 
but also 
of how the effort to understand a radically different paradigm 
embodied in that reading 
provides the first steps 
to resolving broader problems of communication 
across distinct traditions and subdisciplines
in philosophy 
at large.
\end{abstract}

\section{Introduction}
This essay provides 
a broad overview 
of Gyula Klima's writings 
as a philosopher, 
logician, 
translator and historian of medieval thought. 
I begin with an account of 
Klima's pioneering contributions in the field of medieval semantics. 
From there, I examine his work 
editing, 
translating, 
and in engagement with 
the Parisian arts master 
and 
logician 
John Buridan. 
I close with some remarks on the unique character and import of Klima's work as a historian of medieval philosophy within today's' philosophical landscape. 
\section{Gyula Klima's contributions in the history of semantics}
Klima's scholarship stretches 
historically from some of philosophy's best-known figures in Anselm, Aquinas, Ockham and Descartes 
to lesser-known figures including Thomas of Sutton and Henry of Ghent, 
to Frege, Geach, Kenny, and others who have produced some of the most impactful scholarship in the analytic tradition; 
and 
thematically from debates 
on identity, categories, and causation in metaphysics, 
on skepticism in epistemology and theories of mental content in philosophy of mind, 
to others too numerous to mention. 

Klima's influence to date is most felt 
in the history of semantics.
From his earliest contributions to the field, 
Klima recognized 
that 
on account of its 
limited interest in developing the semantics of propositions 
as a precondition for the development of a theory of consequence, 
the traditional apparatus of classical semantics affords less attention 
to the components of propositions themselves. 
%namely names and n-ary predicate relations. 
Klima fills this lacuna by providing some of the earliest and most ambitious applications of restricted quantification in the history and philosophy of logic, 
using it both to formalize the medieval theory of  \emph{supposition}\footnote{See  \autocite{sep-medieval-terms} for an overview.} 
and to provide a general account of quantitatively ambiguous natural language sentences.\footnote{\autocite{Klima1988,Klima1990,KlimaSandu1990,Klima1991b}. Cf. \autocite{Parsons2014}.} 
Elsewhere, 
Klima's formalizations of supposition theory specifically 
and medieval semantics more broadly 
provide us with an account of abstract and connotative terms \autocite{Klima1991}, 
a semantic foundation for Aquinas' theory of the analogy of being \autocite{Klima1996,Klima2002}, 
and 
a clean resolution of the problem of existential import in the Aristotelian square of opposition.\footnote{\autocite{Klima2001}. Cf \autocite{Read2015b}.} 

\subsection{Decoupling \emph{via antiqua} and \emph{via moderna} semantics from metaphysics}
In one sustained thread of that work, 
Klima decouples \emph{via antiqua} and \emph{via moderna} semantics 
from the realist and anti-realist metaphysics with which they are most commonly paired,
contending that neither semantics by itself strictly entails its associated metaphysics \autocite{Klima1999,Klima2011}. 
Rather, archtypical realists were required to adopt non-straightforward semantic accounts of the meanings of terms in at least some cases by their antecedent metaphysical commitments (e.g. to divine simplicity) \autocite{Klima2002b},
while some of the best known nominalist logicians incorporated what today would be regarded as realist elements in their logic \autocite{Klima2005}. 
For Klima, the \emph{via antiqua} and \emph{via moderna} traditions of medieval logic 
differ not in their \emph{quantity} of ontological commitments, 
but in the tools they provide for \emph{handling} ontological commitments, 
which in turn differ from those of the model-theoretic framework dominant today. 

\emph{Via antiqua} semantics takes an affirmative statement to be true when what is signified by its predicate inheres in what is signified by its subject - 
sometimes called the \emph{inherence theory of predication}. 
Within this framework, 
terms predicating common natures or accidental features of a subject are taken to ultimately refer to exactly the categorical entities one might expect. 
But the framework 
avoids full, immediate, fundamental commitment to entities today's nominalists might find objectionable by providing a rich theory according to which being is predicated in different degrees.\footnote{See \autocite{Klima2002}.}

Conversely, the \emph{via moderna} framework that became ascendant after William of Ockham 
takes an affirmative statement to be true when its subject and predicate term refer to the same object 
 - sometimes called the \emph{identity theory of predication}. 
Within this framework, 
terms predicating common natures or accidental features of a subject need not be taken to ultimately refer to different types of objects such as abstract genera or relations, 
but instead refer to familiar objects \emph{differently}. 
For example, the truth of `Socrates is a father' does not require commitment to a distinct entity that is Socrates' fatherhood.
Instead, the sentence's predicate may (non-rigidly) refer to Socrates himself, 
albeit connoting his being a father, 
and hence refer to the same object as that rigidly referred to by the proper name `Socrates'. 
Granting some license for intensional contexts,\footnote{See \autocite{Klima2005}.} 
the verb `is' or `exists' in \emph{via moderna} semantics is equally ontologically committing in its various uses, 
but \emph{what} one is committed to by its uses need not be immediately apparent  \autocite[437-430]{Klima2008a}. 

Both medieval frameworks would reject the object-language metalanguage distinction taken for granted since Tarski in their theory of truth, 
and 
in their use of ampliation for tensed, modal, and intensional contexts,
both provide ample tools for rejecting a na\"{i}ve application of Quine's account of ontological commitment in terms of existential quantification.\footnote{Cf. \autocite{Klima2004}, 
\autocite[171-174]{Klima2009}.}

\subsection{Ontological neutrality and independence}
None of this means that there is \emph{no} relationship between an author's positions in metaphysics and his semantics: 
rather, 
the semantic framework an author adopts conditions what options that author has in metaphysics without fully determining them. 
For example, extreme realism in metaphysics doesn't follow strictly from the \emph{via antiqua}'s inherence theory of predication, 
but it is the most natural fit for that theory 
if one accepts the view that terms signifying accidental being denote their referents rigidly 
while rejecting that framework's insistence on multiple, analogically related senses of `being' \autocite[125]{Klima1999}. 
Conversely, the broad outlines of Ockham's account of the relation between language, thought, and reality 
serve not only as a foundation for Ockham's own metaphysical reductionism, 
but also for the realism of a Descartes, Malebranche, Putnam or a Leibniz \autocite{Klima1991}. 
%In an extreme case, Klima suggests the choice of a mistaken semantic framework may inhibit the speaker from constitutively referring to, 
%and thus believing in, 
%an actually existing God whose existence is only adequately assertable in an alternate framework \autocite[74]{Klima2008b}. %This page reference is an estimate. P. 22 in Klima's online copy.

There is, however, no relationship of \emph{entailment} from purely semantic principles to metaphysical truths. Klima writes: 
\begin{quote}
To be sure, this is not to say that metaphysical principles are to be derived from, or somehow justified 
in a weaker sense on the basis of, semantic principles. Metaphysical principles, being first principles 
using the most general terms, such as the transcendentals and the categories, cannot be derived from 
prior principles, and their terms cannot be defined on the basis of more general terms. What 
semantics can do, however, is that it can provide the principles of interpretation of metaphysical 
principles. On the basis of these principles of interpretation the implications of metaphysical 
principles are more clearly delineated, which then can be used in their evaluation in dialectical 
disputations concerning their acceptability in the interpretations thus clarified. Furthermore, if the 
semantic principles of interpretation are made explicit, they can also be subject to further evaluation, 
in a disputation on a different level, the sort appropriate to the comparison of different logical 
theories  \autocite[49]{Klima2011b}.
\end{quote}
Modern mathematics 
names a relation quite close to this 
in its notion of 
\emph{independence}, 
though as the name implies the fundamental notion itself is by no means a recent one. 
Just as Cantor's continuum hypothesis is neither provable nor refutable from the principles of Zermelo-Fraenkel set theory alone, 
or - to provide a more medieval example - 
truths of revealed theology are neither provable nor refutable from the principles of natural philosophy, 
neither on Klima's account are metaphysical principles provable or refutable from those of semantics alone. 

\subsection{Pluralism, linguistic imperialism, and the problem of cross-cultural communication}
Two complications distinguish the semantic case from those mentioned. 
The first is that while both the set-theoretic and theological case mentioned above concern provability and refutability in a single system, 
the sheer multiplicity of semantic frameworks itself may provide a barrier to a broadly acceptable account of provability across those frameworks. 
The second generalizes a problem nearly the opposite of that established by G\"{o}del in his first incompleteness theorem \autocite{Godel1931}: 
where that theorem established the expressibility of unprovable claims of number theory in any sufficiently robust system, 
the semantic problem we face here is that a claim of metaphysics may be taken to be established or refuted merely on account of the lack of expressibility of the particular semantic framework one occupies. 

Klima's response to these problems is anti-pluralist 
without thereby being dogmatically classical. 
While it would be easy enough to, for instance, 
construct a metalogical account of validity by quantifying over distinct logical systems on the model of possible world semantics and regarding as valid all and only those theorems valid in every system, 
Klima instead recognizes the known limitations of classical semantics 
while also taking the provable equivalence of systems containing distinct logical primitives
as \emph{prima facie} evidence for the possibility of a fundamental diversity at the \emph{conceptual} level 
that nevertheless doesn't entail a despairing or indifferent anti-realism at the \emph{metaphysical} level \autocite{Klima2012}.
Instead, 
Klima's response, 
both technically and philosophically, 
is to \emph{extend the framework}. 
Meeting the tradition where it is, he extends classical semantics to allow for treatment of 
donkey anaphora \autocite{Klima1988,Klima2010}, 
non-existent entities \autocite{Klima2001}, 
and quantificational phenomena \autocite{KlimaSandu1990},
while more broadly appealing (in a rare quote of a `continental' philosopher that shows up in multiple places throughout his \emph{\oe{}uvre}) 
to the possibility of a `fusion of horizons' mentioned by Gadamer 
as a solution to the impasse of communication across distinct semantic frameworks, cultures, or philosophical traditions 
and the attitude of metaphysical anti-realism it encourages \autocite{Klima2000,Klima2009a}.

\section{John Buridan}
Nowhere has this effort been more sustained than in Klima's scholarship on John Buridan', 
which has helped elevate the 14th century arts master from a lesser-known figure 
to one whose stature is closer to that of an Ockham, 
arguably surpassing the Franciscan in his logic.

Like the study of medieval contributions to logic and philosophy of science in the early 20th century more broadly, 
the resurgence of interest in Buridan's writings at that time was stimulated 
by
both 
their promise as a resource for solving contemporary problems in those disciplines 
and the way in which they seemed to herald contemporary developments:  
Duhem, for instance, held out Buridan's theory of impetus as a precursor to Galileo's account of projectile motion; 
Boehner took the logical systems of Buridan and his contemporaries to be closer to the formal work of the Lvov-Warsaw school 
than to the anti-formalist tendencies in neo-scholastic texbooks of the day \autocite{Boehner1952}, 
and Louise Nisbet Roberts took Buridan's solution to the Liar paradox to anticipate that of Tarski \autocite[100]{Roberts1953}. 
This interest paved the way for critical editions of Buridan's writing, beginning with Hubien's edition of his \emph{Tractatus de consequentiis} in 1976 and later extending to his output on logic, philosophy of mind, physics and metaphysics.\footnote{See the extensive bibliography provided in \autocite{sep-buridan}.}

Klima has built on these contributions 
both straightforwardly as an editor and translator 
as well as critically across his articles, books, and various other scholarly contributions. 
The subtlety of Klima's work as a translator is apparent throughout his translation of Buridan's massive \emph{Summulae de dialectica}, 
whose footnotes provide an interesting window not only into the text itself, 
but also into Klima's own decisions on how to translate a text whose parts are frequently concerned with linguistic imprecision and ambiguity.
Take, for instance, 
the following footnote text introducing his translation of the Latin \emph{passio} as `attribute' early on in treatise 1:
\begin{quote}
	The term \emph{passio}, 
	deriving from the verb \emph{pati} 
	(to suffer, 
	to be affected/acted on, 
	to undergo change), 
	has the primary sense of something affecting a subject 
	(which receives the action of an agent). 
	But since the relation between the term signifying such an affection 
	and the term that signifies the subject 
	is analogous to the relation between the affection itself 
	and the subject itself, 
	the term signifying the affection is also called a \emph{passio} 
	(and the term signifying the
	subject is also called \emph{subiectum}). 
	Therefore, 
	in this technical sense, 
	whenever a \emph{passio} is correlated with a \emph{subiectum}, 
	referring to a term that is attributed to a subject term in an act of predication, 
	I will translate \emph{passio} as `attribute'. 
	Whenever \emph{passio} is used to refer to the correlative of some action, 
	however, 
	as is normally the case in the context of Aristotelian physics,
	or to the correlative category of the category of action, 
	or to the third species of the category of quality discussed below (3.5.4), 
	as is usually the case in the context of the theory of
	categories, 
	I will use the customary English transcription `passion'. 
	To be sure, 
	even despite existing translational traditions to this effect, 
	this may occasionally sound odd, 
	given the primary contemporary meaning of the term indicating some strong emotion (which is actually
	quite fitting in the case of the third species of quality, especially in 3.5.4(2)). 
	But this will be very useful when Buridan exploits some of the conceptual relations between the notions of
	`passion' in the technical senses intended here 
	and those of being affected, 
	being acted on, undergoing change, suffering 
	(as the Passion, i.e., the suffering of Christ), 
	and passion in the emotional sense, all of which are conveyed by the Latin \emph{passio} \autocite[5]{BuridanKlimaSD}.
\end{quote}
In this example
(which apart from its discussion of present-day English idiom could well pass for a translation of one among the better specimens of scholastic Latin), 
Klima's parsing out the different significations of the term \emph{passio}, 
relating its meaning to that of its English derivative, 
then justifying different translations for different  contexts 
provides a worthy example of how Klima's long study of Buridan's logic and semantics itself inform his translation of the very Latin texts conveying them.

Klima's Buridan scholarship contrasts with that of preceding generations, 
however, 
in three main respects. 
The first is it's breadth: 
the arc of Klima's scholarship - 
first gaining notoriety in the fields of logic and semantics, 
extending from there into medieval natural philosophy and metaphysics, 
and with an increased focus in recent decades on philosophy of mind and epistemology 
culminating in a critical edition of and companion volume to Buridan's \emph{Quaestiones De Anima}  -  
has over time expanded to cover the whole territory of Aristotle's `semantic triangle'  
mapping out the relations between word and thing through the mediation of concepts. 
Because of this, Klima's body of work provides what is arguably the most integrated and complete account of John Buridan's philosophy to date.

The second is its stance vis-a-vis contemporary analytic philosophy. 
Where much earlier scholarship on Buridan's thought
stressed its proximity to recent discoveries 
to lend it greater credibility, 
Klima has more often used this proximity 
to challenge contemporary positions on their own terms. 
For example, 
where every well-known model theory since Tarski 
both 
identifies truth in a model with satisfaction 
and 
grounds its account of logical consequence on that of truth, 
leading Tarski himself to regard all semantically closed languages 
%- that is, 
%languages whose syntactic elements are themselves members of their target domain, 
%and consequently able to be referred to by other elements in the same domain as themselves - 
as inconsistent \autocite[348-349]{Tarski1943}, 
Klima shows Buridan 
both
rejects the Tarskian identification 
and 
inverts the above grounding relation between consequence and truth. 
Reason for rejecting the first can be found without recourse to semantic paradoxes, 
merely by considering statements like `no sentence is negative' 
whose satisfaction conditions preclude them from being true 
at their time of utterance \autocite[96-100]{Klima2004}. 
Furthermore, 
not only is there is no need for an account of satisfaction grounding that of consequence to do double duty as an account of truth, 
but because statements may fail to be true 
by being inherently inconsistent with what they posit 
(as occurs with `no sentence is negative'), 
or merely contingently so 
(as occurs with reciprocal liar sentences that in other contexts would be merely true or false), 
the semantics for terms like `true' and `false' themselves presuppose a notion of entailment like that hinted at in Buridan's idea of a sentence virtually implying its own truth.\footnote{\autocite[101-107]{Klima2004}. Cf. \autocite[221-225]{Klima2009}, \autocite[22-27]{Hughes1982}.} 
Consequently, 
a sentence meeting its satisfaction conditions constitutes a necessary, 
but not a sufficient 
condition for its truth, 
and the T-Schema for truth is simply mistaken. 
In another example, 
Klima inserts Buridan into the dispute between Quine and his fictional interlocutor Wyman over what exists 
to argue 
both are mistaken 
not in the particularities of their approach to ontological commitment, 
but more broadly 
in accepting a context-insensitive quantifier 
with its tacit assumption of the availability of a metalinguistic `view from nowhere' 
as a criterion for ontological commitment at all:  `the solution Buridan
offers is not an overall split between object-language and meta-language
but a more careful regulation of the reflective uses of the same language' \autocite[174]{Klima2009}. 
More recently, 
Klima has expanded the differences expressed archtypically 
here 
in different attitudes towards the object-metalanguage distinction 
into a concise summary of the different orientations 
of the medieval project Buridan engaged in 
and that which animated Quine, Tarski 
and the tradition after them:

\begin{enumerate}
	\item[(1)] The ``modern project'': to ``cannibalize'' ever greater portions of all
	possible forms of natural language reasoning, expand the expressive
	resources of our formal language(s) for which we can have a uniform
	definition of validity, grounding the construction of a universal
	method for checking validity either in terms of deduction rules or a
	compositional semantics.
	\item[(2)] The ``medieval project'': to ``regulate'' ever greater portions of all
	possible forms of natural language reasoning, regiment the syntax
	of our natural language as much as ordinary usage would tolerate,
	so as to be able to accommodate as many forms of natural language
	reasoning as possible, and thus to be able to separate valid from
	invalid consequences in accordance with a range of different criteria
	of validity.\footnote{\autocite[341]{Klima2016}. Cf. \autocite[429-430]{Klima2008a}.}
\end{enumerate}

This depth of its critical engagement with both medieval and modern philosophy 
has often led to a certain prescience 
in its themes and positions:
his use of restricted quantifiers in formalizing Buridan's logic predates revived interest in these 
in 
work on relevant logic 
and semantic paradox 
by roughly twenty years;\footnote{
	Cf. \autocite{Klima1988}, \autocite{Beall2006}, \autocite{Field2014}. For earlier work, see \autocite{Hailperin1957a,Hailperin1957b}
} 
his formalization of the medieval square of opposition, 
on which the existential import of categorical sentences is determined by their quality (affirmative or negative) 
rather than their quantity (universal or particular), 
significantly predates comparable formal treatments in ancient and medieval logic,\footnote{Cf. \autocite[18-43]{Klima1988}, \autocite{Chatti2013} \autocite{Read2015b}.} 
%by roughly a quarter century,
and his early analysis of suppositional descents 
%as decents to disjuncts or conjuncts of sentences substituting demonstrative pronouns 
is echoed in articles on pronouns and donkey anaphora being published as this paper is being written.\footnote{Cf. \autocite{Klima1990}, \autocite{Blumberg2021}}.

\section{Klima as historian}
The third respect in which Klima's Buridan scholarship, 
and indeed his work as a medievalist as a whole, 
has distinguished itself 
is in its orientation 
towards its source material. 
While the above shows Klima amply capable of 
both 
'pure' historical scholarship 
and 
bringing medievals into engagement with his contemporaries, 
much of his work goes beyond that to 
broader questions of 
how this history has occasioned
the adoption of beliefs widely held today,
and by extension the impasses they lead to.
In this way, 
Klima's reading of Buridan, 
Ockham, 
Aquinas 
and others 
does not so much mine them as resources for arguments and positions 
as it takes their study as part of a kind of philosophical disaster recovery program. 
%In this respect, 
%the introduction to this volume's extended comparison 
%to MacIntyre is an apt one.
In various places 
his writings begin with one question 
only to lead their reader to a higher one: 
moving, 
for instance, 
in his Stanford Encyclopedia article on medieval theories of universals 
from the various questions arising out of Plato's theory of forms 
that the medievals inherited 
from Porphyry 
via Boethius
%to Abelard's resituating of its main questions in semantics
to a consideration of how the debate factored into the disintegration of scholastic discourse \autocite{sep-universals-medieval}; 
or from considering changes to the notion of an efficient cause in the late medieval period 
to the impact of those changes 
on how we continue to think about knowledge and certainty today \autocite{Klima2013}.
In a review of Anthony Kenny's \emph{Aquinas on Mind}, 
Klima contrasts that book's approach of 
making Aquinas' ideas 
`accessible to the philosophically interested contemporary reader 
in terms of such philosophical, scientific and everyday
concepts with which the reader can safely be assumed to be familiar'\autocite[113]{Klima1998}
with his own approach as follows:

\begin{quote}
	First, as should be obvious, we shall never understand properly any of
	Aquinas's theories without first ``learning his language''. However, 
	``learning his language'' does not mean just learning Latin, but rather
	acquiring the radically different conceptual apparatus encoded in his
	language, constantly reflecting on how this different apparatus constitutes its own self-evident truths, while questioning the validity of what
	we take to be self-evident truths on account of the conceptual apparatus
	encoded in our philosophical language. 
	Second, we shall never be able to
	communicate our understanding of Aquinas authentically unless we
	learn how to ``teach his language''  \autocite[115]{Klima1998}.
\end{quote} 

For the kind of teaching that Klima is demanding here, 
let us consider an example from elsewhere in his \emph{corpus}, 
where he leverages a comparison to modern thermodynamics 
into a defense of a 
%\emph{prima facie} implausible 
claim from Aristotle's \emph{Physics}, 
oft-quoted by Aquinas, 
that `man is generated by man and the sun':
\begin{quote}
	[A] universal
	cause as Aquinas thinks about it, is certainly not a universal in its being
	(given that Aquinas rejects Platonic universals), but in its causality: a
	particular cause is the cause of only this particular effect, whereas a
	universal cause is a cause of several particulars of a given kind. However,
	an immediate consequence of this interpretation and the above-demonstrated irreflexivity of per se efficient causality is that a universal
	cause of a given kind of particulars itself cannot be of the same kind; for
	otherwise, being the cause of all particulars of the same kind, it would
	have to be a cause of itself, which is impossible. Therefore, the universal
	cause of a species cannot be a member of the same species: it has to be a
	non-univocal cause, that is to say, the form by virtue of which it acts and
	produces and/or sustains its effects is not the same form that it brings
	about in its effects. This is the reason that talking about more or less
	universal causes, which Aquinas also explicitly identifies with more or
	less remote causes, he means not only that the causality of a more
	universal cause extends to more kinds, but also that the reason why its
	causality covers more kinds of effects is that it is causing them in a more
	universal respect: it has a power and a corresponding activity that can be
	received in so many different ways by different kinds of recipients, as the
	radiation of the sun received as heat in water powers the water cycle
	around the globe, while received in the chloroplasts of plants, it powers
	(most of) the biosphere \autocite[41]{Klima2013}
\end{quote}
The unintuitiveness of this claim itself provides a solid test case for the kind of work that Klima takes to be necessary for understanding medieval philosophy, 
and with it for understanding how its developments both presage and hint at ways out of our own persistent philosophical impasses. 
Within the pages preceding this selection, 
Klima outlines the notion of efficient causality largely taken for granted today as a diachronic relation holding between events, 
contrasts it with the medieval notion as a synchronic relation between individual things, 
and lays the groundwork for showing how the medieval notion is in certain ways closer to the scientific accounts used today in thermodynamics and information theory. 
The charity Klima aims for here is fundamentally higher than that typically afforded to this and other historically discarded theories - 
not merely to explain 
how it could have been believed given the information available and/or the psychological makeup of the inhabitants at the time, 
but to explain how such claims \emph{understood on their own terms} could themselves approximate the truth.\footnote{Cf. \autocite{Rovelli2015}.} 

\section{Conclusion}
Despite the depth and breadth of his work, 
the amount of space Klima devotes to advancing positions that are unambiguously his own, 
rather than to steel-manning positions of historical or contemporary figures 
he may or may not agree with,
is comparatively little.\footnote{Exceptions include his acceptance of both Anselm's proof of God's existence and Aquinas' proof of the immateriality of the intellect as sound \autocite{Klima2000,Klima2009a} 
and his advancing, 
based on an examination of Buridan's treatment of reciprocal liar paradoxes, 
that any adequate semantics for natural language must be semantically closed and token-based \autocite{Klima2004,Klima2008}.} 
Still, 
Klima's solution here provides a window into the answer to a more personal question that his scholarship solicits: 
namely, 
of all the intellectual pursuits to devote oneself to, 
why study medieval philosophy, 
and specifically medieval semantics?

In one uncommonly autobiographical passage, Klima writes: 
\begin{quote}
I remember that when I was at Notre Dame (so this happened in the second
half of the nineties), I asked several of my colleagues, and even the then
visiting David Armstrong, to provide metaphysically non-committal
clarifications of the semantics of the language they were using in
describing their metaphysical theories. In response, I was given puzzled
looks and declarations strongly reminiscent of the way medieval
nominalists characterized the attitude of their realist opponents: we don’t
care about names; we go right to the things themselves!—Well, just look 
at the history of late-scholasticism and early modern philosophy to see
what good that attitude did for them.

So, what can we do to avoid the late-scholastic scenario, going on another
cycle of endless and more and more meaningless metaphysical debates
until the arrival of another Kant declaring the whole enterprise ill-founded
and another Carnap declaring it to be meaningless, to launch another anti-metaphysical cycle of meaningless search for meaning to be abandoned
yet again for metaphysics, etc., etc.? Why don’t we try both in tandem,
i.e., analysis and metaphysics at the same time, as the very designation
“analytic metaphysics” would seem to demand? For then we could start by
laying down our clearly defined semantic principles (instead of making
them up and twisting them around as we go) and engage each other in our
metaphysical debates according to the same principles, instead of talking
past each other, making clear that whoever is talking according to different
semantic principles is just playing a different game \autocite[86-87]{Klima2014}.
\end{quote}

Here, the difficulty that Klima's apology for analysis aims to alleviate remains - 
namely, that in much debate in the core disciplines of analytic philosophy  
and in metaphysics in particular, 
rival participants are often unable or unwilling to state their positions in a linguistic context their opponents would be able to agree to, 
leaving such debates unfruitful from the start. 
%Against this, 
%the independence of metaphysical claims from the semantics in which they are expressed takes on the character not only of a metaphilosophical thesis, 
%but also of a moral one: 
Without the opportunity for common ground that semantics provides, 
not only shared understanding, 
but even proof, refutation, 
and disagreement itself become unattainable.

With this problem in mind, 
Klima's study of medieval semantics, 
as a study of frameworks of meaning remarkably foreign to that of our own time, 
provides an example \emph{par excellence} of the kind of interpretive charity needed to surmount our own crises of meaning and communication. 

\printbibliography

\end{document}
