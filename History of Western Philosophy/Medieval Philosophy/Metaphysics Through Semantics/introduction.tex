\documentclass[]{article}
\usepackage[backend=biber, style=authoryear-icomp]{biblatex}
\bibliography{../../../jacob}

%opening
\title{Introduction}
\author{Jacob Archambault}

\begin{document}

\maketitle

\begin{abstract}

\end{abstract}

\section{Introduction}
\section{Biographical introduction}
\section{Logic and semantics}
The breadth of Klima's scholarship stretches from debates on identity, categories, and causation in metaphysics, 
on skepticism in epistemology and theories of mental content in philosophy of mind, to others too numerous to mention; from some of philosophy's best-known figures in Anselm, Aquinas, Ockham and Descartes 
to lesser-known figures including Thomas of Sutton and Henry of Ghent, 
to Frege, Geach, Kenny, and others who have produced some of the most impactful scholarship in the analytic tradition.

Klima's most widely recognized contributions come in his research on John Buridan - which has helped elevate Buridan from a lesser-known figure to one whose stature is closer to that of an Ockham, arguably surpassing the Franciscan in his logic - and in the field of semantics. 
\subsection{Buridan}

\subsection{Semantics}
From his earliest work in semantics, 
Klima recognized that classical logic, 
being primarily interested in developing an account of the semantics of propositions as a precondition for the development of a theory of consequence, 
affords much less attention to its account of the components of propositions themselves, 
namely names and n-ary predicate relations. 
Klima fills this lacuna by providing a theory not only of simple, but also of complex terms
In this work, Klima provides some of the earliest and most ambitious applications of restricted quantification in the history and philosophy of logic, 
using it both to formalize the medieval theory of supposition and to provide a general account of quantitively ambiguous natural language sentences \autocite{Klima1988,Klima1990,KlimaSandu1990,Klima1991b}. 
Expansions on the same theme - 
namely, formalizations of supposition theory specifically and medieval semantics more broadly as a means to resolve apparently intractable interpretative problems in historical scholarship and debates in contemporary philosophy - 
provide us with an account of the semantics of intensional verbs \autocite{Klima1991}, 
a semantic foundation for Aquinas' theory of the analogy of being in his theory of the copula \autocite{Klima1996,Klima2002}, 
and a clean resolution of the problem of existential import in the Aristotelian square of opposition\autocite{Klima2001}. 

Elsewhere, Klima's work decouples \emph{via antiqua} and \emph{via moderna} semantics from the realist and anti-realist metaphysics with which they are most commonly paired with by showing 
that neither semantics by itself strictly entails its associated metaphysics; 
that archtypical realists were required to adopt non-straightforward semantic accounts of the meanings of terms in at least some cases by their antecedent metaphysical commitments (e.g. to divine simplicity),
and that some of the best known nominalist logicians incorporated what today would be regarded as `realist' elements in their logic \autocite{Klima1999,Klima2005,Klima2011}. 

None of this means that there is \emph{no} relationship between an author's positions in metaphysics and his semantics: 
rather, the semantic framework an author adopts conditions what options that author has in metaphysics without fully determining them: 
for example, extreme realism in metaphysics doesn't follow strictly from the \emph{via antiqua}'s inherence theory of predication, 
but it is easily the most natural fit for that theory if one accepts the view that terms signifying accidental being denote their referents rigidly while rejecting that framework's insistence on multiple, analogically related senses of `being' \autocite{Klima1999}; 
conversely, the broad outlines of Ockham's account of the relation between language, thought, and reality serves not only as a foundation for Ockham's own metaphysical reductionism, but also for the realism of a Descartes, Malebranche, Putnam or a Leibniz \autocite{Klima1991}.

In a particularly drastic example, the choice of a mistaken semantic framework may inhibit the speaker from constitutively referring to, and thus believing in, an actually existing God whose existence is only adequately assertable in an alternate framework (The Grammar of God, World, and Being, 22.).

In \autocite{Klima2008a}, Klima recognizes that representing \emph{via antiqua} semantics would require substantial modifications to modern quantification theory, while representing \emph{via moderna} semantics requires fewer modifications
1) Via antiqua semantics requires a different account of predication, and multiple copula to be introduced to represent the different senses of being
2) Via moderna semantics requires the introduction of restricted quantifiers.
3) both require the rejection of the object-metalanguage distinction.


Klima agrees with Buridan that the notion of truth is not strictly needed for a semantics concerned with formal validity, but it is needed to explain the semantics of sentences that themselves predicate that notion.
Analogy: nobody complains that we don't have a formal definition of the term `red' in our logic, even though a basic grasp of the semantics of that term is needed for using the term in sentences about red things.

Two uses for semantics of truth: 1) as part of a theory of validity, 2) for its own sake.
%The efficient causality model fixes the relation of natural signification on the basis of natural laws systematically connecting causes with their effects. But all that these natural laws guarantee is the systematic correspondence between causes and their effects, supposing the normal course of nature. The effects, however, may be essentially different from their causes, and may, therefore, be produced also by other essentially different causes too, which means that it is clearly possible that an absolute concept be caused in our intellects by a cause which is totally alien from what this concept is supposed to represent. - Ontological Alternatives.
%IMPORTANT!, the immediately above quote is self-defeating for Klima, since his remarks on the relation between semantics and metaphysics are themselves conceived on the efficient, rather than formal, causality model mentioned above.
In `Logic without truth', Klima rejects Buridan's solution to the liar paradox.

`Buridan’s nominalism is obtainable by the adverbialization of Peter
of Spain’s semantics.'
`Nominalism is obtainable by the adverbialization of realist
semantics.'
`Medieval realism and nominalism are just different versions of conceptualism, differing especially in how they handle the problems of
describing and identifying mental content.'\autocite[110]{Klima2011}

why the study of
Buridan (or the history of philosophy in general) can be philosophically so rewarding: this
study can put our own philosophical problems in an entirely different light, providing us with
such theoretical perspectives that otherwise might entirely escape us as we are working in our
set ways determined by the intellectual habits of our philosophical period, which in modern
times tends to stretch to a mere couple of decades. - Klima2005b - Quine, Wyman, Buridan, p. 17.

%In this paper I will attempt to dig further to the roots of
their disagreements, trying to establish those primary logical-semantic differences that may have
motivated their conflicting intuitions concerning these metaphysical principles. - Thomas of Sutton v. Henry of Ghent
%To be sure, this is not to say that metaphysical principles are to be derived from, or somehow justified
in a weaker sense on the basis of, semantic principles. Metaphysical principles, being first principles
using the most general terms, such as the transcendentals and the categories, cannot be derived from
prior principles, and their terms cannot be defined on the basis of more general terms. What
semantics can do, however, is that it can provide the principles of interpretation of metaphysical
principles. On the basis of these principles of interpretation the implications of metaphysical
principles are more clearly delineated, which then can be used in their evaluation in dialectical
disputations concerning their acceptability in the interpretations thus clarified. Furthermore, if the
semantic principles of interpretation are made explicit, they can also be subject to further evaluation,
in a disputation on a different level, the sort appropriate to the comparison of different logical
theories. 3

%The Rises and Falls of Analysis and Metaphysics in Metaphysical Themes, Medieval and Modern - highly autobiographical. Helps to explain his aversion to the idea of metaphysics bleeding into semantics


%Therefore, it should also be clear that the laws of logic in this framework
are supposed to be fundamentally different from the laws of psychology.
For while the former are the laws of the logical relations among objective
concepts, the latter are the laws of the causal relations among formal
concepts. - The problem of universals and the subject matter of logic, Klima 2014, p. 173.

These different theories can be arranged on a `theoretical scale', ranging from extreme realism to extreme nominalism, meaning maximal semantic uniformity along with maximal ontological diversity on the realist end [...], and maximal ontological uniformity with maximal semantic diversity on the nominalist end.- The problem of universals and the subject matter of logic, Klima 2014, p. 176.

%Note. Klima can't get away from the notion of semantics as functions for mapping language to reality. This is part of why he can't accept the idea of metaphysics impinging on logic itself. But this is exactly what the theory of the different main types of supposition, grounded in a theory of analogy, should entail.

Well, conceptual diversity is obviously
a great hindrance to understanding: if we don’t have the same concepts, we can-
not have the same thoughts, which means we are doomed to talking past each
other all the time - Klima 2021 Words and what is beyond, p. 36.

So, what should be our guiding light, in this rational discourse? In one word:
rationality, which is love or goodwill on its active side, on the part of the will, and
understanding on its receptive, theoretical side, on the part of the intellect. -Klima 2021, p. 41
\autocite{Parsons2014,Read2015b}

\subsubsection{inference}


`The primary purpose of a logical semantic theory is to define logical consequence in terms of the truth values of propositions in different interpretations' \autocite[79]{Klima1991b}. 

(1) Natural languages are semantically closed (2) Natural language inference has to be token-based.
Both Klima/Buridan and Tarski come to the conclusion that defining consequence in terms of truth and falsity doesn't work from similar considerations: Tarski's consideration is related to superenumable domains and the possibility that a language may simply an appropriate selection of denoting terms; Klima/Buridan's considerations come from the possibility that a claim may be not exist to even be true or false, or it may be self-falsifying while nevertheless describing a possible state of affairs \autocite[96]{Klima2004}.

The primary impetus behind Klima's work is one of charity.
Examples: 
Positive:
1. His analysis of parasitic reference in his work on Anselm
2. His attempts to translate between via antiqua and via moderna semantics
3. The entirety of his body of work on John Buridan
Negative:
1. The infrequency with which Klima actually reveals his own philosophical positions in his work (exceptions: 
Aquinas' hylomorphism and proof of immortality, 
Anselm's proof, 
Per Buridan, the semantic closure and token-based character of natural language inference)
2. His adopting semantics that build on classical logic while rejecting non-classical semantics.


%Metaphysical questions are undecidable by semantics alone. In the same sense in which the body of truths of revealed theology are undecidable by natural philosophy, or the continuum hypothesis is undecidable by set theory.
%The rises and falls of analysis and metaphysics
%Being, unity, and identity in the Fregean and Aristotelian traditions

%metaphysics \textit{does} determine semantics. Consequently, the attempt to infer metaphysics from semantics deductively, as with all similar cases of causality, would logically have to be a case of affirming the consequent. 
%demonstratio quia
\subsubsection{Why semantics?}
Semantics: 
1) a theory of meaning broadly construed
2) Tarskian/Montaguean mathy stuff
	a semantics is almost never actually this, given that most semantics have a canonical interpretation and a domain to which they are expected to apply (e.g. Model theory handles solids better than liquids or gases).
3) e.g. a dictionary
4) e.g. proof-theoretic semantics
5) a philosophy of language
6) a theory of language, thought, and reality
%Two contradicting claims: 1) The modern turn in metaphysics was conditioned by a prior one in semantics; 2) neither via antiqua nor via moderna semantics strictly entail the metaphysical positions they are usually associated with.

%Two phenomena: 1) the breakdown of communication within analytic philosophy 2) its lack of breadth

%Ontological alternatives vs. alternative semantics - recognition that categories like realism, nominalism, and conceptualism are a poor fit for medieval philosophy generally - The broad outlines of Ockham's account of the relation between language, thought, and reality serves as a foundation for the realism of a Descartes, Malebranche, or a Leibniz, the skepticism of a Hume, or the idealism of a Kant or Hegel.
%Buridan's logic and the ontology of modes - neither via antiqua nor via moderna semantics entailed realism or reductionist metaphysics, respectively.
\section{Philosophy of mind and epistemology}
\subsection{\textit{Via antiqua} and \textit{via moderna} cognizers}
\section{Metaphysics}
\subsection{Hylomorphism, personal identity and immortality}
\subsection{Causation}
\section{Overview of the articles}

\printbibliography

\end{document}
