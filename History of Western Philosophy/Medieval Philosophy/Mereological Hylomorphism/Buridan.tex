\documentclass[]{article}
\usepackage[backend=biber, style=verbose-ibid]{biblatex}
\bibliography{jacob}
\usepackage{amssymb}
%opening
\title{Mereological hylomorphism and the development of the medieval substitutional account of formal consequence}
\author{Jacob Archambault}

\begin{document}

\maketitle

\section{Introduction}
This paper contributes to an explanation of how a sea change in logic from the later 13th to the early 14th century relates to the dialectic of certain metaphysical debates during the same period. The changes we refer to are:
\begin{enumerate}
	\item The development of the concept of \textit{consequence}, first as a replacement for the framework of Aristotle's \textit{Topics}, then as a broader category under which syllogistic, too, was subsumed. 
	\item The division of consequences into natural and accidental, and the later replacement of this division by that into formal and material varieties.
	\item The accompanying growth of hylomorphism, and particularly \textit{mereological} hylomorphism, in thinking about consequence.
\end{enumerate}

Both the development of systematic treatises on consequence and the application of hylomorphic language to consequences first take place in the later middle ages, culminating in the work of the Parisian arts master John Buridan. Buridan is not the first Latin author to implicitly appeal to or systematically explicate a distinction between formal and material consequence;\footnote{The former honor appears to belong to Simon of Faversham; the latter, to William of Ockham. See \autocite[q. 36]{FavershamQE}; \autocite[III-3. 1, p. 589]{OckhamSL}.} but it is Buridan's version of the distinction that becomes widely adopted throughout continental Europe in the later middle ages, ultimately laying the groundwork for today's model-theoretic accounts of formal consequence.

After setting forth the requisite definitions, I introduce Buridan's distinction between formal and material consequence as presented in his \textit{Treatise on Consequences}: first giving the generic structure of the distinction, then placing the distinction in its historical development. I then introduce Buridan's metaphysical views: focusing on the question of whether several substantial forms may be present in the same subject; drawing from Buridan's questions on Aristotle's \textit{Physics} and \textit{De Anima}; and contrasting the views found there with the unicity theory of Aquinas. I then place Buridan's logic specifically, and its development more broadly, against the backdrop of these developments in metaphysics over the same period. We find that Buridan's \textit{logical} hylomorphism reduplicates the peculiarities of his \textit{physical} hylomorphism. In particular, both Buridan's logical and physical hylomorphism are mereological in character, i.e. they both identify the form of a being (or consequence) with a proper part of the being itself - an identification explicitly rejected by Aristotle. Buridan himself subsumes the distinction between the formal and material under that between the natural and accidental in his commentaries on Aristotle's physical treatises. This subsumption may help explain how the distinction between formal and material consequence ultimately supplants that, found in earlier treatises on consequences, between natural and accidental consequences. We conclude that while Buridan's mereological hylomorphism in physics did not necessitate his logical hylomorphism, it did help facilitate it.
\section{Definitions}
Hylomorphism is a thesis in metaphysics originating with Aristotle, according to which terrestrial objects are composites of matter and form. Matter is what an object is made \textit{out of}; form, what it is made \textit{into}. A human being is made almost wholly out of the elements of carbon, oxygen, hydrogen, nitrogen, calcium, and phosphorus. But a human is not these things alone - a corpse is all these things, but no longer a human - but these things conformed into the mould of humanity.
	
Mereological hylomorphism is a kind of hylomorphism according to which matter and form are distinct, proper, and integral parts of a hylomorphic compound. Parts are distinct when they contain no part common to each; proper, when they are not identical with the whole itself; integral, when they are parts in the way the hand or heart is part of the human body. Aristotle himself explicitly rejects this form of hylomorphism.\autocite[Bk. VII. 10, pp. 1034b 33-36a 25]{Metaph}

In its most restricted sense, `logical hylomorphism' refers to the identification of logic \textit{per se} with \textit{formal} logic, where the latter concerns the behavior of certain parts of sentences and arguments deemed 'formal'. In a broader sense, `logical hylomorphism' refers to the application of the concepts of matter and form to logical argument.\footnote{The former understanding appears in \autocite{MacFarlane2000}, \autocite{BeallRestall2006}, and \autocite{Griffiths2013}; the latter, in \autocite{Brumberg-Chaumont2015}. Both are present in \autocite{DutilhNovaes2011,DutilhNovaes2012b,DutilhNovaes2012c}, albeit with greater focus on the former understanding.} Logical hylomorphism is mereological when this application is grounded in a partition of elements in a language into formal and non-formal types. Later medieval logic tends to be hylomorphic only in the second sense. To a greater or lesser degree, the philosophical logic of Tarski and his successors tends to be hylomorphic in both senses.\footnote{Cf. \autocite{Tarski2002,Tarski1986,Sher1991,Ray1996,Gomez-Torrente2000,Bellotti2003,BeallRestall2006,MacFarlane2009}.}

A consequence is a relation of following between objects of an appropriate type. Though in common parlance, consequences may relate actions, objects, or events, in academic discourse the relata of consequences are usually restricted to statements or sets (or multisets, or lists ...) of statements. What follows is called the \textit{consequent}; what it follows, the \textit{antecedent}.

A formal consequence is one valid in virtue of its formal parts; more specifically, a formal consequence is one for which every substitution of like terms by like in its non-formal parts remains a valid consequence. A material consequence is one valid in some other way. The formal parts of a consequence are identified with the logical vocabulary occurring in its sentential parts. In natural language, these include words like `every' and `some', `possible' and `necessary', `is' and `isn't', `if', `and', and 'or'; in artificial languages, their symbolic analogues $\forall$ and $\exists$, $\diamondsuit$ and $\square$, $=$ and $\neg$, $\rightarrow$, $\wedge$, and $\vee$. In modern logic, the (non-)formal parts of a consequence are also called its (non-)logical parts; in medieval logic, the non-formal parts are called \textit{categorematic} terms; the formal parts, \textit{syncategoremata}. Writing in the latter half of the thirteenth century, Nicholas of Paris explains the distinction thus:
\begin{quote}
	As the Philosopher states [Physics II, 2; 194a 21-27], those which are in art and reasoning are taken up in proportion with and imitation of those which are in nature. Now among natural [beings] we thus see that there are some which by nature are disposed in themselves to do something without any outside support, while there are others which are not disposed to move except [when] moved: just as a man moved by himself and not by another writes letters, so a pen is moved not by itself but by a man. It is similar among beings of reason, especially among words: some perform their purpose without the aid of another, i.e. they signify (for every word is for the sake of something to be signified, since, as Aristotle states [De interpretatione I; 16a 3-4], words are indicators of those impressions which are in the soul: that is, they signify ideas which are signs of things; and in this way utterances signify things), and such utterances are called \textit{categoremata}, that is, signifying; there are others which do not signify per se but in connection with others, and these are called \textit{syncategoremata}.\footnote{\autocite[I. 2-15]{Braakhuis1979} \begin{quote}
			Ut dicit Philosophus [Physics II, 2; 194a 21-27], ea que sunt in arte et ratione sumuntur ad proportionem et imitationem eorum que sunt in natura. In naturalibus vero ita videmus quod sunt quedam que per naturam nata sunt in se aliquid agere sine alieno suffragio, alia vero sunt que non sunt nata movere nisi mota, sicut homo a se motus et non ab alio protrahit litteras, calamus non a se sed ab homine motus. Similiter se habet in rebus rationis, maxime in vocibus, quod quedam faciunt id ad quod sunt sine auxilio alterius, scilicet significant, quia omnis vox est ad significandum, quoniam, ut dicit Aristotiles [De interpretatione I; 16a 3-4], voces sunt notae earum que sunt in anima passionum, idest significant intellectus, qui sunt signa rerum; et ita voces significant res; et tales voces dicuntur categoreumata, idest: significantes; alie sunt que per se non significant sed in coniunctione ad alias; et tales dicuntur sincategoreumata.
		\end{quote}}
\end{quote}
\section{Logic}
\subsection{John Buridan's account of formal consequence}
The earliest account of consequence to explicitly distinguish formal from material consequences by appeal to the distinction between categorematic and syncategorematic terms belongs to the fourteenth century Parisian logician John Buridan. Buridan distinguishes formal from material consequences thus:

\begin{quote}
	A consequence is called `formal' if it is valid in all terms retaining a similar form. Or, if you want to put it explicitly, a formal consequence is one where every proposition similar in form that might be formed would be a good consequence.
	
	...A material consequence, however, is one where not every proposition similar in form would be a good consequence, or, as it is commonly put, which does not hold in all terms retaining the same form.\footnote{\autocite[I. 4, p. 68]{Buridan2015}: 
		\begin{quote}
Consequentia `formalis' vocatur quae in omnibus terminis valet retenta forma consimili. Vel si vis expresse loqui de vi sermonis, consequentia formalis est cui omnis propositio similis in forma quae formaretur esset bona consequentia [...] Sed consequentia materialis est cui non omnis propositio consimilis in forma esset bona consequentia, vel, sicut communiter dicitur, quae non tenet in omnibus terminis forma consimili retenta.
		\end{quote}}
\end{quote}

Since Buridan's logic countenances non-formal consequences, it is not hylomorphic in the strict sense. Since it appeals to the distinction between form and matter, it is hylomorphic in the broader sense. Furthermore, since it relies on a distinction between formal and material parts to determine the logical form of a sentence, it satisfies the conditions for a specifically mereological account. Buridan describes the form and matter of consequences thus:

\begin{quote}
	I say that when we speak of matter and form, by the matter of a proposition or consequence we mean the purely categorematic terms, namely the subject and predicate, setting aside the syncategoremes attached to them by which they are [1] conjoined [2] or denied [3] or distributed [4] or given a certain kind of supposition; we say all the rest pertains to form \autocite[I. 7, p. 74]{Buridan2015}.
\end{quote}
Thus on Buridan's account, a consequence is formal if and only if it is good for all uniform substitutions on its categorematic terms. For this reason, it is rightly regarded as a medieval predecessor to the model-theoretic approach to formal consequence found in the work of Alfred Tarski,\autocite{DutilhNovaes2012a} which transforms Buridan's uniform substitution criterion on terms into one over \textit{models}, i.e. over orderings of objects satisfying sentential functions obtained from initial sentences by substituting like non-logical constants with like variables.\footnote{Note that Tarski's understanding of `model' is not the same as that found in model-theory today. For examination of the concept as it appears in Tarski, see \autocite{Bays2001,Gomez-Torrente2009}.}

\subsection{The development of the Buridanian account}
The earliest extant treatises on consequences arise at the turn of the fourteenth century. Of these, two are anonymous, while a third is by Walter Burley, the English `extreme realist' and prominent foe of William of Ockham who later wrote his treatises \textit{On the Essence of the Art of Logic} at Paris in the mid-to-late 1320s.\footnote{The anonymous treatises are edited in \autocite{Green-Pedersen1980a}; Burley's, in \autocite{Green-Pedersen1980b}. English translations of these are found in \autocite{Archambault2017d}. Burley's \textit{De Puritate} is edited in \autocite{BurleyDPAL}, and translated in \autocite{Burley2000}.}

Prior to Buridan's account, one finds two different divisions of consequences. The earlier, that between natural and accidental consequences, is already implicit in Boethius \autocite[835B]{BHS},and later found in Garlandus Compotista, William of Sherwood, Duns Scotus, and Walter Burley.\footnote{\autocite[80]{Sherwood1941}, \autocite[141]{Garlandus1959}, \autocite[I, d. 11, q. 2, pp. 136-137]{ScotusLectura}, \autocite[128-129]{Green-Pedersen1980b}.} The other, that between formal and material consequences, is found in Simon of Faversham's \textit{Questions on Aristotle's On Sophistical Refutations}, implicitly appealed to in the earliest extant treatise on consequences, and first explicated by William of Ockham.\footnote{\autocite[qq. 36-37]{FavershamQE}, \autocite[7, par. 18]{Green-Pedersen1980a}, \autocite[III-3. 1, p. 589]{OckhamSL}.} The relations between these divisions can be garnered to a small degree from Ockham's mention of natural consequence in his discussion of \textit{positio impossibilis}, a debating exercise (\textit{obligatio}) discussed in the \textit{Summa Logicae} section on consequences involving the positing of an impossible proposition;\autocite[III-3. 42]{OckhamSL} and to a greater degree from an exchange in Burley's \textit{De Puritate} where both divisions are used.\autocite[80.13-29, 84.8-86.21]{BurleyDPAL} Though the earlier and later distinctions existed side by side for a short time, the formal/material distinction eventually came to replace the natural/accidental one.

According to the earlier distinction, a natural consequence is one where the antecedent is contained in the consequent, and holds by appeal to an intrinsic topic; an accidental consequence is one which fails to satisfy this containment criterion, but holds in some other way. `Topic' is said in two ways. In the former sense, it refers to some general feature of a thing, appealed to, often as the middle term of a syllogism, to arrive at new knowledge concerning its bearer or something else; in the second sense, it refers to the rule appealed to in an inference pertinent to such a feature. Inferences appealing to intrinsic topics include those from a genus to a species and conversely, from something about a part to something about the whole it is a part of, from ascribing something to a thing under a certain description to ascribing it to the thing directly, from the positing of the cause to the positing of its effect, and others. Consequences invoking extrinsic topics are liable to appeal to more `synthetic' relations, and include inferences involving analogy, geometrical similarity, contrariety and other kinds of opposition, and appeals to authority \autocite[Bk. II]{BDT}. Natural consequences are always relevant consequences; accidental consequences need not be. Burley explains the division thus:

\begin{quote}
A natural consequence is when the consequent is in the understanding of the antecedent, nor can the antecedent be true unless the consequent is true; as `if a man is, an animal is'. An accidental consequence is twofold: some hold on account of the terms or on account of the matter, as `God exists is true, therefore God exists is necessary'; and this holds on account of the terms or the matter, since truth in God and necessity are the same. Other accidental consequences are so: from the impossible anything follows; and the necessary follows from anything. An example of the first: `You are an ass, therefore you are a goat, and a stone...', etc. An example of the second, as `you are running, therefore God exists'. Again, an accidental consequence is when the consequent is not in the understanding of the antecedent \autocite[128-129, par. 70]{Green-Pedersen1980b}.
\end{quote}

In Burley and the early anonymous tracts, the distinction between natural and accidental predication is intended as extending a distinction between \textit{per se} and accidental \textit{predication}, as in `a man is an animal' and `a man is a musician', respectively. There were difficulties in the implementation of the expansion, particularly i) the conflation of accidental predication and essential predication between accidents (e.g. `to run is to move'), and ii) irregularities in the treatment of existential import.\autocite[76-79]{Archambault2017d}

In Ockham's account, the formal/material division of consequences is determined by whether or not a consequence holds by a \textit{middle}: formal consequences hold by middles, material consequences do not. Further, a formal consequence may hold only by an extrinsic middle, Ockham's example being `From a necessary major premise and an assertoric minor follows a necessary conclusion'; or by both an intrinsic and extrinsic middle, as in `Socrates does not run, therefore a man does not run'. Here, an extrinsic middle is a rule licensing an inference; an intrinsic middle, a premise on which the antecedent implicitly depends, `Socrates is a man' in this case. Ockham's examples of material consequences are those from an impossibility or to a necessity: `If a man runs, God exists' and `A man is an ass, therefore God does not exist'.\autocite[III- 3. 1, p. 589]{OckhamSL} Ockham's later \textit{Elementarium Logicae} gives the consequence `an animal is debating, therefore a man is debating' as another example of a good material consequence.\footnote{\autocite[VI. 4, p. 163]{OckhamEL}. For the case for the authenticity of the \textit{Elementarium}, see \autocite{Boehner1958b}.} Simon of Faversham's early example of a material consequence is similar to that in the \textit{Elementarium}.\footnote{\autocite[q. 36, p. 200]{FavershamQE}: 
\begin{quote}
When it is said that `an animal is a substance; therefore a man is a substance' is a good consequence, I reply that this consequence does not hold in virtue of form, but rather in virtue of matter. Because according to the Commentator on the first book of the Physics, an argument which is valid in virtue of form must hold in all matter. This consequence, however, holds only for features which are essential ... and so this consequence is not formal.

Et cum dicitur `Hic est bona consequentia: `animal est substantia, ergo homo est substantia'', dico quod ista consequentia non tenet ratione formae, sed ratione materiae. Non tenet ratione formae quia secundum Commentatorem I Physicorum sermo concludens virtute formae debet tenere in omni materia; ista autem consequentia tantum tenet in essentialibus, et hoc propter identitatem naturae importatam in talibus per antecedens et consequens; et propter hoc consequentia ista non est formalis.
\end{quote} Translation from \autocite[135]{Martin2004}.}

Not all formally valid consequences in Ockham's sense are formally valid in Buridan's sense. Rather, Buridan regards as formally valid only those formal consequences Ockham says hold solely by an extrinsic middle. Furthermore, Scotus and Burley's containment criterion for natural consequence is orthogonal to Buridan's uniform substitution criterion for formal consequence. 

Thus, we have two transformations: first, that from the natural/accidental distinction to the formal/material distinction; then later, from a topical to a substitutional grounding of the latter distinction, with a corresponding restriction of the class of formal consequences.

To better understand the `logic' of these logical developments, let us now embark on a short metaphysical detour.
\section{Hylomorphism}
\subsection{\textit{Parvus error in principio}...}
After briefly discussing the meaning of \textit{ens},\autocite{AquinasDEE} Thomas Aquinas' \textit{de ente et essentia} then transitions to a discussion of essence:  treating first the names of essence\footnote{Namely, form, quiddity, and nature.} and their origin,\footnote{Beginning at `et quia illud per quod res constuitur...'}  then distinguishing the way essence is in substances and accidents.\footnote{At `Sed quia ens absolute et per prius dicitur de substantiis...'}  He then divides substances into simple and composite,\footnote{`Substantiarum vero quaedam sunt simplices et quaedam compositae'}  and proceeds to treat the way essences exist in composite material substances.\footnote{`In substantiis igitur compositis...'}
	
Aquinas' discussion of material substance begins with a refutation of three successive positions: that essence is identical with matter,\footnote{`quod enim materia sola non sit essentia rei planum est...'}  that it is identical with form,\footnote{`Neque etiam forma tantum essentia substantiae compositae dici potest, quamvis hoc quidam asserere conentur'}  and that it is something added to or other than matter and/or form.\footnote{`Non autem potest dici quod essentia significet relationem quae est inter materiam et formam vel aliquid superadditum ipsis'}  In other words, Aquinas first proves that matter and form are \textit{necessary} for a definition of the essence of a material thing, then that they are \textit{sufficient} for it. This position is confirmed by the authority of Boethius,\footnote{`Et huic consonat verbum Boethii...'}  Avicenna,\footnote{Avicenna etiam dicit...'}  and Averroes,\footnote{Commentator etiam dicit super VII metaphysicae...'}  and then by a direct proof from reason.\footnote{Huic etiam ratio concordat...'}

Aquinas then writes the following:
\begin{quote}
But since matter is the principle of individuation, from this it may seem to follow that essence, which contains within itself both matter and form, is of particulars, and not universals; from which it would follow that universals would not have definitions, if essence is that which is signified by a definition.\footnote{\autocite[ch. 1]{AquinasDEE}:
	\begin{quote}
	Sed quia individuationis principium materia est, ex hoc forte videtur sequi quod essentia quae materiam in se complectitur simul et formam, sit tantum particularis et non universalis. Ex quo sequeretur quod universalia diffinitionem non haberent, si essentia est id quod per diffinitionem significatur.
	\end{quote}} 
\end{quote}
This is the first place in the text where Aquinas puts forth an objection to his own position, and it is the only place where an argument contradicting Aquinas is phrased \textit{as} an objection.\footnote{Later in chapter 1, Aquinas will allude to another argument presumably from the same point of view (beginning at the words, `Quamvis autem genus'), and in chapter three he will address the thesis of Ibn Gabirol concerning spiritual matter at length. But in both of these cases the rejection of the thesis is simultaneous with its introduction.} The remainder of the first chapter engages in a refutation of the underlying position, and defense of the position that Aquinas contrasts with it.\footnote{Aquinas' defense of the position he sets out lasts at least until the sentence beginning `Sed diffinitio vel species comprehendit utrumque' \autocite[ch. 1]{AquinasDEE}.}

The argument above can be summarized as the following \textit{reductio}. If matter is the principle of individuation, then whatever contains matter is individuated. But whatever is individuated is particular, and not universal. Therefore, whatever contains matter is particular, and not universal. But by hypothesis, the essence of a composite contains both matter and form. Therefore, essences of composites would be particular, not universal. But an essence is what is signified by a definition. Therefore the definition of composites would signify particulars. But the definition of a composite cannot signify a particular, since particulars are indefinable. Thus, universals corresponding to composite entities, which are only intelligible insofar as they are separated from matter,\footnote{Aquinas makes this claim in the beginning of chapter 3 of the \textit{De ente}. The argument goes through without it, but the presumption that Aquinas' opponent would also hold this belief is supported both by the strength it lends to the argument itself and by the relatively high level of acceptability Aquinas assumes for this premise when he himself uses it.}  would lack definition. But this is false.  

Aquinas' response to the objection begins by further specifying how matter individuates,\footnote{`Et ideo sciendum est quod materia non quolibet modo accepta est individuationis principium, sed solum materia signata' (ibid).}  thus implying that his opponent misconstrues this claim rather than rejects it outright. Given this, Aquinas likely intends the \textit{reductio} to be directed against the thesis that essence embraces both matter and form. Given that the problem in the argument arises from the inclusion of matter in essence, Aquinas' imagined interlocutor cannot hold that essence is identical to matter, nor that it is some relation between or superaddition to matter and form. Therefore, Aquinas likely has in mind an opponent who identifies essence and form. Call the position stated the \textit{form-essence identity thesis}.\footnote{The likelihood of this is bolstered by the presence of `form' on Aquinas' list of names of essence (\textit{`Dicitur etiam forma ...'}); by the explicit recognition, unique among the positions mentioned, of the thesis as a live option advanced by others (`... quamvis hoc quidem asserere conentur'); by the presence, again unique among the three suggested alternatives, of a counterargument to Aquinas' initial rejection: after Aquinas argues that the exclusion of matter from essence would entail that mathematical and natural definitions did not differ from each other, he revises his opponent's position by suggesting that matter is included in the definition of substance as something added to its essence, which revision he then also refutes. This is then followed by a concession to the form-essence identity theorist - `even though the form alone is by its mode the cause of the \textit{esse} of this kind [of being (i.e. a composite)]' -  supported by an analogy to the composite character of tastes: like tastes, an essence i) contains multiple parts, ii) takes its name from only one of those parts, and iii) has only one of those parts, properly speaking, as a cause.}	
	
Aquinas' response is concerned to show how it is possible for matter to be included in essence.  Aquinas' suggests that not prime, but signate matter is the principle of individuation.\footnote{\textit{Pace} \autocite[371]{Lagerlund2004}. Thus, the premise `whatever contains matter is individuated' is ambiguous, blocking the straightforward conclusion `whatever contains matter is particular,' as well as any incongruous conclusions drawn from it.} This is then followed by an exposition of the meaning of `body', and then an analogous shorter exposition of the meaning of `animal'. In each case, Aquinas is at pains to show that predicates such as those in `a tree is a body', `a tiger is an animal', `Plato is rational', `Socrates is human' are not predicating a part of a thing to the integral being to which it belongs -as would be the case in the statement `Socrates is human' if `human' signified only Socrates' form, and not his matter - but instead signify the whole of the subject they are predicated of, albeit in different ways.\footnote{`Sic ergo: genus significat inderminate totum id quod est in specie, non enim significat tantum materiam; similiter etiam differentia significat totum et non significat tantum formam; et etiam diffinitio significat totum, et etiam species' (ibid).}

%\subsection{Error medius}
From here, let us consider an objection alluded to later in chapter 1:
\begin{quote}
Although a genus signifies the whole essence of a species, it does not follow from this that diverse species of which the genus is the same are one essence. For the unity of a genus arises from its very indeterminacy or indifference: not such that that which is signified by a genus be one nature numerically in diverse species, upon which supervenes another thing which is the difference determining it, just as form determines matter which is one numerically; but such that a genus signifies some form, neither determinately the same as nor other than what [its] difference determinately expresses, which [difference] is [itself] nothing other than that which was indeterminately signified by the genus.\footnote{\autocite[ch. 1]{AquinasDEE}: 
	\begin{quote}
	Quamvis autem genus significet totam essentiam speciei, non tamen oportet ut diversarum specierum, quarum est idem genus, sit una essentia, quia unitas generis ex ipsa indeterminatione vel indifferentia procedit: non autem ita, quod illud quod significatur per genus sit una natura numero in diversis speciebus, cui superveniat res alia, quae sit differentia determinans ipsum, sicut forma determinat materiam, quae est una numero; sed quia genus significat aliquam formam, non tamen determinate hanc vel illam quam determinate differentia exprimit, quae non est alia quam illa quae indeterminate significabatur per genus.
	\end{quote}}
\end{quote}	
	
Aquinas' denial of the validity of the inference from i) `a genus signifies the whole essence of a species'; to ii) `diverse species of which the genus is the same are one essence', suggests the affirmation of that connection by an opponent. Similarly, Aquinas' mention of the thesis `that which is signified by a genus is one nature numerically in diverse species' suggests an affirmation of that thesis by his target. Lastly, the stress that Aquinas places on the indeterminate signification of genus contrasts with the determinacy of all signification insisted on by the position with which he is contrasting his own. This, then, suggests the following line of reasoning.

The whole essence of a species is signified by its genus. That which is signified by a genus is one nature numerically in diverse species. But to say that the whole essence of a species is signified by a genus is to say that that species is wholly its genus, which is absurd. For to signify is to designate determinately. Now if every essence in a genus is divided into different species by a difference, and this is only possible if that entity is not wholly determined by its genus, then if a genus signified the whole of an essence, it would follow that that essence was incapable of numerical division into further species, and therefore these diverse species would be themselves one essentially: for instance, to be human would be identical to being a cow, since both are species of the same genus `animal'. Therefore, to avoid this sort of `crowding out' of the differences between species, it is necessary to posit that the genus only signifies part of the essence of a species.

The essence of a species, then, would likely be nothing other than the metaphysical summing of its difference, genus, and any other higher genus to which it belonged; while the entity would be these in addition to individuating matter. Put otherwise, we can state the thesis as the following error: the essence of a species is the combination of the plurality of substantial forms of the entities belonging to that species, abstracted from individuating matter.\footnote{An alternative position might identify the essence of an entity not with the combination of its substantial forms, but exclusively with the highest of those forms, leading to a view reminiscent of (but not identical to) the Cartesian identification of the self with thought. Cf. \autocite{Lagerlund2004}.}

The thesis that multiple substantial forms may inhere in a single substance was widespread in the later thirteenth and early fourteenth century. Popularly associated with Solomon Ibn Gabirol,\footcite[Bk. 2, ch. 8, 37-39; Bk. 3, ch. 3, par. 22, 81]{Avicebron} though already present in Alexander of Aphrodisias,\footcite[Bk. 1, ch. 13, 9]{AlexanderDA} its later advocates include John Pecham, Richard of Mediavilla,\footnote{\autocite[255]{Weisheipl1980}, \autocite{Zavalloni1951}.} Bonaventure,\footcite[index]{Quinn1973} Duns Scotus,\footnote{\autocite[490-497]{Gilson1952}, \autocite[187-229]{Stella1955} \autocite[47-76]{Cross1998}. The traditional reading of Scotus's hylomorphism has been contested in \autocite{Ward2012,Ward2014}. On Ward's reading, mediating forms such as those of corporeity or sensitivity are not admitted on Scotus' hylomorphic pluralism: rather, only the forms of organs are admitted as substantial forms, in addition to the unifying substantial form of the whole. Wards reading, however, is pieced together from a small number of non-central texts; explains away Scotus' use of key phrases associated with the traditional position as figures of speech; and arguably is based on a misreading of one of its central supporting texts. See \autocite[542-545; 548, fn. 58]{Ward2012}; \autocite[772-773]{Pini2016}. In this article, I assume the traditional reading of Scotus' hylomorphism.} and William of Ockham.\footcite[633-670]{Adams1987} The acceptance of a plurality of substantial forms in a single substance was often coupled with a number of other theses.\footnote{\autocite[242-243]{Weisheipl1980}, \autocite[vol. 1, p. 335]{DeWulf1926}.} For our purposes, the most important of these shall be the identity posited, typically in discussions on the soul, between a substantial form and its powers.

The case for the plurality of substantial forms depends on the form-essence identity thesis as follows. One starts with the assumption that matter individuates, without the qualification that signate matter individuates. This then requires that essences not contain matter. Given the association of matter with individuation - that is, with the division of a species into separate individuals\footnote{For this understanding of the problem of individuation, see the essays in \autocite[38-78]{KlimaHall2005}} - form becomes straightforwardly associated with unity: Socrates and Plato are both human because they both partake of the numerically identical form of human. That different substantial forms would, in addition, be separate from each other, and not merely from the supposita which partake of them, would follow from their determinacy and numerical self-identity.
	
In metaphysics, this implies that the positing of a multiplicity of substantial forms is a specification of the form-essence identity thesis; and that according to this thesis, a form is conceived of as having the same sort of unity found in entities, i.e. determinate unity, instead of being conceived of, as Aquinas thinks it should, as having the indeterminate unity appropriate to essences.
	
In logic, this suggests an analysis on which for any entity $x$ to be $F$ is for it to contain the form $F$ as one of its metaphysical parts. Furthermore, since the forms are of their own nature determinately separate from one another, `is' statements predicated of different forms, such as `man is an animal', if taken as statements about forms, must be false. For if one insists i) that every form is identically itself and not any other, and ii) that every form is completely determined, then to posit that `man is an animal' as a statement about forms would be to equate humanity and animality. Therefore, statements like `man is an animal' would need to be analyzed as concerning not containment relations between the significates of essential predicates, but rather the coincidence or non-coincidence of those essences in their supposita.\footnote{Indeed, Aquinas makes precisely this point in the \textit{responsio} of Question 11 of his \textit{Quaestiones de anima}, that statements like `man is an animal' would only be true \textit{per accidens} on the analysis given by adherents to the plurality of substantial forms. \autocite[q. 1, a. 11, co]{AquinasDA}.}

Aquinas' opponents are thus not merely denying that essences contain matter, but are entrenched in their commitment to that claim by a host of other claims: that matter, without qualification, individuates; that every form must be determinately one and numerically identical to itself; that propositions seemingly about the relationship, intersection, and hierarchy of forms are in fact statements about the relationships of those essentially disparate forms to different chunks of matter.

\subsection{...\textit{magnus est in fine}}
Buridan's \textit{Quaestiones de anima} first turns to the plurality of substantial forms in book II, question 4, where Buridan asks `whether in the same animal, the vegetative and sensitive souls are the same'.\autocite[II.4.1]{BuridanQDA}  In spite of agreeing with Aquinas' answer to the question, Buridan differs from Aquinas in his estimation of the character of his proof. While Aquinas regards his answer as proven according to the standards of Aristotelian \textit{scientia},\footnote{`Et hoc consequens est ei quod in praecedentibus ostendimus de ordine formarum substantialium, scilicet quod nulla forma substantialis unitur materiae mediante alia forma substantiali, sed forma perfectior dat materiae quicquid dabat inferior et adhuc amplius' \autocite[q. 1, art. 11, co.]{AquinasDA}.}  Buridan sees the matter as difficult to decide demonstratively;\footnote{`Ista questio bene est difficilis quia difficile est demonstrare aliquam partem' \autocite[II.4.10]{BuridanQDA}.} and apart from an appeal to the authority of Aristotle's \textit{Metaphysics VII}, Buridan decides the question not scientifically, but dialectically.\footnote{Pono igitur rationes probabiles ad probandum quod non sunt sic in equo anime diverse sensitiva et vegetativa' \autocite[II.4.15]{BuridanQDA}.}

Buridan describes the position of those who hold there to be a plurality of substantial forms in one suppositum in the following:
	\begin{quote}
	Those who posit several souls and substantial forms in the same suppositum ground their opinion [by saying] that according to the grade and order of quidditative predicates of genera and the species subordinated successively to them, there are in the same [individual] several subordinate substantial forms: as in Socrates there is a first form by which he is a substance; another by which he is a body; another by which he is living; another by which he is an animal; and another specific [form] by which he is a man. And just as prime matter is naturally in potency toward the first - that is, the most general - of those [forms]; and as this is the first act of this matter, wherefore from these comes something one per se: so also the second form has itself toward the composite of the first matter and form,(\textit{ex materia et forma prima})  such that that composite is per se in potency with respect to the second form. And that [second form] is the formal act of the composite itself. Therefore, from these comes something one per se, just as occurred from the first matter and form.\footnote{`Tenentes enim plures animas et formas substantiales in eodem supposito fundant suam opinionem quod, secundum gradum et ordinem predicatorum quidditativorum generum et specierum subordinatarum sibi invicem sunt in eodem plures forme substantiales subordinate, ut in Sorte est prima forma qua est substantia, alia qua est corpus, alia qua est vivens, alia qua est animal, et alia specifica qua est homo. Et sicut materia prima est naturaliter in potentia ad primam istarum, scilicet generalissimam, et quod ista est primus actus istius materie propter quod ex eis fit unum per se, ita secunda forma se habet ad compositum ex materia et prima forma, scilicet quod compositum illud est per se potentia respectu secunde forme. Et ista est actus formalis ipsius compositi. Ideo fit ex eis unum per se, sicut fiebat ex materia et forma prima' \autocite[II.4.10]{BuridanQDA}.}
	\end{quote}
		
This same pattern is then repeated with a third form, and so on until the final, most specific substantial form is added.

Buridan's description agrees with Aquinas' picture of the multiple-form defender as thinking of determinate forms as determinate additions to each other, and hence with the absence of a notion of the indeterminate containment of a species in its genus. Though Buridan uses the phrase `this matter,' he does not use it as a technical term designating signate matter, but to refer back to the matter he mentions in the preceding clause, i.e. prime matter. Thus, the sentence `this is the first act of this matter, wherefore from these comes something one per se' should be read as asserting that from the union of prime matter and first form - in Buridan's list, substance - comes something one per se. Hence, Buridan, too, takes the defender of multiple substantial forms to hold that prime, not signate matter, individuates. And with the plurality theorist, Buridan likewise assumes a composite is a union of form and prime matter.\footnote{See \autocite[II.7.25; III.6, 1st argument contra]{BuridanQDA}.}

Both the main body and replies to objections in Buridan's response to question 4 consist, in great part, in playing the principles of his opponents against them. For instance, he suggests that multiple-substantial-form positers must be committed to the thesis that an individual horse is composed of an animal and a plant, since it has non-identical vegetative and sensitive forms which could be separated from each other by the divine will.\autocite[II.4.16]{BuridanQDA}  As a result of this strategy, the exact shape of Buridan's own commitments on this question remains largely unpronounced. But his answer to the third objection gives us some clues about his own position.

Objection three argues as follows:
	
	\begin{quote}
	Again, to inhere in something on account of itself is to inhere in it by its nature or essence: therefore, to inhere in an animal insofar as it is an animal is to inhere in it on account of the nature by which it is an animal. And so again to inhere in a living thing insofar as it is a living thing is to inhere in it on account of the nature by which it is a living thing. But sensitive being inheres in Brunellus insofar as [he is an] animal and not insofar as [he is] living; and vegetative being inheres in him insofar as [he is] living, and not insofar as [he is an] animal. So sensitive being inheres in him by the nature by which he is an animal, and not by the nature by which he is living; and conversely with vegetative being. Therefore in Brunellus there is one nature according to which he is an animal, and another according to which he is living, and these are the vegetative and sensitive souls. Therefore, etc.\footnote{`Item inesse alicui secundum se est inesse sibi per suam naturam vel essentiam. Ideo inesse animali secundum quod animal est inesse sibi per naturam per quam est animal. Et sic etiam inesse viventi secundum quod vivens est inesse sibi per naturam per quam est vivens. Sed Brunello inest esse sensitivum secundum quod animal et non secundum quod vivens, et esse vegetativum inest sibi secundum quod vivens et non secundum quod animal. Igitur esse sensitivum inest sibi per naturam per quam est animal et non per naturam per quam est vivens, et esse vegetativum econverso. Ergo in Brunello est alia natura secundum quam est animal et alia secundum quam est vivens, et iste sunt anima vegetativa et sensitiva, igitur et cetera' \autocite[II.4.3]{BuridanQDA}.}
	\end{quote} 
	
Buridan replies:

	\begin{quote}
	Brunellus, by his very essence and nature, is Brunellus and a horse and an animal and a living being and a body. And when it is said that sensitive being inheres in Brunellus insofar as [he is] an animal and not insofar as [he is] a living being, we understand by this that this is true per se and primary: `an animal is sensitive'; and not this: `a living thing is sensitive'; so that by `being on account of itself' a convertible predication of terms is understood (whether primary or immediate or something of this sort) in accordance with how a reduplication is multiply put forth. So the locution was not speaking of a real inherence.\footnote{`Brunellus, per eandem eius essentiam et naturam, est Brunellus et equus et animal et vivens et corpus. Et cum dicitur quod Brunello inest esse sensitivum secundum quod animal et non secundum quod vivens, nos per hoc intelligimus quod hec est vera per se et primo, 'animal est sensitivum' et non hec, 'vivens est sensitivum,' ita quod per 'esse secundum quod ipsum' intelligitur predicatio convertibilis terminorum, vel prima aut immediata aut huiusmodi, prout 'reduplicatio' multipliciter exponitur. Ita quod ista locutio non erat de reali inherentia' \autocite[II.4.22]{BuridanQDA}}
	\end{quote} 
	
The objection is a topical argument, taking `to inhere in something on acccount of itself is to inhere in it by its nature or essence' as a \textit{propositio maxima}, and then applying it to the specific cases of sensitive and vegetative being, to infer that sensitive and vegetative natures are distinct in the same supposit. 

Buridan's response denies the consequence, providing an alternative exposition of the \textit{propositio maxima}. While the multiple form theorist runs the risk of positing as many forms as there are quidditative predicates (such that Brunellus is a horse by equinity, an animal by animality, etc.), Buridan's response suggests that there is only one truth-maker for all of these predicates. If the positers of several substantial forms have made essences determinate and terrestrial, Buridan will retain the element of determinacy in his opponent's account while preventing several such substantial forms from inhering in one suppositum. Thus, the truthmakers for the statements `a pig is an animal' and a `horse is an animal,' are left essentially disparate, leaving two equivocal uses of the term `animal', albeit permissible from the standpoint of common speech.

Elsewhere, it is confirmed that Buridan, like the plurality theorists, retains the image of forms as distinct, integral parts of the composite. In his \textit{Questions on the Metaphysics}, he writes 
\begin{quote}
Aristotle rebuked Plato for wanting to remove matter from the quiddities and dispositions of sensible substances; and \textit{Physics I} posits that sensible substance is essentially a composite of matter and form. And so form is an \textit{integral} part of it.\autocite[VII.12, cor.]{BuridanQM}
\end{quote}

Hence, though Buridan remains an opponent of the multiplicity of substantial forms in the same suppositum, he agrees with its advocate, against both Aquinas and Aristotle, in regarding forms as integral parts of their composites. Buridan's hylomorphism, like that of the plurality view both he and Aquinas argue against, thus satisfies the conditions for being a mereological hylomorphism.

Let us finally examine a passage from the succeeding question from Buridan's \textit{De Anima} questions. 

\begin{quote}
	Again, if a power were an accident of the soul, the soul would be in potency towards it, since a subject is in potency toward all its accidents. Therefore, either it itself would be in potency to this power - and then, by the same reasoning we would be able to say the same thing again from the start - or it is in potency to the other power through another power, and so we would proceed to infinity, which is unfitting \autocite[II.5, obj. 14]{BuridanQDA}.
\end{quote}

Here, Buridan follows the question of whether sensitive and vegetative souls are the same in the same animal with that of whether the powers of the soul are distinct from the soul itself. Buridan will answer in the negative. In the above objection, the objector argues by \textit{reductio} that the soul is not distinct from its powers. The objector argues, from the hypothesis that the powers are accidents, to the dilemma that positing them as separate either leads to nugatoriness or to an infinite regress. In arguing that the soul is distinct from its powers, the objection thus assumes that the only \textit{way} for a power to be distinct is for it to be an accident.\footnote{This is required from the straightforward reading of the proof as a \textit{modus tollens}: if it is to work, it must be assumed that `powers are distinct from the soul' is antecedent to `a power is accidental to the soul'. This is again assumed in the `since' clause of the first sentence: it assumes enthymematically that every power, if distinct from its subject, is an accident (hence that the soul, being a subject, would be in potency towards all its powers).} Nowhere, neither in his reply to the question and the objection nor elsewhere in the QDA, does Buridan deny this assumption that the distinctness of what inheres entails accidentality.

\section{Logical hylomorphism}
With the above laid out, we return to our discussion of the development of formal consequence. The above remarks provide us with the following two modest conclusions, as well as a direction for further research. 

First: the metaphysical thesis of the plurality of substantial forms appears to have provided a dialectical medium for the shift from Aquinas' account of the inherence of a single substantial form in a suppositum, to Buridan's account. In spite of his agreement with Aquinas in rejecting a plurality of substantial forms in a single substance, Buridan agrees with the plurality theorist against Aquinas: 1) in the acceptance of the form-essence identity thesis 2) in the assumption that identity or distinctness of species and genus must be determinate identity; 3) in regarding the union of the composite as one between form and prime matter; 4) in the accompanying \textit{metaphysical} assimilation of the relation between form and matter to one between essence and accident; and 5) in the corresponding \textit{logical} analysis, rejected by Aquinas, of essential predications in terms of reference to forms inhering in a common suppositum.

Second: Buridan's \textit{logical} hylomorphism, according to which the matter and form of a proposition are disjoint integral parts of a proposition - namely, categorematic and syncategorematic terms\footnote{Strictly speaking, the phrase `categorematic terms' is redundant, and 'syncategorematic terms' a contradiction in terms. On Buridan's vocabulary, \textit{termini} are the `ends' of a proposition, in which it `bottoms out'. These are categorematic by construction. Likewise, syncategoremata are never the base units of a proposition, but always functions on smaller units. However, \textit{Nomina sunt ad placitum}, and so the phrase `(syn)categorematic term' is used here in accordance with more recent usage.}  - strictly mirrors his \textit{physical} hylomorphism, according to which matter and form are disjoint integral parts of a composite substance. Thus, while Buridan's physical hylomorphism does not necessitate his logical hylomorphism, the specific shape of the former does seem to have facilitated that of the latter. It is further likely that both the identification of form with essence or nature - excluding matter - and the assumption that distinctness of what inheres entails its accidentality - an assumption equally applicable to the relation of substantial form and prime matter - played some role in the gradual assimilation of the natural/accidental division of consequences to that into formal and material varieties.

Now, what remains for further research is to fill in the middle piece of this puzzle - namely, to further investigate what connections there may be between the pluricity theory and those more `pluralist' accounts of formal consequence in existence prior to Buridan. It is clear enough that those authors, such as Scotus and Ockham, who use a broader sense of 'formal consequence' tend also to be hylomorphic pluralists.\footnote{Further points of connection include: that Scotus held substantial union to be one of prime matter and substantial form\autocite[138]{Cross1995}; and that Scotus sees the extension of a substantial form beyond those corporeal parts manifesting its powers as accidental \autocite[VII.20, par. 19, pp. 383-384]{ScotusMetaph}. That the soul is identical with its powers has also been regarded as closely linked to the plurality thesis. See \autocite[242-243]{Weisheipl1980}, \autocite[I, p. 335]{DeWulf1926}. For Scotus' influence on Walter Burley, see \autocite{Ottman1999}; for an examination of the interconnection between metaphysical and logical disputes in Scotus and Ockham, see \autocite{Martin2004}.} It remains to be determined whether this connection may be more than accidental. That it would, however, appears at present to provide the path of least resistance from the predominantly syllogistic framework of Aquinas to Buridan's substitutional, mereological account of formal consequence.
\end{document}