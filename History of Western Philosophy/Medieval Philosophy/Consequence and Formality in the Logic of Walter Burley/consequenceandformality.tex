\documentclass[]{birkjour}
\usepackage[backend=biber, style=verbose-ibid]{biblatex}
\bibliography{jacob}
%opening
\title{Consequence and Formality in the Logic of Walter Burley}
\author{Jacob Archambault}
\address{Louisville, KY}
\email{jacobarchambault@gmail.com}
\begin{document}
	
	\begin{abstract}
		With William of Ockham and John Buridan, Walter Burley is often listed as one of the most significant logicians of the medieval period. However, Burley's contributions to medieval logic have received notably less attention than those of either Ockham or Buridan. 
		
		To improve upon this situation, I here provide a comprehensive examination of Burley's account of consequences. The first half of the article recounts Burley's enumeration, organization, and division of consequences, with particular attention to the shift from natural and accidental to formal and material consequence. The second half locates Burley's contribution to the theory of consequences within the context of 14th century work on the subject, detailing its relation to the earliest treatises on consequences, then to Ockham and Buridan. 
	\end{abstract}
	\keywords{Walter Burley, William of Ockham, John Buridan, consequence, hylomorphism, logical form}
	\maketitle
	%move section on formal/material consequence to under Burley's division of consequences. re-add the parts on the Parisian anonymous treatise originally in that section to the discussion of the anonymous treatises.
	\section{Introduction}
In his `The medieval theory of consequence', Stephen Read lists Walter Burley (or Burleigh) alongside William of Ockham and John Buridan as one of the most significant logicians of the medieval period.\footnote{\autocite[p. 900]{Read2012} Translations throughout, unless stated otherwise, are my own.} However, Burley has received markedly less attention than either Ockham or Buridan. 
	
Part of the reason for this, already noted by Boehner \autocite[p. VI]{Boehner1955}, is historiographical. While Ockham and Buridan are nominalists, Burley is classified as a realist. Inasmuch as early research into the medieval logic began to accelerate at a time where nominalism was dominant in \textit{philosophical} logic,\footnote{See for instance, the contemporary assessment of Quine's influence on the discipline in \autocite{Lejewski1954}.} it is understandable that historians of the discipline first set their sights on figures whose assumptions were closest to that of its contemporary practitioners.\footnote{That this nominalism was never completely dominant is clear enough from the importance of figures like Frege, G\"{o}del, or even the early Russell for the tradition, not to mention the contributions of early phenomenology or of various members of the Lvov-Warsaw school.} After nearly a century of sustained work in the discipline, a closer examination of Burley's thinking about consequences, including its relation to that of Buridan and Ockham, provides something of a counterweight to this tendency. 

The plan of the essay is as follows. I begin with Burley's divisions and organization of consequences. I then introduce formal and material consequence in Burley's work, and show how Burley relates this division back to that between natural and accidental consequence. After this, I locate Burley's contribution to the theory of consequences within the context of 14th century work on the subject, detailing its relation to the earliest anonymous treatises on consequences, then to the work of Ockham and Buridan. 
\section{Burley's division and enumeration of consequences}
Burley's views on consequences are found mainly in three treatises. The earliest of these is his short tract \textit{De consequentiis}, written prior to 1302 and preserved in six manuscripts \autocite{Green-Pedersen1980b}. The extant manuscripts of the \textit{De consequentiis} divide into two groups \autocite[pp. 104-105]{Green-Pedersen1980b}. The manuscripts of the second group, especially one British Library manuscript, introduces numerous additions and corrections to the more basic text shared by the first group. The deliberate character of these changes may point to a revision originating with Burley himself.

Next is the briefer version of \textit{De Puritate Artis Logicae}, which reached its final, still unfinished state in the early 1320s. The latest work is Burley's \textit{De Puritate Artis Logicae, Tractatus Longior}, which likely reached its current form at the University of Paris between 1324 and 1327 \autocite[pp. 12-13]{Ottman1999}, and is thought to have been revised in light of Ockham's attack on Burley's account of simple supposition in his \textit{Summa Logicae}.\footnote{\autocite[pp. vii-viii]{Boehner1955}. Boehner further conjectured that Burley's treatise draws its title from its opposition to the `impurities' of Ockham's logic. But \autocite{SpadeMenn} shows the title is an allusion to the prologue of Avicenna's \textit{Al-Shif\={a}}, where `purity' means the core, or pith, of a thing.}
\subsection{Consequences in Burley's treatment of hypothetical propositions}
In the longer version of his \textit{De Puritate Artis Logicae}, Burley discusses consequences in the part of the treatise devoted to hypothetical propositions, specifically that devoted to \textit{conditional} hypothetical propositions.
	
Conditionals form one kind of explicit hypothetical proposition, along with conjunctions, disjunctions, and temporal propositions. In addition, Burley counts exceptive, exclusive, reduplicative, and other kinds as implicit hypotheticals.\footnote{\autocite[pp. 106-107]{BurleyDPAL}. Medieval discussion of explicit hypotheticals goes back to Boethius. See \autocite[passim]{BHS}. Discussions of implicit hypotheticals arise later, apparently first toward the end of the twelfth century. For discussion, see \autocite{Giusberti1977}; also, \autocite{DeRijk1966}.}

By `hypothetical proposition', Burley means what today one might call a compound or non-atomic proposition.\footnote{Cf. \autocite[p. 66]{Buridan2015}.} By `implicit hypothetical', he means a proposition analyzable into an explicit hypothetical, even if its surface grammar is categorical. For instance, the sentence `Nothing besides Socrates runs' is called an exceptive proposition, from the function of the exceptive word `besides'. Burley analyzes it into the conjunction `Socrates runs and nothing other than Socrates runs', whose conjuncts are called the \textit{exponents} (\textit{exponentes}) of the original proposition.\footnote{\autocite[p. 121, par. 44]{Green-Pedersen1980b}, \autocite[pp. 164-165]{BurleyDPAL}.} Besides conditionals and other kinds of compound propositions, Burley also refers to conditional \textit{syllogisms}, which are composed of three conditional hypothetical propositions - or, in the cases of \textit{modus ponens} and \textit{modus tollens}, one hypothetical and two categoricals - two of which are premises concluding to the third.\footnote{Cf. \autocite[5.1.3, pp. 308-309]{BuridanKlimaSD}; \autocite{Klima2004b}.}
	
According to Burley, a consequence is an act performed in a conditional hypothetical proposition by words like `if', `thus', and `therefore' - the same act signified by the term `follows' in a categorical proposition.\footnote{For more on the distinction between a signified and performed act, see the following footnote; also, \autocite{Nuchelmans1988}.} Consequences are not themselves conditionals. Neither are they distinguished from conditionals by the use of `therefore' or `follows' rather than `if ... then' to conjoin two propositions.\footnote{In \autocite{HodgesBurley}, Hodges suggests Burley distinguishes consequences from conditionals by using `if' for the latter, and `therefore', `follows', etc. to indicate the former, referring to \autocite[p. 78.10-14]{BurleyDPAL} in support of this interpretation:
	\begin{quote}
		Ulterius sciendum, quod ad omnem conditionalem sequitur categorica, in qua categorica consequens denotatur sequi ad antecedens; et econverso ad omnem categoricam, in qua denotatur quod una propositio sequitur ad aliam, sequitur conditionalis, in qua inter istas propositiones consequentia exercetur.
	\end{quote}
	
But Hodges' interpretation of the passage he cites is incorrect. Burley's point in the passage is that from any conditional follows a \textit{categorical} sentence in which nominalizations of the antecedent and consequent of the conditional are taken as terms, and vice versa - i.e. for any conditional $\phi \rightarrow \psi$, terms $a, b$ respectively naming the conditional's antecedent and consequent, and a binary relation $Follows$: $Follows(a, b)$ if and only if $\phi \rightarrow \psi$. Thus at \autocite[p. 141.26-30]{BurleyDPAL}, Burley states that from a consequence in which the term `therefore' (\textit{ergo}) \textit{performs} its semantic function, there follows a categorical sentence where this function is \textit{signified} by the word `follows', and Burley makes the same point in the \textit{Tractatus Brevior} with respect to the function performed by `if' (\textit{si}) at \autocite[p. 219.1-9]{BurleyDPAL}. Cf. \autocite[p. 143, par. 119]{Green-Pedersen1980b}.} Rather, `conditional' connotes the grammatical or syntactic structure in which a consequence is shown; `consequence', the semantic relation obtaining between a conditional's parts. Because of this close link between consequences and conditionals, one finds the same vocabulary applied to both terms in Burley's work.\footnote{Peter King locates the distinction between consequences and conditionals in that between statements and arguments, arguing that i) the terminology used for conditionals and consequences is not interchangeable, and ii) specifically, conditionals `are true or false whereas consequences are not'. \autocite[p. 120]{King2001}. But the grounds for King's claims are false. Burley calls the conditional `If a man is an ass, you are sitting' a good consequence at \autocite[p. 61]{BurleyDPAL}. At \autocite[p. 114, par. 8]{Green-Pedersen1980b} and \autocite[p. 89.1-31]{BurleyDPAL}, Burley refers to the conditionals linking a larger consequence as `intermediate consequences (\textit{consequentiae intermediae})', and the latter of these passages also calls the conditionals making up a hypothetical syllogism `consequences'. At \autocite[p. 128, par. 68]{Green-Pedersen1980b}, Burley divides conditionals into as-of-now and simple, but immediately afterwards refers to simple and as-of-now consequences, and the same division is found in \autocite[pp. 60.28-61.5]{BurleyDPAL} stated in terms of consequences. At \autocite[p. 78.27-30]{BurleyDPAL}, Burley calls a certain consequence linking a categorical antecedent and hypothetical consequent with `therefore' (\textit{igitur}) a `composite conditional', and Burley calls consequences true and false at \autocite[p. 113, par. 2-3]{Green-Pedersen1980b}; cf. \autocite[p. 15, par. 19]{Green-Pedersen1980a}.} 
				
Burley's consequences are located in a natural, not formal, language \autocite[pp. 71-72]{HodgesBurley}. Because the linguistic limitations of Latin are taken seriously, certain kinds of consequences cannot be unambiguously stated in it: syllogisms cannot be contraposed \autocite[pp. 65.3-17; 207.31-208.9]{BurleyDPAL}; one cannot move from sentential to term negation to negate a complex term, as in `non-(white tree)',\footnote{\autocite[p. 131, par. 80]{Green-Pedersen1980b}, \autocite[pp. 214.14-21, 215.6-21]{BurleyDPAL}.} and one may object to a consequence on the grounds that its premises are well-formed while its conclusion is not.\footnote{\autocite[p. 150, par. 135]{Green-Pedersen1980b}; \autocite[pp. 211.31-33, 212.10-20]{BurleyDPAL}.}
	
\subsection{Burley's division of consequences}
Burley divides consequences into simple and as-of-now, and subdivides the former into natural and accidental consequences.

According to Burley, a simple consequence is one `which holds for every time, such that the antecedent can never be true unless the consequent is true'. An as-of-now consequence, by contrast, `holds for a given time and not always'.\footnote{\autocite[pp. 60.29-61.5]{BurleyDPAL}. Cf. \autocite[p. 199.19-25]{BurleyDPAL}.} This distinction is already present in the earlier \textit{De consequentiis}, where it is also referred to as one between simple and as-of-now \textit{conditionals} \autocite[p. 128, par. 69]{Green-Pedersen1980b}. Burley's definitions for simple and as-of-now consequence do not substantially differ from those in Ockham or Buridan.\footnote{\autocite[III-3. 1, pp. 587-588]{OckhamSL}; \autocite[I. 4, p. 23]{BuridanTC}.}

Burley defines a natural consequence as one where `the antecedent includes the consequent'. Such a consequence `holds through an intrinsic topic', while an accidental consequence is one `which holds by an extrinsic topic'.\footnote{\autocite[p. 61.6-10]{BurleyDPAL}: \begin{quote}
		Consequentia simplex est duplex: Quaedam naturalis, et est quando antecedens includit consequens; et talis consequentia tenet per locum intrinsecum. Consequentia accidentalis est, quae tenet per locum extrinsecum, et est quando antecedens non includit consequens, sed tenet per quandam regulam extrinsecam.
	\end{quote}} Surprisingly, the division between natural and accidental consequences is not found at all in the \textit{Tractatus Brevior}. It is, however, in Burley's earlier \textit{De consequentiis}, which further distinguishes two kinds of accidental consequences: for the one, which `holds on account of the terms or matter', Burley gives the proposition `That God exists is true, therefore that God exists is necessary (since truth in God and necessity are the same)' as an example; for the other, Burley gives as examples propositions with an impossible antecedent or a necessary consequent.\footnote{\autocite[pp. 128-129, par. 70]{Green-Pedersen1980b}: `Consequentia accidentalis est duplex: quaedam tenet gratia terminorum vel gratia materiae, sicut ista `deum esse est verum, ergo deum esse est necessarium'; et tenet gratia terminorum vel materiae, quia veritas in deo et necessitas sunt eadem. Quaedam est consequentia accidentalis sic: ex impossibili sequitur quodlibet; et necessarium sequitur ad quodlibet.'} This division is not presented in the later \textit{De Puritate}, though as we shall see, it shows up tacitly in the solution to a problem there.

Discussion of topics goes back to Aristotle, and its medieval reception is heavily mediated through the work of Boethius.\footnote{For the distinction between intrinsic and extrinsic topics, see the introduction to this volume, and Bianca Bosman, `The roots of the notion of containment in theories of consequence', also this volume.} By the time of Burley, the topical apparatus grounding his distinction between natural and accidental consequences has been expanded and simplified in different ways. It has been simplified by the identification of maximal propositions, in virtue of which topics are traditionally said to hold, with rules: `A maximal proposition is nothing but a rule whereby a consequence holds'.\footnote{\autocite[p. 76.5-7]{BurleyDPAL}: `Nam propositio maxima non est nisi regula, per quam consequentia tenet.} It has been expanded by the claim that such a rule need not be a dialectical one - the traditional purview of topical argument - but may be more broadly logical \autocite[p. 76]{BurleyDPAL}; and by the admission of an indefinite number of differences of maximal propositions, rather than the traditional twenty-five or so admitted by Themistius and Boethius. For Burley,
\begin{quote}
	Not every maximal proposition arises from a difference of a maximal proposition known to us, because not every difference of a maximal proposition has a name. For many maximal propositions are necessary, and still do not have names imposed on the differences of those maximal [propositions].\footnote{\autocite[pp. 76.35-77.1]{BurleyDPAL}: \begin{quote}
			[N]on omnis propositio maxima oritur a differentia maximae nobis nota, quia non omnis differentia maximae habet nomen. Multae enim propositiones maximae sunt necessariae et tamen non habent nomina imposita differentiis illarum maximarum.
		\end{quote}.}
\end{quote}

The most important use Burley makes of extrinsic topics is in his justification of the rules \textit{from the impossible anything follows (ex impossibili quodlibet)} and \textit{the necessary follows from anything (necessarium ex quolibet)}. The former is justified by the extrinsic topic \textit{from the lesser (a minore)}, on the grounds that if something impossible holds, then given it is less likely that something impossible holds than another proposition, that other proposition can be inferred.\footnote{\autocite[pp. 128-129, par. 70]{Green-Pedersen1980b}; \autocite[p. 61]{BurleyDPAL}. Cf. \autocite[III-3. 38, pp. 727-731]{OckhamSL}.} The \textit{Tractatus Brevior} notes that this places some restrictions on the rule itself: for instance, the more impossible cannot be inferred from the less impossible \autocite[pp. 248.19-249.3]{BurleyDPAL}.

\subsection{Burley's enumeration of consequences}
In both the shorter and longer versions of the \textit{De Puritate}, Burley calls certain rules governing conditional hypothetical propositions `principal', thereby distinguishing them from other rules which are said to depend on them. Burley gives ten principal rules in the \textit{Tractatus Brevior}, reduced to five in the \textit{Tractatus Longior}. Nearly all of the rules mentioned are also found in the earlier \textit{De consequentiis}, though it does not mention principal rules. The first, second, and fifth rules of the \textit{Tractatus Longior} are the first three rules of the \textit{Tractatus Brevior}. These are, respectively: (1) that `in every good simple [as-of-now] consequence, the antecedent cannot [now] be true without the consequent';\footnote{\autocite[p. 61.30-37, 199.26-27]{BurleyDPAL}: \begin{quote}
		{In omni consequentia bona simplici, ut consequentia simplex distinguitur contra consequentiam ut nunc, antecedens numquam potest esse verum sine consequente [...]. In consequentia autem ut nunc non potest antecedens ut nunc esse verum sine consequente.}
	\end{quote} Cf. [199.26-27]{BurleyDPAL}.}  (2a) `whatever follows from the consequent follows from the antecedent'; (2b) `whatever entails (\textit{antecedit}) the antecedent entails the consequent';\footnote{\autocite[p. 62, 9-13]{BurleyDPAL}: \begin{quote}
	Secunda regula principalis est, quod quidquid sequitur ad consequens, sequitur ad antecedens. Est autem alia regula quasi eadem cum ista, quae talis est: Quidquid antecedit ad antecedens, antecedit ad consequens.
\end{quote}} and (5) `whenever a consequent follows from an antecedent, the contradictory opposite of the antecedent follows from the contradictory opposite of the consequent' (i.e. contraposition).\footnote{\autocite[p. 64.20-22]{BurleyDPAL}: \begin{quote}
Quinta regula principalis est ista, quod quandocumque ad antecedens sequitur consequens, ex opposito contradictorie consequentis sequitur oppositum antecedentis.
\end{quote}} Several other rules in the \textit{Tractatus Brevior} are not straightforward rules about \textit{consequence}, but rather govern supposition, negation, or denotation. The \textit{Tractatus Longior} moves these to other sections of its text. Rule four of the \textit{Tractatus Brevior}, for instance, is called a `general rule' in the \textit{Tractatus Longior} \autocite[p. 208.12-13, p. 73.29]{BurleyDPAL}. Rule six of the earlier treatise is mentioned in the later treatise's section on supposition \autocite[p. 210.11-12, p. 27.11]{BurleyDPAL}. Rule ten of the earlier treatise appears in connection with the solution to a sophism in the later treatise \autocite[p. 219.1-2, p. 141.26-30]{BurleyDPAL}. 

The rules governing consequences in the \textit{Tractatus Longior}, by contrast, all explicitly concern relations of following, with the two new principal rules in the \textit{Tractatus Longior}, rules three and four, relating following to compatibility and incompatibility. These rules, already present in the \textit{De consequentiis} \autocite[p. 133, par. 88]{Green-Pedersen1980b}, are (3) whatever conflicts with the consequent conflicts with the antecedent, and (4) whatever stands with the antecedent stands with the consequent.\footnote{\autocite[pp. 63.1-2, 7-8]{BurleyDPAL}: \begin{quote}
		Tertia regula principalis est ista: Quidquid repugnat consequenti, repugnat antecedenti [...]. Quarta regula principalis est: Quidquid stat cum antecedente, stat cum consequente.
	\end{quote}} 

Burley gives no justification for principal rules one, two, or five of the \textit{Tractatus Longior}. He does, however, provide arguments for rules three and four \autocite[p. 63.1-14]{BurleyDPAL}. That for the third relies on two applications of the transitivity principle given in rule two, and that for the fourth employs a definition of `able to stand with another' reducing it to the notion `can be true with' found in rule one. We cannot thus construe Burley's principle rules as being strictly independent from each other. Burley also restricts the fifth rule, `from the opposite of the consequent the opposite of the antecedent follows', to non-syllogistic consequences on syntactic grounds: for Burley, the premises of a syllogism do not form one proposition, either simple or complex, and so neither does their negation.\footnote{\autocite[pp. 65.3-17; 207.31-208.9]{BurleyDPAL}. Burley thus rejects the interpretation, found in \L{}ukasiewicz, of the premises of a syllogistic inference as a conjunction. \autocite{Lukasiewicz1957}. For criticism of \L{}ukasiewicz's interpretation, see \autocite{Corcoran1974}.}


\subsubsection{Principal and derivative rules}
Besides these principal rules, Burley gives a small number of rules said to follow from these: from the first, that (1.1) in a simple consequence, the impossible does not follow from the contingent and (1.2) that the contingent does not follow from the necessary;\footnote{\autocite[p. 62.1-8]{BurleyDPAL}: \begin{quote}
		Prima est, quod ex contingenti non sequitur impossibile in consequentia simplici. Secunda est, quod ex necessario non sequitur contingens.
	\end{quote}} from the second, that (2.1) whatever follows from the antecedent and consequent follows from the antecedent by itself (\textit{per se}), and (2.2) that whatever follows from the consequent with something added follows from the antecedent with the same added;\footnote{\autocite[p. 62.20-26]{BurleyDPAL}: \begin{quote}
	Ex ista regula, scilicet quidquid sequitur ad consequens, sequitur ad antecedens, sequuntur duae aliae regulae, quarum prima est: Quidquid sequitur ex antecedente et consequente, sequitur ex antecedente per se. Secunda est, quod quidquid sequitur ad consequens cum aliquo addito, sequitur ad antecedens cum eodem addito.
\end{quote}} from the third and fourth, that (4.1) if the consequences of certain propositions conflict so do those propositions, (4.2) consequences with compossible antecedents have compossible consequences, and (4.3) that in a good consequence, the opposite of the consequent conflicts with the antecedent;\footnote{\autocite[pp. 63.15-32]{BurleyDPAL}: \begin{quote}
Ex istis duabus regulis sequuntur aliae tres regulae. Prima, quod si consequentia ad aliquas propositiones repugnent, illae propositiones repugnant inter se. Unde si consequentia repugnent, antecedentia repugnent [...]. Secunda regula est, quod si antecedentia stent simul, oportet quod eorum consequentia stent simul [...]. Tertia regula, quae sequitur, est, quod in omni consequentia bona oppositum consequentis repugnat antecedenti.
\end{quote}} from the fifth, that (5.1) whatever follows from the opposite of the antecedent follows from the opposite of the consequent, and (5.2) whatever entails the opposite of the consequent entails the opposite of the antecedent.\footnote{\autocite[65.18-21, 31-32]{BurleyDPAL}: \begin{quote}
Ex ista regula, scilicet quod ex opposito consequentis etc., sequuntur duae aliae regulae. Prima est quod quidquid sequitur ad oppositum antecedentis sequitur ad oppositum consequentis [...]. Secunda regula est talis: Quidquid antecedit ad oppositum consequentis, antecedit ad oppositum antecedentis.
\end{quote}}

Already in the \textit{De consequentiis}' discussion of the \textit{Tractatus Longior}'s rule (2.1), Burley makes clear that following can occur in different ways, each of varying strength: the rule is said to follow `at least in the same way \textit{ex impossibili quodlibet} follows' - i.e. accidentally, via an extrinsic topic, and not on account of the terms.\footnote{\autocite[p. 132, par. 85]{Green-Pedersen1980b}: \begin{quote}
		Quia enim ad istam `iste asinus est unus solus homo' cum ista `iste asinus est homo', quod est consequens ad ipsam, sequitur haec conclusio <`iste> homo est unus solus homo' formaliter, ideo ad istam `iste asinus est unus solus homo' sequitur eadem conclusio; saltem sicut ex impossibili sequitur quodlibet.
	\end{quote}} Burley's meaning here does not seem to be that the rule holds at least with the same strength as that consequence, but rather that instances of the rule may admit of varying levels of strength, the weakest of these being that associated with explosive consequences.\footnote{By `explosive consequence', I mean a consequence following from its antecedent merely by virtue of the antecedent's being impossible; by `explosion', a rule according to which one may infer an arbitrary proposition from explicitly or implicitly incompatible propositions.} 

%When contrasted with classical consequence, Burley's assumption of the weakness of explosion may come as a surprise. Burley regards enthymematic consequences such as `if Socrates is a man, Socrates is an animal' as stronger than explosive consequences.\footnote{Cf. \autocite[pp. 333-337]{Klima2016}.} But classical and strict consequence, following Buridan, build explosion into the definition of consequence: a consequence is good just when what the antecedent signifies cannot be so with what the consequent signifies not being so. Since things cannot be as an impossible antecedent signifies, neither can they be so with whatever is consequent to such an antecedent not being so. 
	
Derivative rules are not restricted instances of the principal rules, nor are they derivable solely from those rules. Burley's expositions of derivative rules usually take the following structure: statement of the rule, reason supporting it, optional example, summary statement of the argument from the principal rule to the derivative one. Here, for instance, are Burley's remarks on rule 2.2: 
\begin{quote}
	The reason for the second rule is this: the antecedent with something added implies the consequent with the same added; for `Socrates runs and you are sitting, therefore a man runs and you are sitting' follows. Since, then, whatever follows from the consequent follows from the antecedent, it must be that whatever follows from the consequent with something added, follows from the antecedent with the same added.\footnote{\autocite[p. 62.33-38]{BurleyDPAL}: \begin{quote}
			Ratio secundae regulae est ista: Antecedens cum addito infert consequens cum eodem addito; sequitur enim: Sortes currit et tu sedes, ergo homo currit et tu sedes. Cum ergo quidquid sequitur ad consequens sequitur ad antecedens, oportet quod quidquid sequitur ad consequens cum aliquo addito, sequitur ad antecedens cum eodem addito.
		\end{quote}}
\end{quote}

Here and elsewhere, the reason gives information that must be used with the principal rule to obtain the derived one. In one place, Burley tells us `from one [proposition] nothing follows, neither enthymematically nor syllogistically'.\footnote{\autocite[p. 147, par. 130]{Green-Pedersen1980b}: \begin{quote}
		Alia regula est quod ex uno nihil sequitur, nec enthymematice nec syllogistice. Unde ex ista `homo currit' dempta quacumque alia propositione secundum intellectum et secundum vocem non sequitur ista `animal currit' nec quaecumque alia.
	\end{quote}} Burley's meaning here is that every good consequence relies on premises and/or connections other than those stated explicitly in the premise itself. This holds also for arguments deriving rules from other rules. 

%Consider cutting Hodges section: it's too derivative

Many of the consequences Burley discusses, principally those depending on Burley's second rule, amount to what Hodges has felicitously described as a `calculus of monotonicity'. Briefly, let $T$, $T'$ be noun phrases, $S(T)$ a sentence in which $(T)$ occurs, and $S(T'/T)$ the result of replacing all instances of $T$ in $S$ with instances of $T'$. Now an occurrence of a term $T$ in a sentence $S$ is \textit{upwards monotonic} iff the consequence
\begin{quote}
	Every $T$ is a $T'$. $S(T)$. Therefore, $S(T'/T)$
\end{quote}
is good; and \textit{downwards monotonic} iff the consequence 
\begin{quote}
	Every $T$ is a $T'$. $S(T'/T)$. Therefore, $S(T)$.
\end{quote}
is good \autocite[pp. 93-94]{HodgesBurley}. Burley calls a consequence employing upwards monotonicity one \textit{from an inferior to a superior}; and one employing downwards monotonicity, \textit{from a superior to an inferior}. In the above, `Man' and `runs' are both upwards monotone in the sentence `a man runs', and downwards monotone in the compound proposition `A man runs, therefore an animal moves'. In practice, the sentences of the form `Every $T$ is $T'$ factoring into the above conditions cannot be just any true universal affirmative sentence, but must be necessary truths expressing conceptual containment of the predicate in the subject.\footnote{The exact details are somewhat more complicated, because of issues concerning assumptions about existence built into different kinds of propositions. For instance, for categorical propositions with upwardly monotonic terms, upwards monotonicity holds good for conceptually contained terms without further qualification: `a man is an animal, therefore a man is a substance' holds good regardless of whether humans exist. But consequences proceeding from categorical propositions with downwardly monotonic terms may fail: `Every animal is a substance, therefore every man is a substance' is good, but `every animal is running, therefore every man is running' is not. The former, since it expresses a conceptual truth, does not presuppose the existence of the things it connects. The latter, however, does, and so may be falsified in the situation where every animal is running, but humans do not exist. The existential import of propositions for Burley is thus determined not strictly by their quantity and/or quantity, but also by the kind of predication they exhibit. Cf. \autocite[pp. 61.4-5; 85.3-5, 25-26; 216.15-18]{BurleyDPAL}; \autocite[p. 97]{HodgesBurley}.} 

The development found in Burley's treatises thus suggests two points for the development of his thinking about consequence, mirrored in that of the field as a whole: first, that a large body of rules for consequences had been in place before serious attempts to track relations of dependence between them; second, that the study of consequences only gradually came to be distinguished from related studies, e.g. those into supposition theory and the treatment of syncategorematic functions.
\section{Following formally in Burley's logic}
The division between formal and material consequence, which became central after the work of Ockham and Buridan, plays only a minor role in Burley's work, appearing most prominently in the solution to an objection in the \textit{Tractatus Longior} \autocite[p. 80.13-29, 84.8-86.21]{BurleyDPAL}. Prior to this, the \textit{De consequentiis}, discusses following `on account of the matter' at paragraphs 70 and 168, and following formally at paragraphs 75, 84-85, 106, 118, and 168.  Burley also mentions formal consequence and formal repugnance in the \textit{Tractatus Longior}'s treatise on supposition \autocite[25.21, 39.20]{BurleyDPAL}. In these passages, we find: that a rule fails to hold formally for a specified set of consequences when a consequence assumed to be governed by it fails to hold, i.e. when one can find a bad argument among those having the form specified by the rule \autocite[p. 130, par. 75]{Green-Pedersen1980b}; that for a rule to hold formally is for it to hold generally;\footnote{Cf. \autocite[p. 130, par. 75, 76]{Green-Pedersen1980b}.} that a rule may hold formally for only a restricted class of beings \autocite[p. 157-158, par. 159]{Green-Pedersen1980b}, and that a consequence sharing the same form as a bad consequence does not hold formally \autocite[p. 25.21]{BurleyDPAL}. Burley calls that part of a sentence `formal' which is to be negated in its contradictory - what today one would call the main connective of a sentence.\footnote{\autocite[pp. 73.29; 208.12-30]{BurleyDPAL}; \autocite[p. 120, par. 40]{Green-Pedersen1980b}.}

The earliest mss. of the \textit{De consequentiis} say that rule 2.2 of the \textit{Tractatus Longior} follows `necessarily or formally', apparently intending to equate these notions \autocite[p. 132, par. 84]{Green-Pedersen1980b}. As in the later treatises, one also finds this rule paired with rule 2.1 in the \textit{De consequentiis}, albeit with their order reversed. But by 1302, the \textit{De consequentiis} text found in the London and Cambridge mss. drops the words `necessarily or formally', and this equivalence does not reappear in either version of the \textit{De Puritate}. Thus it seems that early on, Burley equated a formal consequence with a necessary one. This view, however, was dropped by 1302, the date given by the scribe of the London copy of Burley's \textit{De consequentiis}.

From the passages mentioned above, we can see that some rules do not hold for all terms, but only certain kinds of terms.\footnote{Cf. \autocite[p. 214.14-21]{BurleyDPAL}.} We also find that Burley was aware of the substitution test for formal consequence alluded to by Simon of Faversham and explicit in Buridan: to determine whether a consequence is formally good, one should obtain a new sentence by substituting its categorematic terms with other terms; if one can find a consequence of this form whose antecedent can be true without its consequent, then the first consequence was not formally good.\footnote{\autocite[pp. 162-163, par. 168]{Green-Pedersen1980b}; \autocite[p. 150.32-35]{BurleyDPAL}; \autocite[qq. 36-37]{FavershamQE}; \autocite[I. 4]{BuridanTC}.} 

Unlike Buridan, Burley gives no indication that the substitution test provides not only a necessary, but also a sufficient condition for following formally. Nor, more generally, does one find any single kind of consequence defined in terms of the impossibility of the antecedent and contradictory of the consequent being true together. Burley does state that `for a consequence to be good, it suffices and is required that the contradictory of the consequent conflicts with the antecedent'.\footnote{\autocite[p. 64.12-14]{BurleyDPAL}: \begin{quote}
		Et ideo dico, quod ad hoc quod consequentia sit bona, sufficit et requiritur, quod contradictorium consequentis repugnet antecedenti.
	\end{quote}} But in order to obtain the more general definition from this - and with it the consequences \textit{from the impossible anything follows} and \textit{the necessary follows from anything} - one must assume that any proposition conflicts with an impossible one, and that any proposition stands with a necessary one. These assumptions do not appear in Burley's text.

One of Burley's more illuminating passages discussing formal consequence begins with the following objection to rule 2.1 in the \textit{Tractatus Longior}: 
\begin{quote}
	`Brunellus is capable of laughter, therefore, Brunellus is a man' follows; and from these two, it follows formally that a man is capable of laughter. For `Brunellus is capable of laughter, Brunellus is a man, therefore a man is capable of laughter' follows formally. And yet from the antecedent by itself the same does not follow formally. `Brunellus is capable of laughter, therefore a man is capable of laughter' does not follow formally, because then distributing the consequent, the antecedent follows; and then `Every man is capable of laughter, therefore Brunellus is capable of laughter' follows, which is false. And even if `Brunellus is capable of laughter, therefore a man is capable of laughter' were to follow formally, it would follow in the same way placing the negation after. And then `Brunellus is not capable of laughter, therefore a man is not capable of laughter' would follow, where the antecedent is true and the consequent false.\footnote{ \autocite[p. 80.13-29]{BurleyDPAL}:
		\begin{quote}
			Sequitur: Brunellus est risibilis, ergo Brunellus est homo; et ex istis duabus sequitur formaliter, quod homo est risibilis; sequitur enim formaliter: Brunellus est risibilis, Brunellus est homo, ergo homo est risibilis. Et tamen ex antecedente per se non sequitur idem consequens formaliter, quia non sequitur formaliter: Brunellus est risibilis, ergo homo est risibilis, quia tunc distributo consequente sequitur antecedens; et tunc sequitur: Omnis homo est risibilis, ergo Brunellus est risibilis, quod est falsum. Et etiam si sequeretur formaliter: Brunellus est risibilis, ergo homo est risibilis, sequeretur eodem ordine postposita negatione. Et tunc sequeretur: Brunellus non est risibilis, ergo homo non est risibilis, ubi tamen antecedens est verum et consequens falsum.
		\end{quote}}
\end{quote}

The objection in the \textit{Tractatus Longior} is broken into two parts: the first assumes that an undistributed antecedent that formally entails its undistributed consequent is itself entailed by the same consequent with the term distributed; the second, that the presence of a negation should make no difference to whether an antecedent with an undistributed subject entails a consequent with one.\footnote{The terms `distributed' and `undistributed' originally pertained to the medieval theory of supposition, where they were predicable of terms. A distributed term, e.g. `animal' in `no animal is a stone', is one where the term may be replaced by one denoting any individual or type subordinated to it: hence, from `no animal is a stone', one may infer `no man is a stone', or `Socrates is not a stone'. Later medieval logicians, such as Buridan, analyzed a sentence with a distributed term as equipollent to the conjunction of all propositions naming the successive individuals under that term. E.g. `every man is an animal' would be analyzed as `Socrates is an animal and Plato is...' and so on for all men. 
	
	An undistributed term, conversely, is one which may be replaced by one denoting a type to which it is subordinated. Later medievals analyzed a sentence with an undistributed term as equipollent to the sentential disjunction where the term is replaced by each of its instances. E.g. `a man is running' is equipollent to `Socrates is running or Plato is...' etc. for all men.
	
	Analogously, an antecedent [consequent] comes to be called distributed [undistributed] when the inference it belongs to proceeds to the consequent [from the antecedent] via replacement of the distributed [undistributed] term in the way proper to the supposition had by it, as described above. Cf. \autocite[ch. 1]{Klima1988}.} Burley accepts the first assumption already in the \textit{De consequentiis}, and discusses it under the seventh principal rule of the \textit{Tractatus Brevior} - namely, that a consequence from a distributed superior to its distributed \textit{or} undistributed inferior holds, but not conversely.\footnote{\autocite[pp. 117-118, par. 26-31]{Green-Pedersen1980b}; \autocite[pp. 211.21-212.28]{BurleyDPAL}.} Both of the earlier passages consider objections to the rule, and clarify its proper domain of application in the light of them. But neither mentions the notion of formality.

In the \textit{De consequentiis}, Burley responds to the objection to rule 2.1 by distinguishing between two different ways a consequent can follow from its antecedent: some consequences hold `by reason of its incomplex [parts]' - i.e. the \textit{significata} of its terms; others, by reason of the whole complex.\footnote{\autocite[p. 118, par. 31]{Green-Pedersen1980b}: `Ad aliud dico quod habet intellegi regula ratione incomplexorum. Nunc ista consequentia tenet ratione totius complexi. Quare etc.'} The distribution rule applies to consequences of the former, but not the latter type. Furthermore, the distribution must be applied to the terms on account of which the consequence holds. The \textit{Tractatus Longior} text repeats this, adding that the consequence `Brunellus is capable of laughter, therefore a man is capable of laughter' holds formally `by reason of three terms' - that is, by reason of the containment relations between the \textit{significata} of the three terms `Brunellus' `man' and `risible' here present, and not by virtue of any direct relation between the concepts `Brunellus' and `man'. But the distribution rule only holds for consequences that are formal `by reason of two terms'.\footnote{\autocite[pp. 84.11-85.17]{BurleyDPAL}: \begin{quote}
		Ad aliud, quando probatur, quod ista regula non valet: Quidquid sequitur ex antecedente et consequente etc., quia tunc sequeretur formaliter: Brunellus est risibilis, ergo homo est risibilis. - Dico quod haec consequentia est formalis ratione trium terminorum et non ratione duorum terminorum tantum. Unde mutato praedicato non tenet consequentia. Non enim sequitur: Brunellus currit, ergo homo currit, nec sequitur: Brunellus est risibilis, ergo homo est risibilis. Et ideo haec consequentia: `Brunellus est risibilis, ergo homo est risibilis' non tenet propter aliquam habitudinem inter Brunellum et hominem, sed propter habitudinem inter istos tres terminos, scilicet `Brunellum', `hominem' et `risibile' sic in propositionibus ordinatos. Unde sciendum, quod quaedam consequentiae tenent ratione totius complexionis, ut conversiones syllogismi et huiusmodi, et quaedam tenent ratione incomplexorum; et talium consequentiarum quaedam tenent ratione duorum terminorum tantum, ut haec: `Homo currit, ergo animal currit', et quaedam ratione trium terminorum, ut `Brunellus est risibilis, ergo homo est risibilis', et quaedam ratione quatuor terminorum, ut haec: `Homo currit, ergo animal movetur'. Dico ergo, quod haec consequentia: `Brunellus est risibilis, ergo homo est risibilis', est bona et tenet ratione trium terminorum, ut dictum est.

		Et quando dicitur, quod tunc distributo consequente sequitur antecedens vel sequitur antecedens distributum. - Dico, quod haec reguma non est generalis, quia ubi consequentia tenet ratione totius complexionis vel ratione plurium terminorum quam duorum non habet ista regula locum. Quamvis enim sequatur: Animal est homo, ergo homo est animal; tamen ad distributionem consequentis non sequitur antecedens distributum. Non enim sequitur: Omnis homo est animal, ergo omne animal est homo, et hoc, quia prima consequentia tenuit ratione totius complexionis. Nec oportet universaliter, quod distributo consequente sequitur antecedens sine distributione. Quamvis enim sequatur: Homo currit, ergo animal currit, non tamen sequitur: Omne animal currit, ergo homo currit. Dico ergo, quod haec regula: Quando ad antecedens sequitur consequens, distributo consequente sequitur antecedens distribitum vel etiam sine distributione, est intelligenda, quando consequentia tenet ratione duorum terminorum tantum et signum distributivum additur in consequente illi termino, ratione cuius consequens infertur ex antedente. Et si terminus in antecedente, ratione cuius infertur consequens, sit distribuibile, tunc ad consequens distributum sequitur antecedens distributum. Si vero terminus in antecedente, ratione cuius infertur consequens, non sit distribuibile, tunc ad consequens distributum sequitur antecedens sine distributione; quia sequitur: Sortes currit, ergo homo currit; et iste terminus `Sortes' non est distribuibilis, ideo sequitur: Omnis homo currit, ergo Sortes currit.
	\end{quote}} Thus, while both texts distinguish consequences holding by virtue of their parts from those holding by virtue of their structure, the later text further distinguishes among the former type according to the number of terms it holds by, and only the later text calls these different ways of following \textit{formally}.

For Burley, consequences which hold in virtue of their whole structure include conversions, syllogisms \autocite[p. 86.9-12]{BurleyDPAL}, arguments from an exclusive to a universal transposing its terms and vice versa,\footnote{Burley's example is `every man is an animal, therefore only an animal is a man' \autocite[pp. 142-143, par. 118]{Green-Pedersen1980b}.} and presumably consequences such as obversion, contraposition, and the immediate inferences found in the square of opposition. These are contrasted with consequences holding formally in virtue of their terms. What Burley describes by `formal consequence holding in virtue of its simple [parts]' consists chiefly of two things: the first, standard quantificational rules for descent to singulars from quantified common nouns, and for ascent from singulars to the nouns they fall under; the second, rules for replacing terms in an antecedent with terms containing or contained by them in the consequent, what Hodges calls Burley's `calculus of monotonicity' \autocite[pp. 97-99]{HodgesBurley}. `A man runs, therefore an animal runs' is said to hold by two terms; `Brunellus is capable of laughter, therefore, a man is capable of laughter', by three; `a man runs, therefore an animal moves', by four \autocite[p. 84.24-27]{BurleyDPAL}.

From the above, one might infer that a consequence holding in virtue of two terms would be one where those terms are added to its list of invariant parts, its other constants remaining variable. This approach to formal consequence is explicitly championed in the 19th century by Bernard Bolzano \autocite{George1986}, and captures an important part of Burley's practice with respect to consequences holding in virtue of two terms: when Burley wishes to show that a proposition fails to hold formally in two terms, he varies the third to find a counterexample. For instance, `Brunellus is capable of laughter, therefore a man is capable of laughter' with `capable of laughter' as variable part admits the counterexamples `Brunellus is running, therefore a man is running', and `Brunellus is a braying animal, therefore a man is a braying animal' \autocite[84.13-15]{BurleyDPAL}. 

But the problem with ascribing this approach to Burley becomes clear when one considers consequences said to hold in virtue of all of their categorematic terms. As Tarski put it, 
\begin{quote}
	The extreme would be the case in which we treated all terms of the language as logical: the concept of following formally would then coincide with the concept of \textit{following materially}--the sentence $X$ would follow from the sentences of the class $\mathfrak{K}$ if and only if either the sentence $X$ were true or at least one sentence of the class $\mathfrak{K}$ were false \autocite[pp. 188-189]{Tarski2002}.
\end{quote}

In the course of its reply, the \textit{Tractatus Longior} records a third objection making the above Tarskian assumption: a consequence where all terms are held fixed would have to be a material one.\footnote{\autocite[p. 86.4-9]{BurleyDPAL}. The main differences between Tarski's remark and the objection to Burley are two: first, the determination of which parts are fixed takes place at the level of the individual consequence for Burley's objector, but of the language for Tarski; second, Burley's notion of following requires not merely the truth [falsity] of the antecedent [consequent], but its being necessarily so.} 

The proper interpretation of the objection, however, depends on what is meant by `material consequence'. On one reading, a consequence is material in that its antecedent is impossible, or its consequent necessary. This is a modalized variant on Tarski's claim above, and accords with the examples of material consequence given in Ockham's \textit{Summa Logicae}.\footnote{\autocite[III-3. 1, p. 589]{OckhamSL}. Cf. \autocite[p. 7, par. 18]{Green-Pedersen1980a}.} On another reading, a material consequence is one proceeding by application of a rule that would normally be inadmissible, but is admitted in a given context because of the terms in the inference. This account of material consequence is that present in Simon of Faversham \autocite[q. 36]{FavershamQE}, and a clear statement of the view is found in Ockham's late \textit{Elementarium Logicae}:

\begin{quote}
	A material consequence is [one] which does not hold by virtue of the mood of the argument, but thanks to the terms it is composed from. `An animal debates, therefore a man debates' follows in this way, not because there is a good consequence from the superior to the inferior, but because the predicate `debates' cannot accord to any animal besides man.\footnote{\autocite[VI. 4, p. 163]{OckhamEL}: \begin{quote}
			Consequentia materialis est quae non tenet virtute modi arguendi sed gratia terminorum ex quibus componitur. Sicut sequitur `animal disputat, ergo homo disputat', non quia `a superiori ad inferius sit consequentia bona', sed quia istud praedicatum `disputare' non potest competere alicui animali nisi homini.
		\end{quote} The editors of the volume regard Ockham's authorship as doubtful. But the reasons for doubting its authenticity are not strong, mostly based on perceived discrepancies between the views in the text and those expressed in Ockham's \textit{SL}. The external evidence in favor of authenticity is stronger and more concrete. For the case for Ockham's authorship of the work, see \autocite{Boehner1958b}.}
\end{quote}

Ockham's example here is structurally analogous to the objection found in Burley, where the failure of the consequence with the predicate changed is used to press the non-formal character of the initial consequence. It seems likely to me that the objection as Burley construes it is working with an understanding of formal consequence closer to that of Faversham and the \textit{Elementarium} than that of Tarski. Burley replies:
\begin{quote}
	For a consequence to hold in virtue of its terms is twofold: either because it holds materially by reason of its terms, or because it holds formally by reason of its terms, that is, from the formal measure (\textit{ratione formali}) of the terms. I say then, that a consequence can be formal by reason of its terms, and this if it holds \textit{per se} by reason of its terms. If, however, it holds by reason of its terms accidentally, then it is not formal.\footnote{\autocite[86.15-21]{BurleyDPAL}: 
		\begin{quote}
			Unde aliquam consequentiam tenere ratione terminorum potest esse dupliciter, vel quia tenet materialiter ratione terminorum, vel quia tenet formaliter ratione terminorum, hoc est ex ratione formali terminorum. Dico tunc, quod consequentia potest esse formalis ratione terminorum, et hoc si per se teneat ratione terminorum. Si vero teneat ratione terminorum per accidens, tunc non est formalis.
		\end{quote}}
\end{quote}
That Burley rejects Tarski's conclusion shows he rejects the reduction of formal consequence to a consequence's holding good under all substitutions for terms, even in the nuanced form one finds in Bolzano.\footnote{\textit{Pace} \autocite[pp. 83-84]{HodgesBurley}.}

More importantly, the above provides a way of relating the formal/material division to Burley's earlier natural/accidental division. By a consequence holding \textit{per se}, Burley means the same as one where the meaning of the consequent is contained in the antecedent \autocite[p. 158, par. 160]{Green-Pedersen1980b}. This is just the containment criterion found in his definition of natural consequence.\footnote{Cf. \autocite[I, d. 11, q. 2]{ScotusRepPar}.} In contrast, the above response states that a consequence holding in virtue of its terms accidentally is not formal. And as the context makes clear, Burley here intends that such a consequence be understood as a material one.

Thus, for Burley we have 1) formal consequences holding in virtue of their whole complex, including conversions and syllogisms; 2) formal consequences which hold in virtue of their terms, such as `if a man is an animal, a man is a substance'; 3) accidental consequences holding in virtue of their terms, such as `God exists is true, therefore God exists is necessary'; and 4) another class of accidental consequences including instances of the rules \textit{ex impossibili quodlibet} and \textit{necessarium ad quodlibet} \autocite[pp. 128-129, par. 70]{Green-Pedersen1980b}. The above types are listed in order of their strength, with consequences belonging to the first class being the strongest, and those of the last class the weakest. The goodness of consequences holding by their structure is immediate and necessary. Consequences holding by virtue of their terms are enthymematic consequences exhibiting a \textit{per se} containment relation between the terms on whose account the consequence is good. These should be reduced to a syllogism or syllogisms by the assumption of an additional premise or premises as the case requires \autocite[p. 142, par. 117]{Green-Pedersen1980b}. This class may be further divided into consequences holding by virtue of $n$ terms, for any $n$. The third are consequences holding good by a restricted case of a rule that is not normally good, but is allowed in a specific context: `God exists is true, therefore God exists is necessary' holds as an accidental consequence because in the context of the antecedent, the term `true' supposits for the same as what the term `necessary' supposits for, i.e. God's being, thus permitting an otherwise impermissible substitution.\footnote{Burley's point is lost in the most straightforward English reading of \textit{Deum esse est verum, ergo deum esse est necessarium}. i.e. that where \textit{Deum esse} is taken for the sentence `God exists'. To preserve Burley's point, one might better translate the infinitive clause with a gerundive expression, as in `God's being is true, therefore God's being is necessary'.} The final sort hold strictly by the extrinsic topic \textit{from the less}. The first two of the above kinds are called natural consequences, and are said to hold by an intrinsic topic; the latter two, material consequences holding by an extrinsic topic.\footnote{Cf. \autocite[p. 130]{Martin2004}.} Not all consequences holding in virtue of their terms are formal: there are accidental consequences which also hold in virtue of their terms.

\section{Burley's place in the development of consequence}
\subsection{Burley's work among the earliest treatises on consequences}
Burley's \textit{De consequentiis} is among the three earliest extant treatises on consequences. The other two - one extant in a 1302 London manuscript, the other housed in Paris - are anonymous, and have been edited by N. J. Green-Pedersen.\footnote{\autocite{Green-Pedersen1980a}. Translations of these and Burley's treatise are found in \autocite{Archambault2017d}.}

A strong case can be made that Burley knew the text in London, BL, Royal 12 F XIX, ff. 111ra-112rb early on. The manuscript containing the only extant copy of the anonymous work also contains Burley's work at ff. 116ra-122rb. More conspicuous is Burley's treatment of an example found in the anonymous London treatise. In discussing the rule that in a good consequence, the opposite of the consequent cannot stand with the antecedent, the London treatise makes an exception for consequences whose antecedents include opposites, and uses the proposition `no time is' as an example of a statement including opposites. It then gives the following argument:
\begin{quote}
	If no time is, it is not night, and if it is not night, it is day; and if it is day, some time is. Therefore, if no time is, some time is.\footnote{\autocite[p. 7, par. 18]{Green-Pedersen1980a}: 
		\begin{quote}
			Probatio, quia si nullum tempus est, nox non est, et si nox non est, dies est, et si dies est, aliquod tempus est. Ergo si nullum tempus est, aliquod tempus est.
		\end{quote}}
\end{quote}
Burley visits the same example in his discussion of the rule \textit{from the first to the last} (i.e. transitivity). He writes: 
\begin{quote}
	if one argues `if no time is, it is not day; and if it is not day, and some time is, it is night; and if it is night, some time is; therefore \textit{from the first to the last}: if no time is, some time is' - this consequence does not hold from the first to the last, since the consequent of the first conditional is `it is not day', and the antecedent of the second conditional is the whole `it is not day, and some time is'.\footnote{\autocite[pp. 114-115, par 8]{Green-Pedersen1980b}: \begin{quote}
			[S]i sic arguitur `si nullum tempus est, dies non est; et si dies non est, et aliquod tempus est, nox est; et si nox est, aliquod tempus est; ergo a primo ad ultimum: si nullum tempus est, aliquod tempus est.' Haec consequentia non tenet a primo ad ultimum, quia consequens primae condicionalis est `dies non est,' et antecedens secundae condicionalis est hoc totum `dies non est, et aliquod tempus est.'
		\end{quote}}
\end{quote}
Where the anonymous treatise uses the example to prompt an exception to a standard rule, Burley diffuses the example by arguing that it rests on an equivocation.
	
Though Burley's \textit{De consequentiis} has parallels with the treatise of Paris, BN lat. 16130, these parallels are mostly common rules, which are insufficient to establish any dependency of Burley's text on it. Comparing their style and content, the Parisian treatise is more terse, and has a marked preference for the active voice. Its proofs tend to be more detailed, and the connection to supposition theory is more pronounced. The Parisian treatise shows a clear grasp of the difference between downward monotonicity and descent to singulars \autocite[p. 12, par. 2]{Green-Pedersen1980a}. Whether the London author grasped this difference is less clear. In both works, the treatments of upward/downward monotonicity and ascent/descent from singulars are parallel.
	
The anonymous Parisian \textit{De consequentiis} does, however, contain a substantial parallel to an objection to rule 2.1 from the \textit{Tractatus Longior}, discussed above. In the former, we find: 
\begin{quote}
	One should know this rule is invalid: \textit{whatever entails the consequent entails the antecedent}, as is shown if it is so argued: if a [thing] capable of laughter is an ass, then a man is an ass; for here by the rule \textit{whatever entails the consequent entails the antecedent}, one argues that `a [thing] capable of laughter is an ass' entails `an animal is an ass', and thus `a [thing] capable of laughter is an ass' also entails `a man is an ass'. And it is so argued by this rule: \textit{whatever entails the consequent entails the antecedent} since `a [thing] capable of laughter is an ass' entails the consequent `an animal is an ass'.\footnote{\autocite[p. 16, par. 21]{Green-Pedersen1980a}: \begin{quote}
			Sciendum quod ista regula non valet: quicquid antecedit ad consequens antecedit ad antecedens. Sicut patet si sic arguitur `si risibile est asinus, homo est asinus'; hic enim arguitur per istam regulam: quicquid antecedit ad consequens antecedit ad antecedens, quia ista `risibile est asinus' antecedit ad istam `animal est asinus', et ideo ista `risibile est asinus' antecedit ad istam `homo est asinus'. Et ita arguitur per hanc regulam: quicquid antecedit ad consequens antecedit ad antecedens, quia ista `risibile est asinus' antecedit ad hoc consequens `animal est asinus'.
		\end{quote}}
\end{quote}
In Burley's treatise, we find the terms transposed and the common term `\textit{asinus}' replaced with Brunellus, a proper name for a donkey. In both texts, what is resisted is the inference from `an ass [Brunellus] is capable of laughter' to `an ass [Brunellus] is a man'. Burley himself will allow for a sense in which the inference is formally good - namely, by reason of three terms. The anonymous text, like the objection in \textit{De Puritate}, denies this, resisting the inference from a proper accident to its bearer in contexts where \textit{per impossibile}, the proper accident is predicated of something different from its standard bearer.\footnote{Cf. \autocite[q. 11, par. 19]{ScotusQE}.} Neither text subsumes the inference under the rule \textit{ex impossibili quodlibet}. Given the closeness of the objection in Burley's text to the stance in the anonymous Parisian treatise, it is possible that the latter was a source for the objection as it appears in Burley.	
	
The most fruitful comparison with both treatises comes in their approach to suppositional descent, and the corresponding difficulties brought about regarding existential import. None of these treatises adopts the approach to existential import - straightforwardly found in John Buridan, and  attributed by Stephen Read to Aristotle himself\footnote{\autocite[q. I. 38]{BuridanPostAn}, \autocite{Klima2001}, \autocite{Read2015b}.} - according to which affirmative propositions have existential import while negative ones lack it. Rather, whether a proposition presupposes the existence of its supposita is dependent on the mode of predication, and hence on the mode of inherence exhibited by the things named by the terms. \textit{Per se} predication is that wherein a constitutive property is predicated of what is constituted by it, as in `a man is an animal'. \textit{Per accidens} predication involves either the predication of an accident of a substance, as in `a man is white' or `a man is capable of laughter'; or of some quality, be it essential or accidental, itself of an accident, as in `to run is to move'.\footnote{This conflation of accidental predication with \textit{per se} predication of an accident is present in both anonymous texts. See \autocite[pp. 10-11, par. 35-36; 25, par. 66]{Green-Pedersen1980a}.} In one medieval approach, found in both Simon of Faversham and Duns Scotus, \textit{per se} predication does not presuppose existential import, but \textit{per accidens} does.\footnote{See \autocite[q. I. 56]{FavershamQE}; \autocite[I. qq. 5-8. par. 49, 74]{ScotusPeriHerm}; \autocite[q. 11, par. 19]{ScotusQE}. More recently, this approach has been suggested by \autocite{CIFOL1}.} From these approaches to predication, corresponding notions of ascent and descent were developed. A descent from `man' to `white man', for instance, was called \textit{from a superior to inferior per accidens}; from `animal' to `man', \textit{from a superior to inferior per se}. 
	
Both anonymous authors require all propositions where the same is predicated of itself to come out true, including predications involving accidents, such as `a white man is a white man' and those involving impossible objects, such as `a chimera is a chimera' \autocite[p. 8, par. 23; p. 11, par. 36-37; p. 18, par. 31-32]{Green-Pedersen1980a}. Because these propositions are given separate treatment, many ascents and descents to and from such propositions are blocked: one cannot, for instance, ascend from `a white man is a white man' to `a white man is a man', because the latter proposition implies the existence of a (white) man, while the former does not. 
	
Relative to these treatises, Burley's principal contribution seems to have been the rejection of special treatment for statements predicating the same of itself, and with it a more uniform application of ascent and descent rules in \textit{per accidens} predications.\footnote{See \autocite[pp. 116-117, par. 19-20; 134, par. 95; p. 158, par. 160.]{Green-Pedersen1980b}.} On the resulting approach, \textit{per se} ascent and descent work as follows: A universal affirmative categorical proposition is downwardly monotonic in its first term and upwardly monotonic in its second \autocite[p. 211.16-20]{BurleyDPAL}; the opposite is the case for a particular negative categorical; a particular affirmative is upwardly monotonic in both terms; a universal negative, downwardly monotonic in both. Descent \textit{per accidens} (e.g. from `no man is an animal' to `no white man is an animal') holds where descent \textit{per se} does in negatives \autocite[pp. 209.35-210.10]{BurleyDPAL}. Ascent from a \textit{per accidens} inferior to its \textit{per se} superior holds where \textit{per se} ascent does in affirmatives \autocite[pp. 116-117, par. 20]{Green-Pedersen1980b}. Excepting Burley's treatment of propositions predicating the same of itself, the above analysis appears to have been lifted from the London anonymous treatise.\footnote{Cf. \autocite[pp. 10-11, par. 35-37]{Green-Pedersen1980a}, \autocite[pp. 116-117, par. 19-20]{Green-Pedersen1980b}.} Lastly, Burley recognizes that where a class or object descended to can be empty, \textit{per se} descent for affirmatives can fail. But Burley's practice is inconsistent on this point:\footnote{See esp. \autocite[pp. 61.4, 85.16; 85.4, 211.27-28]{BurleyDPAL}. Cf. \autocite{Mora-Marquez2015}.} instead, he employs these descents freely, generally ignoring existential complications \autocite[pp. 23.26, 26.26, 31.21, 67.19, 67.30, 85.16, 85.26, 211.27-28]{BurleyDPAL}.
	
	%Burley's treatment of impossible antecedents: 70.26-72.3
	
	%Cf. 85.3-5 to 211.27-28; also, 218.7-8 to DC par. 19
	
	%17, 19-20, 23-25, 37, 42-43, 95, 102, 120, 155-156, VIII.188, X.69, 
\subsection{Relating Burley's work to that of Ockham and Buridan}
\subsubsection{Disagreements over valid consequences}
Burley's disagreements with Ockham are well-known, with the presence or absence of barbs towards Ockham playing a role in dating the works of Burley's corpus \autocite{Ottman1999,Vittorini2013}. Ockham borrowed liberally from Burley's logic \autocite{Brown1972}, and the parallels between consequences countenanced in Ockham's \textit{Summa Logicae} and Burley's earlier treatises are too extensive to be enumerated.\footnote{The consequences found in \autocite[III-3.38, pp. 727-731]{OckhamSL}, for instance, can be found at \autocite[par. 1, 4, 9, 14-15, 71-72, and 86-88]{Green-Pedersen1980b}.} Though Burley and Ockham's underlying accounts of supposition differ \autocite{Wagner1981}, I've not found substantial disagreements in the consequences they accept or reject.

Buridan's engagement with Burley is less known, but extensive. In his ontology, Buridan adopts Burley's reduction of the real Aristotelian categories to three - namely, substance, quality, and quantity.\footnote{Cf. \autocite[pp. 204]{Klima2009}; \autocite[p. 155]{Read2016b}; \autocite{DutilhNovaes2013}.} In logic, Buridan had direct knowledge at least of the \textit{Tractatus Brevior}: rules 3, 4, and 5 of Buridan's \textit{Treatise on Consequences} are rules 3, 2, and 1.1/1.2 of the \textit{Tractatus Brevior}; Buridan's sixth rule is a restricted instance of Burley's rule 2.1, and Buridan references the content of the eighth rule of Burley's shorter treatise in the eighth rule of the first book of his own treatise on consequences.\footnote{Cf. \autocite[p. 212.29-31]{BurleyDPAL}; \autocite[I. 8]{BuridanTC}.}

One substantial disagreement between Buridan and Burley is over consequences moving between sentential and term negation. Unlike Buridan's, Burley's treatment of term negation does not normally deviate from his treatment of sentential negation. `Socrates is non-white' and `Socrates is not white' have the same truth conditions, with both being true when Socrates does not exist \autocite[pp. 57.17-58.12; 215.6-21; 216.15-18]{BurleyDPAL}. Burley thus accepts the standard rules for obversion and contraposition, both from affirmatives to negatives and \textit{vice versa}, albeit with restrictions on the kinds of terms that can occur therein \autocite[pp. 129-131, par. 73-81]{Green-Pedersen1980b}. This contrasts with Buridan's view, on which the inference from an affirmative to its contraposited or obverted negative is good, but not conversely.\footnote{\autocite[p. 85]{Buridan2015}. The root difference consists in their different treatments of the range of term negation: Burley takes term negation to range over both beings and non-beings; Buridan assumes it only ranges over existent entities.}

\subsubsection{The place of consequences within Burley's vision of logic}
Ockham's treatment of consequence in the \textit{Summa Logicae} does not begin with general rules, but with rules for suppositional ascent and descent for the various qualities and quantities of assertoric, then modal and other kinds of propositions. There, Ockham does not discuss general rules for consequences until the thirty-eighth chapter of his work,\footnote{This was originally the final chapter of the section on consequences. See \autocite[pp. 41*-43*]{OckhamSL}.} and each of the rules mentioned therein is already explicit in Burley's much earlier \textit{De consequentiis}. Thus, as with Burley's earlier \textit{De consequentiis}, general consequences are not yet brought to the fore in Ockham - Burley does not discuss general rules until the last section of the treatise, beginning at paragraph 145. And as with Burley's \textit{Tractatus Brevior}, the differentiation of consequence from supposition theory in Ockham is less pronounced.
	
The divisions and order of Ockham's treatise follow those understood to govern Aristotle's \textit{Organon}. The first two parts of Ockham's treatise, on terms and propositions, address simple concepts and judgment, the subject matters of Aristotle's \textit{Categories} and \textit{On Interpretation}. The third part, broadly concerned with reasoning, discusses the subjects of Aristotle's \textit{logica nova} texts: the forms of reasoning (\textit{Prior Analytics}) and their implementation in demonstrative (\textit{Posterior Analytics}), dialectical (\textit{Topics}), and fallacious argument (\textit{On Sophistical Refutations}).\footnote{Cf. \autocite[prol.]{AquinasPA}} Within this ordering, Ockham's treatise on consequences occupies the place in the \textit{Summa Logicae} corresponding to that of Aristotle's \textit{Topics} in the standard medieval ordering of the Organon. Thus, though the content of Ockham's thought is often radical, the structure wherein he presents it is a deliberately traditional one. In particular, Ockham does not conceive consequences as encompassing traditional syllogistic inference.
	
By contrast, Burley already includes syllogistic under consequence in his \textit{De consequentiis}.\footnote{\autocite[pp. 131-132, par. 82-85]{Green-Pedersen1980b}. Also \autocite[p. 219.19-32]{BurleyDPAL}.} And both the intended structure of Burley's earlier and the actual structure of his later \textit{De Puritate} betray a different understanding than Ockham's of the place of consequences in the Aristotelian curriculum.
	%begin again here. 
According to the prologue of the \textit{Tractatus Brevior}, Burley's plan was to treat in sequence: 1) common rules for the remainder of the work; 2) sophisms, 3) \textit{obligationes}, and 4) demonstration \autocite[p. 199]{BurleyDPAL}. The preface of the earlier \textit{De Puritate}, and the intended ordering of its parts, betray a basic concern with the form of reasoning, which is then examined in the different contexts wherein it is utilized: sophisms corresponding to the traditional subject of the \textit{Sophistical refutations} \textit{obligationes}, to that of the \textit{Topics}, and demonstration to that of the \textit{Posterior Analytics}.\footnote{Cf. \autocite[prol.]{AquinasPA}. The ordering of the \textit{logica nova} materials suggested by the prologue of Burley's earlier treatise, with the \textit{Prior} and \textit{Posterior Analytics} separated by the \textit{Elenchi} and \textit{Topics} (under which Ockham also discusses the \textit{ars obligatoria}), is less common than that suggested by Ockham or Aquinas. That it was nevertheless used is clear from several mss. recorded in the \textit{Aristoteles Latinus} index, including Metz, \textit{Bib. Mun.} 508; Arras, Bib. Mun. 362 (451); Chambery, Bib. Mun. 27. The same separation is found in the ordering of the treatises of Buridan's \textit{Summulae}.} As such, the assimilation found in Burley's earlier \textit{De Puritate} is not to Aristotle's \textit{Topics}, but to the \textit{Prior Analytics}. Its ambition is to provide an account of the forms of reasoning that instead of reducing all reasoning to syllogistic,\footnote{This reduction remains present in Buridan's \textit{Summulae}. See \autocite[6.1.5, pp. 398-400]{BuridanKlimaSD}.} takes up syllogistic as part of a broader work centered on consequence. Burley thus appears to be the first logician to attempt a unified account of consequence including syllogistic as a proper part. It is this vision, rather than Ockham's, which is later adopted in the structure of Buridan's \textit{Treatise on Consequences}.
	
	%Burley's Logic of 1302: 1. De suppositionibus 2. De exceptivis 3. De exclusivis 4. De consequentiis 5. De insolubilibus 6. De obligationibus. Mention that Burley's \textit{Tractatus Brevior} maintains the order of materials in the \textit{Logic of 1302}. On this, see Ottman/Wood. Also, De Rijk, which is issue 1 of its volume.
	
This vision remains in the later version of Burley's treatise, albeit with an important revision. The intended inclusion of a treatise on supposition to end the first part of the \textit{Tractatus Brevior} suggests that Burley earlier regarded supposition theory as part of a broader theory of inference, perhaps supplementing the kinds of inference-justifying rules one finds in topical treatises. This ordering agrees with that found in the \textit{Summulae} of Peter of Spain.\footnote{Cf. \autocite{Klima2003a}.} In Burley's later treatise, by contrast, the part on supposition begins with the words `having laid down the signification of incomplex terms, in this tract I intend to examine certain properties of terms which only belong to them according to their being parts of a proposition' \autocite[p. 1.3-6]{BurleyDPAL}. Now, the signification of incomplex terms was the standard topic associated with the \textit{Categories}.\footnote{See \autocite[p. 65]{Burley2003}.} The mention of the proposition makes clear the association of the treatise on supposition with the \textit{On Interpretation}. Thus, it seems that by the time of the later treatise, Burley thought it more appropriate to treat supposition in closer connection with the content of the \textit{On Interpretation}, in the broader context of subjects discussed under Aristotle's old logic, rather than in connection with discussions of inference traditionally treated under Aristotle's new logic. This relocation provides us with a reason why Burley may have abandoned the \textit{Tractatus Brevior} independently of a need to respond to Ockham's logic: he sought to reorganize his materials to reflect this new understanding of the place of supposition theory within logic, which required him to disentangle supposition theory from the theory of consequence and therefore to rewrite much of his old material to reflect this new organization. This relocation of the treatise on supposition is the one later followed by Buridan's \textit{Summulae}.

\section{Conclusion}
The above reveals a wide array of achievements attributable to the \textit{Doctor Planus et Perspicuus}, which help us better understand his contribution to the development of medieval \textit{consequentiae}.
	
Burley's \textit{De consequentiis} provides the most expansive known treatment of consequences prior to Ockham's \textit{Summa Logicae}. Burley streamlines the treatment of consequences found in the earliest treatises by leaving aside the rule that a proposition predicating the same of itself must always be true. And by providing a parallel treatment of sentential and term negation, he preserves both affirmative-to-negative and negative-to-affirmative directions of obversion and contraposition.
	
In the shorter \textit{De Puritate}, Burley provides the earliest known organization of warrants governing consequences into principal and derivative rules. In the same treatise, he makes explicit that identity statements may fail in cases where the subject does not exist, qualifies the range of the rule \textit{from the impossible anything follows}  (since the less impossible does not entail what is more impossible), and provides a brief treatment of consequences involving modalities. In the movement from the shorter to the longer \textit{De Puritate}, Burley brings supposition theory from a location between topics and syllogisms to one in the context the theory of the proposition, a place it retains in the work of John Buridan.
	
By the time of the longer treatise, Burley has greatly streamlined the treatment of principal and derivative rules first attempted in the \textit{Tractatus Brevior}, and expanded the range of dialectical maxims used in topical arguments to include an indefinite number of \textit{logical} maxims, to be used in consequences. In the same treatise, Burley relates the contrast between natural and accidental consequences found in the \textit{De consequentiis} to that between formal and material consequences, and further subdivides enthymematic formal consequence according to the number of terms they hold by. Burley provides a plurality of different understandings of `follows' with differing levels of strength 
	
There are some failings in Burley's theory. The ambiguity between the different senses of \textit{per accidens} ascent and descent likely contributed to the distinction between natural and accidental consequence falling out of favor. Though Burley recognizes that both downward monotonicity and descent to particulars may fail because of complications involving existence, his practice ignores this insight. The subdivision of enthymematic formal consequences by number is underdeveloped, and without further clarification it is hard to see how some of these consequences should be distinguished from material consequences. 
	
However, the basic import of Burley's work remains: Burley provides a plurality of different understandings of `follows', with differing levels of strength, grounded in the strength of the relations involved in the \textit{significata} in the consequences: natural structural consequences grounded in the meaning of their syncategorematic terms, natural enthymematic consequences grounded in intrinsic containment relations between the significates of their categorematic terms, and accidental consequences grounded on weaker relations. Today's more pluralist logical environment, aiming to give expression to different kinds of relevant containment while still allowing for classical consequence, is much friendlier to Burley's project than it is to the reductive monism that followed in Buridan's reduction of all consequences to formal ones - a reduction mirrored in the classical approach to formal consequence that dominated the last century. As such, now provides an opportune environment for a revival of interest in Burley's logical work, and Burley's work may provide inspiration for a more fruitful grounding of pluralism than those popular at present. At the same time, an examination of the logic of Burley the realist provides much of interest even for understanding the nominalists opposed by - and indebted to - him.

%check HodgesBurley quotations
\end{document}
