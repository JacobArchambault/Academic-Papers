\documentclass[]{article}
\usepackage[utf8]{inputenc}
\usepackage{graphicx}
\usepackage[backend=biber]{biblatex}
\usepackage{lplfitch}
\usepackage{amssymb}
\usepackage{parallel}
\usepackage[all]{xy}
\usepackage{bussproofs}

\bibliography{jacob}

%opening
\title{The development of the medieval Parisian account of formal consequence \\ summary}
\author{Jacob W. Archambault}

\begin{document}

\maketitle

\begin{abstract}

\end{abstract}

\section{Formal consequence and formal logic}
Logic is commonly singled out for special consideration among the sciences by the dictum `logic is formal'. We speak about formal rigor, formalized languages, formal consequence, formal methods - indeed, only the other branches of mathematics come close to logic in the degree to which they are stamped with the language of formality. Other areas of inquiry are sometimes even considered formal to the degree that they incorporate logic into their methodology - formal epistemology, for instance, is an approach to epistemology heavily reliant on the use of logical apparatus.

The formality of logic is especially present in discussions of logical consequence. This is in part because the subject matter of logic itself is often taken to be \textit{what follows from what} - not, of course, in any sense whatsoever, but as a matter of logical \textit{form}; and in part because as modifiers of `consequence', `logical' and `formal' are frequently taken to be synonymous\footnote{See \cite[188, 193]{Tarski2002}. In what follows, I use `formal' and `logical' interchangeably - not to express agreement with this use, but to address the positions exposited on their own terms.} - a synonymy that, if applied consistently, would transform `logical form' and `formal logic' into the emphatically redundant `formal form' and `logical logic'. In short, the very parlance of logicians suggests the concept of form is at the center of logic.

This dissertation provides an account of how this came about, detailing the development of the notion of formal consequence at the time of the appearance of the first treatises on consequence - the first half of the 14th century - and culminating in the account advanced by John Buridan. Buridan's work provides a convenient focal point from both historical and theoretical perspectives: theoretically, because of its close resemblance to the model-theoretic accounts dominant today; historically, because of its lasting influence on treatments of formal consequence up to the advent of modernity.

\subsection{Common elements in modern accounts of formal consequence}
A surprising degree of unity underlies the developments and diversity across post-Tarskian accounts of formal consequence. Partisans of various contemporary accounts typically presuppose 1) that for a consequence to be logical and for it to be formal amount to the same thing.\footnote{For an exception, see \cite{Read1994}.} 2) All post-Tarskian accounts accord a place of prominence to substitutionality, though in different ways. 2a) In the model-theoretic approach, valid consequences are determined by varying the interpretation of the non-logical components of a formalized language, or otherwise by varying those components themselves.\footnote{This is strictly true for the accounts of consequence found in \cite{Tarski2002} and presupposed in the metaphysical projects of \cite{Quine1948} and \cite{Lewis1968}. Later model-theoretic approaches take variability a step further, by allowing variations on the size and elements in the domain, on the set of possible worlds, etc. But when this is done, the invariance of consequence under permutation of non-logical terms becomes only a necessary condition for its holding formally.
	
	The sense of `substitutional' used above is wider than that used to distinguish substitutional from objectual semantics for first-order languages. A substitutional semantics in the more restricted sense is one on which the truth value of its quantified formulae in a model is determined by the truth value of instances of those formulae wherein the formerly bound variables have been replaced by new terms. An objectual interpretation, by contrast, is one on which it is not the terms, but the objects assigned to the variables that are varied. Typically, an objectual semantics is preferred on the grounds that substitutional semantics is not consistent with the intent to quantify over superdenumerable domains, e.g. the real numbers. But from a purely mathematical standpoint, the class of substitutional models can be represented as a subset of the objectual ones, i.e. those where the domain of the model is just the set of terms in the language. See \cite[ch. 14]{Garson2013}.} 2b) in the proof-theoretic approaches surveyed, substitution shows up in a less explicit manner, in the assumption that formally valid consequences hold schematically.\footnote{Cf. Dutilh Novaes (2011). Both Prawitz and Schroeder-Heister make this more explicit than usual: Schroeder-Heister by his use of propositional quantification in his interpretation of formulas as rules; Prawitz doubly so, by his distinction between open and closed arguments, and by his restricting logically valid arguments to those that hold in \textit{every} system of canonical arguments.} 3) In all model-theoretic accounts, substitutionality is taken not merely for a condition on consequence, but rather defines \textit{what it is to be} a formal consequence.\footnote{Cf. \cite[66]{Etchemendy1988}: \begin{quote}
		...as far as extensional adequacy goes, there are a multitude of equally correct (or equally incorrect) definitions of first-order consequence: when we specify any one of the many equivalent proof procedures for first-order languages, we have defined the consequence relation as adequately as when we define the relation model-theoretically. But from among these coextensive definitions, the model-theoretic account is account is typically afforded a special status, a status most clearly reflected in soundness and completeness theorems.
	\end{quote}} 
	And in the Tarski-Quine-Lewis tradition, one finds the assumptions that 4) that precisely those notions which are required to be invariant under all interpretations are the logical notions of a given language; 5) that a consequence is valid \textit{in virtue of} these notions, and it is on account of these that a consequence has its logical form; 6) that a logic is individuated by its class of logical notions; and 7) that, accordingly, without a principled and sharp demarcation criterion for discriminating between the logical and non-logical components of a formal language, we also lack an adequate understanding of the scope and nature of logic \cite{MacFarlane2009}.
	
	To get a better grasp on some of the above points, it is worth reflecting on what formality in logic is most likely to be contrasted with. On the one hand, the formal is said to be the opposite of the \textit{in}formal. In this sense, formality is typically associated with rigor on account of its use symbolization, itself in the service of obscuring the meaning of the matters to which it is applied for the sake of making these formulae more easily or even effectively calculable \cite[321-325]{DutilhNovaes2011}. And so the spirit of informal logic would be typified by an approach to logic working in or otherwise heavily reliant on natural (as opposed to formalized) language, and one making use of the meanings of the terms it treats in determining what follows from them. Such an approach is found, for instance, in the ordinary language tradition of Ryle.
	
	On the other hand, the formal is contrasted with the \textit{material}. In this hylomorphic contrast lifted from the framework of Aristotelian physics, form and matter are constitutive components of every material being (1); form is that which remains invariant in a material being throughout its existence (2, 3, 4); it makes a thing to be what it is, thereby determining its definition and quiddity (3, 5); and on some medieval accounts, serves as a principle of individuation (item 6).
	
	Underlying the multitude of different positions and debates in philosophical logic today is a common core of thinking about logic lifted from this hylomorphic framework, albeit manifested in different ways. This is surprising on several accounts: first, because logic is often thought to be formal precisely inasmuch as it demurs from either any particular content or metaphysical assumptions; second, because even among metaphysically minded logicians (or logically minded metaphysicians), \textit{any} kind of hylomorphism remains a distinct minority opinion.
	
	The most specific kind of hylomorphism, present in Tarski's early account, is what Catarina Dutilh Novaes, following Kathrin Koslicki, calls \textit{mereological} hylomorphism \cite{DutilhNovaes2012b} \cite{Koslicki2006}. Mereological hylomorphism is characterized by the contention not merely that wholes are compounds of form and matter, but also that form and matter are themselves distinct integral parts of the hylomorphic compound. This is reflected in the partition of linguistic signs presupposed in Tarski's notion of formal consequence, where one part - the logical constants  - corresponds to the form, and the rest to the matter.
	
	\section{John Buridan's concept of formal consequence}
	The first known account of formal consequence in the western tradition directly defined in terms of a substitutional criterion is that of John Buridan, the 14th century Master of Arts at the University of Paris. Buridan was not the first of the medievals to distinguish between formal and material consequence - the distinction is first made explicit by Ockham, and implicit in Duns Scotus, Simon of Faversham, and others - but Buridan was the first to distinguish formal and material consequences by varying the categorematic terms of an argument.\footnote{Note that for Buridan, in contrast with today's practice, the \textit{terms themselves} are varied, rather than their interpretations. For instance, he says `A man runs; therefore, an animal runs' is not a valid formal consequence, because `A horse walks, therefore a tree walks' is not \cite[TC 1.4.3]{Buridan2015}.} Buridan's way of distinguishing material from formal consequences was especially influential on the European continent, having been adopted by Marsilius of Inghen, Albert of Saxony, and others \cite{DutilhNovaes2012a}. According to Buridan, 
	
	\begin{quote}
		A consequence is called formal if it is valid in all terms retaining a similar form. Or, if you want to put it explicitly, a formal consequence is one where every proposition similar in form that might be formed would be a good consequence [..]. A material consequence, however, is one where not every proposition similar in form would be a good consequence, or, as it is commonly put, which does not hold in all terms retaining the same form.\footnote{\cite[I.4, p. 68]{Buridan2015}: 
			
			\begin{quote}
				Consequentia `formalis' vocatur quae in omnibus terminis valet retenta forma consimili. Vel si vis expresse loqui de vi sermonis, consequentia formalis est cui omnis propositio similis in forma quae formaretur esset bona consequentia [...] Sed consequentia materialis est cui non omnis propositio consimilis in forma esset bona consequentia, vel, sicut communiter dicitur, quae non tenet in omnibus terminis forma consimili retenta.
			\end{quote}}
			
		\end{quote}
		
		There are some important differences between the way formal consequence is understood by Buridan, and the way it is understood today - to name two, the basic units of Buridan's consequences are written or spoken sentences, and therefore his semantics is token rather than type-based \cite{Klima2004} \cite{DutilhNovaes2005}; and Buridan doesn't assume a material consequence is thereby not a logical one.\footnote{Much of the renaissance of Buridan scholarship in the past half century has been motivated by \textit{prima facie} similarities between Buridan's treatment of various logical topics and contemporary treatments of what are recognizably the same topics and questions. Cf. \cite{Moody1952} \cite{Bochenski1956} \cite{Kretzmann1982} \cite{Parsons2014}.}
		
		But whatever one makes of these lesser differences, there is one difference that makes studying Buridan and his contemporaries an especially fruitful endeavor: the medieval application of hylomorphic language to consequences could not but be a \textit{conscious} one, taking place at the height of critical engagement with both Aristotle's logic and his physics and metaphysics; whereas the contemporary appropriation of hylomorphic language has been by and large uncritical and at times unaware of this appropriation.\footnote{Important exceptions include \cite{Read1994} \cite{Read1995} \cite{MacFarlane2000} \cite{DutilhNovaes2011} \cite{DutilhNovaes2012a} \cite{DutilhNovaes2012b} \cite{DutilhNovaes2012c}.} If, at times, medieval treatments of consequence appear less sophisticated than their contemporary analogues, they're also somewhat less liable to the distractions that accompany the development and long use of a technical vocabulary, and thereby often closer to the matters themselves under discussion.\footnote{This is an application of a broader point frequently ignored in both historical and systematic discussions today: our later standpoint on questions of philosophical importance is not wholly an advantageous one, inasmuch as the development of any body of knowledge brings with it a certain forgetfulness of its origins. To take an especially clear example, the vast proliferation of logics in the past century, while it has brought us a great many proofs, has not brought us a step closer to an understanding of what logic is or of what it is about. Aristotle, whatever one may think of his analytical and topical works, at least had some sense for what he was doing. Our current state regarding the sense of these questions, by contrast, is probably more bewildered than it has ever been.}	

		\subsection{Contrasting Tarskian and Buridanian consequence}
		The dominant contrasts between Buridan's account of formal consequence, on the one hand, and classical and Tarskian accounts, on the other, are as follows. 
		
		Tarski's account of formal consequence transforms Buridan's uniform substitution criterion on terms into one over models, i.e. over orderings of objects satisfying sentential functions obtained from one's initial sentences by substituting non-logical constants with like variables. Though the use of such substitution techniques to check validity goes back to Aristotle, Buridan appears to be the earliest to have seen in it both a necessary and sufficient condition for formal consequence.
		
		Buridan's definition of consequence is a deflationary one: `a consequence is a hypothetical proposition composed of an antecedent and consequent' \cite[I.3]{BuridanTC}. In addition, he offers a criterion for consequence according for which a consequence is good when it is impossible for things to be as the antecedent signifies without being as the consequent signifies. For Buridan, this criterion needs to be reformulated to fit the tense, modality, and quality of propositions present in an antecedent and consequent - for instance, the proposition, `if Socrates was running, Socrates was moving' is good if is impossible for things to \textit{have been} as the antecedent signifies, etc. 
		
		Buridan divides consequence into formal and material, with the latter divided into simple and as-of-now consequence. A formal consequence is one belonging to an equivalence class, determined by its syntactic structure, such that all consequences in that class are good. A material consequence is a good consequence not of this sort. A formal consequence is said to be good in virtue of its formal parts, i.e. its syncategoremata. And though consequences may be valid without being formally valid, a materially valid consequence is only made evident by its reduction to a formally valid one. A simple consequence is one for which Buridan's criterion for a good consequence holds without qualification; an as-of-now consequence, one for which it holds for a given time.
		
		Many of the better-known formal developments since Tarski have brought about a rehabilitation of Buridanian themes. Work on tense, modality, and other intensional operators, for instance, has been the norm since Prior and Kripke; the acceptance of domain variation, since Kemeny. There are also approaches to formal consequence which, with Buridan, take the time of utterance into account, though these remain non-standard.\footnote{See those systems drawn on in \cite{DutilhNovaes2005,DutilhNovaes2007b}.} In other ways, more recent approaches to consequence have moved away from aspects common to both Buridan and Tarski. For instance, in contrast with classical consequence today, neither Buridanian nor Tarskian consequence is schematic, and neither interprets the non-formal parts of the languages to which it applies arbitrarily.
		
		Modern approaches to formal consequence depart from their medieval predecessors in their identification of formal and logical consequence, and their identification of consequence as the subject matter of logic as such. For Buridan and the medievals generally, there are logical consequences which are not formal, including induction and enthymematic consequences. And consequence forms only a small and relatively new part of what medieval logic studies. For Ockham `consequence' is restricted to non-syllogistic argument. For Buridan, it encompasses both syllogistic and other kinds of argument.
		
		\subsection{Medieval consequence as a key to understanding semantic consequence today}
		
		Against the medieval backdrop, the most distinctive aspects of formal consequence today are the concepts of \textit{language} and \textit{function} it employs.
		
		Formal \textit{consequence} is now defined over a formal \textit{language}: typically, a collection of syntactic strings divided into formal and non-formal primitives, with the referent of each formal element fixed to a semantic function, and that of each non-formal element varied arbitrarily; from which a countably infinite collection of formulas is recursively defined. 
		
		The concept of \textit{function} herein employed is a mathematical one: an $n$-ary function $f$ is a mapping from $n$-tuples in some collection of $n$-tuples, called the \textit{domain} of the function to an element in some other, possibly distinct collection of elements, called its \textit{range}. Such a mapping may be \textit{total}, mapping each element in the domain to an element in the range; \textit{partial}, i.e. only from some elements in the domain; \textit{many-to-one}, mapping some distinct elements in the domain to the same element in the range; or \textit{one-to-one}, mapping each element in the domain to a distinct element in the range. But no function is one-to-many. 
		
		The formal consequence relations for classical logic, its extensions, and rivals all operate on a functional understanding of language. In practice, a phrase like `the mother of' is treated as a function from the domain of individuals to itself, mapping each person to his or her mother. $n$-place predicate symbols are treated as functions from $D^{n}$, the $n$th Cartesian product of the domain of a model, to a set of truth values. Names in classical logic are assimilated to functions of arity 0 to the domain of a model; atomic propositions, to predicates of arity 0.
		
		This functional understanding of language leaves a host of dualisms in its wake, which shape the landscape of philosophical logic today: between language and world; object language and metalanguage; names and predicates; use and mention, and between the logical and non-logical parts of a language. Each of these, in turn, brings with it one of modern logic's characteristic \textit{insolubilia}: problems with self-referential terms; Tarski's hierarchy of languages, with its corresponding hierarchy of truth predicates; Frege's `problem of the concept \textit{horse},' concerning the relation between concept and object; over the admissibility of intensional operators into formal languages; and the problem of demarcating the logical from non-logical constants.
		
		Modern approaches to formal consequence thus build on a semantic base that assimilates meaning to function, and eliminates the medieval distinction between signification and supposition, i.e. that between the \textit{meaning} of a lexical item, and its function within a given sentence. The medieval distinction between the syncategorematic and categorematic is thereby transformed from a local one between sentential roles into a global one between linguistic types, generating the demarcation problem for logical constants. Because no function is one-to-many, formal languages are unequipped to distinguish between different kinds of supposition a term might have; nor to deal with problems caused by linguistic ambiguity, which were often at the root of medieval work on fallacies, \textit{sophismata}, and \textit{insolubilia}.
		
\section{Chapter overview}
The aim of this study is to uncover the meanings implicit in our use of the notion of formal consequence by peeling back the layers of meaning imposed at the time when `the main precursor of the modern concept of logical consequence' was first formulated \cite{DutilhNovaes2012a}. The general plan of the work is to begin with Buridan's notion of formal consequence, and from there to move backwards in successive stages to its historically antecedent enabling conditions. 

The questions surrounding the genesis of Buridan's notion, though not all answered, have at least reached a point where they are easily formulable and relatively tractable. The main questions are as follows: 
\begin{enumerate}
	\item What is Buridan's account?
	\item How does Buridan's account relate to that of Ockham, the first to explicitly mention a distinction between formal and material consequence?
	%\item How does Buridan's account differ from that of Tarski, and from those model-theoretic accounts descended from him?
	\item How does the division of consequences into formal and material relate back to the division between natural and accidental consequences, i.e. to the division it seems to have replaced?
\end{enumerate}
There are, of course, more fine-grained questions ensconced within those mentioned, as well as questions that may be asked on either chronological side of these. One may ask, for instance, how the notion of formal consequence is developed by Buridan's followers, or about the development of earlier divisions of consequences. But answering the questions enumerated would yield a philosophically illuminating and relatively self-contained answer to the question of where formal consequence actually came from.

The second chapter provides a more in-depth introduction to Buridan's concept of formal consequence in itself. I begin by reviewing classical formal consequence, then its differences from the account of formal consequence found in Tarski. This is followed by an analysis of Buridan's own approach, examining Buridan's definition and division of consequence into formal and material. The final part compares Buridan's account to Tarski's and its successors in current model-theory. 

Chapter three compares the account of consequence found in Buridan to that of Pseudo-Scotus. Recently, Lagerlund and Read have both held that Pseudo-Scotus' treatment of formal consequence must antedate that of Buridan \cite{Lagerlund2000} \cite{Read2015}. Here, I show this is not the case. A detailed examination of parallels between Buridan's \textit{Tractatus de consequentiis} and the relevant texts from Pseudo-Scotus' \textit{Questions on the Prior Analytics} shows the Scotus text builds on that found in Buridan; while Buridan's own criterion for valid consequence was the likely target of the \textit{Pseudo-Scotus} paradox, a paradox first found in Pseudo-Scotus' text, resembling Curry's paradox today.

Chapter four investigates modal formal consequences in the account of William of Ockham. Ockham distinguishes between two readings of modal propositions: one called `composite'; the other, `divided'. Today, treatments of Ockham's distinction assimilate it to one between wide and narrow-scope modality in first-order classical modal logic. Doing so, however, renders certain consequences Ockham countenances invalid. I provide a formal reconstruction of Ockham's account which validates the arguments Ockham uses in his text, and gives a full account of the formal entailments and oppositions between two term categorical modal propositions according to their quality, quantity, and range of quantification.

%Chapter four compares Buridan's account of formal and material consequence to that of Ockham. The first part of the chapter provides a systematic comparison between the two accounts, while the second part investigates the question of influence. Recent literature has been apt to distinguish Ockham's account from Buridan's - and indeed, British from Continental medieval approaches to formal and material consequence more generally - by saying that while the tradition on the continent formulated the distinction between formal and material consequences substitutionally, the British tradition formulated the distinction in epistemic terms.\footnote{See, for instance, Dutilh Novaes 2012a.}. I show that in the case of Ockham, this isn't quite right. Rather, the formality of Ockham's formal consequence essentially consists in its holding by virtue of an extrinsic rule normatively binding on the thought patterns of actual reasoners. In this way, Ockham's account of formal consequence anticipates both the rules-based accounts of proof-theoretic semantics and the Kantian association of logic with laws of thought. The second part answers the question of whether either the language or the content of Buridan's distinction is in fact derived from Ockham. In short, I show that though Buridan had read Ockham by the time he composed his commentary on the Aristotle's \textit{On Sophistical Refutations}, nothing in the content of Buridan's notion of formal consequence in the \textit{TC} gives us reason to believe he had read Ockham prior to that point. Given that the notion of formal consequence had existed at Paris prior to Ockham's writing the \textit{Summa Logicae}; and given the fairly deep differences between their accounts, it seems more likely than not that Buridan's development of the notion and adoption of the distinction between formal and material consequences is largely an independent development, at best indirectly related to Ockham's distinction.

Chapter five introduces the account of consequences found in Walter Burley's later \textit{De Puritate Artis Logicae (On the Purity of the Art of Logic)}. Burley is perhaps best known for his `realism' in metaphysics; and in part because of this, his own account of consequence has been understudied, and its relation to and influence on nominalists like Buridan has not been thoroughly considered. The later version of Burley's \textit{De Puritate}\footnote{There is also an earlier version, sometimes called the \textit{tractatus brevior}. Though sharing some material with its later counterpart, the earlier treatise is not an abbreviation of the later one, and includes material not included in the later treatise.} was completed while Burley was at Paris, and seems to have been widely read and circulated there. However, by the time the treatise appeared, Burley had a wealth of material from commentaries and short tracts stretching back to his time at Oxford, and Burley's treatise was itself conceived in part as a response to Ockham. Though Burley makes use of a distinction between formal and material consequence, the distinction does not hold the prominence it does in Buridan's account: in its place, we find a distinction between natural and accidental consequence. This chapter shows the relation of Burley's treatise to the earliest work on consequences, and explains the relation of Burley's division of consequences to that which followed it.


%The fifth chapter considers a different influence on the development of theories of consequence: 14th century accounts of hylomorphic composition in commentaries on Aristotle's \textit{Physics}, \textit{Metaphysics}, and shorter physical treatises. Here, I show that Buridan's \textit{logical} hylomorphism reduplicates the peculiarities found in his \textit{physical} hylomorphism: in particular, Buridan's physical hylomorphism is mereological: for Buridan, form is a proper, integral part of a composite substance. Furthermore, Buridan himself assimilates the distinction between form and matter to one between substance and accident, maintaining that the matter of a material substance is united to its form as an accident to a substance. This assimilation helps explain how the distinction between natural and accidental consequences is assimilated to, and ultimately supplanted by, that between formal and material consequences. In short, while the mereological hylomorphism of Buridan's physics need not have necessitated his physical hylomorphism, it did help facilitate it.

%The penultimate chapter examines three early tracts on consequences - two anonymous, one by Walter Burley, all of which are translated in appendices to this dissertation. The chapter vindicates a thesis advocated by Eleonore Stump \cite{Stump1989}, and contested by Niels J. Green-Pedersen \cite{Green-Pedersen1984}: that the earliest treatises on consequences have their main source in treatises on \textit{topics}, particularly in commentaries on Aristotle's \textit{Topics} and Boethius' \textit{De Topicis Differentiis}. However, considerations pertaining to supposition also loom large in these earlier treatises, particularly in the way the validity of different consequences is affected by problems of existential import.

The final chapter takes a synoptic view of the results detailed in the previous chapters, returning to their import for the ways in which logic is said to be formal today.
		\section{Exposition}
		\subsection{Summary of developments up to Buridan}				
		Buridan's account of formal consequence is imperfect in several respects. Its notion of consequence undergirds a solution to the Liar paradox, but remains susceptible to the Pseudo-Scotus paradox. Buridan subdivides material consequence into simple and as-of now consequence, but Buridan's stated criteria for these cannot distinguish simple material consequences from formal ones, or as-of-now consequences from purely invalid ones. Buridan's definitions of simple and as-of-now consequence are later improved upon by Pseudo-Scotus, who distinguishes them by the way they are reduced to a formal consequence: simple consequence, by adding a necessary proposition to the antecedent; as-of-now consequence, by adding a contingent one. The discovery of the Pseudo-Scotus paradox, however, seems to have led to despair over the possibility of providing a simple criterion for valid consequence, and the proliferation of more \textit{ad hoc} criteria in later medieval logic.
		
		Though Buridan admits good consequences that are not formal, any such consequence which is also \textit{evident} is only made so by its reduction to a formal one. Not every material consequence is so reducible: examples and induction are not. For Buridan, the category `consequence' is expansive enough to include both syllogistic and non-syllogistic consequences; formal consequence is the source from which any good consequence is evident; and material consequences are either reducible to formal ones, or lacking in evidence. The resulting picture is what one may call \textit{reductive consequential monism}: there are no ontological differences present in the different ways of following, but only differences in evidence; as far as ontology is concerned, there is only one basic kind of consequence to which others are reducible. And those which are not so reducible have a secondary, imperfect epistemic status.
		
		Buridan adopts the subsumption of syllogistic under consequence from Walter Burley. This treatment, common to Burley and Buridan, differs from that of Ockham, who takes the domain of consequences to be that formerly allocated to topical argument. 
		
		The division of consequences into formal and material varieties goes back to Simon of Faversham, and is also present, albeit not as an explicit taxonomy, in the anonymous London \textit{de consequentiis}. The earliest formal division between formal and material consequence is in William of Ockham's \textit{Summa Logicae}, though Ockham's division differs from that found in Buridan. For Buridan, a consequence is formal if it is good for all uniform substitutions on categorematic terms. Ockham countenances as formal those which, in addition, are reducible to those which are formal in Buridan's sense. 
		
		In Ockham, there remains some distinction between different ways in which something may follow, albeit attenuated from that found earlier in Burley. Ockham calls `material' those consequences which hold `precisely by reason of the terms', and countenances the consequences \textit{from the impossible anything follows} and \textit{the necessary follows from anything} as of this type. In Buridan, by contrast, even this distinction is erased: Buridan appropriates the language of the formal-material division for his own purposes, and builds the validity of Ockham's material consequences into the criterion for consequence as such: since it is impossible, for instance, for things to be as an impossible antecedent signifies, it is also impossible for things to be as it signifies without things being as its consequence signifies. Buridan further takes the consequence to hold formally for explicit contradictions, justifying it by disjunctive syllogism, in the same manner C. I. Lewis would later in his account of strict implication.\footnote{This method for proving anything from an explicit contradiction is first reported by Alexander Neckham and attributed to William of Soissons. See \cite{Martin2012}.} 
		
		Ockham distinguishes Buridan's formal, structural consequences from those enthymematic formal consequences which Buridan classifies as material by appealing to topics: for Ockham, a structural consequence holds by an extrinsic topic; an enthymematic consequence holds by an intrinsic topic immediately, and mediately by an extrinsic topic. By `topic', Ockham means what Boethius means by `maximal proposition': a rule licensing an inference from a premise or premises to a conclusion.\footnote{The straightforward identification of maximal propositions with rules is also found in Burley's longer version of the \textit{De Puritate}, though Burley's identification likely antedates that of Ockham.} An intrinsic topic is a premise added to an enthymematic argument to make it a formal one. It is intrinsic in that it governs things mentioned in the stated premise directly. An extrinsic topic for Ockham turns out to be a rule stated in terms of second intensions, under which the objects mentioned in the premises of a given argument fall accidentally, e.g. in virtue of their being given a certain supposition in a proposition, etc. 
		
		Prior to the distinction between formal and material consequence which gains currency in Ockham's work, one finds a distinction between natural and accidental consequence implicit in Boethius, and present later in William of Sherwood, Scotus, and in Burley's \textit{de consequentiis}. Burley, following Scotus, grounds the distinction in one between intrinsic and extrinsic topics, though by these he means something different than what Ockham does. For Burley, a natural consequence, which holds through an intrinsic topic, is one which satisfies a containment criterion, where `the antecedent includes the consequent' \cite[p. 61.6-10]{BurleyDPAL}. An accidental consequence, which holds by an extrinsic topic, is one holding by some extrinsic relation between the things named therein. Examples include consequences from the positing of one contrary to the denial of the other, and from the positing of a species to its \textit{proprium}, its inseparable attribute. Burley classifies the consequence \textit{from the impossible anything follows} as an accidental consequence holding by the topic \textit{from the less}. 
		
		Burley identifies his natural consequences with formal consequences in Ockham's sense. Against an objection resembling Buridan's position, Burley insists that not all formal consequences are structural consequences. In this much, he is in agreement with Ockham. But against Ockham, he does not classify \textit{ex impossibili quodlibet} as a material/accidental consequence holding by virtue of its terms, but as one justified by appeal to the extrinsic topic \textit{from the less}.
		\subsection{Analysis}
		From Burley and his predecessors to Ockham, and then from Ockham to Buridan, we see two important shifts, each brought about by the appropriation of the language of a prior position for alternative ends. The first terminological shift is in the understanding of a topic; the second, in the understanding of consequence.
		
		In Boethius, a topic represents a real aspect of a thing from which something about it or another is inferred. That is, a topic is the difference of a maximal proposition. From this, an analogous use of the term arises, i.e. to refer to the maximal proposition itself.
		
		In Burley, the focus on the differences of such propositions is diminished, and a topic is more often simply identified with a rule licensing an inference. The distinction between intrinsic and extrinsic topics, however, continues to be made along  traditional lines.
		
		At the time of Burley's \textit{de consequentiis} and the earliest anonymous treatises, the development of supposition theory brought about a great simplification in the number of topics actually appealed to, with a vast number of inferences justified by rules like \textit{from an undistributed inferior to superior}, \textit{from a distributed superior to a distributed inferior}, etc. That is, appeals to qualitative distinctions grounded in aspects of objects gives way to appeals to scope distinctions grounded in supposition of terms.
		
		In Ockham, the distinction between intrinsic and extrinsic topic is no longer drawn along traditional lines: rather, an intrinsic topic is simply a premise added to an antedecent to make it a structurally formal one; and an extrinsic topic is a rule pertaining to second intensions, licensing the movement from antecedent to consequent in a particular kind of inference. 
		
		In Buridan, the language of topics disappears altogether from treatises on consequences, as does the distinction between intrinsic and extrinsic topics. In its stead, we see Buridan arguing that extrinsic topics are reducible to formal ones in exactly the same sense as arguments employing intrinsic topics: 
		\begin{quote}
			If by the topic \textit{from contraries} we argue so: `A is white; therefore A is not black', it is still the case that a syllogism or formal consequence is effected by the addition of the proposition `no white is black'. And so the other dialectical topics, if they are not formalized, do not conclude on account of form, but can be reduced to a form by additions \cite[sec. 6.6]{BuridanLoci}.
		\end{quote}
		
		This disappearance is closely related to the second shift, that from natural/accidental to formal/material consequence. 
		
		In Burley's work, the division of consequences into natural and accidental is a subdivision of simple consequence. The natural/accidental division is clearly intended as an ontological one, grounded in a distinction between intrinsic and extrinsic properties. However, the distinctions between \textit{per se} and \textit{per accidens} predication, on the one hand, and \textit{per se} and \textit{per accidens} ascent and descent, on the other, were poorly worked out in the earliest versions of the theory. 
		
		Ockham lists the two divisions, the one between formal and material, the other between simple and as-of-now, without subordinating either to the other. His example of a formal consequence holding by an intrinsic topic in the \textit{Summa Logicae} - `Socrates does not run, therefore a man does not run' - is deliberately chosen to provide an example of a formal consequence which is nevertheless as-of-now.\footnote{\cite[III-3. 1, p. 588]{OckhamSL}. It is unlikely that Burley would have agreed with Ockham on this classification. For Burley, a simple consequence appears to be one where 1) there is a \textit{per se} relation between the antecedent and the consequent, and 2) in every situation where the antecedent holds, the consequent holds. For Ockham, by contrast, such a consequence additionally requires that the predications in its antecedent and consequent themselves must be intrinsic; and 2) that the antecedent itself holds in every situation. Burley would accept the consequence as formal, given the containment criterion is satisfied for the terms `Socrates' and `man' (names of individuals were taken to imply their natures). Given this disagreement, the number of simple formal consequences seems to be greatly curtailed in Ockham's logic.} More importantly, the distinction between these in terms of topics gives way to a classification according to whether such consequences appeal to universal rules or terms: structural formal consequences hold by an extrinsic rule; enthymematic formal consequence, mediately by a structural rule and immediately by one pertaining to the terms, i.e. by an added premise; material consequences, by the terms alone. In Ockham, we thus see the denial of a distinction between different \textit{ways} of following, relative to different ontological relations, and with it a co-opting of the topical vocabulary standardly used to express such a distinction. Given that Ockham identifies formal consequences with those reducible to formal consequences in Buridan's sense, it is only a natural step from there to Buridan's understanding of formal consequence; and with it, to the identification of the division between simple and as-of-now consequence as one subordinate to material consequence, between two different \textit{ways} of reducing a material consequence to a formal one, i.e. by the addition of a necessary or contingent premise. Lastly, in Buridan, we see even the vestigial diminished status Ockham affords to \textit{ex impossibili quodlibet} erased: for Buridan, the consequence is good not by appeal to an extrinsic topic, not by its terms, but simply by the definition of a good consequence.
		\section{Conclusion}
		The triumph of Buridan's definition of formal consequence was the cumulative result of rapid changes in the theory of topical argument from the late thirteenth to mid-fourteenth century. At the same time, the unification initially wrought by the medieval subsumption of syllogistic and topical argument under the banner of consequence was intimately tied to the development of the notion of formal consequence as consequence \textit{par excellence}, that to which any consequence worth its salt would be reducible. 
		
		
		What was this a triumph \textit{for}? Doing logic without appealing to qualitative ontological distinctions, the norm in the topical framework consequences grew out of and eventually replaced. What was it a triumph \textit{over}? Attempts to ground logic, however loosely, in ontology. 
		
		What was \textit{sacrificed} in this achievement? Logic's relevance; its multifacetedness; its groundedness.
		
		The loss of relevance occurs in two senses. The first, in the introduction of various irrelevant entailments such as \textit{ex impossibili quodlibet}, and their gradual movement from the periphery to the center of the understanding of consequence. 
		
		The second was the loss of relevance to the process of \textit{discovery}. The streamlining of the maximal propositions actually appealed to in logical disputation, wrought via the introduction and gradual improvement of the theory of supposition, was a remarkable achievement. But as with their modern analogues in metatheoretical appeals to set-theoretic containment, these appeals do a much poorer job of providing rules for the development of arguments than the qualitative topical arguments they replaced: for instance, the rule \textit{from the first to the last}, i.e. transitivity, is simply much thinner, and accordingly less useful, than a rule like \textit{from the positing of the effect of an efficient cause follows the positing of the cause}. In this way, the development of the medieval theory of consequence served as the immediate backdrop for the various early modern attempts, such as those of Bacon and Descartes, to develop a method for discovery; for the location of such endeavors, given the tarnished name of `logic', under the banner of not logic, but epistemology; and for the development of rival approaches, from which arises the distinction principally between rationalists and empiricists, but many others as well. Without the transformation of the topical theory into the more supposition-oriented theory of consequences, characteristic early modern complaints about logic's uselessness and inapplicability for discovery would not have found footing. 
		
		The loss of multifacetedness occurs through the replacement of a distinction between two different ontologically grounded ways of following with a purely epistemic distinction. The earlier account is based on two different ways in which something may be predicated of another, i.e. intrinsically or accidentally, and thus has a basis in Aristotle's \textit{Categories}, and is reflected in the distinction between analytics (which, as the name implies, is concerned with breaking down intrinsic components) and dialectic. The account of consequence Buridan advances is a reductively monistic one: all good consequences are reducible to formal ones. Here, one is not far from the identification of consequence with formal consequence as such, and from the identification of the latter with logical consequence. On the one hand, this brings with it a great unification of the previously disparate realms of analytics and dialectic. On the other, its doing so brought with it a forgetfulness of the question of what the different spheres of logic are concerned with, in favor of an understanding of logic as applicable to anything whatsoever.\footnote{Only with Kant's discovery of the synthetic \textit{a priori} do we arrive at anything like a rehabilitation of the traditional domain of topical argument. However, Kant limits the domain of such arguments to bar them any role in metaphysics; and the distinction was strongly rejected by the logicist tradition at the start of early analytic philosophy. Though with \cite{Kripke1980}, the analytic tradition has found a role even for \textit{a posteriori} necessity, it has yet to appropriate, or even to grapple much with, Kant's earlier distinction, let alone to examine its affinity with the basic idea behind Boethian topical argument.}
		
		The loss of groundedness corresponds to this loss in multifacetedness. Where the earlier approach to consequences attempted a basic grasp of logic's nature, subject, and divisions, the later approach to consequences eschewed these concerns in favor of a more conventionalist approach to logical practice. Rules for consequences become those rules which are obvious to everybody, or which are derived from such rules in acceptable ways.
		
		It is against the backdrop of this loss that the crisis in contemporary logic is given its sense. After the collapse of the formalist program and the fading out of the logicist program in the second quarter of the twentieth century, the ensuing widespread adoption of first-order classical logic was never provided with a solid, more than dogmatic foundation. From there, that foundation has been assailed with attacks on \textit{reductio ad absurdum} proof by intuitionism; on \textit{ex falso quodlibet} by the Lewis systems for strict implication; on non-relevant implication by the relevant logics of Belnap and others; on bivalence and excluded middle by many-valued systems and on non-contradiction by paraconsistent systems. Regardless of the strength of the objections, the foundation was weak enough that when the opposition circled about the walls and shouted their objections, the walls of the dogmatic consensus came crumbling down. 
		
		However, the loss of this consensus has not brought with it a serious attempt to understand what formal consequence is, what it is based on, or whether it constitutes a unified thing. Rather, the \textit{de facto} successor to the dominance of the first-order, semantic approach to formal consequence has been the proliferation of a multitude of logics, employed seemingly at whim. As \cite{Mehlberg1960} foresaw, a thousand Carnapian flowers are now blooming in the field of mathematical logic. But the word sown by that Austrian logician is now fallen on rocky ground. Immediately a thousand systems spring up; but since they have no root, they wither away in obscurity and disuse. 
		
		In this period of the proliferation of logics we have now entered, we see the continuation of trends present already in the logic of Buridan and his nominalist followers: on the one hand, the proliferation of a vast array of technical results; on the other, a loss of any real sense of what one is doing when one is doing logic. Where it falls in the literature of the field, work of philosophical depth requiring greater cultivation and care is often choked out by technical results.
		
		Where does this leave us now? 
		
		With much work to do, yes. But with the field ripe for that work. A historical genealogy of the concept of formal consequence serves as a blueprint for a bridge to a better footing for formal logic. From here, the growing acceptance of relevant logics and other deviant logics provides an opportunity to revisit non-reductive approaches to `following', as well as to examine its various kinds and senses. The growth of interest in logical pluralism provides an opportunity for revisiting and improving the real basis on which Burley and earlier logicians grounded early divisions of consequences. Metaphysical work on grounding provides an opportunity to enter into serious discussions concerning on the grounds of consequence.
		
		Surely, let a thousand flowers bloom. But let the seed fall on good soil and bring forth good fruit. Thirty, sixty, a hundredfold.
		\printbibliography
		
\end{document}
