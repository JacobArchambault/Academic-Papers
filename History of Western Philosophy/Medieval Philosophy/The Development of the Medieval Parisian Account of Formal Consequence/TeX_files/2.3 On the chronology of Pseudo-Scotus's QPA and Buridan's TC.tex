\documentclass[]{article}
\usepackage{amssymb}
%opening
\title{On the relative chronology of the Pseudo-Scotus' \textit{Quaestiones Super Libros II Priorum Analyticorum} and John Buridan's \textit{Tractatus de Consequentiis}}
\author{Jacob Archambault}

\begin{document}

\maketitle

It is beginning to be more widely assumed that the Pseudo-Scotus' account of consequence antedates that of Buridan \cite[ch. 6]{Lagerlund2000} \cite[pp. 252-253]{Johnston2015}  \cite[pp. 4-5]{Read2015}, the principal piece of evidence for this being that Buridan considers and rejects an analysis of divided modal propositions Pseudo-Scotus advocates.\footnote{Cf. \cite[In anal. pr. I, q. 26, pp. 143-144]{Pseudo-Scotus1891}, \cite[TC II, 4, p. 97]{Buridan2015}.} On Buridan's analysis, a modal of possibility is equivalent to one where the subject is ampliated to include possible instances of itself. E.g. `A can be B' is analyzed as `What is or can be A can be B'. On Pseudo-Scotus', the aforementioned proposition is equivalent to `What is A can be B or what can be A can be B'.

Closer examination of parallels between Buridan's and Pseudo-Scotus' texts, however, repeatedly shows Pseudo-Scotus' to be more complex than its Buridanian counterpart. This does not mean Pseudo-Scotus' definitions are more correct. But the addition of detail, whether clarifying or convoluting, is a sure sign we are dealing with a later text.\footnote{The later date of Pseudo-Scotus' treatise is assumed, without being supported, by \cite{Boh1982}, \cite{King2001}, \cite{DutilhNovaes2008}, and \cite{Knuuttila2008}.}

Pseudo-Scotus' first criterion for consequence verbally parallels Buridan's first. But a minor change leaves it closer in content to Buridan's second definition. Where Buridan's approach has one proposition antecedent to the other when its impossible for it to be true with the other \textit{not being true}, Pseudo-Scotus' replaces `not being true' with `false'. Since propositions are only true or false when they exist, Buridan's counterexample to the criterion does not apply to Pseudo-Scotus' formulation: he pays heed to the verbal formulation while effectively speeding up the discussion by passing over it's content. Hence, the counterexample Pseudo-Scotus applies to the criterion is similar to Buridan's for the second: Buridan offers `no proposition is negative, therefore no ass is running'; Pseudo-Scotus offers `Every proposition is affirmative, therefore no proposition is negative'. In both cases, the problem addressed is one where correspondence conditions for the truth of the sentences involved in the consequence come apart from the conditions for their being satisfied. The parallel between Buridan's second criterion and Buridan's third is also exact, excepting Pseudo-Scotus' substitution of `false' for Buridan's `not being true'.

The same economy is present in Pseudo-Scotus' dropping the phrase `both being formed together' from the third definition Buridan offers. Since the definition is offered in terms of signification rather than truth, the actual formation of the consequent is irrelevant, and the phrase specifying it becomes otiose.

Other differences include the following. Pseudo-Scotus, mentions besides form, also the disposition of the terms in his definitions of formal and material consequence. Pseudo-Scotus subdivides formal consequence, where Buridan does not. And where Buridan defines simple and as-of-now consequence in terms of the impossibility of the antecedent being true and consequent being false together, Pseudo-Scotus defines it in terms of a criterion Buridan gives later, the ability to be reduced to a formal one by adding some necessary proposition to the antecedent. Pseudo-Scotus employs the same reduction criterion in his definition of as-of-now consequence, where Buridan simply states that an \textit{ut nunc} consequence is one which is not good simply. In these last two cases, Pseudo-Scotus is actually \textit{more} faithful to Buridan's account than Buridan himself is. Buridan's assumed definition of simple consequence conflicts with his division of consequences, the reason being that the definition does nothing to exclude formal consequences from being simple, though Buridan classifies simple under material consequence. And Buridan's definition of \textit{ut nunc} consequence does nothing to distinguish \textit{ut nunc} consequences from those which are in no way valid.

As for Pseudo-Scotus' analysis of divided modals, the analysis shows strong signs of being at worst a misreading of Buridan's position. The structure of question 26 of the \textit{Prior Analytics} commentary is that of a comparison between two approaches to modal conversions: one, an ampliation-based reading where the subject supposits disjunctively; the other, where divided modals of possibility are ambiguous. The first reading is evidently Buridan's; the second, Ockham's.\footnote{For an analysis and extension of Ockham's approach, see \cite{Archambault2017c}} The question describes the manner of conducting conversions on both approaches, without deciding between the two. It is thus somewhat misleading to describe an error Pseudo-Scotus' makes in describing Buridan's analysis as his own position, since he never gives it preference over the Ockhamist analysis.

That the Pseudo-Scotus analysis is intended as one of Buridan's position is clear from its invocation of the same language one finds in Buridan's \textit{Tractatus de Consequentiis} to describe his own position. Pseudo-Scotus' states that `with respect to a verb of possibility in an indefinite or particular proposition, the \textit{subject} supposits disjunctively for those which are or for those which can be'\cite[In anal. pr. I, q. 26, p. 143]{Pseudo-Scotus1891}. This is exactly what one finds in Buridan's own account \cite[TC II, 4, p. 97]{Buridan2015}. This intention is further evident from Pseudo-Scotus' placing explicating the conversion of such a proposition as one with a disjunctive \textit{predicate} \cite[In anal. pr. I, q. 26, p. 145]{Pseudo-Scotus1891}. 

Further analysis, however, shows the error is not Pseudo-Scotus', but \textit{Buridan's}. On Buridan's analysis, `Some A can be B' is analyzed as `Something which is or can be A can be B'. This is formalized in turn as $\exists x((A \vee \diamond Ax) \wedge \diamond Bx)$. Given that an assertoric proposition implies one of possibility, the disjunct from the left-hand side can be eliminated, and this simplifies to $\exists x(\diamond Ax \wedge \diamond Bx$. Now, by disjunction introduction, this implies $\exists x(\diamond Ax \wedge \diamond Bx) \vee \exists x(Ax \wedge \diamond Bx$. And that this is indeed equivalent to the first disjunct alone is shown by cases. If the first disjunct is true, then the conclusion is immediate. If the second, then something is both $A$ and possibly $B$. But since the truth entails possibility, it follows that $\exists(\diamond Ax \wedge \diamond Bx)$, hence, in either case, the entailment follows. Thus, the equivalence holds just as Pseudo-Scotus says it does. The argument Buridan gives in TC II. 4 against the analysis is an \textit{ignoratio elenchus}, since it changes the example to one where the modality is before a negation. This ensures the proposition is not one of the form Buridan says it is, since the example Buridan gives is equivalent to one of the form `Some A is not necessarily B', which isn't at all an affirmative divided modal of possibility, but rather a negative one of necessity.

Further evidence of the later date of Pseudo-Scotus' analysis may be garnered from the following observations. First, it is already accepted that the Pseudo-Scotus knows Buridan's \textit{Questions on the Posterior Analytics} \cite[pp. 4-5]{Read2015}. Second, the author additionally appears to have knowledge of the \textit{Sophismata}, since he uses a sophism Buridan admits as true - `a man will be a boy' - in an objection to the conversion of divided modals into assertoric propositions.\footnote{\cite[in. anal. pr. I, q. 26, p. 143]{Pseudo-Scotus1891}. Cf. \cite[pp. 878, 888]{BuridanKlimaSD}}. Third, one of the objections Buridan raises to the disjunctive-proposition analysis of particular divided modals of possibility is that its contradictory, would have to be not a disjunction, but a conjunction of two universal propositions \cite[TC II, 4, p. 98]{Buridan2015}. But this cannot be used as an objection to Pseudo-Scotus' position, since he explicitly uses such an analysis \cite[in anal. pr. q. 26, p. 144]{Pseudo-Scotus1891}. Lastly, where the analysis on which Pseudo-Scotus' text is earlier leaves us with an unsolved question, namely what Pseudo-Scotus' source is, if not Buridan's \textit{Tractatus}, given that the Pseudo-Scotus presentation is clearly reporting the views of others. This problem doesn't arise, however, on the view that Buridan's text is the earlier one: we can accept the analysis Pseudo-Scotus reports is Buridan's own development.

Now to dissolve the \textit{causa apparentiae}: there is a simpler explanation for the target of Buridan's remarks. According to Buridan, ambiguous written or spoken propositions are mapped to a single mental proposition, i.e. the disjunction of its specified readings \cite[QE 3.2]{BuridanQE}. But according to the early analysis found in Ockham \cite[SL II, ch. 25]{OckhamSLI} \cite{PriestRead1981} and arguably even in Aristotle \cite[Pr. An. 19, 32B 25-32]{AristotlePrA} \cite[pp. 243-245]{Johnston2015}, modals of possibility are ambiguous between just the readings one finds in Pseudo-Scotus listed as the proposition's disjunctive truth conditions. Buridan is reading his own commitments concerning ambiguity back into the earlier analysis. As such, Buridan's rejection of the analysis in his \textit{Treatise on Conseqeunces} does nothing to suggest it postdates Pseudo-Scotus' text.

\bibliography{jacob}
\bibliographystyle{plain}

\section{Appendix: Parallel passages from Pseudo-Scotus and Buridan}
The following provides parallel passages from Buridan's \textit{Treatise on Consequence} and Pseudo-Scotus' \textit{Questions on the Prior Analytics}, all but the last concerning the notion of consequence. Buridan's passages are on the left, Pseudo-Scotus' on the right. Parallels are indicated by underlined text, while discrepancies within parallel passages are bolded.
\subsection{Conditions for consequence}
\subsection{Definition of consequence}
\subsection{Divisions of consequence}
\subsection{Divided modality}

\end{document}
