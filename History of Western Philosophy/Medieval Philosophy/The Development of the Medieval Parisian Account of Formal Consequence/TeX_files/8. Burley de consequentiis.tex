\documentclass[]{article}

%opening
\title{On Consequences}
\author{Walter Burley
\\ translated by Jacob Archambault}

\begin{document}

\maketitle

\begin{abstract}
The following provides an English translation of Walter Burley's \textit{On Consequences}. The translation is made from Green-Pedersen's critical edition \cite{Green-Pedersen1980b}. It follows manuscripts B and P where these make sense, adopting their reading somewhat more frequently than the edition.  Where the underlying text for the translation differs from that of the edition, this is indicated via a footnote. Where the text agrees with the edition, shorter variants are not indicated. %Longer variant readings are given in footnotes.
\end{abstract}

\section{Rules concerning consequences}
\begin{itemize}
\item[1.] Since we use consequences in testing and refuting sophisms, one ought to know many things about the nature of consequences. And so one should know that this rule is good: \textit{whatever follows from the consequent, follows from the antecedent}; and this likewise: \textit{whatever entails (antecedit) the antecedent entails the consequent}. So the consequences following from this rule are necessary.
\item[2.] And so this consequence is good: `If I say you are an animal, I say [something] true; therefore, if I say you are an ass, I say [something] true.' For this is established by this rule: \textit{whatever follows from the consequent follows from the antecedent}. Which is shown thus: for `if I say you are an ass, I say you are an animal' follows; but by the rule: \textit{whatever follows from the consequent follows from the antecedent}; but\footnote{but ... but] Therefore if something follows from the consequent `I say you are an animal', it follows from the antecedent `I say you are an ass,' therefore if C
	
Since if the consequent `I say you are an animal' follows from the antecedent `I say you are an ass,' then L.} `to say [something] true' follows from the consequent `I say you are an animal'; therefore `to say [something] true' follows from the antecedent `I say you are an ass', which is the antecedent. This, then, is a true or good consequence: `if I say you are an animal, I say [something] true; therefore, if I say you are an ass, I say something true.' And it is established by this rule: whatever follows from the consequent, follows from the antecedent.
\item[3.] Solution: I say this must be distinguished, because the act of what is to be said can transfer to the spoken by reason of the thing or by reason of the spoken. In the first way it is false, (f. 19ra) in the second it is true.
\item[4.] Likewise, this rule is good: \textit{whatever entails the antecedent entails the consequent}. And consequences coming down on this rule are necessary, as `if Socrates is, an animal is'. Being Socrates entails (\textit{antecedit}) being man, which is antecedent to being an animal; therefore, being Socrates entails entails being a man, which is the consequent.
\item[5.] From these two rules follow two other rules. One is: \textit{whatever entails the opposite of the consequent, entails the opposite of the antecedent.}. Likewise: \textit{whatever follows from the opposite of the antecedent, follows from the opposite of the consequent}. Since in every good consequence the opposite of the consequent implies the opposite of the antecedent, because always in every good consequence the opposite of the antecedent is implied by the opposite of the consequent; but then the opposite of the consequent becomes antecedent to the opposite of the antecedent; and by the rule: \textit{whatever entails the antecedent, entails the consequent}, thus whatever entails the opposite of the consequent, entails the opposite of the antecedent.
\item[6.] Likewise: whatever follows from the consequent, follows from the antecedent, and the opposite of the antecedent is consequent to the opposite of the consequent, so it is right that if something follows from the opposite of the antecedent, that it follows from the opposite of the consequent.
\item[7.]. Likewise, when one argues \textit{from the first to the last}, one argues by this rule: \textit{whatever follows from the consequent, follows from the antecedent}. As when one argues so: `I a man is, an animal is; and if an animal is, a body si; and if a body is, a substance is; therefore \textit{from the first}, `if a man is, a substance is.'\footnote{is] For here it is argued that `a substance is' follows from `a body is,' and `a body is' follows from `a man is' \textit{add.} CO}
\item[8.] One should know that when one argues \textit{from the first to the last}, where the intermediate consequences are varied, then\footnote{argues ... then] there are many consequences, if the consequences are varied from the first antecedent to the last consequent CO
	
	many consequences occur varied L} the consequence \textit{from the first}, etc. is not valid. Note that consequences which are intermediate are varied when the antecedent in the second conditional is different from what the consequent was in the first conditional. As is shown if one argues: `if no time is, it is not day; and if it is not day and some time is, then it is night; and if it is night, some time is; therefore \textit{from the first to the last}: if no time is, some time is.' This consequence does not hold \textit{from the first to the last}, since the consequent of the first conditional is `it is not day,' and the antecedent of the second conditional is the whole `it is not day, and some time is,' and so \textit{from the first to the last} does not apply.
\item[9.] Again: in a simple consequence it is so that the antecedent cannot be true without the consequent. Hence, if in whatever simple consequence the antecedent can be true in some time without the consequent, whatever the possible case stipulated, the consequent is not valid.
\item[10.] Note that from falsehoods truth can follow, but from truths nothing follows besides truth. Hence the saying: `from falsehoods truth, from truths nothing but truth.'
\item[11.] Likewise: in every good consequence from the contradictory opposite of the consequent the contradictory opposite of the antecedent should be inferred. Because the consequence `a man is, therefore an animal is' is good, and so from the opposite of the consequent the opposite of the antecedent should be inferred, thus `no animal is, therefore no man is' follows.
\item[12.] Likewise: from the contradictory of the consequent the contradictory of the antecedent follows; and then the consequence is good. Hence to know whether a consequence is valid or not, one should see whether from the contradictory of the consequent the contradictory of the antecedent follows or not. If so, the consequence was good; if not, the consequence was not valid.
\item[13.] But for the goodness of a consequence it does not suffice that from the contrary opposite of the consequent the contrary opposite of the antecedent is inferred. For then `every man runs, therefore every animal runs' follows; since from the contrary opposite of the consequent is inferred; for `no animal runs, therefore no man runs' follows.
\item[14.] Likewise, this follows from [what has been] said: from the contingent the impossible does not follow; for if so, the antecedent can be true without the consequent, because that which is contingent can sometimes be true, and that which is impossible never will be true. And so from the contingent the impossible does not follow, since the antecednet can be true without the consequent.
\item[15.] Likewise: from the necessary the contingent does not follow on account of the same cause, since the necessary will always be true and the contingent, sometimes false. Therefore, etc. 
\item[16.] Likewise: in no good consequence does the opposite of the consequent stand with the antecedent: for if it stands, the consequence is not valid. And to stand together with another is to be able to be true with it from the same form.
\item[17.] Likewise, \textit{from the less to the greater without distribution} is a good consequence, if it is not \textit{extraneatio}, as with `Socrates is an individual, therefore man is an individual.'
\item[18.] Likewise if it is not argued with a negation, or with something having the force of negation or distribution, it is a good consequence.\footnote{`or distribution... consequence' added from BFP, against the edition.} For instance,  `only man runs, therefore only man moves' is invalid.\footnote{adding `is invalid', following BFP against the edition.}
\item[19.]Likewise: \textit{from the less per se to the greater per se with a negation placed after} is a good consequence, as here: `man is not, therefore animal is not.' Yet \textit{from the inferior per accidens to its superior per accidens with the negation placed after} is not a valid consequence. Hence, `a white man is not, therefore a man is not' does not follow.
\item[20.] Likewise: \textit{from the distribution of a superior to its inferior per accidens} is not a valid consequence except \textit{per accidens}; because if so, it holds by this extrinsic rule: \textit{positing per se posits per accidens}, which rule is not valid. Yet negating the superior negates the inferior \textit{per accidens}; hence `no man runs, therefore a white man does not run' follows. And this consequence holds by this extrinsic rule: \textit{negating per se, negates per accidens}; which is good, because that which is \textit{per accidens} depends on that which is \textit{per se}.
\item[21.] And one should know that these rules commit the fallacy of the consequent: \textit{whatever follows from the antecedent, follows from the consequent}, as with `if a man is, a risible [thing] is; therefore, if an animal is, a risible [thing] is.' And it is argued by that rule because risible is the consequent if follows from man which is antecedent, therefore it follows from animal which is consequent.
\item[22.] Likewise this rule is not valid: \textit{whatever entails (antecedit) the consequent entails the antecedent}, as `if a man is, an animal is; therefore if a man is, an ass is.' And it is argued by this rule because being man entails being an animal, which is consequent, therefore it entails being an ass, which is antecedent.
\item[23.] Likewise: \textit{from the less to the greater with a negation placed after} is the fallacy of the consequent. Therefore the consequent `no man is, therefore no animal is' is invalid.
\item[24.] Likewise: \textit{from the less to the greater with distribution} is the fallacy of the consequent. Therefore `every man runs, therefore every animal runs' does not follow.
\item[25.] Likewise: \textit{from the greater to the less without distribution} is the fallacy of the consequent. Therefore the consequent `an animal is, therefore a man is' is invalid.
\item[26.] One should know that in every good consequence a distributed consequent follows a distributed antecedent, if the consequent is distributable, and the antecedent likewise.
\item[27.] Against [this]: `Socrates runs, therefore a man runs' is good; and yet distributing the consequent, the distributed antecedent does not follow, because `every man runs, therefore every Socrates runs' does not follow, because Socrates, since he is an individual, is not distributable.
\item[28.] Likewise, it can be argued against the preceding rule so: the consequence `a man is an ass, therefore a man is an animal' is a good consequence. And yet distributing the consequent, the distributed antecedent does not follow, because `every man is an animal, therefore every man is an ass' does not follow.
\item[29.] Likewise, the consequence `an animal is a man, therefore a man is an animal' is good. And yet distributing the consequent the distributed antecedent does not follow.
\item[30.] To the first argument: I say that this rule ought to be understood [so]: distribution is added to those on account of which the first consequence is good. And because the first consequence was good on account of the predicates, therefore the distribution ought to be added to the predicate, arguing so: `man is every animal, therefore, man is every ass.'
\item[31.] To the other I say that the rule has to be understood [of consequences holding] on account of [its] incomplex [parts]. Now this consequence holds on account of the whole complex. Wherefore, etc.
\end{itemize}
\section{On Exclusives}
\begin{itemize}
\item[32.] Likewise, from an exclusive to a universal with the terms transposed and conversely is a good consequence in nominative terms. For `only man is an animal, therefore every animal is man' follows, and conversely.
\item[33.] Likewise: from an exclusive to its prejacent is a good consequence: hence, `only man runs, therefore man runs' is a good consequence. And a prejacent is what remains when the sign of exclusion is removed.
\item[34.] Likewise, any exclusive has two expositions - as `only man runs' has these: `man runs' and `nothing other than man runs.' And if the exclusive is true, each exposition will be true; and if either exposition is false, the whole exclusive will be false. And they are called expositions, because they lay out (\textit{exponunt}) the understanding of the exclusive. Hence,  `expositions' is like saying `laying out'.
\item[35.] And one should note that the opposite of an exclusive proposition having two causes of truth - as `not only man runs' can be verified either because no man runs, or because something other than man runs - can have a third cause, namely [that] no man runs nor does [anything] other than man run.' Hence, if one argues from an exclusive negative to either of these it is the fallacy of the consequent; as here `not only man runs, therefore no man runs'; or `... an other than man runs.' And if the exclusive is true, each exponent will be true, since if either exponent is false, the whole exclusive will be false.\footnote{`And ... false' is added, following BFP.}
\item[36.] And one should know that whenever one argues from a proposition having several causes of truth to one of them it is the fallacy of the consequent.
\item[37.] Likewise one should know that a predicate in an exclusive affirmative stands confusedly and distributively, and the subject only confusedly. Therefore the consequence `only man is animal, therefore every animal is man' holds. Proof: `only man is an animal, therefore every animal is a man' follows, and further `therefore every ass is a man.' Therefore only man is an ass.  Therefore \textit{from the first [to the last]}. The first consequence holds by the rule posited above, namely `from an exclusive...' etc. The second holds by this posited rule: from a distributed superior to its inferior per se distributed is a good consequence. The third holds by this rule: \textit{from a universal to an exclusive}... etc. And the subject stands only confused, because it stands here in just the same was as the predicate in a universal proposition convertible with it.
\item[38.] And one should know that exclusive words are `only' (\textit{tantum}) and `alone' (\textit{solus}). And it is a rule that an exclusive word added to a subject exclusive the predicate from the opposite of the subject; as when saying `only man runs', `to run' is excluded everything other than man, and attributed to man. And so it is exposited thus: `man runs, and nothing other than man runs'.
\item[39.] One should know that when an exclusive word is added to a distributed subject, any such [proposition] must be disambiguated on account of equivocation, because the exclusion can be made general or specific. If the exclusion is made general[ly], then whatever is not included is excluded, and then any [proposition] like this includes opposites, and consequently is impossible. Where the exclusion is made specific, on the other hand, such is possible, because then only species opposite in a certain respect are excluded; and it is excluded\footnote{Reading `excluded' with mss. BFP, rather than `exposited'.} so: `only\footnote{`only' added, following BFP.} every man runs, and not every lion and not every ox...'
\item[40.] Likewise, if an exclusion is added to an integral whole the proposition is ambiguous, because the exclusion can come about on account of the form or on account of to the matter, as in saying `only Socrates is white.' If the exclusion comes about on account of matter, then the proposition is true, and then it has to be exposited thus: `Socrates is white, and nothing that is not Socrates [is]' - namely,\footnote{reading \textit{videlicet} rather than \textit{vel}.} a part of Socrates is white. If the exclusion occurs on account of form, then the proposition is false, and is exposited thus: `Socrates is white, and nothing that is not Socrates is white' - namely\footnote{reading \textit{videlicet} in place of Green-Pedersen's \textit{vel}.} no part of Socrates is white;' and so these would be true together: `no part of Socrates is white' and `some part of Socrates is white.' 
\item[41.] And one should note that any proposition where an exclusive word is added to a word of a proposition is ambiguous according to composition and division. Just as `only man is man is true' (\textit{tantum hominem esse hominem est verum})is ambiguous because the exclusion can come about with respect to the first verb (\textit{infinitivi})or the second (\textit{indicativi}); in the first way true, in the second false.
\item[42.] And one should know that \textit{from the less to the greater with an exclusive word added to the subject} (that is, from the less to the greater on the part of the predicate) is not a valid consequence. Hence `only man runs, therefore only man moves' does not follow.
\item[43.] Likewise: \textit{from the greater to the less with an exclusive word}, arguing so, `only an animal runs, therefore only a man runs' is the fallacy of the consequent. Both in its affirmative expositions is the fallacy of the consequent, as `an animal runs, therefore a man runs'; and in its negative expositions there is the fallacy of the consequent from the destruction of the antecedent, as here: `nothing other than an animal runs, therefore nothing other than man runs'.
\end{itemize}
\section{On exceptives}
\begin{itemize}
\item[44.] What follows is on exceptives. Concerning these, one should know that any exceptive has two expositions, as `nothing besides Socrates runs' has two expositions, namely `Socrates runs' and `nothing other than Socrates runs' /f. 19va/
\item[45.] And one should know that exceptive words are `besides' (\textit{praeter}), `besides that' (\textit{praeterquam}), `except' (\textit{nisi}). Yet there is a difference between these, because `besides' and `besides that' remove from an affirmative quantitative whole, but the word `except' removes from a negative quantitative whole. Hence, `no man except Socrates runs' is properly spoken, and `every man runs except Socrates' is improperly spoken; yet `every man besides Socrates runs' is properly spoken.\footnote{Unfortunately, the syntactic properties of the Latin words mentioned here by Burley don't fully carry over into English.}
\item[46.] And an affirmative quantitative whole is a common term taken with an affirmative universal sign, and a negative quantitative whole is a common term taken with a negative universal sign.
\item[47.] One should know that every exceptive occurs in an affirmative or negative quantitative whole, or from a term standing confusedly and distributively. Hence, if an exception were to occur in something other than these, the proposition would be improper; and thus these propositions and others like them are improper: `some man besides Socrates runs' and `this man besides Socrates runs.'
\item[48.] And one should note that every exception is on account of disparity\footnote{Reading `\textit{indistantiae}' for \textit{instantiae}, with mss. BFP.} or conflict (\textit{repugnantiae}), which are the same, since\footnote{Reading \textit{quia} for \textit{unde}, with mss. BFP.} in every exceptive the excepted part conflicts with the prejacent. And the prejacent is what remains, taking away the sign of exception with the case it effects (\textit{suo casuali}). For instance, the prejacent of `every man besides Socrates runs' is `every man runs', since `besides' is the exceptive word. `Socrates' (\textit{Socratem}) is the case it effects. The excepted part is what is removed (\textit{excipitur}) by the exceptive word. Hence `every man runs' and `Socrates does not run' conflict, since from this [i.e. the latter] follows its [i.e. the former's] contradictory. Hence `Socrates does not run, therefore some man does not run' follows.
\item[49.] One should note as a rule that whenever some proposition implies the opposite of another, they conflict, and conversely; because conflict is of the nature of a contradictory,\footnote{Reading \textit{contradictorie} rather than \textit{contradictionis}, with mss. BFP.} and contradictories conflict most of all.
\item[50.] And one should note that a term following an exceptive word, if it is a common term supposits merely confusedly, since it is neither possible to descend under it conjunctively nor disjunctively, as is clear enough, since the subject in an affirmative exclusive and in a negative exceptive proposition have the same supposition.
\item[51.] And one should note that the consequence \textit{from an exclusive affirmative to a negative exceptive in the same terms} holds. Hence `only Socrates runs, therefore nothing besides Socrates runs'.
\item[52.] It is a rule in exceptives that every proposition true in part and false in part can be verified by excepting the false part; as when every man other than Socrates runs, and Socrates does not run, then `every man runs' is on account of Socrates; and if Socrates is excepted so: `every man besides Socrates runs,' this proposition will be true.
\item[53.] Likewise, one should know that in a negative exceptive the predicate is attributed to the excepted part, and removed from everything else contained under the subject of the prejacent. For instance, if I say `no man besides Socrates runs,' running is attributed to Socrates and removed from everyone other than Socrates.
\item[54.] Likewise: in an affirmative acceptive the predicate is removed from the excepted part and attributed to all the others contained under the subject of the prejacent; as in `every man besides Socrates runs,' running is removed from Socrates and attributed to all others contained, etc. And so it is exposited so: `every man other than Socrates runs, and Socrates does not run.'
\item[55.] And one should note that every exceptive conflicts with its prejacent. Hence, the exceptive is in\footnote{`in' added with mss. BFP.} an instance of the prejacent; for `every man runs' and `every man besides Socrates runs' conflict, because `every man besides Socrates runs, therefore Socrates does not run' follows, and lastly `therefore not every man runs,' which contradicts `every man runs.'
\item[56.] This rule holds generally, that from an exceptive an exclusive [follows], just as `nothing besides Socrates runs, therefore only Socrates runs' follows.
\item[57.] And one should note that in every exceptive four [things] are required, namely that from which the exception is taken, the exceptive word, the excepted part, and that with respect to which the exception occurs; as is shown in `Every man besides Socrates runs': man is that from which the exception is taketn, `besides' is the exceptive word, Socrates is the excepted part, and running is that with respect to which the exception occurs.
\item[58.] With respect to the supposition of the predicate and subject in an exceptive, one should know that that from which the exception is taken, or the subject of the exceptive (these are the same), always stands confusedly,\footnote{Omitting `and distributively', with BFP.} immovably with respect to the exception and movably with respect to the predicate. With respect to the exception it stands immovably, because it is not possible to descend with respect to it; hence`every man besides runs, therefore Plato besides Socrates runs' does not follow. With respect to the predicate it is possible to descend, since `every man besides Socrates runs, therefore Plato runs' follows, and so on for each case (\textit{de singulis}). The predicate in an exceptive stands sometimes movably, confusedly, and distributively; sometimes immovably, confusedly, and distributively. It supposits movably if the exception is taken from a transcendental; e.g. `no being besides an animal is a man.' Here, the exception is taken from a transcendental, and so it is possible to descend under the predicate, since all these are transcendentals: being (\textit{ens}), thing (\textit{res}), something (\textit{aliquid}), and one (\textit{unum}). The predicate in an exceptive supposits in the second way when the exception is taken from something specific, as here: `no animal besides man is running'; hence here it is not possible to descend under the predicate saying this: `no animal is this runner'\footnote{alternative reading: `no animal is a running man' C.}
\item[59.] Likewise one should know that if the prejacent is true, the exceptive will be false, and if the exceptive is true, the prejacent will be false. For instance, if `no man runs' is true, then `no man besides Socrates runs' will be false; and if /f. 19vb/ `no man besides Socrates runs' is true,' `no man runs' will be false. But it need not be that if the prejacent is false the exceptive is true, because `no animal is a man' is false, and likewise `no animal besides an ass is a man.' Hence an exceptive and its prejacent can be false together, but never true together.
\item[60.] One should know that the opposite of any exceptive has two causes of truth, since the contradictory of an exceptive, `not no man besides Socrates runs' has these: `someone other than Socrates runs' and `Socrates does not run.' And so if one argues so: `not no man besides Socrates runs, therefore some man other than Socrates runs,' it is the fallacy of the consequent from a proposition having several causes of truth to one of them. Likewise `not no man other than Socrates runs, therefore Socrates does not run' is the fallacy of the consequent, because it [i.e. the antecedent] has several causes of truth.
\end{itemize} 
\section{On the words `differs,' not the same', etc.}
\begin{itemize}
\item[61.] One should know that this is a general rule: \textit{whatever makes an immovable movable makes a movable immovable, and conversely}. E.g. the negation `not' makes a movable immovable in saying `not every man runs'; here `man' stands immovably, yet before adding the negation it stood movably. The negation `not', then, makes a movable immobile and an immovable movable, as in saying `not man runs', `man' stands movably, and before adding the negation it stood immovably. 
\item[62.] Likewise one should know that when one argues from something standing immovably to it standing movably it is the fallacy of figure of speech, as here: `you differ from anything (\textit{a quolibet}), therefore anything differs from you', because in `you differ from anything', `anything' stands immovably, and in `anything differs from you' `anything' stands movably, since `not' is immovable by virtue of the negation brought in by `differs', since `something you are not the same as' is valid, since `anything' and `something' are not equivalent.\footnote{Here the English does not obviously preserve the semantic properties Burley mentions.}
\item[63.] And one should note that from a difference with respect to [what is] prior follows a difference with respect to [what is] posterior. Hence, `you differ from man; therefore you differ from this man, and from that ...' follows. And that is prior which follows from a subsisting [being] not convertible with it. For instance, from man follows animal and not conversely, and so animal is prior.
\item[64.] From the first rule it follows that whatever differ in genus differ in species; and whatever differ in species differ in number. For instance, man and rock differ in species, and [thus] in number. Numerical difference is between two individuals of the same species.
\item[65.] One should know that these three have the same diffusive force (\textit{vim confundendi}), namely `differs', `not the same' and `other.' Hence `Socrates differs from man, therefore he differs from this man'\footnote{Reading `man' instead of `ass' in both places, with mss. BFP.} follows, and likewise `Socrates is other than an ass, therefore he is other than this ass', and likewise `Socrates is not the same as an ass, therefore he is not the same as this ass.' But there is a difference between `distinct' (\textit{alter}) and `other' (\textit{aliud}), because `distinct' is aptly suited to differentiate an accidental term and `other' a substantial term. Hence `Socrates is not other than an ass, therefore from this ass' follows; and `Socrates is distinct from white, therefore distinct from this white' follows so.
\end{itemize}
\section{On Conditionals}
\begin{itemize}
\item[66.] Concerning the suppositions of terms in a conditional, note that both the subject of the antecedent and the predicate supposit confusedly and distributively, and this with respect to the consequent. For `if Socrates runs, a man runs' follows; and `if a man runs, an animal runs.' Therefore \textit{from the first}, `if Socrates runs, an animal runs.' The consequence holds by this rule: \textit{whatever follows from the consequent, follows from the antecedent}. E.g. Socrates running is the antecedent and man running is the consequent; therefore if from man running follows animal running, it should follow from its antecedent, which is Socrates running.
\item[67.] Likewise, on the part of the predicate both the subject and the predicate in a conditional in a consequent stand particularly, since it can be inferred from singulars. For `if Socrates runs, a man runs; therefore if Socrates runs, an animal runs' follows. And it holds by this rule: \textit{whatever entails the antecedent entails the consequent}.
\item[68.] And one should know that nothing more is required for a good conditional than that if the antecedent is true, the consequent will be true. And so a conditional will be good though the antecedent would be false; as `if man is a stone, man is not an animal.'
\item[69.] On should know that some conditionals hold \textit{as of now}, and some \textit{simply}. An example of the first: `If you are in Rome, then that which is false is true', though when you are not in Rome, the consequence is good, but when you are in Rome it does not hold. And an as-of-now consequence is when the antecedent as of now cannot be true without the consequent.
\item[70.] Simple consequence is divided thus: some is natural and some is accidental A natural consequence is when the consequent is in the understanding of the antecedent, nor can the antecedent be true unless the consequent is true; as `if man is, an animal is.' An accidental consequence is twofold: some holds on account of the terms or on account of the matter, as `God exists is true, therefore God exists is necessary'; and this holds on account of the terms or the matter, since truth in God and necessity are the same. Some accidental consequence is so: from the impossible anything follows; and the necessary follows from anything. An example of the first `You are an ass, therefore you are a goat, and a stone...' etc. An example of the second, as `you are running, therefore God exists.' Again, an accidental consequence is when the consequent is not in the understanding of the antecedent.
\item[71.] Again: from the impossible anything follows by the topic \textit{from the less}. Because if it is argued so - `man is an ass, therefore you are running' - this consequence holds by this rule, namely if that which seems less to inhere also\footnote{Adding \textit{et} with mss. BFP.} inheres, then that which seems to inhere more will inhere; but that ass inheres in man seems less [likely] than running inheres in man.
\item[72.] Likewise: the necessary follows from anything by the topic \textit{from the greater}.\footnote{Reading (maiori) with mss. B, P, against \textit{minori} with C, O, and followed by Green-Pedersen.} Hence `you are running, therefore God exists' follows. And it holds by this rule, namely that given above, because it seems less that running inheres in man than that being inheres in God. But a consequence which is, namely \textit{from the impossible anything follows}, holds by the topic \textit{from the less}. Yet \textit{from the opposite [of the consequent the opposite of the antecedent} holds] by the topic \textit{from the greater}; for instance, if it is so - `you are not running, therefore you are not an ass,' it holds by this rule: if that which seems more to inhere also\footnote{Adding \textit{et}, with mss. BFP.} does not inhere, neither will that which seems less to inhere inhere.
\end{itemize}
\section{On oppositions}
\begin{itemize}
\item[73.] What follows concerns rules for correcting consequences, which same rules are handed down by Aristotle in the book \textit{On interpretation} and in \textit{Prior [analytics]} I. So now we must speak of these. The first rule is: \textit{from the affirmation of an indefinite predicate follows the negation of a definite predicate}; for `man is non-just, therefore man is not just' follows.'
\item[74.] And this rule should be understood thus: if it is argued in nominative, absolute terms, with a present-tense substantive verb; for the consequence is not valid for a future-tense verb, nor for a past-tense verb. Hence, `you were non-white, therefore you were not white' does not follow, since the antecedent can be true and the consequent false, positing that in time past you were white and non-white. Nor is it valid for a future-tense verb, since `you will be non-sighted, therefore you will not be sighted' does not follow, since the antecedent can be true and the consequent false; for this is true `you will be non-signted' positing that a year from now you will be blind; and `you will not be sighted' is false, because tomorrow you will be sighted positing that tomorrow you will see and a year from now [you will] not.
\item[75.] The rule does not hold in declined [terms], as `the eye is a part of a non-man, therefore the eye is not a part of man,' since the antecedent is true and the consequent false. Likewise, neither does it hold formally in transitive verbs (\textit{verbis adjectivis}); since `you see a non-man, therefore you do not see a man' does not follow, since positing that you see a man and an ass, the antecedent is true and the consequent false.
%adjectival verbs? adjacent verbs?
\item[76.] Likewise it doesn't hold generally in respective and relative terms; since `you are a non-father, therefore you are not a father' doesn't follow, since positing that you are a father and a son, the antecedent is true and the consequent is false; for `you are a non-father' is true, since you are a son, and a son is a non-father.
\item[77.] A second rule is that from the affirmation of a definite predicate follows the negation of an indefinite predicate. For `man is just, therefore man is not non-just' follows.
\item[78.] A third rule is that from the affirmation of a privative predicate follows the negation of a definite predicate, and not conversely. Hence, `man is unjust, therefore man is not just' follows; and not conversely.
\item[79.] Note that privative terms are `unjust,' `impossible,' `incongruent,' etc.
\item[80.] A fourth rule is that from the negation of a definite predicate follows the affirmation of an indefinite predicate in simple terms both in utterance and understanding, such as are `man' and `animal' and `white' taken formally and others like them. For `Socrates is not an animal, therefore Socrates is a non-animal' follows. In composite terms, though, it is invalid; since `man is not a white tree, therefore man is a non-white tree' does not follow, since the antecedent can be true and the consequent false. The falsity of the consequent is clear, since it implies a falsehood, namely `man is a tree.' The consequent is false, therefore the antecedent by the rule posited earlier. For the rule posited earlier is that if the consequent is false, then the antecedent is false, since a falsehood does not follow except from falsehoods.
\item[81.] A fifth rule is this: \textit{from the negation of a privative predicate the affirmation of a definite predicate does not follow}. Hence `man is not unjust, therefore man is just' does not follow, since the antecedent can be true and the consequent false, positing that no man exists.
\end{itemize}
\section{On syllogistic consequences}
\begin{itemize}
\item[82.] The rules proving syllogistic consequence are two. The first is \textit{if from the opposite of the conclusion of a syllogism with the other premise the opposite of the other premise follows, the first syllogism was good.}
\item[83.] And the second rule is \textit{what does not follow from the antecedent with something added either syllogistically or in another way does nto follow from the consequent with the same added.}
\item[84.] And from this rule follows another, \textit{whatever follows from the consequent with something added follows from the antecedent with the same added}, reckoned necessarily or formally. For since from `every man is an animal' with `Socrates is a man' follows `Socrates is an animal' so from `every substance is an animal,'\footnote{`Every substance': following mss. LO, against the edition's of (\textit{homo}) and `Socrates' in P.} (which entails `every man is an animal') with `Socrates is a man' follows `Socrates is an animal'; since the first consequence is syllogistic, the second is necessary.
\item[85.] Likewise: whatevever follows from the antecedent and the consequent follows from the antecedent by itself (\textit{per se}). For since from `this ass is only one man' with `this ass is a man' (which is consequent upon it, the conclusion `a man is only one man' follows formally, thus from `this ass is only one man' follows the same conclusion; at least just as from the imposible anything follows. Yet the way something follows from an antecedent and consequent need not be wholly the same as how it follows from the antecedent by itself.
\end{itemize}
\section{On enthymematic consequences}
\begin{itemize}
\item[86.] The rules proving an enthymematic consequence are these: \textit{the opposite of the consequent does not stand with the antecedent, and\footnote{Adding \textit{et} with mss. BFP.} so the consequence is good.} since `no animal is' does not stand with `man is', thus `man is, therefore an animal is' follows. What stands with the antecedent stands with the consequent; for what stands with `man is' stands with `an animal is'.
\item[87.] From this rule follows this: \textit{what conflicts with the consequent conflicts with the antecedent}. For since `no animal is' conflicts with `an animal is,' so it conflits with the antecedent, which is `man is'.
\item[88.] This rule also follows from the prior, \textit{what stands with the antecedent, stands with the consequent.} Likewise from the opposite: \textit{what does not stand with the consequent does not stand with the antecedent.}. And to not stand with another is to conflict with it. Therefore, what conflicts with the consequent conflicts with the antecedent.
\item[89.] Another rule is: \textit{if the antecedent is, the consequent is}. The consequence is clear, since being and truth convert.
\item[90.] Another rule is: \textit{if the antecedent is possible, the consequent is possible}, following to what is proven in the \textit{Prior Analytics}.\footnote{\textit{Anal. Prior.}, 1, 13, 32b 26.}
\item[91.] Another rule is: positing [something] possible in being, the impossible does not occur. Which is proven thus, since if the impossible occured from positing [something] possible in being, then since the antecedent is possible, from the possible the imposible would be able to follow. And so positing this in being `you being in Rome is possible, therefore you are in Rome,' [would follow]. But if it happens to be false, it is not impossible.
\item[92.] Another rule is this: From the differentness (\textit{alietas}) of the consequent follows the differentness of the antecedent. Hence, `Socrates is other than animal, therefore Socrates is other than man' follows. Against this rule it is objected so: `man is man, therefore man is an animal'; yet `something other than man is an animal, therefore something other than man is man' does not follow, since the antecedent is true and the consequent false. One should say to this that the rule has to be understood of a consequent in what follows (\textit{consequens in consequendo}) and is predicated, because animal is consequent upon man: for it is predicated of man, and so is consequent. And so it is not from the consequent `man is an animal', since it is only consequent in what follows.
\item[93.] Another rule is: from any proposition follows its statement to be true, just as `man is, therefore that man is is true' follows. And the statement of a proposition is when the nominative is turned into an accusative, and a verb in the third-person into the infinitive mode, as `man is' (\textit{homo est}) is converted into `that man is' (\textit{hominem esse}).
\item[94.] One should note that these three are the same in reality: something speakable (\textit{enuntiabile}), statement (\textit{dictum}), and infinitive speech (\textit{infinitiva oratio}).
\item[95.] Another rule is that everything having itself addition with respect to another is inferior to it. And so `white man' and the like are less than `man' only\footnote{`only': following mss. BFP, against the edition's \textit{saltem}.}  accidentally. And so in kinds of consequences [such as] `a white man is, therefore a man is' here one proceeds \textit{from the less to the greater}.
\item[96.] Another rule is: for subjects occuring under\footnote{Adding \textit{in} with mss. BFP.} the same predicates varied according to below and above, a consequence holds. For instance, `Socrates was white in \textit{a}, therefore Socrates was white'; and let \textit{a} be yesterday.
\item[97.] Another rule: \textit{A disjunctive follows from either of its parts}; as `Socrates runs, therefore Socrates or Plato runs.'
\item[98.] Another rule is: \textit{the negation of a disjunct is equivalent to a conjunction}. And so this consequence will be good: `not Socrates or Plato runs, therefore neither Socrates nor Plato runs;' since these two propositions are equivalent.
\item[99.] Another rule is: \textit{a negation [of a conjunction] has three causes of truth}. For instance, `not Socrates and Plato runs', [has] either that Socrates does not run, or that Plato does not run, or that neither Socrates nor Plato runs. And if one argues from the negation of this conjunction to either of its [disjuncts] it is the fallacy of the consequent.
\item[100.] Another rule is: the opposite of a conditional is to be given by placing a negation before the whole. Hence the opposite of `if a man is, an animal is' is `not if a man is, an animal is,' which is equipollent to a man being and no animal being. And these do not stand together.
\item[101.] Another rule is that in singulars placing the negation before or after makes no difference. Hence, the proposition `Socrates runs' is contradicted by each of `Socrates does not run' and `not Socrates runs.' The reason for this rule is that singulars do not have many supposita under them, so placing the negation before or after makes no difference.
\item[102.] Another rule is that in necessities speaking universally or particularly makes no difference. Hence it makes no difference saying `man is an animal' and `every man is an animal.' Against this rule it is objected so: `an animal is a man' is necessary,' and yet saying `an animal is man' and `every animal is man' makes a difference with respect to truth and falsity, since the first is true and the second false. One should say that the rule has to be understood where the predicate is direct and proper, and it is not so in the proposed, since in `an animal is a man' an inferior is predicated of its superior.
\item[103.] Another rule is this: \textit{if one of opposites is ambiguous, the other will be ambiguous}. And the rule holds generally in all complex oppositions. And if one of opposites is ambiguous for the reason that in one proposition it is ambiguous, and for the same reason in the other, then consequently [it is] in both. Hence one should know that `every dog runs' is ambiguous, so what appears opposite to it will be ambiguous, namely `some dog does not run.' And the cause is that the reason for ambiguity remains in each of the terms, i.e. in the term `dog'. And note that this rule ought to be understood of complex opposites. If we wish to extend this rule to the opposition that is between complexes,\footnote{Following the ms. FOP `\textit{complexa}' rather than ms. B and the editor's `\textit{incomplexa}'.}  then one objects to this rule similarly with respect to privative and relative opposites so: `none running, you are an ass' must be disambiguated, yet its opposite need not be disambiguated, namely `something running you are an ass.' One should say that the cause of ambiguity does not remain the same in each, since the first must be disambiguated by reason of the negation, so that the negation is the whole cause of the distinction. Since this does not remain in its opposite, therefore etc.
\item[104.] Another rule is: in accidents, /f. 20va/ the consequence \textit{from inherence to being} does not hold. Hence `whiteness is in Socrates, therefore Socrates is white' does not hold, since positing that it is only in Socrates' teeth, then the antecedent will be true with the consequent being false. And note that only in accidents does the consequence \textit{from inherence to being} fail. 
\item[105.] Another rule is: in contradictories the consequence from the contrary holds. And a consequence holds from the contrary when from the opposite of the consequent follows the opposite of the antecedent, since\footnote{Reading \textit{quia} with BFP, rather than the edition's \textit{sicut}.} from the antecedent follows the consequent. For instance, from a man being follows an animal being, so from no animal being follows no man being.
\item[106.] Another rule is: \textit{in contraries and privations and relatives, a consequence holds in these themselves.} And for a consequence to hold in these themselves is when from the opposite of the antecedent follows the opposite of the consequent, just as from the antecedent follows the consequent. So\footnote{Reading \textit{sic}, with BFP, rather than \textit{sicut}.} from `Every man runs' follows `Socrates runs,' so that from `no man runs' follows `Socrates does not run,' which is opposed to the previous consequence. And this rule is good in complex contraries, when that which is inferred from one of contraries is inferred from it formally and on account of distribution. Hence, though `no man is one man alone, therefore Socrates is not Plato' follows, nevertheless `Every man is one man only, therefore Socrates is Plato,' does not follow, since `no man is one man only' implies `Socrates is not Plato' by reason of negation and by reason of distribution.
\item[107.] Another rule is: \textit{if the abstract from the abstract, then the concrete from the concrete}. Hence if `whiteness is color' is true, `what is white is colored' will be true. Likewise, if the concrete from the concrete, then the abstract from the abstract; for instance, `if what is white is colored, white is a color' as is clear. Against this rule one argues so: `what is white is musical' is true, yet `whiteness is music' is false; therefore, etc. One should say that this rule has to be understood thus: if one concrete is posited per se on account of form, then also in concrete things pertaining to matter and form. Hence, the second conditional is lacking here, namely `man is an animal, therefore humanity is animality.' And one should note that a concrete thing pertaining to a subject is as `man' and `animal', a concrete pertaining to matter and suppositum\footnote{Reading \textit{suppositum}, with BFP, rather than the edition's \textit{formam}.} is as `white' and `black.'
\item[108.] Another rule is: \textit{if the medium is a particular object (hoc aliquid), the extremes must be conjoined}. Therefore this syllogism is good `this man is an animal, this man is a man, therefore a man is an animal.' And so a medium is a particular object when the medium is some singular, as `this' or `that' and so for singulars. Against this rule it is argued thus: this syllogism is not valid: `this man is an individual, this man is a man, man is an individual,' since the premises are true and the conclusion false. One should say that the rule has to be understood where the medium is a particular object and not varied; but the medium in the proposed is varied, hence, it is invalid.
\item[109.] Another rule is: \textit{when the extremes convert, it is necessary for the medium to convert with the extremes}. For instance, if one argues `every $b$ is $a$, every $c$ is $b$, therefore every $c$ is $a$'; and if $c$ and $a$, which are the extremes, convert, it is necessary that $b$, which is the medium, converts with them. Against this rule one argues thus: `every man is capable of braying, every ass is a man, therefore every ass is capable of braying.'; the extremes, namely `ass' and `capable of braying' convert, and yet that which is the medium does not convert with them, that is with `ass' and `capable of braying'. One should say that this rule has to be understood of unqualifiedly probative syllogisms (\textit{syllogismo simpliciter ostensivo}), where the premises are true, hence if the premises are true, falsity cannot arise. 
\item[110.] Another rule is this: \textit{a proposition having one cause of truth is true simply, notwithstanding that it have several causes of falsity}. And the reason for this is that a cause of truth entails (\textit{antecedit}) what it is the cause of. But if the antecedent is true, then the consequent will be true. But it does not follow [that] if the antecedent is false that the consequent is false. And so notwithstanding that a proposition have several causes of falsity, if it yet have one cause of truth, it will be called true.
\item[111.] Another rule is this: \textit{when something inheres in many things, and it does not inhere in one by another, it inheres first in something that is common to them.} For instance: risibility inheres in Socrates and Plato, and not in Socrates through Plato or the contrary; therefore, it inheres first in man. Against this rule one argues thus: the property which is `being primarily generated' is in Socrates and Plato, and not in one through the other, for `Socrates is primarily generated' and `Plato is primarily generated' is true; Therefore, it inheres primarily in man. Which is false, since `man is primarily generated' is false, because generation is a real act primarily of singulars. One can say that the rule has to be understood in things ordered essentially. But Socrates and Plato are not of this type. Therefore, etc. Or one can say that the passion `being generated' inheres in Socrates and Plato through something common in them, and inheres in that common [thing] primarily. And that common [thing] can be called a concrete universal (\textit{universale concernens}).
\item[112.] Another rule is: \textit{if the cause is a particular [exercising] its proper act, its effect will be proper}. For instance, if there is building, then a house is being made'; since for a house to be made is the proper effect of building.\footnote{Following LO's reading of \textit{aedificantis}, rather than \textit{aedificatoris}.}
\item[113.] Another rule is: \textit{if [one] of opposites is said of [another] opposite, then the other of the other (\textit{propositum} of \textit{proposito})}. For instance, man and non-man are opposed, ass and non-ass are opposed; if man, then, is a non-ass, an ass will be a non-man. Against this rule one argues so: `every man' and `not every man' are opposed; and Socrates and non-Socrates are opposed; if, then, not every man is non-Socrates, every man will be Socrates; therefore Socrates will be every man; which is impossible. One should say that this rule has to be understood where one of the opposites of one order does not extend itself to two of the other order. But `every man' in the proposed extends itself to Socrates and non-Socrates, for instance, to Plato. So this consequence is not valid.
\item[114.] Another rule is: \textit{one of contradictories is said of anything; both, of none}. For instance, `man' and `non man', which are incomplex contradictories, and one or the other of them is said of anything, and both of nothing. For `an ass is a non man' is true. Therefore, `non-man' is said of anything distinct from man, and `man' is said of any man. Since everything that is is either man or non-man, it follows that one or the other of these is said of anything. Against this rule one argues so: running and non-running contradict; and yet neither of these is true of Socrates alone, since neither `Socrates alone runs' nor `Socrates alone does not run' is true; therefore the rule is invalid. One should say that the rule has to be understood thus: one of contradictories is said of any simple, and both of none. Now `Socrates alone' is not simple, but rather composite. 
\item[115.] Another rule is this: \textit{A good consequence holds by a necessary medium}. Hence, the good consequence `a man is, therefore an animal is' holds by the medium `every man is an animal.' The reason for this rule is this, that the medium is the cause of the consequence, but if the effect is, it must be that the cause of that effect is. And if the consequence is good, it must be that the medium is true at every time, and consequently necessary, since whatever is true and unable to be false is necessary. And this must be understood of simple consequence, since it holds for every time.
\item[116.] One should know that a consequence is always good when it holds through a true medium. But an as-of-now consequence holds through a medium true as of now,\footnote{Reading `\textit{ut nunc}' rather than the edition's `\textit{et nunc}'.} just as `if the Antichrist exists is so, the false is true' holds by the medium `that the Antichrist exists is false.' But a simple good consequence holds through an intrinsic necessary medium; but an accidental consequence holds through an extrinsic medium, such as `an accident is, therefore a substance is'\footnote{Following the reading in L, `\textit{accidens}', rather than P's reading of `\textit{animal}'.}
\item[117.] Another rule is this: \textit{every good consequence has to be reduced to a syllogism}. In different ways, though: for a consequent holding by a descent made under the subject has to be reduced to a syllogism by the assumption of a minor [premise]; as the consequence `every man runs, therefore Socrates runs' has to be reduced to a syllogism by the assumption of the minor `Socrates is a man.' If an ascent occurs above the predicate, it has to be reduced to a syllogism by the assumption of a major [premise]; as `every man is a man, therefore every man is an animal' has to be reduced into a syllgoism by the assumption of the major `Everything that is a man is an animal, every man is a man, therefore every man is animal.'
\item[118.] On the contrary: the consequence `every man is an animal, therefore only an animal is a man' is good; and yet it does not have to be reduced to a syllogism. Just as the consequence `a man runs, therefore an animal moves' is also not reduced to a syllogism, since there are four terms. One should say that consequence is twofold, since some [kind] is which holds with respect to the whole structure \textit{totius complexionis}, and some holds with respect to the terms. The first kind does not need to be reduced to a syllogism. All conversions are of this sort. Likewise, when one argues from an exclusive to a universal with the terms transposed and conversely, it does not need to be reduced to a syllogism, since it holds by reason of the whole structure. To the other, one should say that it has to be reduced to two syllogisms. Hence the sense of the rule is: every consequence holding with respect to the incomplex [parts] or formally of an incomplex [part] has to be reduced to a syllogism or syllogisms. The reason for this rule is that in every genus there is one first, which is the ruler and measure of all existing in that genus, just as the ounce in the genus of weights and whiteness in the genus of colors. In this way, the syllogism is one first in the genus \textit{habit}, from which all habits receive evidence of [their] necessity. So all other habits have to be reduced to a syllogism, as the posterior to what is prior to it.
\item[119.] Another rule is this: \textit{from every exercised act follows the sign of the act}; as `every man besides Socrates runs, therefore Socrates is excepted from man' follows; likewise, `no man is an animal, therefore man is denied of animal.' And one should note that the word `excepted' signifies an exception, which the adverbs `besides' and `alone' exercise; and `none' and `not' exercise negation, which the verb `is denied' (\textit{negatur}) signifies; `if' and `therefore' exercise consequence, which the word `follows' or `implies' signifies. But since these do not seem especially useful, the converse of these rules should be noted; this is valid as a general rule: From any signified act follows the exercised act. Hence, `man implies animal, therefore if a man is, an animal is' follows. Likewise, `man is denied of animal, therefore no man is an animal.' Likewise, `man is denied of white, therefore no man is white', etc.
\item[120.] Another rule is: \textit{from the less to the greater with a sign of alterity is not a valid consequence}. Hence, `Socrates is other than an ass, therefore he is other than an animal' does not follow. Nor is such a consequence with a sign of difference valid, just as `Socrates differs from a stone, therefore he differs from substance' does not follow, since the antecedent is true and the consequent is false. The falsity of the consequent is clear, since `you differ from substance, therefore, you differ from this substance', pointing to your substance, follows, since `differs' has the force of diffusing a common term to which it is immediately joined. The truth of the antecedent is shown, since `Socrates is, and a stone is, and Socrates is not a stone, therefore he differs from a stone' follows; since from the denial of something of another, assuming the being of the extremes, the alterity, that is the privation, of the one from the other follows.
\item[121.] Another rule is this: in relatives, a consequence \textit{from [something in a certain respect] to it simply} holds; since `a mountain is small with respect to a greater one, therefore a mountain is small' follows; since great and small are in the genus of relation. Likewise, `a two cubits is not double with respect to three cubits, therefore two cubits is not double.' Likewise `Two cubits is double with respect to [its] half, therefore two cubits is double and not double',\footnote{Adding `\textit{et non duplum}' with mss. BFP.} since `double' and `not double' are relatives.
\item[122.] Another rule is this: \textit{if the possible is posited, the necessary should not be denied, nor should the impossible be conceded}. For instance, if I posit that you are in Rome, one ought not therefore concede that you are not an animal, nor something else impossible. Neither the expression [posited] nor some other necessary one should be denied, since to deny the necessary is to concede the opposite of the necessary; But the opposite of the necessary is the impossible; therefore, to deny the necessary is to concede the impossible.
\item[123.] Another rule is this: \textit{from the attribution of some predicate to some subject universally follows the removal of the same predicate from the opposite of the subject particularly}; since `Every man is an animal, therefore some non man is not an animal' follows. This consequence can be proven thus: since `every man is an animal, therefore every non-animal is a non-man' follows by conversion by contraposition; and further `therefore a non-man is not an animal'\footnote{`not an animal': following mss. BF, against the edition's \textit{est non animal}.} by this rule: \textit{from an affirmative with an indefinite predicate follows a negative with a definite predicate}. Against this one argues: if this is given, falsehood follows from truth, because then the consequence `Every non man is imaginable, therefore some man is not imaginable'. One should say that the \textit{from the conversion of some predicate}, etc. has to be understood in simples according to utterance and understanding, when it is such that the predicate in a universal does not apply to the opposite of the subject, as is so\footnote{Following mss. BFP against the edition in omitting `\textit{non}'.} in the proposed, since `imaginable' is the same as `imaginable thing'. And so it is invalid.
\item[124.] Another rule is: the consequence \textit{from being predicated as a third adjacent to being predicated as a second adjacent} holds, where the opposite is not [implied] in what is added, as in `a man is dead, therefore a man is'; and where the predication is not according to an accident,\footnote{Adding `non', with mss. BFP.}. For instance, the consequence `Homer is something like a poet, therefore Homer is,' does not hold. Against this: `a man is running, therefore a man is' surely follows; likewise, `a man is white, therefore a man is'; and yet `a man is dead, therefore a man is' does not follow, nor does `Homer is something like a poet, therefore Homer is' follow. And the reason for the disparity is that in the first the opposite is in what is added, and in the second, the predication is [not] according to an accident.\footnote{Adding `non' to conform with the previous.}
\item[125.] And one should note that Aristotle understands by `predication according to an addition or according to an accident'\footnote{Following mss. BP in adding `\textit{secundum adiacens vel}'.} when the predicate is what is not required for the subject to exist, in the way all substantial predicates generally are.
\item[126.] Another rule is this: no accidental destroys its substantial. For instance: the significate of the term `man' is essential to it, and so by no accidental added to it is its significate taken away, so that whether it is posited in speech or outside it or whatever it is matched with, it signifies `rational animal'. Against this rule one argues so: `man' naturally looks to its present, past, and future supposita. Yet if it is matched with a present-tense verb, it supposits only for present [things]. But being matched with a present-tense verb is accidental to it. Wherefore, etc. One should say that though an accidental does not destroy its substantial, though it can destroy its essential, as is the case in the proposed. And hence, if `man' in general (\textit{in communi}) is placed in a phrase, it supposits only for present [things]; yet its significate is not destroyed with respect to past or future [instances].
\item[127.] Another rule is this: \textit{whatever `being' and `one' are added to do not bring about a distinction}, since they convert. Against this one objects thus: that then from truth falsity would follow, for `every man is an animal, therefore being every man is an animal'; and yet the antecedent is true and consequent false. The falsity of the consequent is clear because it implies being every man is something. Likewise, it implies that just as `every man runs' is universal, so `being every man runs' would be universal; the antecedent is true and the consequent false, because the a particular sign can be added to the subject of the consequent, saying `some being every man runs.' One should say that the one has to be understood, that `being' and `one' added to whatever simple do not bring about a distinction.
\item[128.] Another rule is this: \textit{when two functions are included in one word, the one pertains to nothing without the other pertaining to the same.} Against this rule one objects so: the word `not' includes two functions, namely distribution and negation, and yet negation attaches to a predicate and distribution doesn't, as in the phrase `no man is an animal.' One should say that the rule has to be understood of a word having two functions, neither of which is negation. But this is not so in the proposed. Wherefore, etc.
\item[129.] Another rule is this: whenever some are the same as one and the same thing numerically, they are the same as each other, for instance, man and animal are the same as Socrates, /f. 21rb/ and so man and animal are the same as each other. For `man is an animal' is true. Against this rule: man and ass are the same as animal, therefore they are the same as each other. The consequent is false, therefore also the antecedent. One should say that the rule has to be understood of those which are the same as one and the same [thing] numerically. In this way, man and ass are not the same as animal in number, but in genus. Wherefore, etc.
\item[130.] Another rule is that from one nothing follows, either enthymematically or syllogistically. Hence, from `a man runs', removing whatever other proposition following its meaning and utterance, neither `an animal runs' nor anything else follows. Against this: `Socrates runs, therefore a man runs' [follows], and yet only one [proposition] is put in the place of the antecedent. Likewise, from every good enthymeme [composed] of one [antecedent] follows another. One should say that the rule has to be understood of one as one, prescinding from whatever [else] is expressed and understod. To the argument to the contrary one should say that limiting `Socrates is a man', [then] in reality and in utterance `Socrates runs' therefore a man runs' no more follows than `therefore a rock runs' or `a rock runs, therefore a man runs.' Hence, in every good enthymeme, even if only one is expressed, another still is understood in virtue of which the consequence holds; just as in `Socrates runs, therefore a man runs,' `Socrates is a man,' which is really part of the antecedent, is understood.
\item[131.] Another rule is: \textit{a negative proposition has multiple causes of truth}, if it is a negative implication. Hence, `not a man who is white runs' can be verified, either because no man is white, or because no man runs, or because neither some man is white nor does some man run.
\end{itemize}
\section{On propositions with declined [terms]}
\begin{itemize}
\item[132.] Concerning propositions with declined [terms] certain rules should be known. One is: \textit{when a nominative and declined are placed in some proposition in the same part of a sentence, one should consider whether they are placed in the part of the subject or in the part of the predicate}. For instance, `this horse is a man's horse', the whole `a man's horse' is the predicate. Nor does it much matter whether the nominative is placed before the declined, or the declined before the nominative in the part of the predicate. Hence these two do not really differ: `this horse is the horse of a man' and `this horse is a man's horse.' If the nominative and declined are placed in the part of the subject, then one should consider whether the nominative precedes the declined or conversely. If the nominative precedes the declined, the whole is held to be in the part of the subject, as when saying `any ass of a man runs,' the whole `ass of a man' is the subject, and under this one should take up the minor `this ass of a man is an ass.' And note that the whole `ass of a man' is a  subject of distribution, and that under a subject of distribution one should assume [its particulars], therefore etc. But if the declined precedes the nominative, then the nominative is held to be in the part of the predicate, and the declined in the part of the subject. For in saying `of any man an ass runs' `ass' is held [to be] in the part of the predicate, and `of man' is the subject of distribution, and so under man one should make the assumption of one should syllogize from this proposition, for instance `of any man an ass runs, this man is a man, therefore of this man an ass runs.' One should note that any declined can be the subject of distribution, but not any can be the subject of a locution grammatically speaking, since a declined is not able to supposit with respect to a verb, since it depends on a nominative. 
\item[133.] Another rule is this: \textit{any possessive construction, of the sort as `of any man an ass runs,' has to be resolved into a nominative of the posessor, and an accusative of possessing with the verb `posesses'.} Hence the proposition `of any man an ass runs' has to be resolved thus: `any man possesses a running ass.' And one should see that the mode of supposition in the resolving and the resolved declined always remains the same. 
\item[134.] Concerning the mode of syllogizing with declined propositions there are certain rules. The first is this: \textit{if the major occurs in declension, and the minor in the nominative, the conclusion follows in declension}. Thus: `of any ass a man runs, this man is a man, therefore of this man an ass runs.' 
\item[135.] A second rule is: \textit{from a major occuring in the nominative and a minor in declension follows a conclusion in declension}, if the major would be affirmative.\footnote{`if the major ... affirmative': added with mss. BFP.} Thus, `every wisdom is a discipline, of the good is wisdom, therefore of the good is discipline.' If the major were negative, then it is not valid. Thus, `no head is a hand, of Socrates is a head, therefore of Socrates is not a hand.' Against this rule one argues so: the syllogism `every ass runs, of Socrates is an ass, therefore of Socrates runs' is not valid, because the premises are well-formed (\textit{congruae}) and the conclusion badly-formed (\textit{incongrua}), positing the case to be possible. One should say that the conclusion follows provided that the 'running' is taken substantively, and then the conclusion is true just as the premises. 
\item[136.] A third rule is that from both premises occurring in declension a conclusion follows both in the nominative and in declension. Thus, `What the discipline is of, the genus is of. The discipline is of the good. Therefore the genus is of the good.'; and in the nominative, thus: `What the discipline is of, the genus is of, the discipline is of the good, therefore the genus is good.' One should know, though, that these syllogisms are not purely categorical, but hypothetical in a way, being equivalent to conditional syllogisms.
\item[137.] In the second figure these rules are given: \textit{a negative major occurring in the nominative} /f. 21va/ \textit{and affirmative minor in declension, a conclusion follows neither in the nominative nor in declension}. That it does not follow with terms in the nominative is shown thus: `no animal is a head, of any man is a head, therefore no man is an animal';\footnote{Following mss. LO, rather than BP, which have \textit{capra} for \textit{caput}, and \textit{asinus} for \textit{animal}.} That the conclusion does not follow in declension is shown for the same case thus: `no ass is a head, of any man is a head, therefore of no man is an ass'; The premises can be true with the conclusion being false, positing that each man has an ass.
\item[138.] Another rule: \textit{If a negative premise occurs in declension, and an affirmative in declension, a conclusion follows both in the nominative and in declension}. Thus, every ass is an animal, of no man is an animal, therefore of no man is an ass.
\item[139.] Another rule: each premise occurring in declension, a conclusion can follow in the nominative, whether the major or the minor is affirmative. Thus, `of no ass is a horse, of any man is a horse, therefore no man is an ass.' Yet the conclusion does not follow in declension, since the premises can be true and the conclusion false.
\item[140.] In the third figure are rules, the first of which is this: \textit{from each in declension and each affirmatively a conclusion never follows}. Hence `of any man is an ass, of any man is a horse, therefore some horse is an ass' does not follow. 
\item[141.] Another rule is: \textit{from each occurring in declension and one negatively, a good syllogism with respect to a conclusion in the nominative can occur}, like so: `of no man is an ass, of any man is a horse, therefore no horse is an ass.'\footnote{The form in the example is invalid. To see this, substitute `tiger' for `ass', and `cat' for `horse', assuming that each man owns a house cat.}
%Invalid. substitute `ass' for `tiger', `horse' for `cat' 
\item[142.] Another rule: if one occurs in the nominative and the other in declension, if the syllogism is negative, then for such a syllogism to be valid it must be that the premise in declension is negative and the affirmative in the nominative; but if the converse [holds], the syllogism is invalid, since the premises can be true with the conclusion being false, so: `no man is an ass, of each man is this ass, therefore this ass is not an ass.' 
\item[143.] Another rule is: one premise occuring in the nominative and another in declension in an affirmative syllogism, for the syllogism to be good it must be that the proposition which is the major after reducing it to the first figure would be in declension; then it must be that the major extremity in the conclusion is taken in declension. Otherwise the premises can be true with the conclusion being false, arguing so: `every man is an animal, of each man is a hand, therefore a hand is an animal.' 
\item[144.] Hence, to see generally how one ought to syllogize from declined [terms] in the third and second figure, one should see whether the syllogism brought about in each of these after reducing it to the first figure is useful. If so, the syllogism was good; if not, it was not good.
\end{itemize}
\section{General rules}
\begin{itemize}
\item[145.] Besides the rules already mentioned, there are other general rules, the first of which is this: for any possible proposition, one or the other of contradictory [propositions] is compossible with it. Against this: this is a possible proposition, `Socrates sees only every man not seeing himself,' since it is possible that Socrates sees only every blind man, and since every blind man is not seeing himself; and yet this is compossible with neither of these, namely with `Socrates sees himself' nor `Socrates does not see himself.' One should say that the proposition `Socrates sees only every man not seeing himself' is not possible, since it includes opposites, since from this it follows that Socrates does not see himself, nor does he not see himself. To the proof, one should say that the consequence `Socrates sees every blind man, therefore Socrates sees every man not seeing himself' is invalid.
\item[146.] Another rule is: every proposition including opposites is impossible; just as this, `You know you are a stone.' For it includes these two `you are a stone' and `you are not a stone,' since from the fact that it posits you to know, it posits you not to be a stone, since a stone does not know; and from the fact that it posits you to know you are a stone, since noting is known unless true, it posits you to be a stone. And just as it is with this proposition including opposites, so it is with any other.
\item[147.] Another rule is: every true proposition of the present leaves another true of the past. For instance: if `you are keeping vigil' is true of the present, tomorrow `you kept vigil' will be true of the past. Against this rule one argues so: this was at some time true of the present, `every animal is in the ark of Noah', namely in the time of the flood; and yet its past [tense] is false, namely `every animal was in the ark of Noah', on account of the animals corrupted. One should say that this is not its past [tense], since its past [tense] should be given for the same supposita. Thus, this its proposition of the past corresponding to it, `every animal that was in \textit{a} was in the ark of Noah'; and let \textit{a} be the time of the flood, and this is true, justas the other.
\item[148.] Another rule is this: every true proposition of the past is necessary. And if this is true of the past `you were a boy or white', it cannot not be true, and consequently is necessary, since everything which is truth not able not to be true is necessary. Against this: this is true of the past `you've not been to Rome,' and yet it is not necessary, since a year from now it could be /f. 21vb/ false. One should say that this rule ought to be understood: every proposition true of the past and affirmative is necessary. But this is not so in the proposed. Wherefore, etc.
\item[149.] Another rule is this: Every false negative proposition of the past is impossible. And this follows from the the other, because if every affirmative and true proposition of the past is necessary, the opposite of any such proposition is impossible, since the opposite of the necessary is impossible. But the opposite of such is some false negative proposition of the past, and consequently any such is impossible. But the rule does not extend itself to false affirmative propositions of the past, since the proposition `you've been to Rome' is false of the past, and yet a year from now can be true. 
\item[150.] Another rule is that from the contingent the necessary can follow, for `a man is white, therefore a man is a way [of being] (\textit{qualis})'; the antecedent is contingent, and the consequent is necessary. The necessity of the consequent is clear, since it is implied by a necessity: `a man is risible, therefore man is a way [of being]' follows; the antecedent is necessary, therefore the consequent.
\item[151.] Another rule is: \textit{from the possible the impossible does not follow}. Against this: `every man runs, every man is sitting, therefore something sitting is running' follows; the conclusion is impossible, but the premises are possible. One should say that this rule has to be understood this: from possible propositions, if they are compossible with each other, the impossible never follows. So in the proposed, though each of the premises are possible, still they are incompossible with each other.
\item[152.] Another rule is this: when two propositions are incompossible, from one follows the opposite of the other, just as is clear in the argument made above. For these two - `every man runs' and `every man is sitting' - are incompossible, and so from one of them follows the opposite of the other: for `every man runs, therefore every man is not sitting' follows. 
\item[153.] Another rule is this: for any proposition of inherence, a proposition of necessity conflicts with it, since\footnote{Reading \textit{cum} for the edition's \textit{eiusdem}.} the power of the necessity will conflict with that of the [proposition of] inherence. For instance, the proposition `of necessity every man is an animal' conflicts with the proposition of inherence `some man is not an animal' so it will conflict with `it is possible that some man is not an animal'. And the reason for this rule is this, that necessity bring with it perpetual truth, and so what conflicts with the act conflicts with the power of the same act.
\item[154.] One should note that a proposition of inherence as-of-now can be true and false, as `man is running'; and a proposition of inherence simply is always true, as `man is an animal'
\item[155.] Another rule is this: \textit{from an impredicative inferior to the superior containing it} is not a good consequence, but is the fallacy \textit{from qualifiedly to unqualifiedly so} (\textit{secundum quid et simpliciter}). For instance, `Socrates knows himself to know nothing, therefore Socrates knows something knowable'. Here, `knowing oneself to know nothing' is inferior to `knowable.' And one should know that where one argues \textit{from a self-referential inferior}, etc. one of opposites is determined by the other, just as `Socrates knowing nothing' is determined by what I call `knowable', which is something to know.
%revise the above
\item[156.] And one should note that when the same is reflected upon its own privation, if one argues from the privation it is the fallacy of the consequent or \textit{from qualifiedly to unqualifiedly so}. Likewise, when something is reflected upon its own opposite, as `I see myself not to see', `I know myself not to know, or to know nothing'. And generally in all such there is reflection, and in all such the consequence \textit{from the lesser to the greater} is invalid. Against this rule it is argued so: `Plato knows Socrates knows nothing, therefore Plato knows [something] knowable'; therefore by the same reason, `Socrates  knows himself to know nothing, therefore Socrates knows [something] knowable'; and yet here there is reflection. On should say that `Plato knows Socrates knows nothing, therefore Plato knows [something] knowable' follows well; `Socrates knows nothing', with respect to the knowledge of Plato, is one knowable [thing] unqualifiedly. Yet it does not follow of Socrates, since `Socrates knows nothing', with respect to that of Socrates, is one knowable [thing] in a certain respect. And so proceeding thus from that to its common is the fallacy \textit{from qualifiedly to unqualifiedly so}. 
%revisit the above
\item[157.] Another rule is: \textit{every true proposition of the past has another true proposition of the present}. For instance. `you ran' is true of the past, and so implies the proposition of the present `you are running'. Against this one argues so: `you existed yesterday' is true of the past, and yet `you are yesterday' is never true of the present. And likewise, `the day was past' is true of the past, and yet `the day is past' is never true of the present. One should say that this rule has to be understood speaking of a thing \textit{per se} and not \textit{per accidens}, accidentally. 
\item[158.] And one should know that composition is twofold, namely substantial and accidental. Substantial is where the predicate is from the intrinsic nature account of the thing, as `man is an animal'; accidental, where the predicate is not intrinsic etc., but carries some accident in, as `man is white' and [others] of this sort.
\item[159.] Another rule is this: \textit{each for a reason, and that [reason is] greater} (\textit{unumquodque propter quod et illud magis}). For instance, `if water is hot because of fire, fire is more hot.' Likewise, if a boy is loved on account of his nurse, it follows that the nurse is loved more. Against this it is argued: a man is drunk on account of wine, therefore the wine is more drunk.' Likewise, `a sword is hot on account of burning wood, therefore the burning wood is hotter' does not follow. One should say that /f. 22ra/ the rule has to be understood formally in principal causes, in causes both through themselves and univocal. But wine is not a univocal cause of drunkenness in this way. To the other objection, I say that burning wood is not per se the cause of heat, but with respect to a part; wherefore the objection fails. 
\item[160.] Another rule is this: what inhere \textit{per se} inhere of necessity. Since animal inheres \textit{per se} in man, thus `man is an animal' is true of necessity. Against this rule it is argued thus: being mortal inheres \textit{per se} in man, and so `man is mortal' is necessary; the consequent is false, therefore etc. The falsity of the conseqeunt is shown, since a man, for instance Socrates, is able to not be mortal; therefore it is not necessary. Likewise, `a surface is white' is \textit{per se}, and yet `a surface is necessarily white' is false, since a surface is able to be black. One should say that thus rule has to be understood thus: `what inheres \textit{per se} in something inheres of necessity' is true, taking `per se' following how the Philosopher takes it in the book of the \textit{Posterior [Analytics]};\footnote{Aristotle, \textit{Anal. post} I, c. 4 (73a 35ss).} for there he says that those that inhere \textit{per se} inhere of necessity. On account of this, one should know that a proposition is called `per se' in three ways: in one way when the predicate falls in the definition of the subject, as `man is an animal'; in a second way when the subject falls in the definition of the predicate, and in this way `man is risible' is \textit{per se}; in a third way when the subject is the efficient cause of the predicate, as `a killer kills',\footnote{Following mss. BFP here}, from killing. To the reasons: To the first I say that a surface is not \textit{per se} white, taking \textit{per se} as it is taken in the book of the \textit{Posterior [Analytics]}; yet this can indeed be \textit{per se}, because immediate. To the other I say that `man is mortal' is not \textit{per se}, since `mortal' is something undergone resulting in a defect, and such is not predicated \textit{per se} of the species, nor of the individual.
\item[161.] Another rule is this: names and verbs transposed signify the same. Hence, `man is white' and `white is man' signify the same. Against this rule it is argued: `man every is' and `every man is' signify the same, yet one is well-formed and the other malformed. Likewise, this: `of all opposites, the same is the discipline, therefore the same is the discipline of all opposites.' One should say that this rule has to be understood so: where each of the transposed signifies a thing. By this [the response] to the first argument, that `every' does not signify a separate thing as `man' does. Likewise, the rule has to be understood where the transposition occurs from the same part of the extremes. By this [the response] to the second argument, since in the second argument the transposition is of the whole proposition.
%from the same part of the extremes
\item[162.] Another rule is this, that only one is opposed to one - as `no man is an animal' is opposed to the proposition `every man is an animal' - in the same kind of opposition. Against: the proposition `every man runs' is one, and yet these two propositions, `no man runs' and `some man does not run' are opposed to it. One should say that the rule has to be understood in the same kind of opposition, as was already stated. And by this the argument is resolved, since though the two `no man runs' and `some man does not run' are opposed to the proposition `every man runs', this yet occurs in different kinds of opposition, since the one is opposed as contrary and the other as contradictory. 
\item[163.] Another rule is this: when something is predicated of another, whatever is predicated of the predicate is predicated of the subject. For instance, since `animal' is predicated of `man', and `substance' of `animal', so `substance' is predicated of `man'. Against this it is argued so: `proposition' or its contradictory is predicated of this, that is `you are an ass', since this is true, ``you are an ass' is a proposition', or its contradictory. But `true' is predicated of this proposition or its contradictory, and yet `true' is not predicated of this proposition `you are an ass', since `you are an ass'\footnote{The edition, following L, adds `is true' here. My omission follows BFP.} is false. One should say that this rule has to be understood of those which are are ordered above and below in a categorical line, in the way `man' and `animal' are; but it is not so in the proposed.
\item[164.] Another rule is, of different kinds, one of which is not placed under the other nor both under a third, the species and differences are diverse. For instance, `animal' and `science' are diverse kinds not placed under each other, and so of these different kinds, are different species and differences. For the differences of science are like this: `natural' and `moral'; the differences of animal, like this `rational' and `irrational', etc. Likewise the species of these are different, since the species of animal are as ox and lion etc, and the species of science are as grammar and dialectic, etc. Against this: `rational' is a difference of science and animal, and yet science and animal are of different kinds. One should say that `rational' is one and another difference inasmuch as it is a difference of science and animal, for a science to be rational is nothing other than that it be under the act of reason, but `being rational' according to its being a difference of animal is `apt by nature to use reason'. And so `rational' is said equivocally there. Wherefore, etc.
\item[165.] Now we must speak of distributive signs. Note that some are signs distributive of substantial terms. Of this sort are `every' (\textit{omnis}), `whole' (totus), `each [of two]' (\textit{uterque}), and the like; likewise `whatever' (\textit{quicquid}), `any' (\textit{quilibet}), `whoever' (\textit{quicumque}), `on both sides' (\textit{utrumque}), `however many' (\textit{quotienscumque}). And of these signs of distribution, /f. 22rb/ some are distributive with respect to one verb, whose function (\textit{respectu}) is fixed to the\footnote{Omitting \textit{unum}, with mss. BFP.} verb. `Every', `whole', `every single' (\textit{unusquisque}), `each', and the like are of this kind. Hence I say `every man is', `each men is', `every single man is', and `the whole man is'. 

Others of these distribute so, with respect to diverse verbs, and of this sort are nearly all distributive signs which take to themselves the suffix \textit{-cumque}, as `whosesoever' (\textit{cuiuscumque}), `wherever' (\textit{ubicumque}), `however many' (\textit{quotienscumque}), `whenever' (\textit{quandocumque})'; for `whenever man runs' is not well-formed unless some another verb is added, so: `whenever a man runs, a man moves' or some such thing; and so for the others. 

Moreover, among the number of these which distribute with respect to one verb, some distribute for several subjects, as `every', `each', and the like; some for several integral [parts] only, as `whole' and the like. For if I say `the whole Socrates is less than Socrates', so that `whole' is taken significatively there, it will be equivalent to `any part of Socrates is less than Socrates.' 

Moreover, among the number of those which distribute for subjective parts, some are distributed only for two, of such sort being `each' and `both' (\textit{ambo}), etc.; others negatively, as `neither' and the like. Hence `this man runs, and this'--pointing to two--`therefore each of them runs' follows, which would not be true unless `each' distributed for two. And likewise with the word `both'. 

Some distribute not only for two but for all contained under a term. Of this sort are `every, `whichever' (\textit{quaecumque}), which distribute a term according to the possibility\footnote{Reading \textit{possibilitatem}, with BFP, rather than \textit{pluralitatem}, with the edition.} realized in it. Hence they can be added to a term having unactualized supposita (\textit{supposita infinita}), of such sort as `man' and `lion' etc. And according to the philosopher, they can be added to a term having only one suppositum in act; of this sort are `sun' and `moon'; likewise to terms having only one suppositum in act and potency, as `phoenix'. For we say `every lion', `every ox', `every phoenix', `every sun', and `every moon'. For even if `sun' has one suppositum, it can still have several. That is, it is such that there is no conflict in its having several, as much as is on its part. 

Moreover, among the number of signs, some distribute a substantial term for subject parts, and this in two ways. Some distribute a common term for supposita, inasmuch as any is distinguished from another;. The first sign is `every'; A sign of the second type is `every single' and the like. Thus, Aristotle says in \textit{Posterior Analytics} I\footnote{\textit{De caelo} I, c. 1 (268a 15-19).} that one who says `a triangle has three angles', etc. and [says] `every single triangle' speaks according to number, but one who says `every' speaks according to species.

Moreover, of the number of signs which distribute a term, some distribute in the nominative, others in declension. Hence, all signs distributing in the nominative distribute a term in the nominative , signs [distributing] in declension distribute [one] in declension. So saying `of any man, an ass runs', ``of any' which is the sign distibutive in a declined [term] namely of man in the genitive case'
%revisit this paragraph
\item[166.] All those names distributive in some way are called \textit{syncategorematic} words; besides these, there are some syncategorematic words which effect or imply consequence, of this sort are `if' (\textit{si}), `or if' (\textit{sive}), `inasmuch as' (\textit{inquantum}), and `unless' (\textit{nisi}). Among these, `if' implies consequence absolutely. Sometimes it effects an as-of-now consequence, sometimes a simple one. Hence, `if' and `while'  effect the same thing.  `Or if' (\textit{sive}) implies an absolute or as-of-now consequence. And \textit{sive} is put together from \textit{si} and from \textit{-ve}; the meaning of the utterance \textit{-ve} implies disjunction, and the meaning of the utterance \textit{si} implies consequence. `Inasmuch as' states not only a consequence, but the cause of a consequent or consequence. `Unless' (\textit{nisi}) implies a consequence with negation. Hence it is put together from \textit{non} and from \textit{si}. The meaning of the utterance \textit{non} implies negation, the reason of the utterance \textit{si} implies consequence.
\item[167.] Another rule is that to a necessary proposition corresponds a necessary act, and to a proposition of contingency corresponds an act true as-of-now. Hence, to to the proposition contingent toward either `a man runs' corresponds an act true as-of-now, namely `man runs', which is true for now, supposing that some man runs. But to a necessary proposition corresponds a necessary act. For instance, to the proposition of necessity `every man is an animal' corresponds a true necessary act, namely `every man is an animal', which is necessary.
\item[168.] Not that some indefinite proposition is equivalent to a universal, such as `man is an animal' is equivalent to `every man is an animal', since the predicate in each inheres in any contained under the subject. Against this: this chain of reasoning is good: `every man is an animal, Socrates is a man, therefore Socrates is an animal'; therefore this as well, namely `man is an animal, Socrates is a man, therefore Socrates is an animal,' since according to you it converts. The consequent is false, therefore the antecedent, too. One should say that for a chain of reasoning to be good can occur in two ways: either materially or formally. Hence the syllogism `man is an animal' etc. is indeed valid on account of its matter, namely by reason of such propositions, but not formally, since it does not hold in every matter with the premises disposed in this way. Wherefore the objection is not valid.
%check for consistency: 1) per se vs. through itself 2) secundum quid et simpliciter, 3)argues vs. is argued 4) italics vs. no italics for rules. 5) well/badly-formed vs. proper/improper. 
%Do a search for BP. Translate appendices. Provide cross-references to DPAL, Aristotle, etc.
%Currently at I, 22
\end{itemize}
\end{document}
