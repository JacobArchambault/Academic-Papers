\documentclass[]{article}
\usepackage[backend=biber]{biblatex}
\usepackage{bussproofs}
\usepackage{amssymb}
\usepackage{parallel}

\bibliography{jacob}
%opening
\title{Consequence and formality in the logic of Walter Burley}
\author{Jacob Archambault}

\begin{document}

\maketitle

\begin{abstract}

\end{abstract}

\section{Introduction}
In his `The medieval theory of consequence', Stephen Read lists Walter Burley (or Burleigh) alongside William of Ockham and John Buridan as among the most significant logicians of the medieval period \cite[p. 900]{Read2012}.\footnote{Translations throughout, unless stated otherwise, are my own.} However, Burley has received markedly less attention than either Ockham or Buridan. 

Part of the rationale for this lack of attention, already noted in \cite[p. VI]{Boehner1955}, is historiographical. While Ockham and Buridan are nominalists, Burley is classified as a realist. To the degree that logic has been represented as one of the few philosophical subdisciplines where progress undisputedly occurs; and to the degree that the metaphysics associated with this progress is thought to be a nominalist one (or, alternatively, to the degree that logic is thought to be wholly without metaphysical presuppositions), it is natural that historians of logic first set their sights on those figures whose assumptions and methodology appear closest to ours. Such has been the inclination of research in medieval logic for the last half century.

Some of the central developments in the medieval account of consequence were, to be sure, written by nominalist pens: the earliest detailed discussions of modal consequences, for instance, are found in Ockham's \textit{Summa Logicae} and Buridan's \textit{Tractatus de Consequentiis}. But one might question the degree to which the dichotomy between nominalism and realism serves as a useful one for chronicling the history of \textit{logic}. There is, after all, no 14th century nominalist `school' in any sense of that word other than that of a disparate group of thinkers sharing some important ideas, and thinkers from these schools influence each other both directly and through their polemics against each other. Given this, it may make some sense to remove this hermeneutic lens from our eye and look at the matter afresh.\footnote{See the similarities in Burley and Ockham's account mentioned in \cite[p. 46-47]{Normore1999}, \cite[p. 100]{Spade1999a}, \cite[pp. 379-380]{DutilhNovaes2008b}, as well as \cite{Klima1991}'s contrast of later realist \textit{and} nominalist semantics with the earlier \textit{via antiqua} semantic principles.} 

A closer examination of Burley's logic, including its relation to those of Buridan and Ockham, serves this aim. The plan of the essay is as follows. I begin with Burley's divisions and organization of consequences. After this, I locate Burley's contribution to the theory of consequences within the context of the 14th century work on the subject: first detailing its relation to earlier treatises; then to that of Ockham and Buridan. From there, I review the ways formal and material consequence occur in Burley's work, and relate this back to Burley's distinction between natural and accidental consequences. The final section concludes.
\section{Burley's division and enumeration of consequences}
Burley's views on consequences are found mainly in three treatises. The earliest of these is a short tract on consequences written prior to 1302, preserved in six manuscripts, and edited in \cite{Green-Pedersen1980b}. The extant manuscripts of the \textit{de consequentiis} divide into two groups \cite[pp. 104-105]{Green-Pedersen1980b}. The latter group, and among these especially one manuscript housed in the British Library, introduces numerous additions and corrections to the more basic text shared by the first group. The deliberate character of these changes likely point to a revision originating with Burley himself.

Next is the briefer version of \textit{De Puritate Artis Logicae}, which \cite{Boehner1955} suggests reached its final, still unfinished state in the early 1320s prior to Ockham's \textit{Summa}. The latest work is Burley's \textit{De Puritate Artis Logicae, Tractatus Longior}, which provides something of a bridge between Ockham and Buridan: the treatise is thought to have been revised in light of Ockham's treatise, particularly its attack on Burley's account of simple supposition.\footnote{\cite{Boehner1955}. Boehner further conjectured that Burley's treatise draws its title from its opposition to the `impurities' of Ockham's logic. But \cite{SpadeMenn} shows the title is an allusion to the prologue of Avicenna's \textit{Al-Shif\={a}}, where `purity' means the core, or pith, of a thing.} Burley's later version of the \textit{De Puritate} attained its current form at the University of Paris between 1325 and 1328, with Buridan himself rector of the University in 1328.
\subsection{Consequences in Burley's treatment of hypothetical propositions}
In the longer version of Burley's \textit{de puritate}, Burley's discussion of consequences is found in the part of the treatise devoted to hypothetical propositions, specifically that devoted to \textit{conditional} hypothetical propositions. 

Conditionals form one kind of explicit hypothetical proposition, along with conjunctions, disjunctions, and temporal propositions; exceptive, exclusive, reduplicative, and others are counted as implicit hypotheticals \cite[pp. 106-107]{BurleyDPAL}. By `hypothetical proposition', Burley means what today one might call a compound, or non-atomic proposition.\footnote{Cf. \cite[p. 66]{Buridan2015}.} By `implicit hypothetical', he means a proposition analyzable into an explicit hypothetical, even if its surface grammar is categorical.\footnote{For instance, the sentence `Nothing besides Socrates runs' is called an exceptive proposition, from the function of the exceptive word `besides'. Burley analyzes it into the conjunction `Socrates runs and nothing other than Socrates runs', the conjuncts of which are called the \textit{exponents} (\textit{exponentes}) of the original proposition \cite[p. 121, par. 44]{Green-Pedersen1980b}, \cite[pp. 164-165]{BurleyDPAL}.} Besides conditionals and other kinds of compound propositions, Burley also refers to conditional \textit{syllogisms}. These form a proper subclass of conditional hypothetical propositions, whose parts are three conditional hypothetical propositions - or, in the cases of \textit{modus ponens} and \textit{modus tollens}, one hypothetical and two categoricals - two of which are premises concluding to the third.\footnote{Cf. \cite[5.1.3, pp. 308-309]{BuridanKlimaSD} \cite{Klima2004b}.}

According to Burley, a consequence is an act performed in a conditional hypothetical proposition by words like `if', `thus', and `therefore' - the same act signified by the term `follows' in a categorical proposition. Consequences are not themselves conditionals. Neither are they distinguished from conditionals such that a conditional is indicated by the word `if', while a consequence is indicated by `follows' or `therefore'. Rather, `conditional' connotes the grammatical or syntactic structure wherein a consequence is shown; `consequence', the semantic relation obtaining between a conditional's parts. Because of this close link between consequences and conditionals, Burley sometimes uses the terms interchangeably, and one finds the same vocabulary applied to both terms in Burley's work.\footnote{\cite{HodgesBurley} claims Burley distinguishes consequences from conditionals by using `if' for the latter, and `therefore', `follows', etc. to indicate the former, citing \cite[p. 78.10-20]{BurleyDPAL} in support of this interpretation. \cite[p. 120]{King2001} locates the distinction in that between statements and arguments, arguing that i) the terminology used for conditionals and consequences is not interchangeable, and ii) specifically, conditionals `are true or false whereas consequences are not'.
	
	Hodges' interpretation of the passage he cites is incorrect. Burley's point in the passage is that from any conditional follows a \textit{categorical} sentence in which nominalizations of the antecedent and consequent of the conditional are taken as terms, and vice versa - i.e. for any conditional $\phi \rightarrow \psi$, terms $a, b$ respectively naming the conditional's antecedent and consequent, and a binary relation $Follows$: $Follows(a, b)$ if and only if $\phi \rightarrow psi$. At \cite[p. 141.26-30]{BurleyDPAL}, Burley states that from a consequence in which the term `therefore' (\textit{ergo}) \textit{performs} its semantic function, follows a categorical sentence where this function is \textit{signified} by the word `follows'. Burley makes the same point in the \textit{Tractatus Brevior} with respect to the function performed by `if' (\textit{si}) at \cite[p. 219.1-9]{BurleyDPAL}. Cf. \cite[p. 143, par. 119]{Green-Pedersen1980b}.
	
	The grounds for King's claims are false. Burley calls the conditional `If a man is an ass, you are sitting' a good consequence at \cite[II. i. 1, p. 61]{BurleyDPAL}. At \cite[p. 114, par. 8]{Green-Pedersen1980b} and \cite[p. 89.1-31]{BurleyDPAL}, Burley refers to the conditionals linking a larger consequence as `intermediate consequences (\textit{consequentiae intermediae}), with the latter passage also calling the conditionals making up a hypothetical syllogism `consequences'. At \cite[p. 128, par. 68]{Green-Pedersen1980b}, Burley divides conditionals into as-of-now and simple, and immediately thereafter refers to simple and as-of-now consequences without introducing these separately. The same division is found in \cite[pp. 60.28-61.5]{BurleyDPAL} stated in terms of consequences. At \cite[p. 78.27-30]{BurleyDPAL}, Burley calls a certain consequence linking a categorical antecedent and hypothetical consequent with `therefore' (\textit{igitur}) a `composite conditional'. Burley calls consequences true and false at \cite[p. 113, par. 2-3]{Green-Pedersen1980b}; cf. \cite[p. 15, par. 19]{Green-Pedersen1980a}.} 

Burley's consequences are located in a natural, not formal, language \cite[pp. 4-5]{HodgesBurley}. Because the linguistic limitations of Latin are taken seriously, certain kinds of consequences cannot be formed for Burley: for instance, syllogisms cannot be contraposited \cite[pp. 65.3-17; 207.31-208.9]{BurleyDPAL}; one cannot move from sentential to term negation to negate a complex term, as in `non-(white tree)';\footnote{\cite[p. 131, par. 80]{Green-Pedersen1980b}, \cite[pp. 214.14-21, 215.6-21]{BurleyDPAL}.} and it is fair game to object to a consequence on the grounds that its premises are well-formed while its conclusion is not.\footnote{\cite[p. 150, par. 135]{Green-Pedersen1980b}; \cite[pp. 211.31-33, 212.10-20]{BurleyDPAL}.}

\subsection{Burley's enumeration of consequences}
In both the shorter and longer versions of the \textit{De Puritate}, Burley calls certain rules governing conditional hypothetical propositions `principal', thereby distinguishing them from other rules which are said to depend on them. Burley gives ten principal rules in the \textit{Tractatus Brevior}, reduced to five in the \textit{Tractatus Longior}. Nearly all of the rules mentioned are also found in the earlier \textit{de consequentiis}, though this last does not mention principal rules. The first, second, and fifth rules of the \textit{Tractatus Longior} are the first three rules of the \textit{Tractatus Brevior}. These are, respectively: 1) that `in every good simple [as-of-now] consequence, the antecedent cannot [now] be true without the consequent' \cite[p. 61.30-37, 199.26-27]{BurleyDPAL}; 2a) `whatever follows from the consequent follows from the antecedent'; 2b) `whatever entails (\textit{antecedit}) the antecedent entails the consequent' (i.e. cut) \cite[p. 62, 9-13]{BurleyDPAL}; and 5) `whenever a consequent follows from an antecedent, the contradictory opposite of the antecedent follows from the contradictory opposite of the consequent' (i.e. contraposition) \cite[p. 64.20-22]{BurleyDPAL}. Several other rules in the \textit{Tractatus Brevior} are not straightforward rules about \textit{consequence}, but rather govern supposition, negation, or denotation. The \textit{Tractatus Longior} moves some of these to other sections of its text.\footnote{Rule four of the \textit{Tractatus Brevior} \cite[p. 208.12-13]{BurleyDPAL} is called a `general rule' in the \textit{Tractatus Longior} \cite[p. 73.29]{BurleyDPAL}. Rule six of the earlier treatise \cite[p. 210.11-12]{BurleyDPAL} is mentioned in the later treatise's section on supposition \cite[27.11]{BurleyDPAL}. Rule ten of the earlier treatise \cite[p. 219.1-2]{BurleyDPAL} appears in the later treatise in connection with the solution to a sophism \cite[p. 141.26-30]{BurleyDPAL}.} The rules governing consequences in the \textit{Tractatus Longior}, by contrast, all explicitly concern relations of following, with the two new principal rules in the \textit{Tractatus Longior}, rules three and four, relating following to compatibility and incompatibility. These rules, already present in \cite[p. 133, par. 88]{Green-Pedersen1980b}, are 3) whatever conflicts with the consequent conflicts with the antecedent, and 4) whatever stands with the antecedent stands with the consequent \cite[pp. 63.7-10, 64.20-22]{BurleyDPAL}. Ockham lists them as general rules at \cite[III-3. 38, pp. 727-731]{OckhamSL}.

Burley gives no justification for principal rules one, two, or five. He does, however, provide arguments for principal rules three and four \cite[p. 63.1-14]{BurleyDPAL}. That for the third relies on an instance of rule two. Burley restricts the fifth rule to non-syllogistic consequences, albeit on syntactic grounds. The fifth rule holds that from the opposite of the consequent the opposite of the antecedent follows. But for Burley, the premises of a syllogism do not form a proposition, either simple or complex. Hence neither does their negation.\footnote{\cite[pp. 65.3-17; 207.31-208.9]{BurleyDPAL}. Hence, Burley explicitly rejects the interpretation, found in \L{}ukasiewicz, of the premises of a syllogistic inference as a conjunction\cite{Lukasiewicz1957}. For criticism of this approach, see \cite{Corcoran1974}.}

\subsubsection{Principal and derivative rules}
Besides these principal rules, Burley gives a small number of rules said to follow from these: from the first, that 1.1) in a simple consequence, the impossible does not follow from the necessary and 1.2) that the contingent does not follow from the necessary \cite[p. 62.1-8]{BurleyDPAL}; from the second, that 2.1) whatever follows from the antecedent and consequent follows from the antecedent by itself (\textit{per se}), and 2.1) that whatever follows from the consequent with something added follows from the antecedent with the same added \cite[p. 62.22-38]{BurleyDPAL}; from the third and fourth, that 4.1) If the consequences of certain propositions conflict so do those propositions, 4.2) Consequences with compossible antecedents have compossible consequences, and 4.3) that in a good consequence, the opposite of the consequent conflicts with the antecedent \cite[pp. 63.15-64.7]{BurleyDPAL}; from the fifth, that 5.1) whatever follows from the opposite of the antecedent follows from the opposite of the consequent, and 5.2) whatever entails the opposite of the consequent entails the opposite of the antecedent. By examining these in more detail, we can obtain a better understanding of exactly what Burley means by `principal' and `follows'. 

Burley gives as reason for 1.1 and 1.2 that `the contingent can be true without the impossible, and the necessary can be true without the contingent' \cite[p. 62.5-6]{BurleyDPAL}. The argument here is rather truncated, but it may be fleshed out by simultaneously substituting `contingent' [`necessary'] and `impossible' [`contingent'] for `antecedent' and `consequent' in rule one, and recognizing the result of doing so as the contradictory of the independently accepted claim that the contingent [necessary] can be true without the impossible [contingent]. Since contradictories cannot simultaneously be true, and since the reasons for 1.1 and 1.2 are independently accepted, conditionals resulting from instantiating a contingent [necessity] and impossibility [contingent] under `antecedent' and `consequent' in rule 1 must be rejected. Thus, these rules follow from rule one in the sense that their contradictories are not acceptable instances of the schema rule one presents.

Burley licenses rule 2.1 by the recognition that in any consequence, the antecedent and consequent follow from the antecedent \cite[p. 62.27-28]{BurleyDPAL}. If we combine this with an instance of rule 2 - namely, that where the middle term consists of the conjunction of the antecedent and consequent - then rule 2.1 is obtained by an application of cut.

%The remaining derivative rules are treated formally in an appendix. 
From the above, we can already see the following. First, Burley's rules are not, as rules, distinct from another class of things called `propositions'. Rather, rules themselves are propositions no less than others. They are rules because of what they do: they regulate, and thereby place conditions on other classes of propositions that fall under them. 

Burley's rules are \textit{logical} rules: the concepts employed therein - antecedent, consequent, proposition, etc. - are purely logical ones. Thus, these rules do not give substantial answers to questions about the nature of reality, though they do place restrictions on the shape these answers can take.

Derivative rules are not restricted instances of the principal rules, nor are they derivable solely from those rules. Burley's expositions of derivative rules usually take the following structure: statement of the rule, reason supporting it, optional example, summary statement of the argument from the principal rule to the derivative one. Here, for instance, are Burley's remarks on rule 2.2: 
\begin{quote}
	The reason for the second rule is this: the antecedent with something added implies the consequent with the same added; for `Socrates runs and you are sitting, therefore a man runs and you are sitting' follows. Since, then, whatever follows from the consequent follows from the antecedent, it must be that whatever follows from the consequent with something added, it follows from the antecedent with the same added \cite[p. 62.33-38]{BurleyDPAL}.
\end{quote}

Here and elsewhere, the reason gives information that must be used with the principal rule to obtain the derived one. In one place, Burley tells us `from one nothing follows, neither enthymematically nor syllogistically' \cite[p. 147, par. 130]{Green-Pedersen1980b}. Burley's meaning here is that every good consequence relies on premises and/or connections other than those stated explicitly in the premise itself. This holds also for arguments deriving rules from other rules. 

Burley's principal rules are self-applicable. For instance, the argument for rule 2.1 makes use of rule two both as a premise and as an application of cut elimination necessary to obtain the conclusion. 

Many of the consequences Burley discusses, principally those depending on Burley's second rule, amount to what Hodges has felicitously described as a `calculus of monotonicity'. Briefly, let $T$, $T'$ be noun phrases, $S(T)$ a sentence in which $(T)$ occurs, and $S(T'/T)$ the result of replacing all instances of $T$ in $S$ with instances of $T'$. Now an occurrence of a term $T$ in a sentence $S$ is \textit{upwards monotonic} iff the consequence
\begin{quote}
	Every $T$ is a $T'$. $S(T)$. Therefore, $S(T'/T)$
\end{quote}
is good; and \textit{downwards monotonic} iff the consequence 
\begin{quote}
	Every $T$ is a $T'$. $S(T'/T)$. Therefore, $S(T)$.
\end{quote}
is good \cite[pp. 27-28]{HodgesBurley}. Burley calls a consequence employing upwards monotonicity one \textit{from an inferior to a superior}; and one employing downwards monotonicity, \textit{from a superior to an inferior}. In the above, `Man' and `runs' are both upwards monotone in the sentence `a man runs', and downwards monotone in the compound proposition `A man runs, therefore an animal moves'. In practice, the sentences of the form `Every $T$ is $T'$ factoring into the above conditions cannot be just any true universal affirmative sentence, but must be necessary truths expressing conceptual containment of the predicate in the subject.\footnote{The exact details are somewhat more complicated, because of issues arising concerning the existential assumptions built into different kinds of propositions. For instance, for categorical propositions with upwardly monotonic terms, upwards monotonicity holds good for conceptually contained terms without further qualification: `a man is an animal, therefore a man is a substance' holds good regardless of whether humans exist. But consequences proceeding from categorical propositions with downwardly monotonic terms may fail: `Every animal is a substance, therefore every man is a substance' is good, but `every animal is running, therefore every man is running' is not. The former, since it expresses a conceptual truth, does not presuppose the existence of the things it connects. The latter, however, does, and so may be falsified in the situation where every animal is running, but humans do not exist. The existential import of propositions for Burley is thus determined not strictly by its quantity and/or quantity, but also by the kind of predication exhibited therein. Cf. \cite[pp. 61.4-5; 85.3-5, 25-26; 216.15-18]{BurleyDPAL}; \cite[p. 32]{HodgesBurley}.} 

The development found in Burley's treatises thus suggests two points for the development of his thinking about consequence, mirrored in that of the field as a whole: first, that a large body of rules for consequences had been in place before serious attempts to track relations of dependence between them; second, that the study of consequences only gradually came to be distinguished from related studies, e.g. those into supposition theory and the treatment of syncategorematic functions.
\subsection{Burley's division of consequences}
Burley divides consequences into simple and as-of-now, the former being divided into natural and accidental consequences. 

According to Burley, a simple consequence is one `which holds for every time, such that the antecedent can never be true unless the consequent is true'. An as-of-now consequence, by contrast, `holds for a given time and not always'\footnote{\cite[pp. 60.29-61.5]{BurleyDPAL}. Cf. \cite[p. 199.19-25]{BurleyDPAL}.} The distinction is already present in the earlier \textit{de consequentiis}, where it is also referred to as one between simple and as-of-now \textit{conditionals} \cite[p. 128, par. 69]{Green-Pedersen1980b}. Burley's definitions for simple and as-of-now consequence do not substantially differ from those in Ockham or Buridan [SL III-3. 1], \cite[I. 4]{BuridanTC}.

Burley defines a natural consequence as one where `the antecedent includes the consequent'. Such a consequence `holds through an intrinsic topic', while an accidental consequence is one `which holds by an extrinsic topic' \cite[p. 61.6-10]{BurleyDPAL}. Surprisingly, this division is not found at all in the \textit{Tractatus Brevior}. It is, however, in Burley's earlier \textit{de consequentiis}, which further distinguishes two kinds of accidental consequences: one, which `holds on account of the terms or matter', Burley's example being the proposition `That God exists is true, therefore that God exists is necessary (since truth in God and necessity are the same)'; the other, for which Burley gives as examples propositions with an impossible antecedent or a necessary consequent \cite[pp. 128-129, par. 70]{Green-Pedersen1980b}. This division is not presented in the later \textit{De Puritate}, though as we shall see, it shows up tacitly in the solution to a problem there.

Natural and accidental consequence are already distinguished, albeit in passing, by Boethius \cite[1.3.6]{BHS}. Later, their division is found in Duns Scotus, where as in Burley, we also find it grounded in a division between intrinsic and extrinsic topics. Prior to Scotus, the notion of a natural consequence is found in William of Sherwood's \textit{Syncategoremata}, where it is juxtaposed with non-natural consequence.\footnote{\cite[I, d. 11, q. 2, p. 136-137]{ScotusLectura}, \cite[p. 80]{Sherwood1941}; see \cite{Martin2012}.} At [SL III-3. 1], The same topical distinction Burley uses to distinguish natural from accidental consequence is employed by William of Ockham to distinguish two kinds of formal consequences.

In the \textit{de consequentiis}' discussion of the \textit{Tractatus Longior}'s rule 2.1, Burley makes clear that following can occur in different ways, each of varying strength: the rule is said to follow `at least in the same way \textit{ex impossibili quodlibet} follows' - i.e. accidentally, via an extrinsic topic, and not on account of the terms \cite[p. 132, par. 85]{Green-Pedersen1980b}. Burley's meaning here does not seem to be that the rule holds at least with the same strength as that consequence, but rather that instances of the rule may admit of varying levels of strength, the weakest of these being that associated with explosive consequences.

Burley's assumption of the weakness of explosion may come as a surprise, given classical and strict consequence, following Buridan, build explosion into the definition of consequence: a consequence is good just when the antecedent cannot be true with the consequent being false. Since an impossible antecedent cannot be true, neither can it be true with whatever is consequent to it being false. But for Burley, enthymematic consequences, such as `if Socrates is a man, Socrates is an animal' are stronger than these.\footnote{Cf. \cite[pp. 11-14]{Klima2016}.} The strength of a given consequence is thus not merely a syntactic matter, but dependent on those genuine relations between things mentioned in it. 
%find correct pages for Klima 
\section{Burley's place in the development of consequence}
\subsection{Burley and the Boethian tradition}
Burley takes the language of topics (\textit{loci}), used to ground his distinction between natural and accidental consequences, from Aristotle's \textit{Topics}, as mediated by Boethius' \textit{On Differential Topics}. Following the Aristotelian commentator Themistius, Boethius divides topics into intrinsic, extrinsic and middle topics. By the later 13th century, little emphasis is placed on middle topics: Radulphus Brito's commentary on the second book of Boethius' work (c. 1300), for instance, only devotes two of its twenty three questions to middle topics, with one of these concerning whether the category is a sensible one \cite{BritoDDT}.

According to the definition Boethius takes over from Cicero, a topic is the seat of an argument (\textit{sedes argumenti}), and an argument an account making a doubtful matter sure (\textit{ratio rei dubiae faciens fidem}) \cite[I]{BDT}. The term `topic' can refer to either what Boethius calls a \textit{maximal proposition}; or what such a proposition concerns, called the \textit{difference} of such a proposition. A maximal proposition is one not in need of further justification, e.g. `equals added to equals are equal'. The difference of such a proposition is what it is in the maximal proposition that justifies the conclusion, i.e. that from which the deduction of the desired conclusion proceeds. This will become clearer with some examples.

Boethius divides intrinsic topics into those in substance, and those consequent upon substance. The former include the topics \textit{from a definition}, \textit{from a description}, and \textit{from the interpretation of a name}. The latter include arguments from a genus to its species and conversely, from a part to its whole and conversely, and arguments from formal, efficient, material, and final causes, among others. In an intrinsic topic, the argument proceeds by eliciting some property, description, or relation belonging to what is signified in the minor term as a way to confirm or remove what is signified in a major term of it. The following is his example for the topic \textit{from a description}. Suppose you wish to know whether whiteness is a substance. To prove that it is not, one argues as follows: A substance is what can be the subject of accidents; but whiteness cannot be the subject of accidents; so whiteness isn't a substance. The topic appealed to is that \textit{from a description}, since the description serves as the medium - in this case, the middle term - whereby the question is determined. The maximal proposition, or rule appealed to, is \textit{that the description of which does not belong to a species is not the genus of that species}.\footnote{A more famous use of the same topic is found in Anselm's `ontological' argument. There, one proceeds from God to his existence via the description `that than which nothing greater can be thought'. The maximal proposition licensing the inference being `what is predicated of the description is predicated of its bearer'.}

In an extrinsic topic, by contrast, one exploits a relation that the subject term of the desired conclusion bears to some other concept, constructs an argument about the latter, and thereafter leverages the original relation to infer something about the original subject one was inquiring about. For instance, to determine whether being two-footed is proper to human beings, Boethius offers the following argument: `being two-footed belongs to human beings in the same way as being four-footed belongs to a horse; but being four-footed is not proper to a horse; neither, then, is two-footedness proper to humans.' The question here concerns a proper accident, i.e. one belonging to a thing by virtue of its specific nature; the maximal proposition is `if what inheres in a similar way does not do so properly, neither does what one is inquiring about inhere properly.' The topic is that from a likeness, or similitude. Other types of extrinsic topics include arguments from authority, from analogy, from various kinds of opposites, and \textit{a fortiori}/\textit{a minore} arguments. The most important use Burley makes of extrinsic topics is in his justification of the rules \textit{from the impossible anything follows (ex impossibili quodlibet)} and \textit{the necessary follows from anything (necessarium ex quolibet)}. The former is justified by the extrinsic topic \textit{from the lesser (a minore)}, on the grounds that if the impossible holds, then given it is less likely that the impossible holds than another proposition, that other proposition can be inferred.\footnote{\cite[V. 70, pp. 128-129]{Green-Pedersen1980b}, \cite[II. i. 1, p. 61]{BurleyDPAL}. Cf. SL III-3. 38. The \textit{Tractatus Brevior} notes that this places some restrictions on the rule itself: for instance, the more impossible cannot be inferred from the less impossible \cite[pp. 248.19-249.3]{BurleyDPAL}. }

By the time of Burley, the topical apparatus grounding his distinction between natural and accidental consequences is both expanded and simplified in different ways. It is simplified by the straightforward identification of maximal propositions with rules: `A maximal proposition is nothing but a rule whereby a consequence holds' \cite[II. i. 2, p. 76]{BurleyDPAL}. It is expanded by the claim that such a rule need not be a dialectical one - the traditional purview of topical argument - but may be more broadly logical \cite[II. i. 2, p. 76]{BurleyDPAL}; and by the admission of an indefinite number of differences of maximal propositions, rather than the traditional twenty-five or so admitted by Themistius and Boethius. For Burley,
\begin{quote}
	Not every maximal proposition arises from a difference of a maximal [proposition] known to us. For many maximal propositions are necessary, and still do not have names imposed on the differences of these maximal [propositions] \cite[II. i. 2, pp. 76-77]{BurleyDPAL}
\end{quote}

The resulting picture contrasts strongly with modern logical practice. Where today it is common to construct a logical calculus out of a minimal number of rules, defining familiar connectives in terms of others in order to achieve maximal economy,\footnote{E.g. disjunction in terms of conjunction and negation, or all classical connectives in terms of the Sheffer stroke. For discussion, see \cite{Paseau2016}.} Burley's theory admits an indefinitely large number of topical rules, with only some concern for their hierarchical organization.

\subsection{Burley's work among the earliest treatises on consequences}
If we compare the content of Burley's \textit{de consequentiis} with the early anonymous treatises on consequences edited in \cite{Green-Pedersen1980a}, it becomes clear that Burley's is not the first work on the topic. A strong case can be made that Burley knew the text in London, BL, Royal 12 F XIX, ff. 111ra-112rb early on. The ms. containing the only extant copy of the anonymous work also contains a copy of Burley's work at ff. 116ra-122rb. 

More conspicuous is Burley's treatment of an example found in the anonymous London treatise. In discussing the rule that in a good consequence, the opposite of the consequent cannot stand with the antecedent, the London treatise makes exception for consequences whose antecedents include opposites, and uses the proposition `no time is' as an example of a statement including opposites. It then gives the following argument:
\begin{quote}
	If no time is, it is not night, and if it is not night, it is day; and if it is day, some time is. Therefore, if some time is. \cite[p. 7, par. 18]{Green-Pedersen1980a}
\end{quote}
Burley visits the same example in his discussion of the rule \textit{from the first to the last} (i.e. transitivity). He writes: 
\begin{quote}
	if one argues `if no time is, it is not day; and if it is not day, and some time is, it is night; and if it is night, some time is; therefore \textit{from the first to the last}: if no time is, some time is' - this consequence does not hold from the first to the last, since the consequent of the first conditional is `it is not day', and the antecedent of the second conditional is the whole `it is not day, and some time is' \cite[pp. 114-115, par 8]{Green-Pedersen1980b}
\end{quote}
Where the anonymous treatise uses the example to prompt an exception to a standard rule, Burley diffuses the example by arguing that it rests on an equivocation.

Though Burley's \textit{de consequentiis} has parallels with the treatise of Paris, BN lat. 16130, these parallels are mostly common rules, which are insufficient to establish any dependency of Burley's text on it. Comparing their style and content, the Parisian treatise is more terse, and has a marked preference for the active voice. Its proofs tend to be more detailed, and the connection to supposition theory is more pronounced. The Parisian treatise shows a clear grasp of the difference between downward monotonicity and descent to singulars \cite[p. 12, par. 2]{Green-Pedersen1980a}; whether the London author grasped the difference is less clear. In both works, the treatments of upward/doward monotonicity and ascent/descent from singulars are parallel. Burley's \textit{Tractatus Longior}, as we shall see, contains a more substantial engagement with the Paris text.

The most fruitful comparison with both treatises comes in their approach to suppositional descent, and the corresponding difficulties brought about regarding existential import. None of these treatises adopt the approach to existential import - straightforwardly found in John Buridan, and  attributed by Stephen Read to Aristotle himself\footnote{\cite[q. I. 38]{BuridanPostAn}, \cite{Klima2001}, \cite{Read2015b}.} - according to which affirmative propositions have existential import while negative ones lack it. Rather, whether a proposition presupposes the existence of its supposita is dependent on the mode of predication, and hence on the mode of inherence exhibited by the things named by the terms. \textit{Per se} predication is that wherein a constitutive property is predicated of what is constituted by it, as in `a man is an animal'. \textit{Per accidens} predication, the predication of an accident of a substance, as in `a man is white' or `a man is risible'; or of some quality, be it essential or accidental, itself of an accident, as in `to run is to move'.\footnote{This conflation of accidental predication with \textit{per se} predication of an accident is present in both anonymous texts. See \cite[pp. 10-11, par. 35-36; 25, par. 66]{Green-Pedersen1980a}.} In one medieval approach, found in both Simon of Faversham and Duns Scotus, \textit{per se} predication does not presuppose existential import, but \textit{per accidens} does.\footnote{See \cite[q. I. 56]{FavershamQE}; \cite[I. qq. 5-8. par. 49, 74]{ScotusPeriHerm}; \cite[q. 11, par. 19]{ScotusQE}. More recently, this approach has been suggested by Nuel Belnap and Thomas Mueller \cite{CIFOL1}.} From these approaches to predication, corresponding notions of ascent and descent were developed, so that a descent from `man' to `white man', for instance, was called \textit{from a superior to inferior per accidens}, but from `animal' to `man' was \textit{per se}. 

Both anonymous authors require all propositions where the same is predicated of itself to come out true, including predications involving accidents, such as `a white man is a white man' and those involving impossible objects, such as `a chimera is a chimera' \cite[p. 8, par. 23; 11, par. 36-37; 18, par. 31-32]{Green-Pedersen1980a}. Because these propositions are given a separate treatment, most ascents and descents to and from such propositions are blocked by existence complications: for instance, one cannot ascend from `a white man is a white man' to `a white man is a man', because the latter proposition is existence implying, while the former is not. 

Relative to these treatises, Burley's principal contribution seems to have been the rejection of special treatment for statements predicating the same of itself, and with it a more uniform application of ascent and descent rules in \textit{per accidens} predications.\footnote{See \cite[pp. 116-117, par. 19-20; 134, par. 95; p. 158, par. 160.]{Green-Pedersen1980b}.} On the resulting approach, \textit{per se} ascent and descent work as follows: A universal affirmative categorical proposition is downwardly monotonic in its first term and upwardly monotonic in its second;\footnote{\cite[p. 211.16-20]{BurleyDPAL}.} the opposite is the case for a particular negative categorical; a particular affirmative is upwardly monotonic in both terms; a universal negative, downwardly monotonic in both. Descent \textit{per accidens} (e.g. from `no man is an animal' to `no white man is an animal') holds where descent \textit{per se} does in negatives \cite[pp. 209.35-210.10]{BurleyDPAL}. Ascent from a \textit{per accidens} inferior to its \textit{per se} superior holds where \textit{per se} ascent does in affirmatives \cite[pp. 116-117, par. 20]{Green-Pedersen1980b}. Excepting Burley's treatment of propositions predicating the same of itself, the above analysis appears to have been lifted from the London anonymous treatise.\footnote{Cf. \cite[pp. 10-11, par. 35-37]{Green-Pedersen1980a}, \cite[pp. 116-117, par. 19-20]{Green-Pedersen1980b}.} 

Lastly, Burley recognizes that where a class or object descended to can be empty, \textit{per se} descent for affirmatives can fail. But Burley's practice on this point is inconsistent.\footnote{See esp. \cite[pp. 61.4, 85.16; 85.4, 211.27-28]{BurleyDPAL}. Cf. \cite{Mora-Marquez2015}.} Instead, Burley generally employs these descents freely, ignoring existential complications \cite[pp. 23.26, 26.26, 31.21, 67.19, 67.30, 85.16, 85.26, 211.27-28]{BurleyDPAL}.

%Burley's treatment of impossible antecedents: 70.26-72.3

%Cf. 85.3-5 to 211.27-28; also, 218.7-8 to DC par. 19

%17, 19-20, 23-25, 37, 42-43, 95, 102, 120, 155-156, VIII.188, X.69, 
\subsection{Relating Burley's work to that of Ockham and Buridan}
\subsubsection{The place of consequences within Burley's vision of logic}
If we compare the structure of Burley's later treatises with Ockham's \textit{Summa Logicae}, the following differences become conspicuous. 

Ockham's treatment of consequence does not begin with general rules, but rather with rules for suppositional ascent and descent for the various qualities and quantities of assertoric, then modal and other kinds of propositions. Ockham does not discuss general rules for consequences until the thirty-eighth chapter of his work,\footnote{This was originally the final chapter of the section on consequences. See \cite[pp. 41*-43*]{OckhamSL}.} and each of the rules mentioned therein is already explicit in Burley's much earlier \textit{de consequentiis}. Thus, as with both Burley's earlier \textit{de consequentiis}, the generality of consequence is not yet brought to the fore;\footnote{Burley does not discuss general rules until the last section of the treatise, beginning at par. 145.} and as with Burley's \textit{Tractatus Brevior}, the differentiation of consequence from supposition theory is less pronounced in Ockham.

Furthermore, the divisions and order of Ockham's treatise follow those understood to govern Aristotle's \textit{Organon}. The first two parts of Ockham's treatise, on terms and propositions, address simple concepts and judgment, the subject matters of Aristotle's \textit{Categories} and \textit{On Interpretation}. The third part, broadly concerned with reasoning, discusses the subjects of Aristotle's \textit{logica nova} texts: the form of reasoning (\textit{Prior Analytics}) and their implementation in demonstrative (\textit{Posterior Analytics}), dialectical (\textit{Topics}), and fallacious argument (\textit{On Sophistical Refutations}).\footnote{Cf. \cite[prol.]{AquinasPA}} Within this setup, Ockham's treatise on consequences occupies the place of Aristotle's \textit{Topics}. Thus, though the content of Ockham's thought is often radical, the structure wherein it is presented is a deliberately traditional one. In particular, Ockham does not yet conceive consequences as encompassing traditional syllogistic inference.

By contrast, Burley already includes syllogistic under consequence in his \textit{de consequentiis}.\footnote{\cite[pp. 131-132, par. 82-85]{Green-Pedersen1980b}. See also \cite[p. 219.19-32]{BurleyDPAL}.} Burley's works all begin with general consequences, then move to more specific ones later. And both the intended structure of Burley's earlier and the actual structure of his later \textit{De Puritate} betray a different understanding than Ockham's of the place of consequences in the Aristotelian curriculum.

According to the earlier treatise's prologue, Burley's plan was to treat in sequence: 1) common rules for the remainder of the work; 2) sophisms, 3) \textit{obligationes}, and 4) demonstration \cite[p. 199]{BurleyDPAL}. The preface of the earlier treatise, and the intended ordering of the treatises parts, betray a fundamental concern with the form of reasoning, which is then examined in the different contexts wherein it is utilized: demonstration corresponding to the traditional subject of the \textit{Posterior Analytics}; sophistry, that of the \textit{Sophistical Refutations}.\footnote{Cf. \cite[prol.]{AquinasPA}. The ordering of the \textit{logica nova} materials suggested by the prologue of Burley's earlier treatise, with the \textit{Prior} and \textit{Posterior Analytics} separated by the \textit{Elenchi} and \textit{Topics}, is less common than that suggested by Ockham's. That it was nevertheless used is clear from some mss. recorded in the \textit{Aristoteles Latinus} index, such as Metz, \textit{Bib. Mun.} 508; Arras, Bib. Mun. 362 (451); Chambery, Bib. Mun. 27. The same separation is found in the ordering of the treatises of Buridan's \textit{Summulae de Dialectica}.} As such, the assimilation found in Burley's earlier treatise is not to topical argument, but rather, to the \textit{Prior Analytics}: the ambition of Burley's treatise is to provide an account of the forms of reasoning - an account that, instead of reducing all reasoning to syllogistic,\footnote{This reduction remains present in Buridan's \textit{Summulae}. See \cite[6.1.5, pp. 398-400]{BuridanKlimaSD}.} takes up syllogistic only as a part in a work centered on consequence, and expanded to address syncategorematic functions found in hypotheticals other than conditionals. Burley thus appears to be the first logician to attempt a unified account of consequence including syllogistic as a proper part. It is this vision in Burley's treatise, rather than Ockham's, which is adopted later in the structure of Buridan's \textit{Treatise on Consequences}.

This vision remains in the later version of the treatise, albeit with an important revision. The intended inclusion of a treatise on supposition in the first part suggests Burley early regarded supposition theory as part of a broader theory of inference, perhaps as supplementing the kinds of inference-justifying rules one finds in topical treatises. This ordering agrees with that found in the \textit{Summulae} of Peter of Spain.\footnote{Cf. \cite{Klima2003a}.} In Burley's later treatise, by contrast, the part on supposition begins with the words `having laid down the signification of incomplex terms, in this tract I intend to examine certain properties of terms which only belong to them according to their being parts of a proposition' \cite[p. 1.3-6]{BurleyDPAL}. Now, the signification of incomplex terms was the standard topic associated with the \textit{Categories}.\footnote{See \cite[p. 65]{Burley2003}.} The mention of the proposition makes clear the association of the treatise on supposition with the \textit{On Interpretation}. Thus, it seems that by the time of the later treatise, Burley thought it more appropriate to treat supposition in connection with the content of the \textit{On Interpretation}, in the broader context of subjects discussed under Aristotle's old logic, rather than in closer connection with discusssions of inference traditionally treated under Aristotle's new logic. This relocation provides us with a reason why Burley may have abandoned the \textit{Tractatus Brevior} independent of an alleged urgent need to respond to Ockham's logic: he sought to reorganize his materials so as to reflect this new understanding of the place of supposition theory within logic, a reorganization which then required a rethinking, and disentangling, of supposition theory from the theory of consequence, which in turn required rewriting much of his old material to reflect this new organization. This relocation of the treatise on supposition is the one later followed by the \textit{Summulae} of John Buridan.

\subsubsection{Disagreements over valid consequences}
Burley's disagreements with Ockham are well-known, with the presence or absence of barbs towards Ockham playing a role in the dating of the works of Burley's corpus.\footnote{See \cite{Ottman1999}, \cite{Vittorini2013}.} Buridan's engagement with Burley is less known, but extensive. Buridan's knowledge of Burley's commentary on the physics is confirmed in \cite[p. 439]{Michael1985b}; his polemic against Burley's theory of universals, described in \cite{Markowski1982}. In his ontology, Buridan adopts Burley's reduction of the real Aristotelian categories to three - namely, substance, quality, and quality - albeit understanding the reduction in a manner different from that in Burley.\footnote{Cf. \cite[pp. 57-59, 204, 269-270]{Klima2009}; \cite[p. 13]{Read2016b}; \cite{DutilhNovaes2013}.} In logic, Buridan had direct knowledge at least of the \textit{Tractatus Brevior}: rules 3, 4, and 5 of Buridan's \textit{Tractatus de consequentiis} are rules 3, 2, and 1.1/1.2 of the \textit{Tractatus Brevior}; Buridan's sixth rule is a restricted instance of Burley's rule 2.1; and Buridan references the content of the eighth rule of Burley's shorter treatise in the eighth rule of the first book of his own treatise on consequences.\footnote{Cf. \cite[p. 212.29-31]{BurleyDPAL}, \cite[I. 8]{BuridanTC}.}

Ockham borrowed liberally from Burley's logic,\footnote{See \cite{Brown1972}.} and the parallels between consequences countenanced in Ockham's \textit{Summa Logicae} and Burley's earlier treatises are too extensive to be enumerated.\footnote{The consequences of SL III-3, ch. 38, for instance, can be found at \cite[par. 1, 4, 9, 14-15, 71-72, and 86-88]{Green-Pedersen1980b}.} Though Burley and Ockham's underlying accounts of supposition differ,\footnote{See \cite{Wagner1981}.} I've not found substantial disagreements in the consequences they accept or reject.\footnote{Thus, \cite[p. 276]{Read2007} regards Ockham's SL as a radical ontological sheep in a traditional logic's clothing: `Ockham uses the context of a work of logic ... to defend his central reductive ontological theme'.} With Buridan, however, one finds a more substantial disagreement, over consequences involving sentential and term negation. Unlike Buridan's, Burley's treatment of term negation, i.e. infinitizing negation, does not normally deviate from his treatment of sentential negation. `Socrates is non-white' and `Socrates is not white' have the same truth conditions, with both being true when Socrates does not exist \cite[pp. 57.17-58.12; 215.6-21; 216.15-18]{BurleyDPAL}. Burley's thus accepts the standard rules for obversion and contraposition, both from affirmatives to negatives and \textit{vice versa}, albeit with restrictions on the kinds of terms that can occur \cite[pp. 129-131, par. 73-81]{Green-Pedersen1980b}. This contrasts with the approach one later finds in Buridan, on which only the inference from an affirmative to its contraposited or obverted negative is good, but not conversely.\footnote{\cite[p. 85]{Buridan2015}. The root difference consists in their different treatments of the range of an infinite negation: Burley takes an infinite negation to range over both beings and non-beings; Buridan assumes it only ranges over existent entities.}
%Change pages for Read citation
\section{Following formally in Burley's logic}
The division between formal and material consequence, which became central after the work of Ockham and Buridan, plays only a minor role Burley's work, appearing most prominently in the solution to an objection in the \textit{Tractatus Longior} \cite[p. 80.13-29, 84.8-86.21]{BurleyDPAL}. Prior to this, Burley mentions formal consequence at \cite[p. 25.21]{BurleyDPAL} and formal repugnance at \cite[p. 39.20]{BurleyDPAL} in the \textit{Tractatus Longior}'s treatise on supposition. The \textit{de consequentiis}, discusses following `on account of the matter' at \cite[pp. 128-129, par. 70; 162-163, par. 168]{Green-Pedersen1980b}; following formally, at \cite[pp. 130, par. 75; 132, par. 84-85; 137, par. 106; 143, par. 118; 163, par. 168]{Green-Pedersen1980b}. 

In these passages, we find: that a consequence sharing the same form as a bad consequence does not hold formally \cite[p. 25.21]{BurleyDPAL}; that a rule fails to hold formally for a specified set of consequences when a consequence assumed to be governed by it fails to hold, i.e. when one can find a bad argument among those having the form specified by the rule \cite[p. 130, par. 75]{Green-Pedersen1980b}; that for a rule to hold formally is for it to hold generally;\footnote{Cf. \cite[p. 130, par. 75, 76]{Green-Pedersen1980b}.} and that a rule may hold formally for only a restricted class of beings \cite[p. 157-158, par. 159]{Green-Pedersen1980b}. Burley calls that part of a sentence `formal' which is to be negated in its contradictory - what today one would call the main connective of a sentence.\footnote{\cite[pp. 73.29; 208.12-30]{BurleyDPAL}; \cite[p. 120, par. 40]{Green-Pedersen1980b}.} 

From these, one can see that some rules do not hold for all terms, but only certain kinds of terms;\footnote{Cf. \cite[p. 214.14-21]{BurleyDPAL}.} and that Burley was aware of the substitution test for formal consequence, alluded to by Simon of Faversham and explicit in Buridan: to determine whether a consequence is formally good, one should obtain a new sentence by substituting its categorematic terms with other terms; and if one can find a consequence of this form whose antecedent can be true without its consequent, then the first consequence was not formally good.\footnote{\cite[pp. 162-163, par. 168]{Green-Pedersen1980b}; \cite[p. 150.32-35]{BurleyDPAL}; \cite[qq. 36-37]{FavershamQE}; \cite[I. 4]{BuridanTC}.} 

Unlike Buridan, Burley gives no indication that the substitution test provides a not only necessary, but also sufficient condition for following formally. Nor, more generally, does one find any single kind of consequence defined in terms of the impossibility of the antecedent and contradictory of the consequent being true together. Burley does state that `for a consequence to be good, it suffices and is required that the contradictory of the consequent conflicts with the antecedent' \cite[p. 64.12-14]{BurleyDPAL}. But in order to obtain the more general definition from this - and with it the consequences \textit{from the impossibile anything follows} and \textit{the necessary follows from anything} - one must assume that any proposition conflicts with an impossible one, and that any proposition stands with a necessary one. But these assumptions do not appear in Burley's text.

The objection mentioned from the \textit{Tractatus Longior} is directed at rule 2.1 of the treatise. One can see the development of Burley's thinking about formality by tracing his remarks on this rule. 

In the earliest mss. of the \textit{de consequentiis}, the rule is likewise paired with rule 2.2, albeit with their order the reverse of that in the later treatises. Burley says of the latter rule that it follows `necessarily or formally', apparently intending to equate these notions \cite[p. 132, par. 84]{Green-Pedersen1980b}. In the revised texts of the same, one finds the words `necessarily or formally' dropped, not to appear again in either version of the \textit{De Puritate}. Thus, it seems that Burley early equated a formal consequence with a necessary one. This view, however, was dropped by 1302, the date given by the scribe of the London script.

In the \textit{Tractatus Longior}, Burley voices the following objection to rule 2.1: 
\begin{quote}
	`Brunellus is risible, therefore, Brunellus is a man' follows; and from these two, it follows formally that a man is risible. For `Brunellus is risible, Brunellus is a man, therefore a man is risible' follows formally. And yet from the antecedent by itself the same does not follow formally. `Brunellus is risible, therefore a man is risible' does not follow formally, because then distributing the consequent, the antecedent follows; and then `Every man is risible, therefore Brunellus is risible' follows, which is false. And even if `Brunellus is risible, therefore a man is risible' were to follow formally, it would follow in the same way placing the negation after. And then `Brunellus is not risible, therefore a man is not risible' would follow, where the antecedent is true and the consequent false \cite[p. 80.13-29]{BurleyDPAL}.
\end{quote}

The anonymous Parisian \textit{de consequentiis} contains a substantial parallel to this objection from the \textit{Tractatus Longior}. In the former, we find: 
\begin{quote}
	One should know this rule is invalid: \textit{whatever entails the consequent entails the antecedent}, as is shown if it is so argued: if a risible [thing] is an ass, then a man is an ass; for here by the rule \textit{whatever entails the consequent entails the antecedent}, one argues that `a risible [thing] is an ass' entails `an animal is an ass', and thus `a risible is an ass' also entails `a man is an ass'. And it is so argued by this rule: \textit{whatever entails the consequent entails the antecedent} since `a risible [thing] is an ass' entails the consequent `an animal is an ass' \cite[p. 16, par. 21]{Green-Pedersen1980a}.
\end{quote}
In Burley's treatise, we find the terms transposed and the common term `\textit{asinus}' replaced with Brunellus, a proper name for a donkey. In both texts, what is resisted is the inference from `an ass [Brunellus] is risible' to `an ass [Brunellus] is a man'. Burley himself will allow for a sense in which the inference is formally good - namely, by reason of three terms. The anonymous text, with the objection, denies this, both resisting the inference from a proper accident to its bearer in contexts where \textit{per impossibile}, the proper accident is predicated of something different from its standard bearer.\footnote{Cf. \cite[q. 11, par. 19]{ScotusQE}.} Neither text subsumes the inference under the rule \textit{ex impossibili quodlibet}. From this, it seems likely to me that the author of the anonymous Parisian treatise was also a source for the objection as it appears in Burley.

The objection in the \textit{Tractatus Longior} is broken into two parts: the first assumes an undistributed antecedent that formally entails its undistributed consequent is itself entailed by the same consequent with term distributed; the second, that the presence of a negation should make no difference to whether an antecedent with an undistributed subject entails a consequent with one. The first assumption is accepted as a rule by Burley at \cite[pp. 117-118, par. 26-31]{Green-Pedersen1980b}, and discussed under the seventh principal rule\footnote{Namely, that a consequence from a distributed superior to its inferior, taken either with or without distribution, holds. But not conversely.} of the \textit{Tractatus Brevior} \cite[p. 211.21-212.28]{BurleyDPAL}. Both passages consider objections to the rule, in the light of which its domain of application is clarified. But neither mentions the notion of formality, nor objects in the manner found in the \textit{Tractatus Longior}. 

In the \textit{de consequentiis}, Burley responds to the rejection of the first assumed rule by distinguishing between two different ways a consequent can follow from its antecedent: some consequences hold `by reason of its incomplex [parts]' - i.e. the \textit{significata} of its terms; others, by reason of the whole complex \cite[p. 118, par. 31]{Green-Pedersen1980b}. The distribution rule applies to consequences of the former, but not the latter type. Furthermore, the distribution has to be applied to the terms on account of which the consequence holds. 

In the \textit{Tractatus Longior} text, Burley adds that the consequence `Brunellus is risible, therefore a man is risible' holds formally `by reason of three terms'; and that the distribution rule only holds for consequences that are formal `by reason of two terms' \cite[pp. 84.11-85.17]{BurleyDPAL}. Thus, while both texts distinguish between consequences holding by virtue of their parts from those holding by virtue of their structure, the later text further distinguishes among the former type according to the number of terms it holds by, and only the later text considers these to be different ways of following \textit{formally}.

For Burley, consequences which hold in virtue of their whole structure include conversions, syllogisms \cite[p. 86.9-12]{BurleyDPAL}, arguments from an exclusive to a universal transposing its terms and vice versa,\footnote{Burley's example is `every man is an animal, therefore only an animal is a man' \cite[pp. 142-143, par. 118]{Green-Pedersen1980b}.} and presumably consequences such as obversion, contraposition, and the immediate inferences found in the square of opposition. These are contrasted with consequences holding formally in virtue of their terms. What Burley describes by `formal consequence holding in virtue of its simple [parts]' consists chiefly of two things: the first, standard quantificational rules for descent to singulars from quantified common nouns, and for ascent from singulars to the nouns they fall under; the second, Hodges `calculus of monotonicity'. `a man runs, therefore an animal runs' is said to hold by two terms; `Brunellus is risible, therefore, a man is risible', by three; `a man runs, therefore an animal moves', by four \cite[p. 84.24-27]{BurleyDPAL}.

The above approach to formal consequence bears a \textit{prima facie} similarity to one found in Bolzano, and explored at some length by Rolf George \cite{George1986}. On this approach, the notion of `following formally' is relativized not merely to a list of logical constants, but also to an explicit demarcation of variands. Bolzano writes: 
\begin{quote}
	Propositions $M, N, O$ ... \textit{follow} from propositions $A, B, C, D$ ... with respect to variable parts $i, j,$ ... if every class of ideas whose substitution for $i, j,$ ... makes each of $A, B, C, D,$ .. true also makes all of $M, N, O,$ ... true \cite[p. 209]{Bolzano1972}.
\end{quote}

On this approach, then, a consequence holding in virtue of two terms, say, would be one where those terms are, with its syncategoremata, added to its list of invariant parts, with its other constants being among the variable parts. This does appear to capture Burley's practice with respect to consequences holding in virtue of two terms: when Burley wishes to show that a proposition fails to hold in two terms, he varies the third to find a counter example. For instance, `Brunellus is risible, therefore a man is risible' with `risible' as variable part admits the counterexamples `Brunellus is running, therefore a man is running', and `Brunellus is a braying animal, therefore a man is a braying animal' \cite[84.13-15]{BurleyDPAL}. But the problem with ascribing this approach to Burley becomes clear when one considers consequences said to hold in virtue of all of their categorematic terms. As Tarski put it, 
\begin{quote}
	The extreme would be the case in which we treated all terms of the language as logical: the concept of following formally would then coincide with the concept of \textit{following materially}--the sentence $X$ would follow from the sentences of the class $K$ if and only if either the sentence $X$ were true or at least one sentence of the class $K$ were false \cite[pp. 188-189]{Tarski2002}
\end{quote}

In the course of its reply, the \textit{Tractatus Longior} records a third objection making the above Tarskian assumption: a consequence where all terms are held fixed must be a material one.\footnote{\cite[p. 86.4-9]{BurleyDPAL}. The main differences between Tarski's remark and the objection to Burley are two: first, the determination of which parts are fixed takes place at the level of the individual consequence for Burley's objector, but of the language for Tarski; second, Burley's notion of following requires not merely the truth [falsity] of the antecedent [consequent], but its being necessarily so.} The objection, however, is ambiguous, depending on what is meant by `material consequence'. The first reading assumes that such a consequence is material in the sense that its antecedent is impossible, or consequent necessary. This is a modalized variant on Tarski's claim above, and accords with the examples of material consequence given in Ockham's \textit{Summa Logicae}.\footnote{\cite{OckhamSL}. Cf. \cite[p. 7, par. 18; ]{Green-Pedersen1980a}.} The second, version assumes the consequence is material in the sense that precisely because of the terms in it, application of a rule that would not normally be admissible becomes so in that specific context. This understanding of material consequence is that present in Simon of Faversham \cite[q. 36]{FavershamQE}. A clear statement of the view is found in Ockham's late \textit{Elementarium Logicae}:

\begin{quote}
	A material consequence is [one] which does not hold by virtue of the mode of argument, but thanks to the terms it is composed from. In this way, `an animal debates, therefore a man debates' follows, but because the predicate `debates' cannot accord to any animal besides man.\footnote{\cite[VI.4, p. 163]{OckhamEL}. The final editors of the work regard Ockham's authorship as doubtful. But the reasons for doubting its authenticity are not strong, mostly based on perceived discrepancies between the views in the text and those expressed in Ockham's SL. The external evidence in favor of authenticity is stronger and more concrete. For the case for Ockham's authorship of the work, see \cite{Boehner1958b}.}
\end{quote}

Ockham's example here is structurally analogous to the objection found in Burley, where the failure of the consequence with the predicate changed is used to press the non-formal character of the consequence. It seems likely to me that the objection as Burley construes it is working with an understanding of formal consequence closer to that of Faversham and the \textit{Elementarium} than that of Tarski. Burley replies:
\begin{quote}
	`for a consequence to hold in virtue of its terms is twofold: either because it holds materially by reason of its terms, or because it holds formally by reason of its terms, that is, from the formal measure (\textit{ratione formali}) of the terms. I say then, that a consequence can be formal by reason of its terms, and this if it holds \textit{per se} by reason of its terms. If, however, it holds by reason of its terms accidentally, then it is not formal'
\end{quote}
That Burley rejects Tarski's conclusion shows Burley rejects the reduction of formal consequence to a consequence's holding good under all substitutions for terms, even in the nuanced form one finds in Bolzano.\footnote{\textit{Pace} \cite[pp. 16-17]{HodgesBurley}.} 

More importantly, the above passage provides a way of relating the formal/material division to his earlier natural/accidental division. By a consequence holding \textit{per se}, Burley means the same as one where the meaning of the consequent is contained in the antecedent.\footnote{\cite[p. 158, par. 160]{Green-Pedersen1980b}.} This is just the containment criterion found in his definition of natural consequence.\footnote{Cf. \cite[I, d. 11, q. 2]{ScotusRepPar}.} In contrast, the above response states a consequence holding in virtue of its terms accidentally is not formal. And as the context makes clear, Burley here intends that such a consequence be understood as a material one.

Thus, for Burley we have 1) formal consequences holding in virtue of their whole complex, including conversions and syllogisms; 2) formal consequences which hold in virtue of their terms, such as `if a man is an animal, a man is a substance'; 3) accidental consequences holding in virtue of their terms, such as ``God exists' is true, therefore `God exists' is necessary'; and another class of accidental consequences including \textit{ex impossibili quodlibet} and \textit{necessarium ad quodlibet} \cite[pp. 128-129, par. 70]{Green-Pedersen1980b}. The above types are listed in order of their strength, with consequences belonging to the first class being the strongest, and those of the last class, the weakest. The goodness of consequences holding by their structure is immediate and necessary. Consequences holding by virtue of their terms are enthymematic consequences, which `have to be reduced to a syllogism' \cite[p. 142, par. 117]{Green-Pedersen1980b}. This class may be further divided into consequences holding by virtue of $n$ terms, for any $n$. The third are consequences holding good by a restricted case of a rule that is not normally good, but is allowed in the given context: `God's being is true, therefore God's being is necessary' holds as an accidental consequence because in the context of the antecedent, the term `true' supposits for the same as what the term `necessary' supposits for, i.e. God's being, thus permitting an otherwise impermissible substitution.\footnote{Burley's point is lost in the most straightforward English translation of \textit{Deum esse est verum, ergo deum esse est necessarium}. i.e. that where \textit{Deum esse} is taken for the sentence `God exists'. To preserve Burley's point, one must take the infinitive clause as done above.} The final sort hold strictly by the extrinsic topic \textit{from the less}. The first two of the above kinds are called natural consequences, and are said to hold by an intrinsic topic; the latter two, material consequences holding by an extrinsic topic.\footnote{Cf. \cite[p. 130]{Martin2004}.} Not all consequences holding in virtue of their terms are formal: there are accidental consequences which also hold in virtue of their terms.

\section{Conclusion}
What the above has revealed is a wide array of achievements attributable to the \textit{Doctor Planus et Perspicuus}, which help us better understand his place in the developments that took place in the theory of consequence during his lifetime. 

Burley's \textit{De consequentiis} provides: the most expansive known treatment of consequences prior to Ockham's \textit{Summa Logicae}; the earliest explicit taxonomy of kinds of consequences; and the earliest inclusion of syllogistic under the banner of consequences, an inclusion never adopted in Ockham's works. Burley streamlines the treatment of consequences found in the earliest treatises by leaving aside the rule that a proposition predicating the same of itself must always be true; and with the later treatises, he preserves both affirmative-to-negative and negative-to-affirmative directions of conversion and contraposition by providing a parallel treatment of sentential and term negation.

In the shorter \textit{De Puritate}, Burley provides the earliest known organization of warrants governing consequences into principal and derivative rules. In the same treatise, he makes explicit that identity statements may fail in cases where the subject does not exist; qualifies the range of the rule \textit{from the impossible anything follows}, since the less impossible does not entail what is more impossible; and provides a brief treatment of consequences involving modalities. In the movement from the shorter to the longer \textit{De Puritate}, we see the relocation of supposition theory from being treated between topics and syllogisms, to being placed in the context of treatments of the proposition, a place it retains in the later work of Buridan.

By the time of the longer treatise, Burley has greatly streamlined the treatment of principal and derivative rules first attempted in the \textit{Tractatus Brevior}; and expanded the range of dialectical maxims used in topical arguments to an indefinite number of \textit{logical} maxims, to be used in consequences. In the same treatise, Burley relates the contrast between natural and accidental consequences found in the \textit{De consequentiis} to that between formal and material consequences, and further subdivides enthymematic formal consequence according to the number of terms they hold by.

There are, of course, a few failings in Burley's theory. The ambiguity between the different senses of \textit{per accidens} ascent and descent is not resolved, which likely contributed to the distinction between natural and accidental consequence falling out of favor. Though Burley recognizes both downward monotonicity and descent to particulars may fail because of complications involving existence, his practice ignores this insight. The subdivision of enthymematic formal consequences by number is underdeveloped, and without further clarification it is hard to see how some among these consequences should be distinguished from material consequences. 

However, the basic import of Burley's work remains: Burley provides a plurality of different understandings of `follows', with differing levels of strength, grounded in the strength of the relations involved in the \textit{significata} in the consequences; natural structural consequences when grounded in the meaning of the syncategorematic terms; natural enthymematic consequences, grounded in an intrinsic containment relation between the significates of its categorematic terms; and accidental consequences grounded on weaker relations. Today's more pluralist logical environment, aiming to give expression to different kinds of relevant containment while still allowing for classical consequence, is much friendlier to Burley's project than it is to the monism that followed in Buridan's reduction of all consequences to formal ones - a reduction mirrored in the classical monist approach to formal consequence that dominated the last century. As such, now provides an opportune environment for a revival of interest in Burley's logical work, while Burley's work itself should provide inspiration for a better, more fruitful grounding of pluralism than that available at present. At the same time, an examination of the logic of Burley the realist provides much of interest even for understanding the nominalists opposed by - and indebted to - him.
\printbibliography 
\end{document}
