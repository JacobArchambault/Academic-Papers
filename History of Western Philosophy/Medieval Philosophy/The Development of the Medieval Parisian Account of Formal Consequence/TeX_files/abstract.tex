\chapter{Abstract}
Jacob Archambault

\indent BA, Franciscan University of Steubenville

\indent MA, University of Houston

\indent \textit{The development of the medieval Parisian account of formal consequence}

\noindent Dissertation directed by Gyula Klima, Ph.D.
\bigskip

The concept of formal consequence is at the heart of logic today, and by extension, plays an important role in such diverse areas as mathematics, computing, philosophy, and linguistics. In this dissertation, I trace the roots of this concept in medieval logic from Pseudo-Scotus and John Buridan back to the earliest treatises on consequences, and provide translations of the three earliest known treatises on consequences.

Chapter one introduces the reader to the dominant philosophical approaches to formal consequence from the turn of the twentieth century to today. After this, I introduce the account of formal consequence advanced by John Buridan, the medieval predecessor to the semantic account advocated by Tarski and his followers. 

Chapter two provides a detailed contrast of Buridan's account of formal consequence with those of Tarski, on the one hand, and later classical logic, on the other. 

Chapter three examines the account of formal consequence in Pseudo-Scotus. I show that Pseudo-Scotus' account is dependent on that of Buridan, and therefore most post-date it. 

Chapter four examines the account of divided modal consequence in William of Ockham. I show that Ockham's divided modalities are not fully assimilable to narrow-scope propositions of classical modal logic; formalize Ockham's account in an extension of first-order modal logic with restricted quantification; and provide a complete account of relations between two-term divided modal propositions on Ockham's account. 

Chapter five introduces Walter Burley's thinking about consequences, examining: Burley's division and enumeration of consequences; his distinction between principal and derivative rules licensing good consequences; the relation of the division of consequences into formal and material varieties to Burley's preferred division between natural and accidental consequences; the relation Burley's work bears to Buridan, to the Boethian reception of Aristotle's Topics, and to the earliest treatises on consequences. 

The final chapter concludes: highlighting the characteristic marks of medieval and modern approaches to consequences relative to each other; summarizing the various developments that led to the adoption of the account of formal consequence epitomized in Buridan's work; and suggesting prospects for recovering the most promising aspects of the medieval treatments of the topic.
