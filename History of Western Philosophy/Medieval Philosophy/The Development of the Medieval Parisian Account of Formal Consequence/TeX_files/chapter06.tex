\chapter{Conclusion}
\section{Introduction}
Now is the appropriate time to place the genesis of Buridan's theory in its proper context. 

The renaissance of scholarship on medieval logic broadly and John Buridan specifically took place against the backdrop of two movements. The earlier movement involved a renewed look at the multiplicity of medieval Catholic thought after the cooling down of the modernist controversy. The later movement involved not a recovery of the middle ages as a source of Catholic thought, but rather its reappropriation for an increasingly non-religious academic sector.

The first Vatican council brought about a greater unification of Roman Catholicism than had existed at any time prior to it. This was secured not only institutionally via its proclamations concerning the authority and infallibility of the papacy, but also intellectually via a renewed focus on Catholic philosophy, in particular on Augustine and Aquinas, later encapsulated by Pope Leo XIII's encyclical \textit{Aeterni Patris}.\footnote{This renewal still governs the historiography of medieval philosophy as a whole, considering that there is really no good reason to consider Augustine a medieval thinker apart from the understanding of the medieval first wrought by this revival.} The intellectual aspect of this unification reached its apex in Pius X's condemnation of modernism in the 1907 encyclical \textit{Pascendi Domini Gregis}, in whose wake one should locate the Neo-Thomistic revival of the work of Father Garrigou-Lagrange and others. 

In the security brought about by the early 20th-century condemnation of modernism, one finds a reexamination of the diversity of Catholic thought. Part of this was the \textit{nouvelle theologie} of Congar, De Lubac, and others who effected a return to patristic and early medieval theology and biblical exegesis. But another part was the recovery of non-Thomistic strands of scholasticism, first in the work of Thomists like Gilson, but later in its own right. To this movement we would owe the founding of the Pontifical Institute of Medieval Studies at the University of Toronto, the Franciscan Institute at St Bonaventure University, and the journal \textit{Franciscan Studies}. Through the work of Fathers Philotheus Boehner, Allan Wolter, and others, the last-mentioned institution and journal played a critical role especially in the recovery of the logical contributions of Duns Scotus, Walter Burley, and William of Ockham.

The recovery of medieval logic generally, and that Buridan in particular, is in great part the work of the later movement. Through the early 20th century, anthologies and overviews of medieval intellectual history focused largely on natural theology and metaphysics. This focus was understandable, since it was contributions to these areas that medieval thought was best known for. However, with the ill-repute into which metaphysics fell in the first half of the 20th century and in which natural theology largely remains, it became necessary, rather than leaving the medieval period as an intellectual vacuum to be abhorred, to find some other manner of redeeming it in the light of the values and achievements of the philosophical work of the time. This movement coalesced with an organic interest in the history of logic, instanced in the work of \L{}ukasciewicz and Kneale and Kneale, to bring about an interest in medieval theories of consequence, with the task of reinforcing the image of medieval theories as an imperfect herald of modern approaches to the topic. To this directive, we owe the early work of Moody and Broadie on the topic; the founding of the \textit{Corpus philosophorum Danicorum}, the editions and studies in the \textit{Cahiers de L'institut du Moyen \^{A}ge Grec et Latin}, and the work of Pinborg, Ebbesen, Green-Pedersen, and others at the University of Copenhagen; the critical editions of De Rijk, his founding of the journal \textit{Vivarium}, and the establishment of the \textit{Artistarium} series exploring the work of secular medieval arts masters; and the work of Kretzmann, Stump and others at Cornell University, culminating in the 1982 \textit{Cambridge History of Later Medieval Philosophy}, which devoted considerably more space to discussion of logic than any prior survey of the medieval period, deliberately sacrificing space devoted to metaphysics and philosophical theology. The recovery of Buridan's logic formed the centerpiece of work on secular masters, rather than those belonging to religious orders; and the assimilation of Buridan's theory of consequence to modern model-theoretic approaches to the topic, the centerpiece of the reappropriation of the medieval period \textit{sans} its metaphysical and theological outlook. If you are reading this book, then chances are you owe a great debt to the shape this recovery took, even if you may doubt its starting premises.

With a quarter century passed since the release of the aforementioned \textit{Cambridge History}, we are now in a better place to more exactly assess both Buridan's relation to modern model-theory and to his contemporaries, and the veracity of the theories motivating his rediscovery as a major figure of medieval thought.

We begin by relating Buridan's theory to its modern model-theoretic counterparts. From there, we summarize{\tiny } Buridan's place in the development of formal consequence, and its relation to the broader development of consequence up to Buridan. Lastly, we offer some remarks on how the definition of formal consequence arrived at where it is today.
\section[From semantic to formal consequence]{From semantic consequence to medieval formal consequence}
\subsection{Contrasting Tarskian and Buridanian consequence}
Buridan's definition of consequence is a deflationary one: `a consequence is a hypothetical proposition composed of an antecedent and consequent' \cite[I.3]{BuridanTC}. In addition, he offers a criterion for consequence according for which a consequence is good when it is impossible for things to be as the antecedent signifies without being as the consequent signifies. For Buridan, this criterion needs to be reformulated to fit the tense, modality, and quality of propositions present in an antecedent and consequent - for instance, the proposition, `if Socrates was running, Socrates was moving' is good if is impossible for things to \textit{have been} as the antecedent signifies, etc. 

Buridan divides consequence into formal and material, with the latter divided into simple and as-of-now consequence. A formal consequence is one belonging to an equivalence class, determined by its syntactic structure, such that all consequences in that class are good. A material consequence is a good consequence not of this sort. A formal consequence is said to be good in virtue of its formal parts, i.e. its syncategoremata. And though consequences may be valid without being formally valid, a materially valid consequence is only made evident by its reduction to a formally valid one. A simple consequence is one for which Buridan's criterion for a good consequence holds without qualification; an as-of-now consequence, one for which it holds for a given time.

Many of the better-known formal developments since Tarski have brought about a rehabilitation of Buridanian themes. Work on tense, modality, and other intensional operators, for instance, has been the norm since Prior and Kripke; the acceptance of domain variation, since Kemeny. There are also approaches to formal consequence which, with Buridan, take the time of utterance into account, though these remain non-standard.\footnote{See those systems drawn on in \cite{DutilhNovaes2005,DutilhNovaes2007b}.} In other ways, more recent approaches to consequence have moved away from aspects common to both Buridan and Tarski. For instance, in contrast with classical consequence today, neither Buridanian nor Tarskian consequence is schematic, and neither interprets the non-formal parts of the languages to which it applies arbitrarily.

Modern approaches to formal consequence depart from their medieval predecessors in their identification of formal and logical consequence, and their identification of consequence as the subject matter of logic as such. For Buridan and the medievals generally, there are logical consequences which are not formal, including induction and enthymematic consequences. And consequence forms only a small and relatively new part of what medieval logic studies. For Ockham `consequence' is restricted to non-syllogistic argument. For Buridan, it encompasses both syllogistic and other kinds of argument.

\subsection{Medieval consequence as a key to understanding semantic consequence today}

Against the medieval backdrop, the most distinctive aspects of formal consequence today are the concepts of \textit{language} and \textit{function} it employs.

Formal \textit{consequence} is now defined over a formal \textit{language}: typically, a collection of syntactic strings divided into formal and non-formal primitives, with the referent of each formal element fixed to a semantic function, and that of each non-formal element varied arbitrarily; from which a countably infinite collection of formulas is recursively defined. 

The concept of \textit{function} herein employed is a mathematical one: an $n$-ary function $f$ is a mapping from $n$-tuples in some collection of $n$-tuples, called the \textit{domain} of the function to an element in some other, possibly distinct collection of elements, called its \textit{range}. Such a mapping may be \textit{total}, mapping each element in the domain to an element in the range; \textit{partial}, i.e. only from some elements in the domain; \textit{many-to-one}, mapping some distinct elements in the domain to the same element in the range; or \textit{one-to-one}, mapping each element in the domain to a distinct element in the range. But no function is one-to-many. 

The formal consequence relations for classical logic, its extensions, and rivals all operate on a functional understanding of language. In practice, a phrase like `the mother of' is treated as a function from the domain of individuals to itself, mapping each person to his or her mother. $n$-place predicate symbols are treated as functions from $D^{n}$, the $n$th Cartesian product of the domain of a model, to a set of truth values. Names in classical logic are assimilated to functions of arity 0 to the domain of a model; atomic propositions, to predicates of arity 0.

This functional understanding of language leaves a host of dualisms in its wake, which shape the landscape of philosophical logic today: between language and world; object language and metalanguage; names and predicates; use and mention, and between the logical and non-logical parts of a language. Each of these, in turn, brings with it one of modern logic's characteristic \textit{insolubilia}: problems with self-referential terms; Tarski's hierarchy of languages, with its corresponding hierarchy of truth predicates; Frege's `problem of the concept \textit{horse},' concerning the relation between concept and object; over the admissibility of intensional operators into formal languages; and the problem of demarcating the logical from non-logical constants.

Modern approaches to formal consequence thus build on a semantic base that assimilates meaning to function, and eliminates the medieval distinction between signification and supposition, i.e. that between the \textit{meaning} of a lexical item, and its function within a given sentence. The medieval distinction between the syncategorematic and categorematic is thereby transformed from a local one between sentential roles into a global one between linguistic types, generating the demarcation problem for logical constants. Because no function is one-to-many, formal languages are unequipped to distinguish between different kinds of supposition a term might have; nor to deal with problems caused by linguistic ambiguity, which were often at the root of medieval work on fallacies, \textit{sophismata}, and \textit{insolubilia}.

\section[From formal to natural consequence]{From formal and material to natural and accidental consequence}
\subsection{Summary}
John Buridan provides the earliest account of formal consequence in terms of a substitution criterion. On Buridan's criterion, a consequence is formal if and only if it is good for all uniform substitutions on its categorematic terms. This is the ground for its claim to be a predecessor of Tarski's definition, which transforms Buridan's uniform substitution criterion on terms into one over models, i.e. over orderings of objects satisfying sentential functions obtained from one's initial sentences by substituting non-logical constants with like variables. Though the use of such substitution techniques to check validity goes back to Aristotle, Buridan appears to be the earliest to have seen in it both a necessary and sufficient condition for formal consequence.

Buridan's account is imperfect in several respects. Its notion of consequence undergirds a solution to the Liar paradox, but remains susceptible to the Pseudo-Scotus paradox. Buridan subdivides material consequence into simple and as-of now consequence, but Buridan's stated criteria for these cannot distinguish simple material consequences from formal ones, or as-of-now consequences from purely invalid ones. Buridan's definitions of simple and as-of-now consequence are later improved upon by Pseudo-Scotus, who distinguishes them by the way they are reduced to a formal consequence: simple consequence, by adding a necessary proposition to the antecedent; as-of-now consequence, by adding a contingent one. The discovery of the Pseudo-Scotus paradox, however, seems to have led to despair over the possibility of providing a simple criterion for valid consequence, and the proliferation of more \textit{ad hoc} criteria in later medieval logic.

Though Buridan admits good consequences that are not formal, any such consequence which is also \textit{evident} is only made so by its reduction to a formal one. Not every material consequence is so reducible: examples and induction are not. For Buridan, the category `consequence' is expansive enough to include both syllogistic and non-syllogistic consequences; formal consequence is the source from which any good consequence is evident; and material consequences are either reducible to formal ones, or lacking in evidence. The resulting picture is what one may call \textit{reductive consequential monism}: there are no ontological differences present in the different ways of following, but only differences in evidence; as far as ontology is concerned, there is only one basic kind of consequence to which others are reducible. And those which are not so reducible have a secondary, imperfect epistemic status.

Buridan adopts the subsumption of syllogistic under consequence from Burley. This treatment, common to Burley and Buridan, differs from that of Ockham, who takes the domain of consequences to be that formerly allocated to topical argument. 

The division of consequences into formal and material varieties goes back to Simon of Faversham, and is also present, albeit not as an explicit taxonomy, in the anonymous London \textit{de consequentiis}. The earliest formal division between formal and material consequence is in William of Ockham's \textit{Summa Logicae}, though Ockham's division differs from that found in Buridan. For Buridan, a consequence is formal if it is good for all uniform substitutions on categorematic terms. Ockham countenances as formal those which, in addition, are reducible to those which are formal in Buridan's sense. 

In Ockham, there remains some distinction between different ways in which something may follow, albeit attenuated from that found earlier in Burley. Ockham calls `material' those consequences which hold `precisely by reason of the terms', and countenances the consequences \textit{from the impossible anything follows} and \textit{the necessary follows from anything} as of this type. In Buridan, by contrast, even this distinction is erased: Buridan appropriates the language of the formal-material division for his own purposes, and builds the validity of Ockham's material consequences into the criterion for consequence as such: since it is impossible, for instance, for things to be as an impossible antecedent signifies, it is also impossible for things to be as it signifies without things being as its consequence signifies. Buridan further takes the consequence to hold formally for explicit contradictions, justifying it by disjunctive syllogism, in the same manner C. I. Lewis would later in his account of strict implication.\footnote{This method for proving anything from an explicit contradiction is first reported by Alexander Neckham and attributed to William of Soissons. See \cite{Martin2012}.} 

Ockham distinguishes Buridan's formal, structural consequences from those enthymematic formal consequences which Buridan classifies as material by appealing to topics: for Ockham, a structural consequence holds by an extrinsic topic; an enthymematic consequence holds by an intrinsic topic immediately, and mediately by an extrinsic topic. By `topic', Ockham means what Boethius means by `maximal proposition': a rule licensing an inference from a premise or premises to a conclusion.\footnote{The straightforward identification of maximal propositions with rules is also found in Burley's longer version of the \textit{De Puritate}, though Burley's identification likely antedates that of Ockham.} An intrinsic topic is a premise added to an enthymematic argument to make it a formal one. It is intrinsic in that it governs things mentioned in the stated premise directly. An extrinsic topic for Ockham turns out to be a rule stated in terms of second intensions, under which the objects mentioned in the premises of a given argument fall accidentally, e.g. in virtue of their being given a certain supposition in a proposition, etc. 

Prior to the distinction between formal and material consequence which gains currency in Ockham's work, one finds a distinction between natural and accidental consequence implicit in Boethius, and present later in William of Sherwood, Scotus, and in Burley's \textit{de consequentiis}. Burley, following Scotus, grounds the distinction in one between intrinsic and extrinsic topics, though by these he means something different than what Ockham does. For Burley, a natural consequence, which holds through an intrinsic topic, is one which satisfies a containment criterion, where `the antecedent includes the consequent' \cite[p. 61.6-10]{BurleyDPAL}. An accidental consequence, which holds by an extrinsic topic, is one holding by some extrinsic relation between the things named therein. Examples include consequences from the positing of one contrary to the denial of the other, and from the positing of a species to its \textit{proprium}, its inseparable attribute. Burley classifies the consequence \textit{from the impossible anything follows} as an accidental consequence holding by the topic \textit{from the less}. 

Burley identifies his natural consequences with formal consequences in Ockham's sense. Against an objection resembling Buridan's position, Burley insists that not all formal consequences are structural consequences. In this much, he is in agreement with Ockham. But against Ockham, he does not classify \textit{ex impossibili quodlibet} as a material/accidental consequence holding by virtue of its terms, but as one justified by appeal to the extrinsic topic \textit{from the less}.
\subsection{Analysis}
From Burley and his predecessors to Ockham, and then from Ockham to Buridan, we see two important shifts, each brought about by the appropriation of the language of a prior position for alternative ends. The first terminological shift is in the understanding of a topic; the second, in the understanding of consequence.

In Boethius, a topic represents a real aspect of a thing from which something about it or another is inferred. That is, a topic is the difference of a maximal proposition. From this, an analogous use of the term arises, i.e. to refer to the maximal proposition itself.

In Burley, the focus on the differences of such propositions is diminished, and a topic is more often simply identified with a rule licensing an inference. The distinction between intrinsic and extrinsic topics, however, continues to be made along  traditional lines.

At the time of Burley's \textit{de consequentiis} and the earliest anonymous treatises, the development of supposition theory brought about a great simplification in the number of topics actually appealed to, with a vast number of inferences justified by rules like \textit{from an undistributed inferior to superior}, \textit{from a distributed superior to a distributed inferior}, etc. That is, appeals to qualitative distinctions grounded in aspects of objects gives way to appeals to scope distinctions grounded in supposition of terms.

In Ockham, the distinction between intrinsic and extrinsic topic is no longer drawn along traditional lines: rather, an intrinsic topic is simply a premise added to an antedecent to make it a structurally formal one; and an extrinsic topic is a rule pertaining to second intensions, licensing the movement from antecedent to consequent in a particular kind of inference. 

In Buridan, the language of topics disappears altogether from treatises on consequences, as does the distinction between intrinsic and extrinsic topics. In its stead, we see Buridan arguing that extrinsic topics are reducible to formal ones in exactly the same sense as arguments employing intrinsic topics: 
\begin{quote}
If by the topic \textit{from contraries} we argue so: `A is white; therefore A is not black', it is still the case that a syllogism or formal consequence is effected by the addition of the proposition `no white is black'. And so the other dialectical topics, if they are not formalized, do not conclude on account of form, but can be reduced to a form by additions \cite[sec. 6.6]{BuridanLoci}.
\end{quote}

This disappearance is closely related to the second shift, that from natural/accidental to formal/material consequence. 

In Burley's work, the division of consequences into natural and accidental is a subdivision of simple consequence. The natural/accidental division is clearly intended as an ontological one, grounded in a distinction between intrinsic and extrinsic properties. However, the distinctions between \textit{per se} and \textit{per accidens} predication, on the one hand, and \textit{per se} and \textit{per accidens} ascent and descent, on the other, were poorly worked out in the earliest versions of the theory. 

Ockham lists the two divisions, the one between formal and material, the other between simple and as-of-now, without subordinating either to the other. His example of a formal consequence holding by an intrinsic topic in the \textit{Summa Logicae} - `Socrates does not run, therefore a man does not run' - is deliberately chosen to provide an example of a formal consequence which is nevertheless as-of-now.\footnote{\cite[III-3. 1, p. 588]{OckhamSL}. It is unlikely that Burley would have agreed with Ockham on this classification. For Burley, a simple consequence appears to be one where 1) there is a \textit{per se} relation between the antecedent and the consequent, and 2) in every situation where the antecedent holds, the consequent holds. For Ockham, by contrast, such a consequence additionally requires that the predications in its antecedent and consequent themselves must be intrinsic; and 2) that the antecedent itself holds in every situation. Burley would accept the consequence as formal, given the containment criterion is satisfied for the terms `Socrates' and `man' (names of individuals were taken to imply their natures). Given this disagreement, the number of simple formal consequences seems to be greatly curtailed in Ockham's logic.} More importantly, the distinction between these in terms of topics gives way to a classification according to whether such consequences appeal to universal rules or terms: structural formal consequences hold by an extrinsic rule; enthymematic formal consequence, mediately by a structural rule and immediately by one pertaining to the terms, i.e. by an added premise; material consequences, by the terms alone. In Ockham, we thus see the denial of a distinction between different \textit{ways} of following, relative to different ontological relations, and with it a co-opting of the topical vocabulary standardly used to express such a distinction. Given that Ockham identifies formal consequences with those reducible to formal consequences in Buridan's sense, it is only a natural step from there to Buridan's understanding of formal consequence; and with it, to the identification of the division between simple and as-of-now consequence as one subordinate to material consequence, between two different \textit{ways} of reducing a material consequence to a formal one, i.e. by the addition of a necessary or contingent premise. Lastly, in Buridan, we see even the vestigial diminished status Ockham affords to \textit{ex impossibili quodlibet} erased: for Buridan, the consequence is good not by appeal to an extrinsic topic, not by its terms, but simply by the definition of a good consequence.

\section{There and back again}
The triumph of Buridan's definition of formal consequence was the cumulative result of rapid changes in the theory of topical argument from the late thirteenth to mid-fourteenth century. At the same time, the unification initially wrought by the medieval subsumption of syllogistic and topical argument under the banner of consequence was intimately tied to the development of the notion of formal consequence as consequence \textit{par excellence}, that to which any consequence worth its salt would be reducible. 


What was this a triumph \textit{for}? Doing logic without appealing to qualitative ontological distinctions, the norm in the topical framework consequences grew out of and eventually replaced. What was it a triumph \textit{over}? Attempts to ground logic, however loosely, in ontology. 

What was \textit{sacrificed} in this achievement? Logic's relevance; its multifacetedness; its groundedness.

The loss of relevance occurs in two senses. The first, in the introduction of various irrelevant entailments such as \textit{ex impossibili quodlibet}, and their gradual movement from the periphery to the center of the understanding of consequence. 

The second was the loss of relevance to the process of \textit{discovery}. The streamlining of the maximal propositions actually appealed to in logical disputation, wrought via the introduction and gradual improvement of the theory of supposition, was a remarkable achievement. But as with their modern analogues in metatheoretical appeals to set-theoretic containment, these appeals do a much poorer job of providing rules for the development of arguments than the qualitative topical arguments they replaced: for instance, the rule \textit{from the first to the last}, i.e. transitivity, is simply much thinner, and accordingly less useful, than a rule like \textit{from the positing of the effect of an efficient cause follows the positing of the cause}. In this way, the development of the medieval theory of consequence served as the immediate backdrop for the various early modern attempts, such as those of Bacon and Descartes, to develop a method for discovery; for the location of such endeavors, given the tarnished name of `logic', under the banner of not logic, but epistemology; and for the development of rival approaches, from which arises the distinction principally between rationalists and empiricists, but many others as well. Without the transformation of the topical theory into the more supposition-oriented theory of consequences, characteristic early modern complaints about logic's uselessness and inapplicability for discovery would not have found footing. 

The loss of multifacetedness occurs through the replacement of a distinction between two different ontologically grounded ways of following with a purely epistemic distinction. The earlier account is based on two different ways in which something may be predicated of another, i.e. intrinsically or accidentally, and thus has a basis in Aristotle's \textit{Categories}, and is reflected in the distinction between analytics (which, as the name implies, is concerned with breaking down intrinsic components) and dialectic. The account of consequence Buridan advances is a reductively monistic one: all good consequences are reducible to formal ones. Here, one is not far from the identification of consequence with formal consequence as such, and from the identification of the latter with logical consequence. On the one hand, this brings with it a great unification of the previously disparate realms of analytics and dialectic. On the other, its doing so brought with it a forgetfulness of the question of what the different spheres of logic are concerned with, in favor of an understanding of logic as applicable to anything whatsoever.\footnote{Only with Kant's discovery of the synthetic \textit{a priori} do we arrive at anything like a rehabilitation of the traditional domain of topical argument. However, Kant limits the domain of such arguments to bar them any role in metaphysics; and the distinction was strongly rejected by the logicist tradition at the start of early analytic philosophy. Though with \cite{Kripke1980}, the analytic tradition has found a role even for \textit{a posteriori} necessity, it has yet to appropriate, or even to grapple much with, Kant's earlier distinction, let alone to examine its affinity with the basic idea behind Boethian topical argument.}

The loss of groundedness corresponds to this loss in multifacetedness. Where the earlier approach to consequences attempted a basic grasp of logic's nature, subject, and divisions, the later approach to consequences eschewed these concerns in favor of a more conventionalist approach to logical practice. Rules for consequences become those rules which are obvious to everybody, or which are derived from such rules in acceptable ways.

It is against the backdrop of this loss that the crisis in contemporary logic is given its sense. After the collapse of the formalist program and the fading out of the logicist program in the second quarter of the twentieth century, the ensuing widespread adoption of first-order classical logic was never provided with a solid, more than dogmatic foundation. From there, that foundation has been assailed with attacks on \textit{reductio ad absurdum} proof by intuitionism; on \textit{ex falso quodlibet} by the Lewis systems for strict implication; on non-relevant implication by the relevant logics of Belnap and others; on bivalence and excluded middle by many-valued systems and on non-contradiction by paraconsistent systems. Regardless of the strength of the objections, the foundation was weak enough that when the opposition circled about the walls and shouted their objections, the walls of the dogmatic consensus came crumbling down. 

However, the loss of this consensus has not brought with it a serious attempt to understand what formal consequence is, what it is based on, or whether it constitutes a unified thing. Rather, the \textit{de facto} successor to the dominance of the first-order, semantic approach to formal consequence has been the proliferation of a multitude of logics, employed seemingly at whim. As \cite{Mehlberg1960} foresaw, a thousand Carnapian flowers are now blooming in the field of mathematical logic. But the word sown by that Austrian logician is now fallen on rocky ground. Immediately a thousand systems spring up; but since they have no root, they wither away in obscurity and disuse. 

In this period of the proliferation of logics we have now entered, we see the continuation of trends present already in the logic of Buridan and his nominalist followers: on the one hand, the proliferation of a vast array of technical results; on the other, a loss of any real sense of what one is doing when one is doing logic. Where it falls in the literature of the field, work of philosophical depth requiring greater cultivation and care is often choked out by technical results.

Where does this leave us now? 

With much work to do, yes. But with the field ripe for that work. A historical genealogy of the concept of formal consequence serves as a blueprint for a bridge to a better footing for formal logic. From here, the growing acceptance of relevant logics and other deviant logics provides an opportunity to revisit non-reductive approaches to `following', as well as to examine its various kinds and senses. The growth of interest in logical pluralism provides an opportunity for revisiting and improving the real basis on which Burley and earlier logicians grounded early divisions of consequences. Metaphysical work on grounding provides an opportunity to enter into serious discussions concerning on the grounds of consequence.

Surely, let a thousand flowers bloom. But let the seed fall on good soil and bring forth good fruit. Thirty, sixty, a hundredfold.


%In Ockham, by contrast, \textit{all} consequences are said to hold by an extrinsic topic, save those with necessary consequents or impossible antecedents; while those holding by an intrinsic topic are either of the latter kind or enthymemes. The changes wrought  in this shift are as follows: 1) \textit{ex impossibili quodlibet} and \textit{necessarium a quolibet} are no longer held to be justified by extrinsic topics, but strictly by their terms; 2) those accidental consequences Burley classifies as holding by their terms disappear; 3) All consequences which Burley says hold formally or naturally hold by an extrinsic topic in Ockham; and 4) among Burley's formal consequences, only enthymematic consequences hold by an intrinsic topic for Ockham; 5) the division of formal and material consequences is not presented as a subdivision of simple consequence, as that between natural and accidental consequence is in Burley;\footnote{The Pseudo-Ockham \textit{Elementarium Logicae} is more explicit on this, stating that formal consequences may be \textit{ut nunc}. \cite[pp. 339-340]{Boehner1958}.}

%Burley's recognition of the complications caused by existential import, however, appears to have helped pave the way for the erasure of the distinction between natural and accidental consequence. This distinction was grounded on that between \textit{per se} and \textit{per accidens} ascent and descent, which built on an earlier distinction between \textit{per se} and \textit{per accidens} predication, itself grounded in the different modes of being. Thus, whether a consequence held or failed to hold was grounded in the ontological relations embedded in the suppositional ascents and descents it carried out \cite[p. 75.28-30]{BurleyDPAL}. But once one accepts that sentences like `Caesar is Caesar' may be falsified under certain empirical conditions (i.e. when Caesar is dead); and once one recognizes, as Burley does, that \textit{per se} descent may fail for essentially the same reasons \textit{per accidens} descent does (e.g. where the class or object descended to does not exist); it appears that both types of consequences end up invalid for fundamentally the same reasons, having to do with existential import, rather than ontological relations. As such, Burley's pluralism about consequences gives way to a uniform approach to suppositional descent grounded in existential import - one Burley suggests but does not follow - which in turn contributes to the formation of a purely monistic approach to simple consequence: one where all simple consequences are either good or not, with the evidence of their goodness being their reducibility to structurally formal consequences.


%For Ockham, structurally formal consequences are those which hold by an extrinsic rule; those formal consequences holding by an intrinsic medium are enthymemes; material consequences hold `strictly by reason of the terms, and not by reason of any extrinsic medium strictly pertaining to the general conditions of the proposition'

%different ways of `following': rule 70 as pluralistic For Buridan, `follows' is univocal. The distinctions among consequences (TC Bk. III)are, for Buridan, not ontological, but epistemic and syntactic.

%In Buridan, Ampliation appears as a way of getting rid of different readings of the same sentence, e.g. formal vs. material. This is a more radical application of Ockham's program than Ockham himself provides. See the Ockham-Quine quote. 

%In the \textit{de consequentiis}, Burley states that consequences holding by virtue of their terms `have to be reduced to a syllogism'. Same test found in Pseudo-Scotus for a material consequence. 

%Is natural consequence a subdivision of accidental consequence? Probably not. Only if everything related through an intrinsic topic is also related through an extrinsic one. Ockham embraces this. This opens the way for Buridan's approach to formal and material consequence grounded in a single notion that assumes explosion in its definition of consequence.

%For Buridan, only formed consequences count. For Ockham and Burley, types.

%Where Burley's distinction is presented as a subdivision of simple consequence, Ockham's is merely presented alongside that between simple and as-of-now consequence. By the time of Buridan's treatise, the dependency between simple and natural/formal consequence is reversed, with Buridan presenting simple and as-of-now consequence as subdivisions of material consequence. 

%For Ockham, the difference in terminology makes a difference (see Chris Martin)

%W. r. t. Buridan, the inability to accomodate relevance. The shift towards evidence. Burley does not countenance irreducibly material consequences, though Buridan does.



%relation between divided/composite and material/formal.

% tB, p. 212.1-28 parallels 80.19-23. the objection to 2.1. , par. 85, esp. remarks at `non tamen...'; 118 - all consequences reduced to syllogisms; different kinds of formal consequences.

%Form: 84 (= rule 2.2), 85 (= rule 2.1),  168
%materia: 70, 168
%80.13-29, 
%84.8-86.21: with 39.20, par. 106, 118, 168.

%40, 159, 
%already used: DPAL: pp. 25.21, DC: 75, 
%comparison with Ockham: 1) Ockham's approach to as-of-now consequence is clearly an irrelevant one; it is unclear whether Burley's is. I think it unlikely.

%Ockham's approach appears to derive from William of Sherwood; Burley's, from Scotus.

%For Ockham, every formal consequence holds by an extrinsic topic; some hold by an intrinsic one as well. Does every consequence hold by an extrinsic topic for Burley as well?

%\section{Appendix 1: The \textit{tractatus longior}'s rules for consequences}
%\begin{enumerate}
%	\item In every good simple consequence, the antecedent cannot be true without the consequent; in an as-of-now consequence, the antecedent cannot now be true without the consequent.
%	\begin{enumerate}
%		\item[1.] In a simple consequence, the impossible does not follow from the contingent.
%		\item[2.] The contingent does not follow from the necessary.
%	\end{enumerate}
%	\item Whatever follows from the consequent follows from the antecedent; Whatever entails the antecedent entails the consequent.
%	\begin{enumerate}
%		\item[1.] Whatever follows from the antecedent and consequent follows from the antecedent per se.
%\item[2.] Whatever follows from the consequent with something added follows from the antecedent with the same added.
%	\end{enumerate}
%	\item Whatever conflicts with the consequent conflicts with the antecedent.
%\item Whatever stands with the antecedent stands with the consequent.
%	\begin{enumerate}
%		\item[1.] If the consequences of certain propositions conflict, those propositions conflict with each other.
%		\item[2.] If antecedents stand together, so do their consequents.
%		\item[3.] In every good consequence the opposite of the consequent conflicts with the antecedent.
%	\end{enumerate}
%	\item Whenever a consequent follows from an antecedent, the opposite of the antecedent follows from the contradictory opposite of the consequent.
%	\begin{enumerate}
%		\item[1.] Whatever follows from the opposite of the antecedent follows from the opposite of the consequent.
%		\item[2.] Whatever entails the opposite of the consequent entails the opposite of the antecedent. 
%	\end{enumerate}
%\end{enumerate}

%Reasons for not attributing the TB to Burley: 1) The TB does not discuss natural vs. accidental consequences, 2) the TB's discussion of \textit{ex impossibili quodlibet} and \textit{necessarium a quolibet} is more sophisticated, with more qualifications. 3) The treatment of syllogisms differs completely from that of either the TL or the DC, 4) The TB contains a discussion of modal syllogisms, while the TL does not. If the TL were later, one would expect this to be included. 5) nothing follows from negatives: affirmed at 92.12, 151.15, and 151.30. Denied at 213.1-217.5; affirmed of syllogisms at 219.23-24; implicitly denied at par. 80. 6) TL 85.4-5 holds an example invalid that is held valid at TB 211.27-28. 7) The TB's frequent allusions to `my opinion' suggest a contrast with someone else - i.e. an authority from which the text was compiled, viz. the authentic Burley. 8) Burley's TL is better organized, but TB is more sophisticated in its content. If both were by the same author, we would expect uniform improvement of both form and content. The more sophisticated content of the TB suggests a later date; its less clear organization, a different author. 
