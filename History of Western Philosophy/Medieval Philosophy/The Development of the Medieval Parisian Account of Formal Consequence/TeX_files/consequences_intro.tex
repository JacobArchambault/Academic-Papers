\documentclass[]{article}
\usepackage[backend=biber, style=verbose-ibid, firstinits=true]{biblatex}
\usepackage{amssymb}
\addbibresource{jacob.bib}
\DeclareBibliographyCategory{shorthand}
%opening
\title{Consequences in medieval logic: an introduction}
\author{Jacob Archambault}

\begin{document}

\maketitle

\begin{abstract}
This article introduces consequences in medieval logic, summarizing medieval definitions and divisions of consequences and explaining the import of the medieval development of the theory of consequence for logic today. I then introduce the various contributions to this special issue of \textit{Vivarium} on consequences in medieval logic.
\end{abstract}

\section{Introduction}
Logic is commonly said to be about what follows from what.\footnote{Cf. \autocite[156]{Kneale1948}; \autocite[74]{Etchemendy1988}; \autocite[99]{Read1995}; \autocite[3]{BeallRestall2006}.} Within this common expression lies an understanding of logic with a concept of following, inference, or consequence, at its center. Consequences have arguably existed in logic from its beginning: Aristotle's \textit{Topics} classifies various kinds of inferences employed in debate, and his assertoric syllogistic constitutes a well-defined class of formally valid consequences with certain distinctive properties. Yet for all this, ancient logicians arguably lacked a unified theory of what we would call `consequence'. Rather, one finds certain one-premise inferences treated in connection with the \textit{Peri Hermeneias}, other, two-premise inferences in syllogistic, and other inferences still in discussions of hypothetical syllogisms, dialectical topics, and elsewhere. One first finds consequences \textit{thematized} as such in the later middle ages. 

Early discussion of consequences appears in school texts from the later twelfth and early thirteenth century, in commentaries on Aristotle and Boethius, in discussions of fallacies, and in treatises on \textit{syncategoremata} - roughly analogous to what are today called logical constants. The first treatises explicitly devoted to consequences then appear at the turn of the fourteenth century, with discussions of the topic becoming more numerous and more sophisticated from the 1320s onward.

\section{Medieval and modern definitions of consequence}
In common English, `consequence' usually refers to the result or outcome of an action, `inference', to a subject's act of asserting or coming to believe something on the basis of something else, and `implication', to a suggestion communicated in a veiled manner through something else stated explicitly. In logicians' English, these terms are naturally used interchangeably to refer to none of these things. In logic today, `consequence' `inference' and `implication' refer to an ordered pair whose first element, called the \textit{antecedent}, is usually a set (or multiset, or list)\footnote{A set is a grouping of elements without respect to their order or repetition. \{A, B\} and \{B, A\} and \{A, A, B\} all name the same set. A list respects both the order of elements and the number of times an element occurs. Hence, $\langle A, B\rangle, \langle B, A \rangle$ and $\langle A, A, B \rangle$ are all different lists. Multisets are like sets but respect the number of times a given element occurs. Hence, though [A, B], and [B, A], and [A, A, B] all name the same set, only the first two refer to the same multiset. See \autocite[300]{Ripley2015}.} of sentences, propositions, or even arguments \autocite{Garson2013}, and whose second element, called the \textit{consequent}, is a single object of the same type. A consequent is said to \textit{follow from} its antecedent, and an antecedent is said to \textit{entail} its consequent.\footnote{Both consequence and inference have been suggested as appropriate translations for the Latin \textit{consequentia}. See \autocite{King2001} and \autocite{DutilhNovaes2005} for discussion.}

In medieval logic, `consequence' (\textit{consequentia}) usually refers to a relation between an antecedent and a consequent, variously described as a habit (\textit{habitudo}), inference (\textit{illatio}), or a following (\textit{sequela}).\footnote{\autocite[I, par. 1, p. 4]{Green-Pedersen1980a}: `Consequentia est habitudo inter antecedens et consequens. Antecedens est illud ad quod sequitur aliud. Consequens est illud quod sequitur ex alio'. \autocite[1.1.01, p. 1]{StrodeConsequentiis}: `Consequentia dicitur illatio consequentis ex antecedente'. Rodolphus Anglicus: `Consequentia est quaedam habitudo vel sequela in qua consequens se habet ad antecedens', in \autocite[p. 306]{Green-Pedersen1983}.} Some medieval logicians define a consequence according to its part of speech,\footnote{\autocite[I, c. 3, p. 22.60-62]{BuridanTC}: `Consequentia est propositio hypothetica ex antecedente et consequente composita, designans antecedens esse antecedens et consequens esse consequens'. \autocite[I. q. 10, pp. 104-105]{Pseudo-Scotus1891}: `Consequentia est propositio hypothetica, composita ex antecedente, et consequente, mediante conjunctione conditionali, vel rationali, quae denotat, quod impossibile est ipsis, scilicet antecedente, et consequente simul formatis, quod antecedens sit verum, et consequens falsum'.} others in terms of its function,\footnote{\autocite[A.I, par. 2, p. 92]{Green-Pedersen1982}: `Circa definitionem nota quod consequentia est argumentatio composita ex antedente et consequente. `Argumentatio' ponitur in definitione consequentiae, quia omnis consequentia sumitur ad aliquod argumentum producendum. `Composita' dicitur, quia nullum incomplexum est consequentia. `Ex antecedente et consequente' additur, quia in omni consequentia adminus requiruntur duae propositiones categoricae'.} and still others, seemingly throwing up their hands, regard it as a clustering of its various parts.\footnote{See no. 6 in \autocite[p. 300]{Green-Pedersen1983}, : `Consequentia est quoddam aggregatum ex antecedente et consequente ad idem consequens cum nota consequentiae. Et sunt notae consequentiae `ergo', `ideo', `quia', `igitur', `idcirco''. Also nos. 7, 9, and 15, \autocite[300-306]{Green-Pedersen1983}.} Others pass over the definition of consequence altogether and begin by listing good consequences or divisions of consequences.\footnote{\autocite[II, par. 1, p. 11]{Green-Pedersen1980a}: `In omni consequentia bona quicquid sequitur ad consequens sequitur ad antecedens; ut sequitur `Socrates currit, ergo animal currit' et sequitur `animal currit, ergo substantia currit'; ergo a primo ad ultimum sequitur `Socrates currit, ergo substantia currit'. \autocite[III-3, c. 1, 587.4-9]{OckhamSL}: `Habito de syllogismo in communi et de syllogismo demonstrativo, agendum est de argumentis et consequentiis quae non servant formam syllogisticam. Et primo ponam aliquas distinctiones quae sunt communes aliis consequentiis multis, quamvis non sint enthymemata, ex quibus omnibus faciliter patere poterit studioso quid de omnibus syllogismis non demonstrativis est tenendum'. Cf. \autocite[262]{Pozzi1978}.}

Modern logicians make a strong distinction between consequences and conditionals: a conditional is a connective appearing in formulae within a regimented language, called the \textit{object language}, employed in a proof system. Consequence is a relation of following asserted to hold between [schemata for] formulae in the object language, and whose written expression does not appear in the object language but in a second, more expressive language called the \textit{meta-language} (usually a natural language augmented with various mathematical symbols) which is used to evaluate the expressions of the object language. 

Medieval logicians do not strongly distinguish consequences from conditionals, and certainly do not do so in the above way. Humanist scruples aside, medieval logicians worked within natural language. In accord with the range of source material from which it arises, the medieval concept of consequence comes to include conditional statements, categorical and hypothetical syllogisms, conversions, enthymemes, and other argument forms.\footnote{\autocite[A.I, par. 3, p. 92]{Green-Pedersen1982}: `Etiam ex ista definitione sequitur quod omnis argumentatio generaliter potest vocari consequentia, sive sit syllogistica sive inductiva sive exemplaris sive enthymematica'. \autocite[V. c. 1, p. 31.4-5]{OckhamTML}: `Sic syllogismus et inductio, conversio et multi alii modi considerandi sunt consequentiae formales'. \autocite[I. q. 20, p. 130]{Pseudo-Scotus1891}: `Notandum est, quod quaedam est consequentia enthymematica, et quaedam syllogistica'.}

None of this yet tells us what \textit{counts} as a good consequence, either for medieval logicians or their modern counterparts. Today, the two most common ways of providing criteria for determining when a consequence exists are one, relying on the techniques of proof theory and called \textit{proof-theoretic}, and the other relying on model theory, called \textit{model-theoretic} or \textit{semantic}.\footnote{Cf. \autocite{Tarski2002}, \autocite{Prawitz1974}.} 

In the semantic approach to consequence, a consequence from a premise set $\Gamma$ to a conclusion $\phi$ - written $\Gamma \vDash \phi$ - is valid if and only if every model of $\Gamma$ is, at the same time, a model of $\phi$. In early model theory, e.g. that of Tarski, a model $M$ of a sentence $\phi$ [set of sentences $\Gamma$] in a recursively-defined language $L$ presupposes a division of the basic elements of $L$ into logical and non-logical kinds, and is a sequence of objects \textit{satisfying} (roughly, making true) each \textit{sentential function} obtained by uniformly replacing each non-logical element in the sentence $\phi$ [set of sentences $\Gamma$] with a variable - like variables replacing like constants, unlike replacing unlike. In classical model theory today, a model $M$ is a pair $\langle D,  I \rangle$ consisting of a (possibly infinite, possibly empty) set of objects $D$, called the \textit{domain}, and an interpretation $I$ that assigns non-logical constants in $L$ to elements in $D$, and thereby provides the basis for recursively determining the truth value on $M$ of each sentence $\phi$ in $L$. Modal, non-classical, and other model theoretic approaches to consequence generally arise by expanding the number or adjusting the interpretation of the logical constants of a language, and/or by modifying the notion of a model in interesting ways e.g. by the addition of Kripke frames in modal logic, or of further truth values in many-valued logics.

In modern proof-theoretic approaches to consequence, a consequence $\Gamma \vdash_{N} \phi$ occurs from premises $\Gamma$ to conclusion $\phi$ in a proof system $N$ if and only if there exists a derivation of $\phi$ in $N$ from (open) assumptions $\Gamma$.\footnote{The definition is taken from \autocite[152]{Francez2017}.} Here, a proof system consists of a set of rules (and possibly, axioms) for obtaining certain formulae of a language $L$ from others,\footnote{Or, in the case of sequent calculi, arguments from arguments. The array of proof systems in logic today is vast, and the above description only captures a fraction of them.} and commonly consists of rules for introducing and eliminating logical connectives, along with structural rules governing matters like the introduction of assumptions and repetition of formulae. The definition of an open or closed assumption, and the corresponding notion of an open or closed argument, are given in a manner parallel to the treatment variables and formulae in syntax: just as a variable $x$ [formula of a language $\phi$] is said to be bound [closed] just if it occurs within the scope of a quantifier $Qx$ [all of its variables are bound], and free/unbound [open] otherwise, an assumption [argument] is said to be bound - that is, discharged - just if it occurs within the `scope' of an inference [all of its assumptions are bound].\footnote{The definition given here is a simplification which leaves aside problems pertaining to the normal form of inferences. For fuller discussion, see \cite{Prawitz1985}.} Consequently, the basic idea behind modern proof-theoretic approaches to consequence is that a consequence exists when, given a certain set of premises as inputs, there is a precise, rule-governed procedure for obtaining the consequent as output. The proof-theoretic approach to consequence traces its origins back to David Hilbert's work on mathematical proof and Gerhard Gentzen's on natural deduction, and has since been heavily influenced by the contributions of Dag Prawitz, Michael Dummett, and others \autocite{Gentzen1934,Prawitz1974,Dummett1991,Schroeder-Heister2006,Franks2010}.

Medieval ways of explicating consequence have parallels to both of these approaches. Instead of models, some medieval accounts of consequence rely on a notion of \textit{causes of truth}, which shares similarities with the modern notion of a truthmaker.\footnote{\autocite[I. c. 2, pp. 19-20]{BuridanTC}. An early application of model-theoretic consequence has recently been traced to the 12th century Arabic logician Ab\={u} al-Barak\={a}t. See \autocite{Hodges2018}.} And instead of rule-governed proof systems, several medieval logicians appeal to inference-licensing rules called \textit{maximal propositions} which arise out of discussions of topical argument \autocite[p. 76.5-7]{BurleyDPAL}. 

But what likely strikes the non-logician about these accounts is their abstraction from the actual content of an inference \autocite{DutilhNovaes2011}. Since Tarski's advances in model theory, for instance, classical models for standard formal representations of `Socrates runs' have also served to model `Plato jumps'.\footnote{Should this entail that `Plato jumps' follows from `Socrates runs'? No: though every model of the former is a model of the latter, it is not so `at the same time', and hence fails the semantic criterion for following.} And since Hilbert's advances in geometry, proof systems have been constructed with the intent of making the interpretation of the symbols occurring in them a matter of indifference. In accord with this characteristic, modern theories of consequence tend to focus almost entirely on formally valid inference.

Medieval approaches to consequence generally lack this feature. Rather, the earliest medieval criteria for consequence state that a consequence is good, true, valid, or holds when it is impossibile for the antecedent to be true and the consequent false, and later criteria for consequence generally consist of more sophisticated variations on this same theme - to avoid problems with self-falsifying propositions, for instance, several authors state a version on which a consequence is good when things cannot be \textit{as the antecedent signifies} without being as the consequent signifies.\footnote{\autocite[I, c. 3, pp. 20-22]{BuridanTC}, \autocite[I, q. 10, pp. 103-105]{Pseudo-Scotus1891}.} Another popular criterion, now called the `containment criterion' and later used in characterizations of \textit{formal} consequence, holds that a consequence is good when [the understanding of] the consequent is `contained' in [the understanding of] the antecedent \autocite[pp. 283-284]{AbelardDialectica}. Some authors appeal to versions of both criteria, while others use one to the exclusion of the other. Both continue to be regarded as basic, intuitive criteria for consequence and to serve as the basis for its modern formalizations.\footnote{\autocite[esp. pp. 218-219]{Saguillo1997}, \autocite{Fine2017a,Halbach2018,Prawitz2018}.}

\section{Medieval divisions of consequence}
In earlier medieval writings on logic, two divisions of consequence predominate: one, derived from Boethius' treatment of hypothetical syllogisms, between natural and accidental consequences \autocite[I, c. 3, par. 6, p. 219]{BHS}; the other, apparently stemming from the twelfth century \textit{Ars Meliduna}, between simple and as-of-now consequences.\footnote{See Christopher J. Martin, `The Theory of Natural Consequence', this issue.} A simple consequence is one where the antecedent cannot ever be true without the consequent; an as-of-now, one where the antecedent cannot be true at a specified time without the consequent being so. A natural consequence is one where there is a natural or causal relationship between what the antecedent and consequent express; an accidental, one where there is a mere relationship of temporal accompaniment between these \autocite[pp. 60.28-61.25]{BurleyDPAL}. In this earlier division, natural and accidental consequences come to be considered types of simple consequences. In later writers, the distinction between natural and accidental consequence is supplanted by one between formal and material consequences, and simple and as-of-now consequences come to be regarded as types of material consequences. In both the earlier divisions and the later one into formal and material consequence, the former type is more intensionally basic and extensionally restricted than the latter, with consequences of the second type somehow falling short of, or even being reducible to, those of the first.\footnote{\autocite[152]{SherwoodNewSyncategoremata}, \autocite[I, q. 10, p. 105]{Pseudo-Scotus1891}. Cf. \autocite[VIII, par. 117, p. 142]{Green-Pedersen1980b}.}

%\footnote{In recent secondary literature, the containment criterion has been associated with a characteristically `British' approach to consequences, while the impossibility criterion has been more closely associated with continental authors. This division can be accepted with some qualification: several early fourteenth century authors, including Ockham and Burley, appeal to both criteria, and a version of the containment criterion later became very popular among Italian logicians.}

%Consequentia necessaria/probabilis - Summa Lamberti
%Other divisions also exist - Peter of Spain divides consequences into simple and composite, and further subdivides composite consequences, while the \textit{Summa Lamberti} proposes a division into \textit{consequentia necessaria/naturalis} and \textit{probabilis/ut in pluribus} - but these 
\subsection{Earlier divisions of consequence}
\subsubsection{Natural and accidental consequence}
Boethius mentions the distinction between natural and accidental consequence when discussing the syncategorematic terms \textit{si} and \textit{cum} in his treatment of hypothetical syllogisms. There, Boethius gives an example of following accidentally (`if fire is hot, the heavens are spherical'), and two examples of following by nature.\footnote{\autocite[I, c. 3, par. 6-7, pp. 219-220]{BHS}: `Sed quoniam dictum est idem significare `si' coniunctionem et `cum' quando in hypotheticis propositionibus ponitur, duobus modis conditionales fieri possunt: uno secundum accidens, altero ut habeant aliquam naturae consequentiam. Secundum accidens hoc modo, ut cum dicimus: `Cum ignis calidus sit, caelum rotundum est'. Non enim quia ignis calidus est, caelum rotundum est sed id haec propositio designat, quia quo tempore ignis calidus est, eodem tempore caelum quoque rotundum est. Sunt autem aliae quae habent ad se consequentiam naturae; harum quoque duplex modus est, unus cum necesse est consequi, ea tamen ipsa consequentia non per terminorum positionem fit; alius uero cum fit consequentia per terminorum positionem. Ac prioris quidem modi exemplum est, ut ita dicamus: `Cum homo sit, animal est' non enim idcirco animal est quia homo est, sed fortasse a genere principium ducitur, magisque essentiae causa ex uniuersalibus trahi potest, ut idcirco sit homo quia animal est. Causa enim speciei genus est. At qui dicit: `Cum homo sit, animal est', rectam ac necessariam consequentiam facit, per terminorum uero positionem talis consequentia non procedit. Sunt autem aliae hypotheticae propositiones in quibus et consequentia necessaria reperitur, et ipsius consequentiae causam terminorum positio facit, hoc modo: `Si terrae fuerit obiectus, defectio lunae consequitur'. Hic enim consequentia rata est, et idcirco defectio lunae consequitur, quia terrae interuenit obiectus'.} According to his discussion there, an accidental consequence is one where though the antecedent does not occur without the consequent, it doesn't explain it, either. Consequences of nature, by contrast, exhibit a proper cause-effect relationship: Boethius suggests his first example  (`if a man is, an animal is') moves from effect to cause, while the second (`if the earth is obstructing [it], an eclipse of the moon follows') moves from cause to effect. Boethius also tells us that while the following is necessary in both examples, in the first example the consequence itself does not occur through the positing of the terms, while in the second it does. What Boethius means by this suggests a number of confusions, which are nevertheless extremely illuminating. In his remarks on the kind that does posit its terms, he apparently means to suggest by this that this kind of consequence holds exactly when the thing described by its parts, e.g. an eclipse, is actually occurring in the world. These kinds of consequences are thus `existence implying' with respect to their parts. The other kind, which includes the inference from a species to its genus which becomes the classic example of a consequence holding by containment, does not set forth the things described by its terms in the same way, but apparently holds in a manner indifferent to the existence of e.g. actual humans. Though Boethius' account is at odds here with modern intuitions concerning both when a consequence holds and when it implies the existence of its subjects, one finds a similar approach to existential import in consequences in the earliest fourteenth century treatises.\footnote{See Jacob Archambault, `Consequence and Formality in the Logic of Walter Burley', this issue.}

Accidental consequences are called temporal consequences by some authors, and non-natural or unnatural by others.\footnote{Cf. \autocite[141]{Garlandus1959}, \autocite[p. 472]{AbelardDialectica}, \autocite[152]{SherwoodNewSyncategoremata} \autocite[vol. 2, pp. 199-200]{Braakhuis1979}.} A clear statement of the division between natural and accidental consequences is found in Robert Kilwardby in the mid-thirteenth century, and at the turn of the fourteenth Duns Scotus and Walter Burley ground the division between these in one between intrinsic and extrinsic topics.\footnote{\autocite[Part 2, 1, 55, p. 1140.79-83]{KilwardbyPriorum}, \autocite[I, d. 11, q. 2, pp. 136-137]{ScotusLectura}, \autocite[pp. 60.26-61.15]{BurleyDPAL}.} This topical grounding is especially important from the time of the first treatises on consequences through the first half of the fourteenth century.

Following the Aristotelian commentator Themistius, Boethius had divided topics into intrinsic, extrinsic and middle topics, in accordance with the channels, or \textit{middles}, they employ. In an intrinsic topic, the argument proceeds by eliciting some property, description, or relation belonging to what is signified in the minor term to confirm or remove what is suggested of it by a major term. To prove that whiteness is not a substance, for instance, Boethius argues as follows: A substance is what can be the subject of accidents; but whiteness cannot be the subject of accidents; so whiteness isn't a substance. The topic appealed to is that \textit{from a description}, since the description serves as the medium - in this case, the middle term - whereby the question is determined. The maximal proposition, or rule appealed to, is \textit{that the description of which does not belong to a species is not the genus of that species}.\footnote{\autocite[II, pp. 1187C-D]{BDT}. A more famous use of the same topic is found in Anselm's `ontological' argument, where one proceeds from God to his existence via the description `that than which nothing greater can be thought' and the maximal proposition `what is predicated of the description is predicated of its bearer'. \autocite{Holopainen2007}.} In an extrinsic topic, by contrast, one exploits a relation that the subject term of the desired conclusion bears to some other concept, constructs an argument about the latter, and thereafter leverages the original relation to infer something about the original subject one was inquiring about. For instance, to determine whether being two-footed is proper to human beings, Boethius offers the following argument: being two-footed belongs to human beings in the same way as being four-footed belongs to a horse; but being four-footed is not proper to a horse; neither, then, is two-footedness proper to humans. The question here concerns a proper accident, i.e. one belonging to a thing by virtue of its specific nature; the maximal proposition is `if what inheres in a similar way does not do so properly, neither does what one is inquiring about inhere properly'. The topic is that from a likeness, or similitude \autocite[II, pp. 1190C-D]{BDT}. Other types of extrinsic topics include arguments from authority, from analogy, from various kinds of opposites, and \textit{a fortiori}/\textit{a minori} arguments.\footnote{By the later 13th century, little emphasis is placed on middle topics. Radulphus Brito's commentary on Boethius' \textit{De Differentiis Topicis} (c. 1300), for instance, only dedicates two of its twenty three questions to middle topics, with one of these devoted to defending the category as a sensible one \autocite[II, qq. 22-23, pp. 78-84]{BritoDDT}. For further discussion of topics in Boethius, see Bianca Bosman, `The Roots of the Notion of Containment in Theories of Consequence: Boethius on Topics, Containment, and Consequences', this issue.}

By the time the earliest treatises on consequences appear at the turn of the fourteenth century, the use of topics in discussions of consequence has been fundamentally altered by the development of \textit{supposition theory} - roughly analogous to modern theories of reference, but better understood as a theory governing the interpretation of terms in propositional contexts \autocite{DutilhNovaes2007,DutilhNovaes2008b}. This shift brought about a simplification in the number of topics actually appealed to, with a vast number of inferences justified by rules like \textit{from an undistributed inferior to superior}, and \textit{from a distributed superior to a distributed inferior}. In brief, appeals to qualitative distinctions grounded in aspects of objects give way to appeals to quantitative distinctions grounded in the supposition of terms \autocite{Stump1982}.
%Explain distribution
\subsubsection{Simple and as-of-now consequence}
In his discussion of following by nature in the \textit{De hypotheticis syllogismis}, Boethius' intent seems to have been to distinguish consequences that genuinely follow \textit{from} their premises from those that don't. But some later versions suggest a different reading of the distinction. The \textit{Logica Lamberti}, composed around 1250, contrasts natural or necessary consequence - note the identification - with a \textit{consequentia probabilis vel consequentia ut in pluribus}.\footnote{\autocite[VII, pp. 195-196]{SummaLamberti}: `Sciendum autem quod duplex est consequentia: una scilicet in qua, posito antecedente, de necessitate ponitur consequens, et hoc potest dici naturalis vel necessaria; alia vero est consequentia in qua, posito antecedente, non propter hoc de necessitate ponitur consequens, sed ut frequentius concomitatur antecedens consequens et hoc potest dici consequentia probabilis vel consequentia ut in pluribus. Hoc autem habet fieri inter accidentia que se concomitantur in aliquo et raro se derelinquunt. 
	
In utraque istarum consequentiarum possunt fieri paralogismi iuxta primum modum istius figure secundum Aristotelem in primo \textit{Elenchorum}. In consequentia necessaria sic paralogizatur: `si mel est, rubeum est, sed hoc est rubeum' demonstrato felle, `ergo fel est mel'. Item alius paralogismus similis huic: `si terra est depluita, terra est madida, sed est madida, ergo est depluita'. [...] In consequentia probabili sic paralogizatur: `si aliquis est adulter, ipse est comptus, sed iste est comptus, ergo est adulter'. Item alius paralogismus similis huic: `si aliquis est fur, est errabundus de nocte, sed iste est errabundus de nocte, ergo est fur'. Reading `comptus' for `corruptus'. Cf. \autocite[V. p. 169.11-13]{PetrusTractatus}.} Furthermore, while Lambert's second example of natural consequence (`if the earth has been rained on, it's moist') still fits Boethius' description of a consequence moving from cause to effect, his first (`if it's honey, it's red') no longer exhibits a relation of intrinsic containment, but at best has to be read as moving from a substance to its necessary accident. 

The existence of this identification of natural or necessary consequence is already critiqued by the much earlier \textit{Intentio Boethii}. The author writes: 
\begin{quote}
	Natural and necessary consequence, according to some, are the same, and they take `necessary' for `indubitable'. And they make this division of the hypothetical proposition: that one is accidental, and the other natural, that is, necessary. This division is not sufficient. For there are some hypothetical propositions which are neither temporal nor necessary, as this, `if a crow is, a black [thing] is' is neither temporal nor necessary.\footnote{Orl\'{e}ans Bib. Municipale ms. 266, ff. 78-118a, f. 87a: `Naturalis vero consequentia et necessaria secundum quosdam sunt paria, et accipiunt necessarium <pro> indubitabili. Et faciunt hanc divisionem hypotheticae propositionis quod alia accidentalis, alia naturalis, id est necessaria. Quae divisio non est sufficiens. Quaedam enim hypotheticae propositiones sunt quae neque temporales sunt nec necessariae, ut ista `si corvus est, niger est' nec temporalis est nec necessaria.' The text is further discussed in Martin, `The Theory of Natural Consequence', this issue.}
\end{quote}
%why is the reading a misreading?
Here, the text distinguishes between two criteria for natural consequence - one based on temporal inseparability, the other based on indubitability - and shows via the example that some consequences are temporally necessary, but not intrinsic. The author is clearly drawing on Porphyry's \textit{Isagoge}, and transferring a division originally applied to predications to hypothetical propositions wherein those predications are found: predications can be either intrinsic, as those of genus, species, and difference, or extrinsic, i.e. accidental. Among extrinsic predications, some are inseparable, while others are not. We thus have two conflated readings of `natural consequence', highlighting two different conditions, one appealing to containment, indubitability, or intrinsicality, the other to temporal inseparability, which will later come to characterize different approaches to consequences in the fourteenth century. 

It is in light of these two readings of the natural/non-natural distinction that I think we should understand the development of the dichotomy between simple and as-of-now (\textit{ut nunc}) consequences, and the corresponding subordination of natural and accidental consequence to simple consequences. A simple consequence is necessary in the sense of being always good; an as-of-now consequence is contingent or accidental in the sense of being sometimes good. Of those that are always good, some are necessary in the sense that they intrinsically contain their consequents in their antecedents, while others are accidental in that they do not.

Some early notions stretching toward \textit{ut nunc} consequence, e.g. Lambert's \textit{consequentia probabilis}, suggest a kind of consequence whose holding is based on the presence of significant, albeit not absolute, statistical coincidence.\footnote{Cf. \autocite[q. 35, 196.48-80]{FavershamQE}.} But the settled notion of \textit{ut nunc consequence} comes to suggest a consequence holding by virtue of a suppressed premise true at the time the consequence is asserted to hold for, but not at every time. In his \textit{de consequentiis}, Walter Burley gives the example `You are in Rome, therefore that which is false is true' relying on the premise ``You are in Rome' is false' \autocite[V. par. 69, p. 128]{Green-Pedersen1980b}. And in the \textit{Summa Logicae}, William of Ockham's example is `every animal runs, therefore Socrates runs', relying on the premise `Socrates is an animal' \autocite[III-3, c. 1, pp. 587.11-588.18]{OckhamSL}.

In ancient and medieval logic, a proper division of a subject into its parts should be both exclusive and exhaustive. The division of consequences into simple and as-of-now varieties thus implies that no consequence can belong to both of these, and that every consequence belongs to one of them. Thus, every good consequence either holds at some time or at every time. Upon reflection, it is clear that good consequence cannot here mean what we mean by `valid' - validity is indifferent to time. Furthermore, `some time' here must mean `some, but not every time', since otherwise every simple consequence would also be a good as-of-now consequence, vitiating the exclusivity of the division.\footnote{Cf. \autocite[141.11-15]{RichardAbstractiones}.} Furthermore, the division implies that there is no kind of consequence that holds at \textit{no} time. 

The force of this last point is made clearer when we consider that some authors, e.g. Matthew of Orl\'{e}ans, also distinguish simple consequences from those that hold \textit{ex hypothesi} or \textit{per concessionem aliquam}, apparently meaning to include in this any consequence involving an impossibility \autocite[IV, 141-142, pp. 339-340]{MatthewSophistaria}. The implication here seems to be that nothing follows from an impossibility in reality: the impossible doesn't obtain, and consequently cannot be a cause of anything that would putatively follow from it. This early division of consequences thus suggests an understanding of consequence on which the place of impossibilities in the theory is unclear.\footnote{This is not to say that medieval logicians did not discuss such inferences. As is well-known, inferences involving impossibilities factor extensively into treatises on Sophisms, \textit{Insolubilia}, and \textit{Obligationes}. But, at least in earlier stages, these seem not to have been universally considered to belong to a theory of consequence.}

It is not yet clear exactly when this changes. The rule `from the impossible anything follows' was apparently first discovered by the logician William of Soissons in the 1150s, and was promulgated by the followers of Adam of the Little Bridge, called \textit{Adamiti} or \textit{Parvipontani}, in the 1170s \autocite{Martin1986,Martin2012}. William of Sherwood incorporates the rule into his theory of non-natural consequence \autocite[152]{SherwoodNewSyncategoremata}, and Henry of Ghent, who requires every good conditional to conform to the requirement for simple consequence, allows for inferences from impossibilities in certain restricted cases.\footnote{See \autocite{Spruyt1993}. Also, Joke Spruyt, `Consequence and `Cause': Thirteenth-century Reflections on the Nature of Consequences', this issue.} But by the turn of the fourteenth century, numerous cases of reasoning with impossibilities, in both simple natural and accidental consequences, are included in Burley's \textit{De consequentiis}, one of the three earliest extant tracts on consequences. Among these are straightforward examples of what we would classify as formally valid consequences, e.g. `every donkey is a man, therefore only man is a donkey' \autocite[II. par. 37, p. 119]{Green-Pedersen1980b}.



%Kirchoff, syncategoremata 152.

%Gradual incorporation of impossible into the treatment of simple consequence forces a rethinking of simple consequence.
%Bellucci 2016 gives the Boethius reference I need for DHS
%Could the simple/ut nunc distinction go back to the division between dialectical and rhetorical topics?

\subsection{The later division of consequences into formal and material}
\subsubsection{Material consequence}
Material consequence is understood in two ways, one earlier which is more common among British writers, and another associated with John Buridan and others on the European continent. The earlier criteria for material consequence parallel those for accidental consequence, and demarcate inferences from rules whose application is not completely general. These include inferences from an impossibility to anything or from anything to a necessity, as well as certain inferences where the interpretation of a term required in the antecedent licenses a term substitution in the consequent, e.g. `an animal debates, therefore a man debates'.\footnote{Cf. \autocite[VIII. par. 70, p. 128; X. par. 168, pp. 162-163.]{Green-Pedersen1980b}, \autocite[VI, c. 4, 163.8-13]{OckhamEL}.} The early identification of material and accidental consequence would seem to suggest that every material consequence is a simple one. But the \textit{Elementarium Logicae} ascribed to Ockham witnesses to (and rejects) the converse assumption, that every material consequence is \textit{ut nunc}.

According to the later criterion provided by Buridan, a material consequence is one which does not hold for every substitution of its \textit{categorematic} terms - terms so called from their signifying beings in the ten Aristotelian categories, and contrasted with syncategoremata - but rather holds in virtue of its content.\footnote{\autocite[I, c. 4, p. 23.10-13]{BuridanTC}: `Sed consequentia materialis est cui non omnis propositio consimilis in forma $\langle$quae formaretur$\rangle$ esset bona consequentia, vel, sicut communiter dicitur, quae non tenet in omnibus terminis forma consimili retenta'. Cf.  \autocite[I, q. 10, p. 105]{Pseudo-Scotus1891}.} This remark is unfortunate, as it does nothing to distinguish materially good consequences from merely bad ones. Buridan elaborates with the assumption that `no material consequence is evident in inferring except by its reduction to a formal one. It is reduced to a formal one by the addition of some necessary proposition or some necessary propositions, whose addition to the assumed antecedent yields a formal consequence'.\footnote{\autocite[I. c. 4, p. 23.15-19]{BuridanTC}: `Et videtur mihi quod nulla consequentia materialis est evidens in inferendo nisi per reductionem eius ad formalem. Reducitur autem ad formalem per additionem alicuius propositionis necessariae vel aliquarum propositionum necessariarum quarum appositio ad antecedens assumptum reddit consequentiam formalem'.} But Buridan does not require this reduction as a condition for being a material consequence, but only for being one that is able to be made evident, and he makes clear elsewhere that not every material consequence can be made evident in this way \autocite[338]{Klima2016}. 

Unlike Buridan, another early witness to Buridan's criterion, the \textit{Questions on the Prior Analytics} commentary falsely attributed to Duns Scotus, does define simple and as-of-now material consequences in terms of their ability to be reduced to a formal one by the addition, respectively, of a necessary or contingently true premise \autocite[I, q. 10, p. 105]{Pseudo-Scotus1891}. Importantly, a material consequence is not merely one to which any proposition can be added to produce a formal consequence. If this were so, \textit{every} consequence would be good materially on the later account: given the thesis that anything follows formally from a formal contradiction, one could make any consequence formal simply by adding the contradictory of its antecedent to it \autocite[I, c. 8, p. 36.160-37.181]{BuridanTC}. Instead, both Buridan and Pseudo-Scotus, require the assumed premise to be necessary (in the case of simple consequence), or true (in as-of-now consequence).

But put this way, it becomes rather unclear what material consequence is meant to be a theory \textit{of}. Since it includes inferences from impossible and false antecedents, the theory is too broad to be a theory of consequences that can be made sound. But it is too restricted to be a theory of consequences that can be made into formally valid ones, since it excludes inferences implicitly relying on impossible or false auxiliary assumptions. Nor, at least given the principle that an impossibility entails anything, does it plausibly capture a theory of what \textit{would} follow from a given assumption were it to obtain in the real world at some or every time. Happily, the theory of formal consequence fares better.
\subsubsection{Formal consequence and the origins of a general theory of formally valid inference}
By the time Ockham wrote his \textit{Summa Logicae} in the 1320s, his first example of a good formal consequence, one holding through an extrinsic middle, is that from an exclusive to a universal with the terms transposed - the converse of the rule Burley employs in his example above \autocite[III-3, c.1, p. 589.46-50]{OckhamSL}. Immediately before this, Ockham gives an argument with an impossible antecedent, `Only a donkey is a man, therefore every man is a donkey', as his example of a consequence holding through such an extrinsic middle \autocite[III-3, c. 1, p. 588.28-35]{OckhamSL}. 

Ockham himself takes `formal consequence' to be ambiguous between a consequence holding through an extrinsic middle, indifferent to the truth value or modal status of its antecedent, and one where the consequent may be derived from the antecedent via an intrinsic middle - i.e. via a proposition, taken up as an additional assumption, expressing an essential truth. Earlier, Burley had used this same criterion, i.e. holding through an intrinsic middle,  to characterize simple natural consequence,\footnote{Cf.  \autocite[V, par. 70. p. 128; VIII, par. 116, pp. 141-142]{Green-Pedersen1980b}.} Comparison to both Burley's division in his later \textit{De Puritate Artis Logicae} and his earlier \textit{de consequentiis} suggests that Ockham uses formal and material consequence to mean what Burley generally does by natural and accidental consequence.\footnote{See Archambault, `Consequence and Formality', this issue. Cf. the parallels in \autocite[VIII. par. 188, pp. 142-143]{Green-Pedersen1980b}, \autocite[p. 86.9-21]{BurleyDPAL}, \autocite[II, c. 7, par. 8, 204.124-205.154]{OckhamExpositio} \autocite[IV, c. 1, 15.1-17]{OckhamSentences}.} 

Ockham appears to have chosen his example of a formal consequence in the second sense - `Socrates does not run, therefore a man does not run' relying on the intrinsic middle `Socrates is a man'\footnote{Ockham's example of an intrinsic middle has an interesting history. Peter of Spain rejects the necessity of the proposition, arguing that its necessity would entail Socrates' necessary existence. The proposition is later accepted as necessary by Henry of Ghent, but the entailment is denied. See Spruyt, `Consequence and `Cause'', this issue. If the \textit{Tractatus minor logicae} is taken to be authentic, the discussion of a similar example there - Every man is an animal, therefore Jacob is an animal - would suggest Ockham takes the proposition to be intrinsic without thereby being necessary. \autocite[V, c. 1, p. 31.10-15]{OckhamTML}.} - in order to undermine the subordination, found in Scotus and Burley, of natural consequence to simple consequence.\footnote{Cf. \autocite[VI, c. 4, p. 163.3-8]{OckhamEL}: `Item consequentiarum quaedam vocatur formalis, quae tenet ratione formae propositionum, ita quod tenet quandocumque talis modus arguendi servatur, sicut `ab inferiori ad superius sine distributione est consequentia bona'. Sequitur enim `Sortes currit, ergo homo currit'. Et potest esse consequentia ut nunc, et ita non omnis consequentia formalis est consequentia simplex'; also \autocite[A.III, par. 16, p. 95]{Green-Pedersen1982}: `Ex istis conclusionibus patet falsitas dicti ipsius Ockham, super consequentias, capitulo primo, ubi dicit quod ista sit formalis `Sortes non currit, ergo homo non currit', et tenet per illud medium intrinsecum `Sortes est homo''.}

By the 1320s, one thus has essentially two accounts of formal consequence: one based on structural properties of inferences, indifferent to the truth and modal status of the propositions it employs; another, which is closer to the early theory of natural consequence, not indifferent to truth or modal status, since it minimally appears to require the auxiliary assumptions it employs to be true and necessary. But the initial concern of the latter account - for producing good arguments in the sense that they reliably led to truth - was gradually eroded by its expansion into the realm of the impossible. Hence, by the time of Burley's later \textit{De Puritate}, one finds an example using a proper name for a donkey - `Brunellus is able to laugh, therefore a man is able to laugh'  relying on the impossible middle `Brunellus is a man' - described as a good formal consequence.\footnote{\autocite[p. 86.4-21]{BurleyDPAL}.}

Once the theory of consequences has progressed to this point, the ground for the earlier division of consequences has been largely undercut by its internal developments. Already in Boethius, the pairing of natural consequence with accidental consequence encouraged confusing the necessity of a consequence with the necessity of its components.\footnote{It is instructive that Boethius' only example of following accidentally, the sentence `if fire is hot, the heavens are spherical', involves inferring a necessity from a necessity.} Early accounts of simple natural consequence seem to have intended a kind of following actually occurring in the nature of things, but the gradual expansion of the theory to include impossibilities occluded this aim. Ockham's recognition that inferences relying on intrinsic middles could still generate contingent conclusions undermined the subordination of natural consequence to simple consequence, and the inclusion of syllogisms and other formally valid arguments within the theory of natural, then formal consequence left the aim of the theory ambiguous. In short, the earlier aim of giving a theory of following \textit{intrinsically} was being gradually supplanted by a theory of following \textit{by form}.

The aim and assumptions of the earlier division of consequences persist for some time in a number of authors. For instance, a treatise on \textit{consequentiae} dated between 1325 and 1340 and attributed to Thomas Bradwardine appeals to a \textit{dictum} of Aristotle to argue that in every good consequence the modality of the conclusion is at least as strong as that of the premises, leverages this assumption to reject as-of-now consequence altogether, and later explicitly contests Ockham's example of an as-of-now formal consequence \autocite[A.II, par. 4-6, pp. 92-93;  A.III, par. 16-17, p. 95]{Green-Pedersen1982}. This intertwining of the earlier and later aims seems to have been encouraged especially by the topical, quasi-proof-theoretic approach to consequences, and most British (and later, Italian) logicians continue to define a formal consequence as one where the antecedent virtually contains the consequent.\footnote{\autocite[476]{DutilhNovaes2008}.For the reception of British logic in the Italian Renaissance, see \autocite{Maieru1982}.} Among these logicians, formal and material consequences appear to be distinguished as those which do and don't, respectively, proceed by the application of universally valid rules. Within such a theory, the difference in character between rules like \textit{from a universal to an exclusive with terms transposed} and \textit{from a superior to an inferior with distribution} is not immediately obvious, and perhaps not so important as it is in the continental theory.

After John Buridan, most authors on the European continent regard a formal consequence as one for which every consequence similar in form is good, i.e. one holding for all uniform substitutions on its categorematic terms. Combining this with his account of a good consequence, it follows that every good formal consequence is one where every consequence like it in form is such that it is impossible for things to be as its antecedent signifies without being as signified by its consequent. On this account, it is clear that any consequence with an explicitly contradictory antecedent or a formally necessary consequent is a good one. In this way, Buridan and his followers arrive at a completely general theory of formally valid inference containing syllogistic as a proper part - what I have elsewhere called `The medieval substitutional account of formal consequence' - surprisingly close to modern semantic theories of formal consequence \autocite{Archambault2018}.


\section{Introduction to the articles}
The articles collected in this issue survey a wide array of topics in the theory of consequence, beginning with the contributions of Boethius (which properly antedate the theory but are central to its later development) and following with discussions of subjects from the mid-twelfth through the later fourteenth century. Following the scholastic commitment to the question of the right order of reading \autocite[104]{Ebbesen}, the articles in this collection are ordered to facilitate reading from beginning to end, moving from material likely to be more basic for the understanding of other authors, more widely discussed in secondary literature, and more accessible from the standpoint of modern logic, to less familiar figures and broader thematic discussions.

%But to echo a remark from the rule of Saint Benedict on the order for the weekly reading of the Psalms, if anyone finds a better order, let him follow it.


Much of the basic material out of which the doctrine of consequence arose comes from Boethius. In `The Roots of the Notion of Containment in Theories of Consequence: Boethius on Topics, Containment, and Consequences', Bianca Bosman addresses the question of whether and to what extent the containment criterion for consequence, common in both earlier discussions of natural consequence and later British discussions of formal consequence, is anticipated in the logical works of Boethius. Bosman argues that, while the containment criterion does draw on Boethian source texts, those sources are different from those standardly assumed. Bosman shows that the later criterion draws much from texts \textit{not} devoted to conditionals, including Boethius' discussions of \textit{per se} predication in his treatment of the Porphyrian predicables, and of the \textit{locus} \textit{from a genus} in his commentary on Cicero's \textit{Topics}.

The best-known medieval accounts of consequences are those of William of Ockham and John Buridan. But the relation between these accounts remains obscure. In particular, Ockham classifies certain consequences as formal which Buridan admits only as material, and the exact reason for these differences has not been sufficiently explored. In `The Distinction between Formal and Material Consequences in Ockham and Buridan', Milo Crimi provides a classification of consequences both figures treat as formal, those both treat as material, and those which Ockham calls formal and Buridan calls material. Crimi then shows that the taxonomical discrepancy between Ockham and Buridan's accounts is not due to differences in their propositional hylomorphism, but to Ockham's endorsement of relational characterizations of formal consequences.

One of the more outstanding continental authors writing on consequences after Buridan is Marsilius of Inghen, later founder and rector of the University of Heidelberg. Marsilius calls Buridan `my teacher'\autocite[fol. 106va]{MarsiliusDeGen} and with Albert of Saxony Marsilius is traditionally regarded as a prominent member of a Buridanian school of logic. Though neither Marsilius nor Albert held such a relation to Buridan in any institutional sense,\autocite{Courtenay2004,Thijssen2004} their approach to consequences share some broad similarities when compared to that of later British writers, particularly in their use of a substitutional criterion for formal consequence. In `Marsilius of Inghen on the Definition of \textit{consequentia}', Graziana Ciola compares Marsilius' account of consequences with those of Buridan and Albert, and finds that Marsilius diverges from Buridan and Albert in several important respects. Specifically, Buridan and Albert affirm, while Marsilius denies, that a consequence is a \textit{propositio hypothetica}. Instead, Marsilius characterizes a consequence as an \textit{oratio}, further distancing the theory of consequences from that of the conditional and more clearly establishing it as an entailment relation. In addition, Marsilius rejects, where Buridan and Albert accept, \textit{ut nunc}, or as-of-now consequence. This rejection is found with some frequency among British and Italian logicians \autocite[A.II, par. 5, pp. 92-93]{Green-Pedersen1982}, and the adoption of the position in Marsilius suggests the interaction between British and continental traditions may be more complex than currently recognized.\footnote{Cf. \autocite[46]{Green-Pedersen1981}.}

With Ockham and Buridan, Walter Burley is often regarded as one of the most influential logicians of the later middle ages. One of the earliest extant \textit{consequentiae} treatises, and the earliest with a known author, belongs to Burley. In addition, Burley is one of the few early authors to discuss both the natural/accidental division and formal/material division of consequences at some length. In `Consequence and Formality in the Logic of Walter Burley', Jacob Archambault provides a comprehensive overview of Walter Burley's account of consequences. After reviewing Burley's division and enumeration of consequences, Archambault shows how Burley relates his own theory of natural and accidental consequence to the division into formal and material consequence found in Ockham. The article then compares Burley's work to the earliest anonymous treatises on consequences and to Ockham and Buridan's treatises on the subject. Archambault highlights Burley's advances over the former treatises' treatment of existential import in consequences, his disagreements with Ockham and Buridan on rules governing consequences, and his influence on the broader place of the study of consequences in logic. 

Next, Joke Spruyt provides an overview of consequences in the thirteenth century. Successively examining thirteenth-century discussions of syllogisms, \textit{syncategoremata}, and \textit{sophismata}, Spruyt shows that across these genres, thirteenth-century work on consequences often assimilated the relation of a consequent to its antecedent(s) to that of an effect to its cause(s). Though thirteenth-century logicians typically regarded premises as causes not of being, but of following, the assimilation played an important role in thirteenth-century treatments of inferences from impossible antecedents or to necessary consequents. Many thirteenth-century logicians rejected the validity of these inferences, and those who admitted them to be valid in some respect did not regard them as unqualifiedly so.

This issue closes with Christopher Martin's analysis of the development of the theory of natural consequence from Peter Abaelard to the turn of the fourteenth century. Martin argues that the early theory of natural consequence provides a medieval theory of relevant consequence, specifically one conforming to principles which today characterize connexive logic, and he shows the crucial role played by changes in the account of disjunction in the shift away from this relevantistic account of consequence. According to Martin, Abaelard distinguishes between extensionally-defined predicate disjunction for categorical propositions, on the one hand, and propositional disjunction, on the other, and employs an intensional account of propositional disjunction on which this kind of disjunction is equivalent to a conditional with the negation of the first disjunct as antecedent and the second disjunct as consequent. Like his account of the conditional, Abaelard's account of propositional disjunction thus also conforms to connexive principles. As logic texts shifts from twelfth century manifestos for the doctrines of rival schools to more irenic thirteenth century textbooks, Abaelard's distinction is lost, and largely replaced by an extensional account of propositional disjunction. But the need for a stronger form of consequence than that holding merely in virtue of a standard semantic requirement - namely, the impossibility of the antecedent holding with the consequent not holding - is found in authors through the thirteenth and into the fourteenth century, and was especially acute in a species of disputational exercises, or \textit{obligation}, involving the positing of an impossible proposition, called \textit{positio impossibilis}. It is in this disputational context, and specifically in the different treatments of impossible \textit{positio} in Scotus and Ockham, that the seeds of Ockham's alternative analysis of consequence, and the replacement of the earlier one, would be sown.\footnote{Special thanks to Milo Crimi for spotting errors in a previous draft.}

%\nocite{BTC}
%\nocite{OckhamSentences}
%\nocite{BuridanPostAn}
%\nocite{BuridanKlimaSD}%Note that I need all the Summulae critical editions, and that all of them are separate.
%\nocite{MarsiliusConsequentiae}
%\nocite{AlbertPL}
%incertorum auctorum quaestiones
%Faversham, Quaestiones Veteres
%Tractatus
%Explain connexive logic
%\section{Conclusion: consequences and the historiography of medieval logic today}

%Much early research into medieval logic was motivated, then accelerated by the surprising ways in which it anticipated advances made by Tarski and his contemporaries.\footnote{Cf. \autocite{Moody1952,Bochenski1956,Boehner1958a,Kneale1962}.} Historical research into medieval consequences formed a core component in the recovery of medieval logic, and in the broader development of the history of logic as a field. Central to this development was an increased attention to the work of William of Ockham, and a near-wholesale revival of interest in the work of the Parisian Arts Master John Buridan. 

%This work has, in its turn, fundamentally changed the shape of medieval philosophy today. Examining earlier overviews and anthologies on medieval philosophy by Gilson, Copleston, Wolter, and others, one finds them focused largely on questions of natural theology and metaphysics. One could hardly then anticipate a future in which in the leading journals, encyclopedias, and companions devoted to medieval philosophy, logic would frequently claim greater attention than these traditional foci.\footnote{Cf. \autocite{Copleston1962,Gilson1955,Kretzmann1982,Kretzmann1988,}.}



% Sherwood (natural and non-natural), Intentio Boethii, Kilwardby, Richard Sophista (natural & accidental) Peter of Spain (natural), Nicholas of Paris (natural and unnatural) 

%In Burley: Simple = always sound, ut nunc = sometimes sound, natural = essentially always sound, accidental = accidentally always sound. 

%In Ockham: material = not proceeding by a general rule, formal = proceeding by a general rule, 

%In Buridan: Formal = valid, material = sound-make-able, ut nunc = sound-makeable by a contingent premise, simplex = sound-makeable by a necessary premise


%Formal consequence is not one following solely in virtue of form: if it were, then \textit{ex impossibili quodlibet} would not be in need of proof. Rather, it is a consequence derivable solely by using rules manipulating the syncategoremata, irrespective of truth. Related question: how does formal/material relate to the distinction between primitive and derived rules, either in its modern or its medieval versions? Burley's notion of logical loci, as contrasted with dialectical \textit{loci}, as prefiguring the modern notion of a rule.

%comparison of simple-as-of-now to global-local distinction between consequences.

% If a material consequence is one which can be reduced to a formal one, then every consequence is a good consequence - simply add the contradictory, whatever its truth value, to the material consequence.

%Ut nunc consequence presupposes a synchronic, i.e. Scotus', account of possibility. Those who reject Scotus' account of possibility will, accordingly, reject ut nunc consequence.

%Tarski build consequence from truth from satisfaction. Buridan and his followers cut the notion of truth out altogether (e.g. 'every sentence is negative')

% `omnis consequentia est formalis in qua ex posito antecedenti potest deduci suum consequens vel consequentias intermedias formales nullo extrinseco coassumpto' Here, Marsilius' account of formal consequence appears to be completely in accord with Schupp's assumption about the critical text of Ockham

%Ockham takes the language of middles, used to ground his distinction between two different kinds of formal  consequences, from Aristotle's \textit{Topics}, as mediated by Boethius' \textit{On Differential Topics}. Earlier, Walter Burley uses the same topical distinction to distinguish simple natural from accidental consequences \autocite[VIII. par. 116, pp. 141-142]{Green-Pedersen1980b}. 

%One of the unacknowledged curiosities of research into  medieval logic generally, and consequences specifically, has been its near-inversion of the standard bell-curve history of medieval philosophy. According to a once-common historiography, medieval philosophy rapidly developed from the time of Saint Anselm, reached its height in the time of Aquinas and Bonaventure, and began its decline with Ockham (or, perhaps, Scotus).\footnote{Cf. \autocite{Gilson1955}.} The historiographical perspective implicit in much work in medieval logic, by contrast, hints that the age of the famous theological \textit{summae} marked an ebb between the output of twelfth-century schools and the great nominalists of the fourteenth century.\footnote{On consequences, cf. \autocite{Read2012,DutilhNovaes2012a}; on logic broadly, \autocite{GabbayWoods2008}, roughly only three of whose thirteen essays touch on thirteenth-century developments.} The latter perspective has at times brought with it an overemphasis on nominalist, reductive, and quasi-classical over realist, non-reductive pluralist, or more relevantistic approaches to consequence; magnified the influence of some logicians while diminishing that of others; exaggerated connections between like-minded thinkers while ignoring less straightforward, but more concrete connections, including those wrought in polemic. With this in mind, this collection aims to resituate medieval work on consequences within a broader perspective, providing greater attention to earlier work on consequences, a more nuanced account of the relations between well-known figures, and a more complete picture of the influences of each era on the next. 

%medieval logicians thus use `consequence' to refer to either: a relation of following between an antecedent and a consequent, the relation in addition to the antecedent and the consequent, or a statement or argument communicating this following through words like `if' (\textit{si}), `because' (\textit{quia}), or `therefore' (\textit{ergo}, \textit{ideo}, \textit{igitur}).


%Page cites/complete references needed for \autocite{AbelardDialectica}  \autocite{Garlandus1959}
%Go through, fix overuse of `is' and `has', and `will' wherever reasonable
\end{document}
