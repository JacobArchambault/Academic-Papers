\chapter[Anonymous 1, on consequences]{On consequences
\\London, BL, Royal 12 FXIX, ff. 111ra-112rb (\textit{ante} 1302)}
The following provides a translation of one of the earliest treaties on consequences, an anonymous treatise found in London, BL, Royal 12 FXIX, ff. 111ra-112rb and edited in \cite[pp. 4-11]{Green-Pedersen1980a}. Numbering follows the edition.
\begin{enumerate}
\item[1.] A consequence is a relation (\textit{habitudo}) between an antecedent and a consequent. An antecedent is that from which another follows. A consequent is what follows from another. For instance, in `if a man is, an animal is', `an animal is' is the consequent, `a man is' is the antecedent. And the antecedent is what immediately follows the sign [indicating] the connection.
\item[2.] One should know that in every good consequence \textit{whatever follows from the consequent follows from the antecedent}. And in every good consequence \textit{from the opposite of the consequent the opposite of the antecedent follows}. For instance, this consequence is good: `a man runs, therefore an animal runs'. Therefore from the opposite of the consequent the opposite of the antecedent follows so: 'no animal runs, therefore no man runs'.
\item[3.] If it is asked why `a man runs, therefore an animal runs' is good, one should say because it argues \textit{from an inferior to a superior without distribution}, since `man' is inferior and `animal' is superior.
\item[4.] Hence one should know that the consequence \textit{from an inferior to a superior} is valid in two ways, and in two ways it isn't. Hence, the consequence holds without distribution and without a negation prefixed, as in `a man runs, therefore an animal runs'.
\item[5.] And likewise, \textit{from an inferior to a superior without distribution and with a negation placed after} is a good consequence, as in `a man does not run, therefore an animal does not run'.
\item[6.] But the consequence \textit{from an inferior to a superior with distribution} is invalid - as in `every man runs, therefore every animal runs' - since the antecedent can be true without the consequent; when it is posited that every man runs, then the antecedent `every man runs' is true, and the consequent `every animal runs' is false, supposing in that case that an ass does not run. And in a good consequence the antecedent cannot be true without the consequent. Therefore, this consequence, `every man runs, therefore every animal runs', is invalid. 
\item[7.] Likewise, the consequence \textit{from an inferior to a superior with a negation prefixed} is invalid - as in `not a man runs, therefore not an animal runs' - since `not a man runs' is equivalent to `no man runs', and `not an animal runs' is equivalent to `no animal runs'. Now one should know that the consequence `no man runs, therefore no animal runs' is invalid, since the consequence \textit{from an inferior to a superior with distribution} is invalid, as was previously stated.\footnote{Par. 6.} In the same way, the consequence `not a man runs, therefore not an animal runs' is invalid.
\item[8.] But one should examine in what way the consequence \textit{from a superior to an inferior} holds. For this, one should know that it holds in two ways, and in two ways it doesn't. Hence, the consequence \textit{from a superior to an inferior with distribution} holds - for instance, `every animal runs, therefore every man runs'; and likewise in the negative `no animal runs, therefore no man runs'.
\item[9.] And one should know that the consequence \textit{from a superior to an inferior with a negation prefixed} holds, as in `not an animal runs, therefore not a man runs'. That this is good is shown: since `not an animal runs' is equivalent to `no animal runs', and `not a man runs' is equivalent to `no man runs', and the consequence `no animal runs, therefore no man runs' is good, in the same way the consequence `not an animal runs, therefore not a man runs' is good.
\item[10.] Note that the consequence \textit{from a superior to an inferior} fails in two ways. One way is when one argues from a superior to an inferior without distribution /f.111rb/ or negation, as in `an animal runs, therefore a man runs'. This consequence is invalid, since the antecedent can be true without the consequent, because positing that an ass runs and that no man runs, `an animal runs' is true, but `a man runs' is false, since its contradictory, namely `no man runs', is true by the case.
\item[11.] In another way the consequence \textit{from a superior to an inferior} fails, with a negation placed after as here: `an animal does not run, therefore a man does not run'. This consequence is invalid, since the antecedent can be true without the consequent, because positing that an ass does not run and that every man runs, `an animal does not run' is true, since an ass does not run, yet `a man does not run' is false, since its contradictory, namely `every man runs', is true by the case.
\item[12.] One should know that in every good consequence \textit{if the antecedent is true, the consequent is true}, since the false does not follow from the true. But it is not necessary that if the antecedent is false, that the consequent is false, since what is true can well follow from what is false, as is clear from Aristotle's \textit{Prior Analytics} I:\footnote{\cite[II. 2, p. 53b 7sq]{AristotlePrA}.} from falsehoods, truth; from truths, nothing but truth.
\item[13.] The second rule is: in a good consequence \textit{if the consequent is false, it must be that the antecedent is false}, since the false does not follow except from falsehood or falsehoods. But it is not necessary in a good consequence that if the consequent is true that the antecedent is true, since the true can follow from the false.
\item[14.] Another rule is this: \textit{whatever follows from the consequent follows from the antecedent}. For instance, `Socrates runs, therefore a man runs' follows; and `a man runs, therefore an animal runs' follows: therefore, `Socrates runs, therefore an animal runs' follows by the rule \textit{whatever follows from the consequent follows from the antecedent}.
\item[15.] Likewise, in every good consequence \textit{whatever entails (antecedit) the antecedent entails the consequent}. For instance, `a substance runs, therefore something runs' follows; and `a body runs, therefore a substance runs' follows: therefore, `a body runs, therefore something runs' follows by the rule \textit{whatever entails the antecedent entails the consequent}.
\item[16.] One should know that these two rules make a good argument:\footnote{or: `make an argument good'.} \textit{whatever follows from the consequent follows from the antecedent}; and \textit{whatever entails the antecedent entails the consequent}. But these two rules make the fallacy of the consequent: \textit{whatever follows from the antecedent follows from the consequent}; and \textit{whatever entails the consequent entails the antecedent}.
\item[17.] One should know that when the opposite of the consequent cannot stand with the antecedent, then there is a good consequence, as is clear in the consequence, `a man runs, therefore an animal runs' - since `a man runs' and `no animal runs' cannot stand together, but `no animal runs, therefore no man runs' follows. What, then, stands with the consequent stands with the antecedent. Thus, if `a man runs' and `no animal runs', were to stand together, `a man runs' and `no man runs' would stand together; and so these two contradictories would stand together, which is impossible.
\item[18.] Hence there is this rule: in every good consequence \textit{the opposite of the consequent cannot stand with the antecedent} (except where the antecedent includes opposites), since in every good consequence \textit{from the opposite of the consequent the opposite of the antecedent follows}. Whatever, then, stands with the antecedent stands with the consequent. If then the opposite of the consequent stands with the antecedent, the antecedent and the opposite of the consequent can stand together. But the antecedent and the opposite of the antecedent\footnote{Folling the ms. \textit{antecedentis}, rather than Green-Pedersen's \textit{consequentis}.} /f.111va/ are two contradictories. Therefore two contradictories would stand together. But this is impossible. Therefore, it is impossible for the opposite of the consequent to stand with the antecedent in a good consequence where the antecedent does not include opposites. If the antecedent does include opposites, then the opposite of the consequent can stand with the antecedent, as is clear in these consequences: `if nothing is, something is', and `if no proposition is true, some proposition is true' and `if no time is, some time is'. In each of these consequences the opposite of the consequent can stand with the antecedent, since the antecedent in each of these includes opposites: `if nothing is, something is'; `if no proposition is true, some proposition is true'; and likewise `if no time is, some time is'. Proof: if no time is, it is not night, and if it is not night, it is day; and if it is day, some time is. Therefore, if no time is, some time is. Hence [this] rule for seeing when a consequence is good and when not: one should see whether the opposite of the consequent can stand with the antecedent or not; if not, the consequence is good; if the opposite of the consequent can stand with the antecedent, the consequence is not valid from [its] form.
\item[19.] There is another rule [amounting] to the same: one should see whether the antecedent can be true without the consequent or not. If so, then the consequence is  invalid; but if the antecedent cannot be true without the consequent, then the consequence is good.
\item[20.] It should be known as a rule that \textit{if any two propositions are incompatible, one implies the opposite of the other}, as is clear in these: these two are incompatible - `Socrates runs' and `no man runs' - and so one implies the opposite of the other, since if Socrates runs, then a man runs, and `a man runs' is opposed to `no man runs'. Likewise, `no man runs, therefore Socrates does not run' follows, and `Socrates does not run' is opposed to `Socrates runs'.
\item[22.] For giving a contradiction in singular propositions, placing the negation before or after doesn't matter, since for giving the contradictory of `Socrates runs' it doesn't matter whether `Socrates does not run' or `not Socrates runs' is said, because either contradicts `Socrates runs'. But in indefinite, particular, and universal propositions, placing the negation before matters much. Hence if it is asked what the contradictory of `a man runs' is, one should say `not a man runs' and not `a man does not run'. In the same way you ought to place the negation before in universals and particulars. And if it is asked why `a man runs' and `a man does not run' do not contradict, one should say that they can be true together, and contradictories can't; therefore they aren't contradictories. If it is asked how `a man runs' and `a man does not run' can be true together, one should say that positing that Plato runs and Socrates does not run, then each of them is true. Hence positing this case, `a man runs' is true, since it has one true singular, namely `that man runs', pointing to Plato.\footnote{Correcting the ms. \textit{Socrate}.} And for the truth of an indefinite it suffices that one singular be true. And likewise, `a man does not run' is true supposing this case, since it has one true singular, namely `that man does not run', pointing to Socrates.\footnote{Correcting the ms. \textit{Platone}.} And for the particular to be true /f.111vb/, as well as the indefinite, it suffices that one singular be true.
\item[23.] One should know that for a universal to be true, it must be that each of its singulars is true, if it has singulars. But there is some universal which does not have singulars, e.g., `every chimera is a chimera'. This is true, since it predicates the same of itself, and no proposition is more true than that in which the same is predicated of itself. And yet none of its singulars is true, since it doesn't have singulars. Likewise, positing that no man is white, then the proposition [`every white man is a white man'] is true and yet none of its singulars is true, since it does not have singulars in the case supposed. Hence in brief, for a universal that has singulars to be true, it must be that each of its singulars is true. And if it is asked what is required for a universal to be false, it must at least be that one singular is false, if it has a singular. Positing that no man is white, `no white man is a white man' is false, as is clear, and yet it does not have a false singular, since it does not have a singular.
\item[24.] What is required for a particular or indefinite to be true? That one singular be true, if it has a singular. What is required for a particular or indefinite to be false? It must be that each of its singulars is false, if it has singulars. Hence, though a particular or indefinite may have a false singular, an indefinite need not be false on account of this; but it must be that each of its singulars is false. But if an indefinite or particular has one true singular, such propositions are true. 
\item[25.] This should be known as a rule: \textit{whatever one convertible is predicated of, the other is also predicated of}. For instance, `man' and `risible' convert, so whatever the one is predicated of, the other is predicated of the same.
\item[26.] One should know that convertibility is twofold, namely convertibility between terms and between propositions. Convertibility between terms is when two terms convert, as with `man' and `risible'. And this rule, `whatever one, etc.' has to be understood of this sort of convertibility. Convertibility between propositions is when two propositions convert, as with the two propositions `a man runs' and `a mortal rational animal runs'; and this rule is not to be understood of these. But rules concerning convertibility between propositions include: \textit{if one among convertibles is true, the other will be true}; and \textit{if one is false, the other will be false}; and \textit{if one is necessary, the other will be necessary}; and \textit{if one is contingent, the other will be contingent}.
\item[27.] Hence to know when terms convert and when they don't, one should see whether each term is predicated universally of the other or not; if so, then the terms convert. As is clear, this term `man' and this term `risible' convert, since each is predicated universally of the other; for `every man is risible' is universal and true, and likewise the converse `every risible /f.112ra/ is a man'. And consequently, `man' and `risible' convert with respect to their supposita, since all the things that are supposita of the one are supposita of the other; and if this were not so, they wouldn't convert.
\item[28.] One should know that \textit{from a universal to a singular} - both conjunctively and separately - is a good consequence, as `every man runs, therefore Socrates runs' makes clear. And likewise this consequence is good: `every man runs, therefore this man runs and that one...' and so on for each singular. 
\item[29.] Likewise, \textit{from all singulars taken conjunctively to a universal} is a good consequence. Thus, `this man runs, and that one...', and so on for each singular, `...therefore every man runs'. But \textit{from one singular to a universal} is not a good consequence. As is clear, `this man runs, therefore every man runs' is not valid, but is the fallacy of the consequent \textit{from the positing of the consequent}, since it follows conversely - `every man runs, therefore this man runs' - and does not follow in this way.
\item[30.] One should know that \textit{from one singular to an indefinite} is a good consequence, as here: `this man runs, therefore a man runs'. Likewise, \textit{from all singulars to an indefinite} is a good consequence, as here: `that man runs, and that one...' and so on for each singular, `...therefore a man runs'. But the consequence \textit{from an indefinite to a singular} is not valid, neither conjunctively nor separately, but [is valid] taking the singulars disjunctively; hence `a man runs, therefore this one ... or that one runs' follows.
\item[31.] One should know that \textit{from a universal to an indefinite} is a good consequence, as in `every man runs, therefore a man runs'; but the converse consequence, as in `a man runs, therefore every man runs', is invalid.
\item[32.] One should know that from the impossible anything follows, that is, from the impossible any proposition follows. For instance, `a man is an ass, therefore a man is a she-goat', since the antecedent is impossible and from the impossible anything follows. Likewise: `a man is an ass, therefore you are a bishop'. Likewise `a man is an ass, therefore God does not exist' follows, and this is because the antecedent is impossible, and from the impossible anything follows. 
\item[33.] Another rule is this, that the necessary follows from anything. For `a man runs, therefore a man is an animal' follows in its terms, because the consequent is necessary, and the necessary follows from anything. Likewise `a man is an ass, therefore God exists' follows, and this is because the consequent is necessary, and the necessary follows from anything - that is, from any proposition.
\item[34.] Against one stated before:\footnote{Par. 8.} when it is said `\textit{from a superior to an inferior with distribution} is a good consequence,' I prove that it is invalid. Since `every man runs, therefore every white man runs' argues from a superior to an inferior with distribution, and yet this consequence is invalid, because the antecedent can be true without the consequent; for positing that every man runs and that no man is white, then this antecedent, `every man runs', is true, and this consequent, `every white man runs', is false in the posited case, because `every white man runs, therefore a man is white' follows. The consequent [is] false by the case; therefore, the antecedent is false by the case. 
\item[35.] I respond to this argument and say that `inferior' is twofold, namely inferior \textit{per se} and inferior \textit{per accidens}. /f.112rb/ An inferior \textit{per se} is just as `man' is \textit{per se} inferior to `animal', and `animal' \textit{per se} to `substance'. An inferior \textit{per accidens} is as `white man' is inferior \textit{per accidens} to `man', and `running' is inferior \textit{per accidens} to `moving'. And through this I respond to the argument, and say that the consequence \textit{from a superior to an inferior per accidens with distribution}, and this affirmatively, is invalid, as here: `every man runs, therefore every white man runs', since it argues from a superior to an inferior \textit{per accidens}. But \textit{from a superior to an inferior per se with distribution}\footnote{Correcting the ms. \textit{ab inferiori ad superius}.} is a good consequence, as is clear here: `every animal runs, therefore every man runs'. But \textit{from a superior to an inferior per accidens with distribution} negatively is a good consequence, as here: `no man runs, therefore no white man runs', since the negation negates for whichever, \textit{per se} or \textit{per accidens}. 
\item[36.] Likewise, one should know that \textit{from an inferior \textit{per accidens} to a superior without distribution} is not a good consequence \textit{simpliciter}, as is shown: the consequence `a white man is a white man, therefore a white man is a man' is invalid. Neither is `a white man is a white man, therefore a white man is this white [thing]' valid, since the antecedent can be true without the consequent: for positing that no man is white, then `a white man is a white man' is true, because it predicates the same of itself, and no proposition is more true than that in which the same is predicated of itself; therefore, etc. And yet `a white man is a man' is false, and likewise `a white man is this white [thing]', on account of a false implication, since it implies a man is white, which is false by the case.
\item[37.] Likewise, \textit{from an inferior per accidens to a superior} is a good consequence, provided that one argues from an inferior to a superior where there is not a predication of the same of itself. This consequence is good, as is clear: `a white man runs, therefore a man runs'. But where there is a predication of the same of itself the consequence is invalid. As is clear here, this consequence is invalid: `a white man is a white man, therefore a man is a white man'.
\end{enumerate}