\chapter[Anonymous 2, on consequences]{On consequences \\ Paris, BN, lat. 16130, ff. 118va-120vb}
The following translates an early anonymous treatise on consequences found in Paris, BN, lat. 16130, ff. 118va-120vb, and edited in \cite[pp. 12-28]{Green-Pedersen1980a}. Numbering follows the edition.


%Par. 39-40.
%The ambiguity of `per accidens suppositum': either a per accidens suppositum, or what is placed under accidentally.
%`distributo per se distribuitur per accidens'?
%`posito per accidens, ponitur per se'
\begin{enumerate}
\item[1.] In every good consequence \textit{whatever follows from the consequent follows from the antecedent}. For instance, `Socrates runs, therefore an animal runs' follows, and `an animal runs, therefore a substance runs' follows; therefore, \textit{from the first to the last}, `Socrates runs, therefore a substance runs' follows. Rule: \textit{whatever follows from the consequent follows from the antecedent}.
\item[2.] I show the consequence `Socrates runs, therefore an animal runs': the opposite of the consequent implies the opposite of the antecedent; the opposite of the consequent is `no animal runs'; the opposite of the antecedent is `Socrates does not run'; now `no animal runs, therefore Socrates does not run' follows. The proof that `no animal runs, therefore no man runs' follows by this rule: \textit{from the distribution of the superior follows the distribution of the inferior}; and `animal' is superior to `man'; therefore, from the distribution of `animal' follows the distribution of `man'. And `no man runs, therefore Socrates does not run' follows by this rule: the consequence \textit{from the universal to its singulars} holds. Therefore \textit{from the first} [\textit{to the last}], `no animal runs, therefore Socrates does not run' follows. And consequently, from the opposite of the consequent follows the opposite of the antecedent.
\item[3.] Furthermore, I show the consequence `Socrates runs, therefore an animal runs', since the opposite of the consequent does not stand with the antecedent, because these do not stand together: `no animal runs', therefore `Socrates runs' (because `two do not stand together' is nothing other than that these two conflict). But these two - `no animal runs' and `Socrates runs' - conflict, because from the one the opposite of the other is implied, since `no animal runs, therefore Socrates does not run' follows, as is shown above.\footnote{Par. 2.} And `Socrates runs' implies the opposite of `no animal runs', since `Socrates runs, therefore an animal runs' or `some animal runs' (which are the same, since the judgment of an indefinite and a particular are the same) follows. But `no animal runs' and `some animal runs' contradict. Therefore, `Socrates runs' implies the opposite of `no animal runs'.
\item[4.] I show the consequence `Socrates does not run, therefore a man does not run' by this rule: a consequence \textit{from a inferior to a superior with the negation placed after} holds; but it is so argued here; therefore, the consequence `Socrates does not run, therefore a man does not run' is good. And consequently, removing the negation from both, it follows affirmatively thus: `Socrates runs, therefore a man runs'. And lastly `Socrates runs, therefore an animal runs' follows, since from the positing of the inferior follows the positing of the superior. And consequently from the opposite of the consequent follows the opposite of the antecedent.
\item[5.] Furthermore, I show the consequence `Socrates runs, therefore an animal runs', since the opposite of the consequent does not stand with the antecedent; therefore the first consequence [is] good. I show that they conflict by this rule: neither of `no animal runs' and `Socrates runs' is impossible; the impossible follows from them; therefore they conflict. I show that neither of them is impossible, for any singular of this universal is possible, therefore the universal is possible. For `no animal runs' is not impossible, nor is `Socrates runs' impossible; therefore neither of these is impossible; the antecedent is true; therefore the consequent. I show that the impossible follows from these, for I argue thus in the second figure: no animal runs; Socrates runs, therefore Socrates is not an animal. The conclusion is impossible, and neither premise is impossible; therefore they are incompossible,\footnote{Ms. \textit{impossibilia}; cf. \cite[p. 155, par. 151]{Green-Pedersen1980b}.} and consequently conflict. I show that the conclusion is impossible, since the impossible follows from it: therefore it is impossible (since from `Socrates is not an animal' the impossible follows). I show it: for `Socrates is not an animal, therefore Socrates is not Socrates' follows; that in which the same is removed from itself is impossible; such is `Socrates is not Socrates'; therefore it is impossible. I show the consequence `Socrates is not an animal, therefore Socrates is not Socrates', since `Socrates is not an animal, therefore Socrates is not a man' follows by this rule: \textit{from the denial (negatio) of a superior of something follows the denial of an inferior of the same}. And finally, `Socrates is not a man, therefore Socrates is not Socrates' follows by the same rule.
\item[6.] One should know that this rule is good: \textit{whatever entails (antecedit) the antecedent /f. 118vb/ entails the consequent}. For `a man runs, therefore an animal runs' follows, since whatever entails `a man runs' entails `an animal runs'; and because `Socrates runs, therefore a man runs' follows, `Socrates runs, therefore an animal runs' follows.
\item[7.] Hence one should know that the consequence `if a man runs, then an animal runs; therefore if Socrates runs, then an animal runs' is good, since it is argued by this rule: \textit{whatever entails the antecedent entails the consequent}.
\item[8.] Hence one should see that one may always descend from a term standing under a condition with respect to the consequent, as is shown: `if a man runs, then an animal runs; therefore if Socrates runs, then an animal runs' follows, for the consequent is always the same here, - that is, `an animal runs' - and the descent occurs from a term placed under a condition, that is, from `man'.
\item[9.] And one should know that a consequence of this sort always holds by this sort of rule: \textit{whatever entails the antecedent entails the consequent}. One should know that the consequence `if a man runs, an animal runs, therefore if a man does not run, an animal does not run' is good' [***]
\item[10.] [***] since here it is argued by this rule: \textit{whatever follows from the consequent follows from the antecedent}, since `a substance runs' follows from `an animal runs', so it follows from `a man runs'. 
\item[11.] Hence one should know that when \textit{from the inferior with respect to [something] following to the superior with respect to the same anteceding} is argued, the consequence is good, because it is argued by this rule: \textit{whatever follows from the consequent follows from the antecedent} as is shown in the example above.\footnote{Par. 10.}
\item[12.] One should know that this rule is good for disproving a consequence: \textit{something entails the antecedent that does not entail the consequent}; therefore, the consequence is invalid, as is clear in the example `an animal runs, therefore a man runs'.
\item[13.] I show that this consequence does not hold: something entails the antecedent that does not entail the consequent; therefore the consequence is invalid. Something entails `an animal runs' that does not entail `a man runs' since `an ass runs, therefore an animal runs' follows, yet `an ass runs, therefore a man runs' does not follow; and consequently, the consequence is invalid.
\item[14.] Furthermore, I show that the consequence `an animal runs, therefore a man runs' is invalid, because something follows from the consequent that does not follow from the antecedent; therefore, the consequence is invalid. For `a man runs, therefore a rational animal runs' follows, yet `an animal runs, therefore a rational animal runs' does not follow.
\item[15.] Furthermore, I show that `an animal runs, therefore a man runs' does not follow, since the antecedent can be true without the consequent; therefore the consequence is invalid. I show that the antecedent can be true without the consequent, since `an animal runs' can be true if an ass runs and if no other animal runs, since `an ass runs, therefore an animal runs' follows, and this can be true even if no man runs. Therefore `an animal runs' can be true with `a man runs' being false.
\item[16.] Furthermore, I show that `an animal runs, therefore a man runs' does not follow, since something stands with the antecedent that does not stand with the consequent; therefore the consequence is invalid. For `no man runs' stands with the antecedent `an animal runs', but\footnote{Agreeing with Green-Pedersen's \textit{sed} against the ms. \textit{vel}.} does not stand with the consequent `a man runs'. I show that `an animal runs' and `no man runs' may stand together. For `no man runs' and `an ass runs' stand together; and `an ass runs, therefore an animal runs' even follows. Therefore `no man runs' and `an animal runs' stand together, since what stands with the antecedent stands with the consequent.
\item[17.] Furthermore, I show that `an animal runs' and `no man runs' may stand together. Since if they won't stand together, then one implies the opposite of the other; and consequently, `no man runs, therefore no animal runs' follows. But this consequence is invalid, since it argues \textit{from a distributed inferior to a distributed superior}.
\item[18.] Furthermore, the opposite of the consequent stands with the antecedent. Therefore `an animal runs, therefore a man runs' does not follow. For `an animal runs' and `no man runs' stand together, since for some things to stand together is nothing other than for them to be able to be true together. But they can be true together - supposing, for instance, that no man runs and an ass runs.
\item[19.] One should know that the rule \textit{whatever follows from the antecedent follows from the consequent} is invalid, but brings about a false consequence, as is shown if it is so argued: `If a risible [thing] is an ass, then an animal is an ass'; but /f. 119ra/ by \textit{whatever [follows from the antecedent] follows from the consequent} one argues that since `a man is an ass' follows from the antecedent `a risible [thing] is an ass', so `a man is an ass' follows from the consequent `an animal\footnote{Agreeing with Green-Pedersen's `\textit{animal}' against the ms. `\textit{homo}'.} is an ass'. This consequence is in no way valid.
\item[20.] Hence one should know that the consequence `if a risible [thing]\footnote{Reading `\textit{risible}' in place of `\textit{homo}'.} is an ass, an animal is an ass; therefore if an animal is an ass, a man is an ass' is invalid, because it argues by this rule: \textit{whatever follows from the antecedent follows from the consequent}.
\item[21.] One should know this rule is invalid: \textit{whatever entails the consequent entails the antecedent}, as is shown if it is so argued: if a risible [thing] is an ass, then a man is an ass; for here by the rule \textit{whatever entails the consequent entails the antecedent}, one argues that `a risible [thing] is an ass' entails `an animal is an ass', and so `a risible is an ass' also entails `a man is an ass'. And it is so argued by this rule: \textit{whatever entails the consequent entails the antecedent} since `a risible [thing] is an ass' entails the consequent `an animal is an ass'.\footnote{Cf. \cite[pp. 80.13-29, 84.8-86.21]{BurleyDPAL}.}
\item[22.] Hence a descent under the consequent of a conditional with respect to the same antecedent, as occurs in the example above, is never valid, because\footnote{Reading `\textit{quia}' for the ms. `\textit{qui}', and instead of Green-Pedersen's `\textit{ubi}'.} it is always argued by the rule \textit{whatever entails the consequent entails the antecedent}.
\item[23.] Against the rule \textit{whatever follows from the consequent follows from the antecedent}\footnote{Par. 1.} one argues thus: the consequence `the Antichrist is, therefore the Antichrist is right now' is good. Yet something follows from the consequent that does not follow from the antecedent: for `The Antichrist is right now, therefore the false is true' follows; yet `the Antichrist is, therefore the false is true' does not follow. Therefore not whatever follows from the consequent follows from the antecedent.
\item[24.] I show that the consequence `the Antichrist is right now, therefore the false is true' follows. The antecedent is impossible, for `the Antichrist is right now' is impossible; and from the impossible anything follows, whether it be impossible\footnote{Following Green-Pedersen's reading of `\textit{impossible}' instead of `\textit{possibile}'.} or it be necessary. Therefore `the Antichrist is right now, therefore the false is true' follows.
\item[25.] I show that the consequence `the Antichrist is, therefore the false is true' is invalid, because the antecedent is possible (for `the Antichrist is' is possible) and the consequent, namely `the false is true',\footnote{Reversing the ms. `\textit{verum est falsum}'.} is impossible; and from the possible the impossible does not follow; therefore `the false is true' does not follow.
\item[26.] One should say that this rule is good: \textit{whatever follows from the consequent follows from the antecedent}. 

To the argument one should say that `the Antichrist is, therefore the Antichrist is right now', should be disambiguated for the reason that `the Antichrist...' can denote an as-of-now (\textit{ut nunc}) consequence or a simple consequence. If it denotes an as-of-now consequence, then `The Antichrist is, therefore the Antichrist is right now' is good, because the antecedent, as of now, cannot be true without the consequent, for `the Antichrist is' cannot be true as of now unless `the Antichrist is right now' is true. And so it follows as of now, because an as-of-now consequence is nothing else than that the antecedent can[not] be true as of now unless the consequent is true. If `the Antichrist...' denotes a simple consequence, then `The Antichrist is right now' doesn't follow, since in a simple consequence it is required that the antecedent never can be true unless the consequent is true. But `The Antichrist is, therefore the Antichrist is right now' is not so, because if right now is \textit{a}, then it is the same to say `The Antichrist is right now' and `The Antichrist is at \textit{a}', but `the Antichrist is' can be true tomorrow with `the Antichrist is at \textit{a}' being false, because tomorrow it will be the past, and then `The Antichrist is at \textit{a}' is false. And so\footnote{Agreeing with Green-Pedersen's `\textit{sic}' against the ms. `\textit{sicut}'.} it is clear that `The Antichrist is, therefore the Antichrist is right now' does not follow as a simple consequence. Still, `the Antichrist is, therefore the Antichrist is right now' does follow as of now. And so the consequence `The Antichrist is, therefore the false is true' does follow as of now, because the antecedent cannot be true unless the consequent is true. Yet the consequence `The Antichrist is, therefore the false is true' does not follow simply, because in a simple consequence it is required that the antecedent never is /f. 119rb/ true unless the consequent is true. But it is not so in the proposed, as it is clear that `the Antichrist is' could be true and `the false is true' never could be true; and consequently, the antecedent could be true without the consequent; and consequently the consequence is invalid spoken as a simple consequence.
\item[27.] Hence one should understand that never in a simple consequence does the impossible follow from what is contingent toward either. But `The Antichrist is' is contingent toward either, and `the false is true', impossible.
\item[28.] Then to the reasons for the opposite: when\footnote{Par. 23.} it is argued the consequence `the Antichrist is, therefore the false is true' is invalid because the antecedent can be true without the consequent, one should say that this consequence is good spoken as an as-of-now consequence, and `the antecedent can be true without the consequent' is invalid spoken of an as-of-now consequence. But in a simple consequence the reason [given] proves hence that this consequence is not good spoken as a simple consequence.
%`the antecedent can be true without the consequent' is invalid: i.e. the \textit{criterion} `the antecedent...' is invalid.
\item[29.] Against the rule\footnote{Par. 27.} \textit{from the contingent toward either the impossible does not follow} spoken of a simple consequence, one argues so: from the necessary follows the contingent; therefore, from the opposite of the contingent follows the opposite of the necessary. But the opposite of the contingent is always contingent, and the opposite of the necessary is impossible. Therefore from the contingent the impossible follows.
\item[30.] To this one should say that from the necessary the contingent does not follow, because then from the opposite of the contingent - which is contingent - follows the opposite of the necessary, which is impossible. The consequent is false; therefore also the antecedent.
\item[31.] Against [this]: I prove that from the necessary the contingent follows, because `a white man is a white man' is necessary, since it affirms the same of itself, and from this the contingent follows, for `a white man is a white man, therefore a man is a white man' follows. The consequent is contingent, and the antecedent is necessary as is proven; therefore from the necessary the contingent follows. And consequently, from the opposite of the consequent follows the opposite of the antecedent, i.e. `no man is a white man, therefore no white man is a white man'; and `no man is a white man' is contingent, and `no white man is a white man' is impossible, since the same is denied (\textit{negatur}) of itself.
\item[32.] To this it is said that `a white man is a white man, therefore a man is a white man' does not follow, because `a white man is a white man' is necessary, since it predicates the same of itself, and `a man is a white man' is contingent.
\item[33.] I show that `a man is a white man' [is] contingent, since it can be true and can be false; therefore it is contingent. The consequence is clear from the definition of the contingent. It is certain that it can be true. And that it can be false, proof: for I suppose that no man is white, and then `a man is a white man, therefore a man is white' is true.\footnote{Substituting \textit{verum} for the ms. \textit{falsum}.} The consequent is false in the case posited, and the case, possible. Therefore, `a man is a white man' can be false.
\item[34.] I show the consequence `a man is a white man, therefore a man is white' by this rule: \textit{positing per accidens, also posits per se}. But `white man' is a suppositum \textit{per accidens} of something. With respect\footnote{Following Green-Pedersen's \textit{respectu} in place of the ms. \textit{regula}.} to the same, `white' is also posited.
\item[35.] Furthermore: from the opposite of the consequent follows the opposite of the antecedent. For `no man is white, therefore no man is a white man' follows by this rule: \textit{a suppositum distributed per se, is distributed per accidens}.
\item[36.] Against this I show the consequence `a white man is a white man, therefore a man is a white man' by this rule: \textit{positing per accidens posits per se}; but `white man' is supposed \textit{per accidens}, and so `man' is supposed \textit{per se};\footnote{Reading `suppositum' instead of the ms. `superius'.} therefore positing 'white man' with respect to some predicate, `man' is posited with respect to the same. 
\item[37.] Furthermore: from the opposite of the consequent follows the opposite of the antecedent; for `no man is a white man, therefore no white man is a white man' follows by this rule: \textit{distributing per se, distributes per accidens}.
\item[38.] One should say that the consequence `a white man is a white man, therefore a man is a white man' is invalid, for /f. 119va/ the reasons given above.\footnote{Par. 32.} On account of this, one should understand that any proposition in which a superior is predicated of its inferior \textit{per accidens} is contingent. And [those] placed under \textit{per accidens} are those that are combined out of two inhering in each other contingently, as \textit{white man} is combined \textit{per accidens} because it is put together out of \textit{man} and \textit{white}, which inhere in each other contingently. So `a man is a white man', etc. is contingent. This is contingent, and its equivalents (\textit{convertibilia}) are contingent, because in all of these a superior is assumed of an inferior \textit{and} white\footnote{Unbracketing \textit{et} and \textit{album}.} \textit{per accidens} (since whatever is constituted by an addition with respect to another is inferior to it. Thus, `white man' is inferior to `man' and `white').
\item[39.] Another thing that should be understood is that the consequence \textit{from an inferior \textit{per accidens} to its superior} is never valid when the same is compared to itself.
\item[40.] Because of this, to the reason\footnote{Par. 36.} for accepting `a white man is a white man, therefore a man is a white man' one should say that it doesn't follow for the reason given above.\footnote{Par. 38.} To the proof: when it is accepted that `positing \textit{per accidens} the \textit{per se} is also posited', one should say that it is true if that which is \textit{per accidens} is not compared to itself, etc. `Distributing \textit{per se},\footnote{Par. 37.} the \textit{per accidens} is also distributed', unless that which is \textit{per accidens} is compared to itself, as in `a white man is a white man' what supposits\footnote{Following the edition's \textit{suppositum} against the ms. \textit{sumptum}.} \textit{per accidens} is compared to itself. And thus it need not be that `a white man is a white man, therefore man is a white man' follows, because \textit{from the positing of something suppositing}\footnote{Following the Green-Pedersen's \textit{suppositi} against the ms. \textit{sumpti}.} \textit{per accidens with respect to itself} the positing of the superior does not follow. Nor \textit{from the distribution of a superior} does the distribution of what is \textit{per accidens} inferior with respect to itself follow. And thus `no man is a white man, therefore no white man is a white man' does not follow'. Hence these two rules: \textit{positing per accidens posits per se}, and \textit{distributing per se, distributes per accidens} should be understood where the \textit{per accidens} is not compared to itself.
\item[41.] One should know that this rule is good: \textit{if one contradictory is true, the other will be false}, and conversely. And similarly, this rule will be good: \textit{if one of conflicting [propositions] is true, the other will be false}, [but not conversely] because two conflicting [propositions] can be false together; for `every animal is a man' and `no animal is a man' conflict, and yet are false together.
\item[42.] I show that `every animal is a man' is false for this reason: \textit{every proposition where something false follows from it and something true is false}. And this [proposition] is of this type, for I argue thus: every animal is a man, an ass is an animal, therefore an ass is a man. The conclusion is false and the minor isn't, so the major [is].
\item[43.] Furthermore, I show that it is false because every proposition from which something false follows is false. But it is so in the proposed, for `every animal is a man, therefore an ass is a man' follows by this rule: the consequence \textit{from a distributed superior to its per se inferior} holds.\footnote{Par. 2.}
\item[44.] Furthermore, I show that `Every animal is a man' is false because if one among contradictories is true, the other will be false. But `some animal is not a man' is true'. I show that this is true because a proposition which follows from something true is [true]. `An ass is not a man, therefore some animal is not a man' follows just from this rule: the consequence \textit{from an inferior to a superior with the negation placed after} holds.\footnote{Par. 4.}
\item[45.] Furthermore, I show that `every animal is a man' is false, because what conflicts with it is true; therefore it is false. For `an ass is not a man' is true; and `an ass is not a man' and `every animal is a man' conflict; therefore `every animal is a man' is false.
\item[46.] I show that these conflict because one implies the opposite of the other, as is shown above,\footnote{Par. 43-44.} so they /f. 119vb/ conflict.
\item[47.] Against the second\footnote{Reading the ms. \textit{secundam} rather than Green-Pedersen's \textit{illam}.} rule\footnote{Par. 15; cf. par. 3.} \textit{an antecedent cannot be true without its consequent}, I argue thus: the consequence `every man runs, therefore John the grammarian runs' is good, and yet the antecedent can be true without the consequent. I show that the antecedent can be true without the consequent, for supposing that every man runs, and that John is not a grammarian, then the antecedent is true and the consequent false. And consequently the antecedent can be true without the consequent, because the consequent is false, as is shown: for `John the grammarian runs' is false because it implies a falsehood, that John is a grammarian. I prove the consequence both because it argues \textit{from the universal to its singular} and because from the opposite of the consequent follows the opposite of the antecedent. For `John the grammarian does not run, therefore some man does not run' follows, because the  consequence \textit{from an inferior to a superior with the negation placed after} holds. 
\item[48.] To this one should say that the rule is good: the antecedent can be true without the consequent, therefore the consequence is invalid.

To the argument: when it is accepted that the consequence `every man runs, therefore John the grammarian runs' is good, I say that this consequence is invalid, because the antecedent can be true without the consequent, as has been shown. To the proof: when it is accepted that the consequence \textit{from the universal to its singular} holds, one should say that it is true spoken of a singular \textit{per se}, but is not true spoken of a singular \textit{per accidens}; but John the grammarian is a singular \textit{per accidens} of `man'; therefore `Every man runs, therefore John the grammarian runs' does not follow.
\item[49.] Hence one should know that spoken as a simple consequence, the consequence \textit{from the universal affirmative to its singular per accidens} is never valid. But the consequence \textit{from the universal negative to its singular per accidens}  - and \textit{to [its] singular per se} - always holds. For `no man runs, therefore John the grammarian does not run' follows. 
\item[50.] To the other proof: when `John the grammarian does not run, therefore a\footnote{Deleting Green-Pedersen's addition of \textit{aliquis}.} man does not run' is accepted, one should say that it does not follow. When it is argued that the consequence \textit{from an inferior to its superior with the negation placed after} holds, one should say that this is true spoken of what is inferior \textit{per se}, but is not true spoken of what is inferior \textit{per accidens}. But `John the grammarian does not run, therefore some man does not run' argues \textit{from an inferior per accidens to a superior with the negation placed after}. and so the consequence is invalid.
\item[51.] Hence one should know that the consequence \textit{from an inferior per accidens to a superior with the negation placed after} never holds.
\item[52.] One should know, then, that any proposition in which the sign of disjunction `or' is placed should be disambiguated according to composition and division. E.g. in `only a man runs or doesn't run', `runs or doesn't run' should be disambiguated for the reason that consequences assert the `or' [either] disjunctively (\textit{disiunctive}) [or disjunctly (\textit{disiunctim}). It is asserted disjunctly] if it holds disjunctly in the consequent, since it distinguishes (\textit{disiungit}) between terms. If it is asserted disjunctively, then it holds disjunctively [in the consequent], so it is understood as `only a man runs or only a man does not run'. And so the `or' holds disjunctively, since it distinguishes between propositions. But since the term `or' holds disjunctively, it forms a disjunctive proposition.
\item[53.] Similarly, the word `and' is sometimes held conjunctly (\textit{copulatim}) and sometimes conjunctively (\textit{copulative}). Conjunctly when it conjoins terms, and so forms a conjunction from conjoined extremes; but when it is held conjunctively it conjoins propositions, and so forms a conjunctive proposition.
\item[54.] I ask whether this consequence is good: `every man besides Socrates runs, therefore every man besides Socrates /f. 120ra/ or Plato runs'. I say that the consequent should be disambiguated according to composition and division; and it should be disambiguated thus: in the composite sense, it is understood as `every man besides Socrates or Plato [runs]'; in the divided sense, as `Every man besides Socrates runs or every man besides Plato runs'; and in either sense `every man besides Socrates runs, therefore every man besides Socrates or Plato runs' indeed follows.

I prove the consequence in the divided sense, since a disjunctive follows from either of its parts; so this consequence is good in that sense.

I prove the consequence in the composite sense: because a disjunction is superior to either part of a disjunction, and thus `Socrates' is inferior to `Socrates or Plato'; and because the exposition of the antecedent implies the exposition of the consequent. The affirmative implies an affirmative, since the middle is necessary; the negative implies a negative, because the consequence \textit{from an inferior to a superior with a negation placed after} holds.
\item[55.] I prove that the middle is necessary, because its opposite is impossible. For `every man other than Socrates or Plato is not a man other than Socrates' is impossible. And I prove that it is impossible, because from this the impossible follows, therefore it is impossible. Because `a man other than Socrates or Plato is not a man other than Socrates; therefore some [man] other than Socrates or Plato is not [a man] other than Socrates or Plato' follows. But the consequent is impossible, therefore the antecedent [is, too].
\item[56.] I prove the consequence: from the opposite of the consequent follows the opposite of the antecedent, since `every man other than Socrates or Plato is a man other than Socrates or Plato; therefore every man other than Socrates or Plato is a man other than Socrates' follows, because from the positing of the inferior follows the positing of the superior. And consequently `every man other than Socrates or Plato is a man other than Socrates' is necessary.
\item[57.] I show that the consequence `nothing besides a man is an animal, therefore nothing besides an animal is a man' is good. Because `nothing besides a man is an animal, therefore only a man is an animal' follows (because a negative exceptive in which the exception is from a transcendental\footnote{ms. \textit{transmutatione} (?) Cf. \cite[pp. 124-125, par. 158]{Green-Pedersen1980b}.} converts with an exclusive affirmative). And again `therefore only a man is a man' follows, because the consequence \textit{from a superior to an inferior on the part of the predicate, with an exclusive term occurring on the part of the subject} holds. And again `Therefore only an animal is a man' follows [because] the consequence \textit{from an inferior with an exclusive term immediately added} holds. And lastly `Therefore nothing besides an animal is a man' is given, because an affirmative exclusive [is equivalent to] an exceptive negative in which the exception is taken from a transcendental.\footnote{Reading Green-Pedersen's \textit{transcendente} instead of the ms. \textit{transmutatione}, as above.} Therefore \textit{from the first [to the last]}, `nothing besides a man is an animal, therefore nothing besides an animal is a man' follows.
\item[58.] It is asked whether the consequence `if a man runs an animal runs; [therefore] if no substance runs, no man runs' is good or not. [I prove] that the consequence is good by this rule: \textit{whatever entails (antecedit) the opposite of the consequent entails the opposite of the antecedent}. And because `no substance runs' [entails] the opposite of the consequent, which is `no animal runs', thus it entails the opposite of the antecedent, which is `no man runs'. 
\item[59.] I show that the consequence `you are other than a man, therefore you are not a man' is good. Because the consequence \textit{from an affirmative with an indefinite predicate to a negative with a definite predicate} holds. 
\item[60.] It is asked whether the consequence `nothing besides Socrates is an animal, therefore nothing besides Socrates is a white man' is good. I say that is [in]valid, because something entails the antecedent that does not entail the consequent; therefore the consequence is invalid. Because `Only Socrates is an animal, therefore nothing besides Socrates is an animal' follows; yet `only Socrates is an animal, therefore nothing besides Socrates is a white man' does not follow; because something /f. 120rb/ follows from the consequent that does not follow from the antecedent. `Nothing besides Socrates is a white man, therefore only Socrates is a white man' follows; yet `Only Socrates is an animal, therefore only Socrates is a white man' does not follow, because something entails the antecedent that does not entail the consequent; therefore the consequence is invalid. Because `Every animal is Socrates, therefore only Socrates is an animal' follows; but `every animal is Socrates, therefore only Socrates is a white man' does not follow; because something follows from the consequent that does not follow from the antecedent. `Only Socrates is a white man, therefore every white man is Socrates' follows; but `every animal is Socrates, therefore every white man is Socrates' does not follow, since the consequence \textit{from a universal affirmative to an inferior per accidens} does not hold.
\item[61.] Note that whatever follows from the consequent [with something added] follows from the antecedent with the same added. Because `every man is black, therefore no man is white' follows, thus whatever follows from `no man is white' with something added follows from `every man is black' with the same added. But because from `no man [is] white' with `every animal is white' added, `no animal is a man' follows, thus from `every man is black' with the same added, `therefore, no animal is a man' follows by this rule: \textit{whatever follows from the consequent with [something] added follows from the antecedent with the same added}.
\item[62.] It is asked whether the consequence `only every [man] is an animal, therefore only every man is white' is good. I say that `only every man is white' and any [proposition] like it should be disambiguated, for the reason that the word `only' can effect its exclusion either on account of matter or on account of form. If it effects [it] on account of form, then by effecting an affirmation it is said to remove the predicate from anything other than the subject in that very respect; and so it is understood as `only every man is white, therefore nothing other than \textit{man} is white'; so it includes opposites.\footnote{Cf. \cite[p. 120, par. 29-40]{Green-Pedersen1980b}.} If it brings about its effect on account of matter, then it is given\footnote{Following the ms. \textit{detur} rather than Green-Pedersen's \textit{dicitur}.} by effecting the affirmation that the predicate inheres in the subject under the mode `all', and is removed from anything other than man; hence these are exposited so: `every man is white, and\footnote{Reading \textit{et} instead of the ms. \textit{ergo}.} nothing other than a man is white'. 
\item[63.] It is asked whether the consequence `only Socrates is moving, therefore only Socrates is running' is good. I say that this consequence is invalid.

On the contrary: the consequence \textit{from a superior to an inferior on the part of the predicate with an exclusive term occurring on the part of the subject} holds; therefore the previous consequence was good.
\item[64.] Furthermore: a universal in transposed terms implies a universal, because `every moving is Socrates, therefore every running is Socrates' follows; from the distribution of the superior the distribution of the inferior follows; therefore the consequence is good.
\item[65.] To the first I say that the consequence `only Socrates is moving, therefore only Socrates is running' is invalid. To the proof \textit{from a superior to an inferior}, etc. I concede it for an inferior \textit{per se}, and not for an inferior \textit{per accidens}.
\item[66.] To the other: when it is accepted that a universal implies a universal, I deny [it]. To the proposition\footnote{Reading the ms. \textit{propositionem} rather than Green-Pedersen's \textit{probationem}.} that `every mover is Socrates, therefore every runner is Socrates follows, I say that [it is] not [so]. To the proof \textit{from the distribution of a superior}, etc., I concede it for an inferior \textit{per se}, and not \textit{per accidens}. But `runner' is inferior \textit{per accidens}; therefore the consequence is not valid, nor others like it. 
\item[67.] Concerning disjunctions and conjunctions: one should know that a disjunction either of whose parts is true is true, because a disjunction follows from either part, and if the antecedent is true, the consequent will be true. And so if part of a disjunction is true, the whole disjunction will be true, as `a man is white or a man is an animal' is true because one part of the disjunction is true.
\item[68.] Hence one should understand that any disjunction has two causes of truth. And the proposition `if a proposition has several causes of truth, it has one cause of truth'\footnote{Omitting Green-Pedersen's \textit{si} from the consequent.} is true. And if a disjunction has one true part and another false, that disjunction which has one /f. 120va/ cause\footnote{Omitting Green-Pedersen's \textit{veram}.} is true, and any cause implies that of which it is the cause.
\item[69.] One should know that a disjunction implies neither of its parts, because a disjunction has many causes of truth. And thus arguing to one of them is the fallacy of the consequent. Therefore arguing \textit{from a disjunction to either part} is the fallacy of the consequent.
\item[70.] One should know that for a conjunction to be true it is required that each part is true: because a conjunction implies each of its parts; and if the antecedent is true, the consequent will be true. Thus, if a conjunction is true it is required that each of its parts is true.
\item[71.] One should know that a conjunction follows from neither of its parts. For `Socrates runs, therefore Socrates runs and Plato runs' does not follow, because something follows from the consequent that does not follow from the antecedent; therefore the consequence is invalid. For `Socrates runs and Plato runs, therefore Plato runs' follows, but `Socrates runs, therefore Plato runs' does not follow.
\item[72.] One should know that the opposite of a disjunction is equivalent to a conjunction having parts contradictory to the parts of the disjunction. For example, the opposite of the disjunction `Socrates runs or Plato runs' is `not Socrates runs or Plato runs', which is equivalent to the conjunction\footnote{Reading the ms. \textit{copulative} rather than Green-Pedersen's \textit{copulativa}.} `Socrates does not run and Plato does not run', since the conjunction has parts contradictory to the parts of the disjunction `Socrates runs or Plato runs'.
\item[73.] And one should know that the opposite of this disjunction implies the negation of each part of the disjunction. For `not Socrates runs or Plato runs, therefore Plato does not run' follows. And so it is clear how the negation of a disjunction implies the negation of each part.
\item[74.] One should know that the opposite of a conjunction is equivalent to a disjunction having parts contradictory to the parts of the conjunction. For instance, the opposite of the conjunction `Socrates runs and Plato runs' is `not Socrates runs and Plato runs'; but `not Socrates runs and Plato runs' is equivalent to the disjunction `Socrates does not run or Plato does not run'; because the disjunction has parts contradictory to the parts of the conjunction `Socrates runs and Plato runs'.
\item[75.] Hence one should know that from the negation of a conjunction the negation of either part never follows, because the negation of a conjunction is equivalent to one disjunction, which implies neither of its parts. For instance, `Socrates does not run or Plato does not run, therefore Socrates does not run' [does not follow]. All these consequences are clear for the reasons given above.
\item[76.] Hence one should understand that the opposite of a conjunction has as many causes of truth as a disjunction convertible with it. And just as the consequence \textit{from the disjunction to either of its parts} is invalid, so the negation of a conjunction does not imply the negation of either conjunctive part.
\item[77.] Against where it is stated ``a disjunction [implies] neither of its parts'',\footnote{Par. 69.} and against where it is stated ``a conjunction does not follow from either of its parts''\footnote{Par. 71.} I argue thus: `Socrates runs or a man runs, therefore a man runs' follows, because the opposite of the consequent implies the opposite of the antecedent. For `No man runs, therefore neither Socrates runs nor a man runs' follows. And lastly `therefore not Socrates runs or a man runs' follows; the consequence is clear, because `not Socrates runs or a man runs' and `neither Socrates runs nor a man runs' are equivalent. `Socrates runs, therefore a man runs'\footnote{Deleting Green-Pedersen's addition of \textit{Socrates currit et} in the consequent.} follows. Therefore `Socrates runs' implies each part of the conjunction\footnote{ms. \textit{consequentiae}. If the ms. is followed, then in what follows the `and' (\textit{et})  should be changed to `therefore' (\textit{ergo}), and `conjunction' (\textit{copulativam}) should be changed to `conjunctively' (\textit{copulative}). But cf. par. 83.} `Socrates runs and a man runs', and consequently implies the whole conjunction.
\item[78.] To this it is said that the rule \textit{a disjunction implies neither of its parts} and the rule \textit{a conjunction does not follow from either of its parts} should be understood [of] when the parts of a disjunction and when the parts of a conjunction are not ordered as antecedent and consequent.
\item[79.] On account of this one should note that the consequence \textit{from a disjunction to its other part} always holds when the  parts of a disjunction are ordered as antecedent and consequent, when\footnote{Following the ms. \textit{quando}, rather than Green-Pedersen's \textit{quoniam}.} one part follows from the other part; just as when the parts are ordered as antecedent and consequent, the consequence \textit{from one part of a conjunction to the whole conjunction} holds.
\item[80.] One should know that the consequence does not hold when the parts are /f. 120vb/ so ordered indifferently, but only to the part that is the consequent.
\item[81.] And one should understand that only when the parts are ordered from the part which is the antecedent to the conjunction does the consequence hold. And as it is with a conjunction and disjunction, so should it be understood of the opposite of a conjunction (which is equivalent to a disjunction) and the opposite of a disjunction (which is equivalent to a conjunction).
\item[82.] From this to the first argument:\footnote{Par. 77.} when `Socrates runs or a man runs, therefore a man runs' is accepted, I say that it indeed follows, because `a man runs' is a consequent of `Socrates runs'; therefore from the disjunction to `a man runs' which is the consequent, the consequence holds. But `Socrates runs or a man runs, therefore Socrates runs' does not follow, because something implies the antecedent that does not imply the consequent; for `a man runs, therefore Socrates runs or a man runs' follows; but `a man runs, therefore Socrates runs' does not follow.
\item[83.] To the other argument:\footnote{Par. 77.} when `Socrates runs, therefore Socrates runs and a man runs' is accepted, I say that it indeed follows, because in this consequence the parts are ordered as antecedent and consequent; for `a man runs' is a consequent of `Socrates runs'. Hence when the parts of a conjunction are ordered as antecedent and consequent, then the consequence from the part of the conjunction which is antecedent to the conjunction holds. Then `Socrates runs, therefore Socrates runs and a man runs' does follow. But `a man runs, therefore Socrates runs and a man runs' does not follow, because something follows from the consequent that does not follow from the antecedent; for `Socrates runs and a man runs, therefore Socrates runs' follows; but `a man runs, therefore Socrates runs' does not follow.
\item[84.] On the contrary: I show that `Socrates runs, therefore Socrates runs and a man runs' does not follow, because from the opposite of the consequent the opposite of the antecedent does not follow; for `not Socrates runs and a man [runs], therefore not Socrates runs' does not follow, because the negation of a conjunction, as stated previously,\footnote{Par. 75.} does not imply the negation of either part.

I say that it indeed follows, and from the opposite the opposite follows. And when `\textit{from a negation of a conjunction}', etc. is accepted, I say that it is true unless the conjunctive parts are ordered as antecedent and consequent. Therefore, etc.
\end{enumerate}