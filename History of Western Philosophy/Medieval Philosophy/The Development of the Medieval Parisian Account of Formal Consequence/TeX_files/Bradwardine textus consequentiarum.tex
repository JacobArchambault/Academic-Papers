\documentclass[]{article}

%opening
\title{Text of consequences}
\author{Thomas Bradwardine}

\begin{document}

\maketitle

\begin{abstract}

\end{abstract}

\section{}
	\subsection{Definition of consequences}
\begin{enumerate}
	\item[1.] Concerning the knowledge of consequences, certain prerequisites must be prefaced.\footnote{put forth L.} To be set forth in the first place is the definition of consequences; in the second, their division, and then some conclusions to be included under them.
	\item[2.] Concerning the definition, note that a consequence is an argumentation composed out of an antecedent and consequent. `Argumentation' is put in the definition of consequence, because every consequence is taken up for establishing\footnote{producing V.} some argument. It is called `composite' because nothing incomplex is a consequence. `from an antecedent and consequent' is added, because in every consequence at least two categorical propositions are required.
	\item[3.] Indeed, from this definition it follows that every argumentation generally can be called a consequence, whether it be syllogistic or inductive or by example or enthymematic, because whatever precedes a final proposition brought forth can be called antecedent; specifically, though, no argumentation regulated by mood and figure is called a consequence.
\end{enumerate}
\subsection{Division of consequences}
\begin{enumerate}
	\item[4.] The division of consequence: one is necessary and formal, another is necessary and non-formal. Necessary and non-formal is that when the antecedent cannot be true without the consequence, retaining the proper\footnote{retaining the proper: retaining the same L.} signification of words, holding not by an intrinsic middle, but by some other necessary proposition.\footnote{rule L.} Hence, an intrinsic middle is a proposition composed from the subject of the antecedent and the subject of the consequent. For instance: this consequence is necessary and non-formal: `the stick stands in the corner, therefore God exists', since this consequent does not follow formally from the antecedent, and /f. 118ra/ still it is impossible for the antecedent to be true without the consequent; and it holds by the necessity by the general name: \textit{if the first cause does not exist, then it is impossible for another to exist}. A formal consequence is nearly <the same> as the aforementioned, but that such\footnote{nearly-such: that which} holds through an intrinsic middle, of which sort is `a man runs, therefore an animal runs', which holds through this middle `man is an animal'. 
	\item[5.] Another division of consequences is put forth by some, which is this: of consequences one [is] simple,\footnote{and necessary \textit{add.} L.} another\footnote{necessary \textit{add.} L.} as-of-now. A simple one, according to them, is when it is impossible for the antecedent to be true with the consequent being false, and this retaining the primary\footnote{\textit{om.} L.} signification of the words. But they call a consequence `as-of-now' when the antecedent for now can<not> be true <unless> the consequent be true, though it can be true at some time when the consequent is not true.
	\item[6.] But this division does not hold (\textit{non valet}), which is established so: if there were such a consequence, then from the merely possible the impossible would follow, namely, such as would be impossible with respect to whatever its significate. The consequent is false. Therefore. The falsehood is evident from Aristotle: positing the possible to be, nothing impossible follows,\footnote{\textit{Aristot. Anal. prior.} I,13,32a 19-20.} since just as from the true nothing follows except the true, so from the possible, etc. [i.e. nothing follows except the possible]. And this is established more clearly in that place by [this] reason: since positing only an ass runs, then so `Every running is an ass; every man is running; therefore, every man is an ass'. Taken for now, it is impossible for the antecedent to be true unless the consequent be true, since it holds by this true proposition: `every running is an ass', which is true by the case. But the antecedent is possible and the consequent impossible. It follows, then, that an as-of-now consequence is not valid, since [then] the impossible follows simply from the possible, which is against Aristotle.
\end{enumerate}
\subsection{Conclusions}
\begin{enumerate}
	\item[7.] These things set forth, some conclusions follow. First, that every consequence good simply is necessary. It is established: every simply true conditional is necessary. Every consequence good simply\footnote{\textit{om. L.}} is equivalent to one true conditional. Therefore, every consequence is necessary.\footnote{simply \textit{add.} L.} The antecedent\footnote{major L.} is granted by all good logicians, both ancients and moderns, since  `a man is, therefore an animal is' is equivalent to `if a man is, an animal is'.
	%`major L': should this be `minor'?
	\item[8.] Second: every good consequence is either a consequence as-of-now, or is a consequence necessary simply. But nothing\footnote{no good consequence L.}is a consequence as-of-now. Therefore, every consequence is necessary simply. The antecedent\footnote{major L.} is evident by exhaustive disiunction\footnote{\textit{om.} L.} and\footnote{\textit{om.} L.} division. The minor is established from the aforementioned.
	\item[9.] Against the conclusion: nothing that can be or not be is simply necessary. Every consequence is of this sort. Therefore. The antecedent\footnote{major L.} can be garnered from the words of Aristotle, \textit{Posterior Analytics} I: those are called `necessary' which cannot have themselves otherwise.\footnote{\textit{Aristot. Anal. post.} I,6, 74b 5-6.} The minor is established, because every consequence is mental, spoken, or written. But every such can be or not be.
	\item[10.] It is replied: I deny the consequence `nothing that can' etc., especially speaking of the necessity of the proposition, since that some proposition be necessary, it is not prevented\footnote{reading \textit{obstat} for \textit{oportet}.} that it could not be or be,\footnote{or be \textit{om.} L.} but suffices that it could not be unless it be true, no change of terms being made. And so to Aristotle's `those are called `necessary', etc., I say that Aristotle understands that some such proposition is necessary /f. 118rb/ which when it is true cannot be false except making a change of terms.
	\item[11.] Against this solution it is argued, and established that it is necessary for every proposition to be: everything that is when it is, it is necessary for it to be. Every proposition is when it is. Therefore, etc. The antecedent is evident from Aristotle, \textit{On interpretation} I.\footnote{\textit{Aristot. De interp.} 9, 19a 23-24.} The minor is established, because otherwise a plain contradiction would follow, namely that something would be when it would not be, which is impossible. The sequence is clear, since it is in the first mood of the first figure, etc.
	\item[12.] One replies by distinguishing the antecedent\footnote{major L.} according to composition and division. In the composite sense, the antecedent is false.\footnote{major L.} And so the `while it is' is added on the part of the subject, and expressed thus: `everything that is when it is, it is necessary for it to be'. In the divided sense it is true, and with the aforementioned addition added to the predicate: `everything that is, it is necessary for it to be while it is', then the antecedent is true. And the previous sequence does not hold, since more is predicated in the minor than is subjected in the major. 
	\item[13.] The second conclusion: that in every good consequence the antecedent can be true though the consequent is not true: Proof: sometimes the antecedent can be when\footnote{though L.} the consequent is not, the antecedent can be true, though the consequent is not true. But the antecedent of any consequence can be though the consequent not be. Therefore. The antecedent is evident, since every truth presupposes being. The minor is evident, since\footnote{The minor...since: the second(?) V.} when some are such that one is not part of the other, nor something which is part of the first part of the other, nor one of them the cause or effect\footnote{or accident or subject \textit{add.} L.} of the other, one can be with the other not existing. But the antecedent and consequent of any consequence are this way. Therefore.
	\item[14.] Against that conclusion it is argued: when two propositions are such that one can be true without the other, the truth of the one does not follow from the truth of the other.\footnote{The consequence holds \textit{add.} V.}. If, then, in every good consequence the antecedent can be true such that the consequent not be true, from the truth of the antecedent the truth of the consequent does not follow. The antecedent\footnote{major L.} is established, since the truth of the antecedent is related to the truth of the consequent just as the antecedent is related to the consequent. Therefore, it so happens that this\footnote{the aforementioned V.} rule, most common among all logicians, is proven to be false:\footnote{namely \textit{add.} L.} the consequence is good, and the antecedent is true, therefore the consequence is true.\footnote{Therefore \textit{add.} V.}
	\item[15.] One responds, conceding that the truth of the consequent does not follow from the truth of the antecedent. But still for juniors,\footnote{those coming into their third year L.} one should distinguish that something follows from another in two ways: either absolutely or conditionally. Absolutely, the truth of the consequent no more follows from the truth of the antecedent than the being of the consequent follows from the being of the antecedent, or conversely.\footnote{or conversely: \textit{om.} L.} But conditionally it does follow, as if it is stated: `if\footnote{\textit{om.} L.} this is true ``man runs'', then\footnote{therefore L.} this is true ``an animal runs'''. Furthermore, I say that the rule\footnote{\textit{om.} L.} `the consequence is good and the antecedent is true, therefore the consequent is true' is most necessary,\footnote{most necessary: good L.} since it is impossible in a good consequence that the antecedent be true unless the consequent be true, since the antecedent and consequent are relatives, which posited, posit each other, and taken away, take away each other. Therefore, it is not necessary that one adds the condition. But still, with this it holds good that the antecedent may true though the consequent may not be true. But when it is so that the antecedent is true without /f. 118va/ the consequent, then the antecedent is not.
	\item[16.] From these conclusions the falsehood of that saying of Ockham on consequences in the first chapter is evident, where he says that `Socrates does not run, therefore a man does not run' is formal, and holds by the intrinsic middle `Socrates is a man'.\footnote{\textit{Guil. Ockham, SL} III-3, c. 1, 50-54; cf. 23-28.}
	\item[17.] Against this: If that were good, it would be possible for two contradictories to be true with respect to the same significate. The consequent is false.\footnote{impossible L.} Therefore. The consequence is established, and I posit the case that every man runs and Socrates does not run. Then according to him, the consequence `Socrates does not run, therefore a man does not run', is good, and the antecedent is true, therefore also the consequent. And furthermore, `therefore some man does not run', since an indefinite and particular\footnote{singular V.} convert. But this is true by the case `every man runs'. Those two `every man runs' and `some man does not run'\footnote{therefore \textit{add.} L.} will stand together. And if it is said that this consequence is formal\footnote{as-of-now and not simply \textit{add.} L.}, then this is refuted above by the first conclusion. 
\end{enumerate}
\section{General rules}
\begin{enumerate}
	\item[18.] These things being held, some general rules pertaining to all consequences are to be set forth, of which the first is this: \textit{whatever follows from the consequent follows from the antecedent}. And this rule is taken from the words of Aristotle in the \textit{Categories}, where he says: when one is predicated of another, whatever is affirmed of the predicate, also [affirmed] of\footnote{\textit{om.} V, \textit{defecit} L.} the subject.\footnote{\textit{Aristot. Categoriae} 3,1b 10-11.} It is established by reason thus: if something follows from the consequence `an animal is a substance' which does not follow from the antecedent `man is a substance', then the same proposition would be true and false with respect to the same significate. The consequent is impossible. Therefore.  The consequence is established: Let \textit{a} be the proposition which follows from the consequent\footnote{consequent V, \textit{om.} L.} `an animal is a substance' and not from the antecedent\footnote{the antecedent \textit{add.} L.} `a man is a substance'. Then this: \textit{a} does not follow from man being a substance, therefore man being a substance can be true without \textit{a}. Therefore \textit{a} does not follow from an animal being a substance. And before, it was stated that \textit{a} does follow from an animal being a substance. Therefore the proposition `\textit{a} follows from an animal being a substance' is both true and false with respect to the same significate.
	\item[19.] Against this one argues in three ways: whenever any two are such that one of them can be true without the other, the truth of the one does not follow from the truth of the other. But the antecedent\footnote{antecedent \textit{add.} L.} and consequent\footnote{consequent \textit{add.} L.} are of this type. Therefore.
	\item[20.] Second: if the rule were true, it would follow that whoever would state a man to be an ass would state a truth. The consequent is false. The consequence is established: whosoever states a man to be an ass states a man to be an animal. Therefore this consequence is good: `Plato states a man to be an ass, therefore he states a man to be an animal. But whoever states a man to be an animal states a truth. Therefore, whoever states a man to be an ass states a truth.'
	\item[21.] Third: if the rule were good, then from one contradictory the other would follow. The consequent is impossible. Therefore. The consequence is established, and argued by the previously stated rule: `if no time is, it is not day. And if it is not day, it is night. And if it is night, some time is'. Therefore, from the first to the last: if no time is, some time is'. 
	\item[22.] To the first one responds by conceding the proposition that the consequent\footnote{consequent V, \textit{om.} L.} can be without the antecedent.\footnote{antecedent \textit{add.} L.} And furthermore, the truth of the consequent does not follow absolutely from the truth of the antecedent, but conditionally.
	\item[23.] To the second: both the antecedent and consequent are denied, since /f. 118vb/ `whosoever' etc. is false, since there is not an indicative proposition. `Whoever states a man' etc., is also false, since the two statements\footnote{two statements \textit{om.} V.} are distinct, and one can be true without the other; and if the statements of the propositions supposit for propositions, then the falsehood of the argument is more clearly evident, since it need not be that one saying the proposition `a man is an ass' says `a man is an animal'. 
	\item[24.] To the third, one responds by conceding the consequence `if no time' etc., since from the impossible anything follows, since\footnote{but L.} it is impossible, following the statement of Aristotle, that no time be. And consequently, anything\footnote{anything: any proposition L.} follows from it, whether it be true or false. In a like way from `the heaven is not' follows `a star is not, nor an ass'. Again, `A man is an ass' follows by the same rule.\footnote{In-rule: \textit{om.} L.} 
	\item[25.] Another rule: \textit{whatever entails (\textit{antecedit})the antecedent entails the consequent}. This rule follows from the preceding, for if the consequent `a man is' follows from the antecedent `Socrates is', and the consequent `an animal is' follows likewise from the same antecedent `Socrates is'; so that the consequence `Socrates is, therefore a man is' is good, so also `Socrates is, therefore an animal is'. 
	\item[26.] One can argue against this rule just as [against] the preceding, which\footnote{so it V.} converts with it, and in a like manner the arguments are resolved. 
	\item[27.] But against this: arguing thus - `If Socrates is,\footnote{therefore \textit{add.} V.} a man is' - here, nothing comes before the antecedent. Therefore the rule is false. 
	item[28.] Second: Let it be posited in the case that all beings of the world are a substance or an accident. Then one argues so: `an accident is inhering, therefore a substance is'. Here nothing comes before the antecedent, from which nothing is in the world besides these two. 
	\item[29.] One responds: `If Socrates is', I say that the rule ought to be restricted and understood conditionally, and in a good consequence where the antecedent of the antecedent is given. And it is not so here. 
	The second reason is resolved from the same root.\footnote{But-root: \textit{om.} V.}
	\item[30.] Another rule: \textit{from the opposite of the consequent follows the opposite of the antecedent}; as `if a man is, then an animal is; if an animal is not, then a man is not'. 
	\item[31.] Against this: if so, then this would be good\footnote{\textit{om.} V.} - `every man runs, therefore every animal runs'. For `no animal runs, therefore no man runs' follows, from which the falsehood of the consequence\footnote{consequent \textit{ed}.} appears. 
	\item[32.] One responds that the rule ought to be understood of the contradictory opposite, and not the contrary.Hence it is generally true, whether one argues the affirmatively or negatively, the contradictory opposite of the consequent always\footnote{\textit{uncertain reading} V, \textit{mut.} L.} conflicts with the antecedent.
	\item[33.] From this rule follows\footnote{is V.} another rule convertible\footnote{which is converted L, conditional V.} with it, that in no good consequence does the contradictory opposite of the consequent stand with the antecedent in truth, nor can it stand without a new assignment of terms. 
	\item[34.] Another rule following from these: whatever stands in truth with the antecedent stands, too, in truth with the consequent, just as whatever stands with the proposition `a man runs' stands with `an animal runs'. 
	\item[35.] But against this:\footnote{if \textit{add.} V, \textit{defecit} L.} `Socrates is' stands with the proposition `a singular is', and `man is' does not stand with this, since `man' is a universal term, and not particular.
	\item[36.] One responds that the rule is understood of terms suppositing subordinately with the same supposition. But in the reasoning, there is a difference in supposition.\footnote{But-supposition: \textit{om.} L.}
	\item[37.] The final general rule: what conflicts with the consequent conflicts with the antecedent, just as whatever conflicts with the proposition `an animal runs' conflicts with `a man runs'. 
	\item[38.] Against this: if the defined is, the the definition is. But to have only one term conflicts /f. 119ra/ with the consequent, therefore it conflicts with the antecedent; which is false, since every defined is only one term, and a definition includes several terms. 
	\item[39.] One responds: the rule ought to be understood of terms suppositing uniformly, or being taken significatively.\footnote{Against-significatively: \textit{om.} L.}
	
	And these are the general rules by which all consequences are established.\footnote{establish V.}	
\end{enumerate}
\section{Special rules}
\subsection{Rules by which an affirmative follows from an affirmative}
\begin{enumerate}
	\item[40.] Seeing these things, we should continue to special rules, the first chapter on which contains rules by which an affirmative follows from an affirmative.
	
	The first is that from a distributed superior to a distributed inferior is a good consequence, as `every animal runs,\footnote{is a substance L.} therefore every man runs'\footnote{is a substance L.- \textit{Cf. Guill. Ockham, SL} III-3, c.2, 9-11.}.
	\item[41.] But against this rule: it follows that this would be a good consequence: `every man runs, therefore every white man runs'. The consequent is false. The falsehood is evident, for the antecedent `every man runs' can be true without a new assignment of terms\footnote{of terms: of signification V.} apart from it that the consequent `every white man runs' be true. For positing that all men are black and none white, and that every man runs, the rule doesn't hold then. That last consequence is established by the general rule accepted above, namely that in no\footnote{every V.} good consequence does the opposite of the consequent stand with the antecedent.
	\item[42.] Second: if the rule were true, it follows that the consequence `every animal besides man runs, therefore every man runs' would be good. The consequent is false, therefore that from which it follows [is false]. The consequence holds, since here one argues \textit{from a distributed superior to an inferior} etc. The falsehood of the consequent is evident, since if it were good, contraries would stand together, namely\footnote{every \textit{add.} L.} `man runs' and `man does not run'.
	\item[43.] Third, it seems that the aforementioned rule holds in no matter, for if it were to hold in any matter, it would especially in this: `every animal runs, therefore every man runs'. But in that consequence it is not valid. Proof: positing the antecedent and consequent in the nature of things without a new assignment of terms, the antecedent can be true without the consequent. Therefore the consequence is not valid. The consequence is established, since positing that no man is, and that every animal runs, then the antecedent is true and\footnote{\textit{om.} V.} the consequent false. And this is strengthened by that instance `every man runs, therefore whatever is Socrates runs'. Here, one argues \textit{from a distributed superior} etc., and yet it is evident that the consequence is not valid, since\footnote{\textit{om.} V.} when Socrates does not exist, the antecedent is true and the consequent false. 
	\item[44.] To these,\footnote{this V.} Ockham\footnote{\textit{Guill. Ockham, SL} III-3, c.2,33-36.} says that the previously stated propositions are good as-of-now, just as `every man runs, therefore every white man runs'. 
	\item[45.] Against those statements it is argued: no consequence is good as-of-now which is not good simply. Therefore, the aforementioned consequences are not good as-of-now. The consequence holds\footnote{is evident L.} in itself. The antecedent is held from things stated above.
	\item[46.] Second: no consequence necessary simply is as-of-now. But the consequence `every animal runs, therefore every man runs' is necessary simply. Therefore it is not as-of-now. The antecedent\footnote{major L.} is granted by all. The minor is established, since in no way could the aforementioned consequence be bad unless it could happen that no man existed. But this is impossible. Therefore, etc. That this is impossible philosophically speaking is evident\footnote{in itself and \textit{add.} L.}  following the path of Aristotle, as shown in \textit{De Caelo} I: the world was from eternity,\footnote{\textit{Cf. Arist. De caelo} I,3, 270a 12-14.} and so every man is preceded by some man, and after every man is some man. It also is evident theologically, for God could not destroy himself, but he himself is a man, and consequently could not make it, /f.119rb/ retaining the same signification of words, that  no man existed.
	\item[47.] Here it is said that by his absolute power he could remove\footnote{leave behind L.} his human nature, and so would no longer be human\footnote{by nature \textit{add.} V.}, just as separating whiteness from Socrates in some way, Socrates is not white, for just as Socrates is not white except by reason of whiteness, so God is not man except by reason of the human nature united to him.
	\item[48.] Against this: it follows that a blessed\footnote{a blessed: from a good V.} could become not blessed\footnote{become not good: come a non-good V.}, and consequently that this proposition would be possible `what was blessed is not blessed'. This includes a contradiction, since perpetuity of duration is of the essence of blessedness. The consequence is established, since by the same reason by which the assumed nature could become not assumed, by that same reason a blessed nature could become not blessed; and it follows that God could bring all the saints to nothing, including the Blessed Virgin. All these things are absurd. It is evident, then, both philosophically and theologically speaking, that the consequence `every animal runs, therefore every man runs' is necessary. 
	\item[49.] Therefore,\footnote{To the other argument L.} one should say that the rule ought to be understood of an inferior \textit{per se}, since though white man is inferior to man, this still is not \textit{per se}, but \textit{per accidens}. 
	\item[50.] Furthermore: the rule ought to be understood in strictly affirmative categorical propositions, and not hypotheticals, nor those equivalent to hypotheticals. 
	
	By this the resolution of the argument `every animal besides man runs, therefore every man runs' is shown, since `every animal besides man runs' is exceptive, and equivalent to a hypothetical, since\footnote{`every...since \textit{om.} V.} it is equivalent to the conjunction `every animal other than man  runs, and man does not run'. 
	\item[51.] From the aforementioned it is clear that consequences like [this] are good: `every man is a being \textit{per se}, and every white man is a man, therefore every white man is a being \textit{per se}'; `every man is\footnote in a genus \textit{per se}, and a white man is a man, therefore white man\footnote{is-man: \textit{om.} V.} is in a genus \textit{per se}'; And if the premises are true, a true conclusion necessarily follows. And they can be reduced to the syllogistic form: `every man is in a genus \textit{per se},\footnote{every \textit{add.} L.} white man is a man, therefore\footnote{every \textit{add.} L.} white man is in a genus \textit{per se}'. Therefore.
	\item[52.] The second rule: from a convertible distributed to its convertible distributed is a good consequence, just as `every man runs, therefore every risible runs' follows, and conversely.\footnote{\textit{Cf. Guill. Ockham, SL} III-3, c.2, 69-71.}
	\item[53.] Against this: it follows that a man who is not would be a man. Which is established so: `every risible is a man; everything\footnote{risible-man: \textit{om.} V.} which can laugh is risible; therefore everything that can laugh is a man'. Then so: `Everything that can laugh is a man; but the Antichrist\footnote{the Antichrist: A man who is not L.} can laugh; therefore'. And consequently, a non-being would be a man. 
	\item[54.] Second: it follows that the consequence `that every man is a mortal rational animal is \textit{per se} in the first mode,\footnote{\textit{om.} V.} therefore that every risible is a mortal rational [animal] is\footnote{true \textit{add.} V.} \textit{per se} in the first mode'\footnote{\textit{om.} V.} would be good. The consequent is false. The consequence is granted, since it is established that the rule holds. 
	\item]55.] Following Ockham,\footnote{\textit{Guill. Ockham, SL} III-3, c.2,74-92.} it is responded that a proper passion can be taken in two was: broadly, inasmuch as it extends itself both to being and to non-being; and so it can supposit for non-being, just as for being. And taking it so, /f. 119 va/ then the consequence from a distributed subject to its distributed proper passion  is not valid. In another way, it is taken properly\footnote{it is taken properly: `proper passion' is taken strictly L.} inasmuch as it extends itself to beings only, and so the consequence is valid. 
	\item[56.] But that statement of Ockham\footnote{of Ockham: simply L.} is not valid. Proof: since nothing can supposit for non-being, because such a term suppositing such either supposits materially, simply, or personally. Not materially or simply, such as is granted by all. Nor personally, since no non-being is the significate of any term, since sign and significate, and all things of this sort, are relatives, and posited, posit each other, etc. And the whole of that is granted in the treatise on supposition.
	\item[57.] Besides, a proper passion does not extend itself further than that of which it is the proper passion. But risible is a proper passion of man. The major and minor are granted by Porphyry.\footnote{\textit{Porphyry, Isagoge} 12.13-22.}
	\item[58.] So\footnote{\textit{om. V.}} one speaks to the argument in another way, denying the proposition `everything which can laugh is risible' but only that which is born apt to laughing - as Porphyry states plainly, not what actually laughs, but what has an aptitude toward laughing, is risible.\footnote{\textit{Porphyry, Isagoge} 12.18-19.} And so that which is not is not risible. But risible is sometimes taken so broadly that it is equivalent to the term `what can laugh'; [taken] so, it is not a proper passion of man, and so not convertible with man. 
	\item[59.] To the second, Ockham says that the aforementioned rule must be understood of\footnote{such \textit{add.} V.} modal propositions taken in the divided, and not the composite, sense. Hence, one concedes all such consequences: `that every animal is a substance is necessary' therefore that every man is a substance is necessary'; `that every man runs is true, therefore that every risible runs is true'; `that every man is a mortal rational animal is \textit{per se} in the first mode, therefore that every risible is a mortal rational animal is \textit{per se} in the first mode'. But in the composite sense, he says the aforementioned consequences are not valid. The composite sense is this: `this is \textit{per se} in the first mode: ``every man is a mortal rational animal'', therefore this is per se in the first mode ``every risible is a mortal rational animal'''. 
	\item[60.] But that is not valid, for the aforementioned consequences are valid neither in the composite sense nor in the divided sense. That the are not in the composite sense he concedes. But that they are not valid in the divided sense is established, because\footnote{wherefore V.} retaining the signification of words, it is impossible for any \textit{dictum} to be true or false, necessary or impossible without a new assignment, since no infinitive clause is true or false. Every \textit{dictum} of a proposition is an infinitive clause. Therefore. It is clear, then, that all such propositions are false: `that man is an animal is true', `that God is is necessary', `that every man is an animal is \textit{per se} in the first mode', because whatever is \textit{per se} presupposes truth, whether it be in the first or in the second mode. 
	\item[61.] So one speaks to the argument in another way, that the aforementioned rule\footnote{aforementioned rule: proposition V.} ought to be understood of propositions concerning inherence with terms suppositing\footnote{taken L.} significatively, since if they were to be taken otherwise, the consequence would not be valid, for `risible is a proper passion of man, therefore man is a proper passion of man' does not follow. /f. 110vb/
	\item[62.] Another rule: from the distributed interpretation of a name\footnote{distributed-name: interpretation of a distributed name L.} to the [name] interpreted distributed is a valid consequence.\footnote{\textit{Cf. Guill. Ockham, SL} III-3, c. 2,54-56.} For example: `every lover of wisdom is a man, therefore every philosopher is a man'. And if it is said that a layman can love wisdom, and yet is not a philosopher, I say that `philosopher' is taken in two ways: in one way, `one who lives philosophically and wisely', and in this way a pure layman can be called a philosopher, taking `layman' for one who has not acquired knowledge by technique (\textit{artificialiter}); and the rule is\footnote{not \textit{add.} V.} understood in that way. In another way, one is called a philosopher who, beyond what was said, closes in on the perfection of the acquired science by technique. The rule is understood in the first way\footnote{is-way: is not understood thus L.}.
	\item[63.] Another rule: from a description distributed to what is described distributed is a good consequence,\footnote{\textit{Cf. Guill. Ockham, SL} III-3,c.2,50-53.} just as `every two-footed animal that can walk runs,\footnote{is V.} therefore every man runs' follows.\footnote{is V.} And the rule is to be understood as previously, when the terms supposit significatively. 
	\item[64.] Against this: `every irrational animal runs, therefore every ass runs'. Here one argues as instructed, yet the consequence is not valid, since positing that all irrational animals runs besides the ass <the antecedent is true and the consequent false.>
	\item[65.] Here I say that that description\footnote{definition V, \textit{defecit} L.} is not proper to to an ass as such, since `irrational' is something like a difference under another [difference], for specific differences are hidden from us.\footnote{Against-us: \textit{om.} L.} 
	\item[66.] Another rule: from a distributed definition to the [term] defined distributed is a good consequence, and conversely; just as `every rational mortal animal runs, therefore every man runs', and conversely.\footnote{consequence-conversely: \textit{om.} V.}
\end{enumerate}
%sicut: currently translated as `just as'; consider changing this to something more like `for instance'.
 %gressibile: animal that can walk - consider `plantigrade', or alternative translation
	%probatur: established, shown, proven - check for consistency across translations
	%dictum: phrase, statement
	%oratio: clause? \textit{oratio infinitiva}
	%does the use of theological proofs suggest a theologian, rather than an arts faculty member?
	%\textit{om. v} change to \textit{add.} L; fix text accordingly
\end{document}\textit{}
