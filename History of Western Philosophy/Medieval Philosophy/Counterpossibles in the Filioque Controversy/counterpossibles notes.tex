\subsection{Counterpossibles: Arguments against}
\printbibliography
\section{List of counterpossibles in Anselm's text}
\subsection{Chapter 1}
1. Necesse est ergo spiritum sanctum esse de filio, si potest monstrari filium non esse de illo. Nam si quis dicit non sequi fillum esse de patre et de spiritu sancto, idcirco quoniam unus deus est pater et spiritus sanctus, etiam si non opponitur aliud; aut spiritum sanctum esse de patre et de filio, /184/ quondam unus est deus pater et filius, quamvis non sit filius de spiritu sancto: consideret quia, cum est deus de deo, aut est totus de toto, aut pars de parte, aut totus de parte, aut pars de toto. Sed deus nullam habet partem. Impossibile igitur est ut sit deus de deo ut totus de parse, aut ut pars de toto, aut ut pars de parse. Necesse est igitur ut, si est deus de deo, totus sit de toto. Cum ergo filius dicitur esse de deo, qui est pater et spiritus sanctus, aut alius totus erit pater, alius totus spiritus sanctus, ut de toto patre sit et non de toto spiritu sancto; aut si idem totus deus est pater et spiritus sanctus, ex necessitate, cum est de deo toto, qui unus totus est pater et spiritus sanctus, est pariter de patre et de spiritu sancto, si non repugnat aliud. Eodem modo cum dicitur spiritus sanctus esse de toto deo qui est pater et filius, aut alius totus erit pater, alius totus filius, ut sit de toto patre et non de toto filio spiritus sanctus; aut cum est de patre spiritus sanctus, non potest non esse de filio, si non est filius de spiritu sancto. Nulla enim alia ratione potest negari spiritus sanctus esse de filio.

2 (against). Dicet aliquis: Si -- quoniam unus deus est pater et spiritus sanctus -- cum filius est de patre, sequitur eum esse de spiritu sancto, aut cum spiritus sanctus est de patre, quia idem deus est pater et filius, est etiam de filio: cum pater gignit fillum, necesse est eum gignere quoque spiritum sanctum, quia unus idemque deus est filius et spiritus sanctus; et cum spiritus sanctus procedit de patre, propter eandem unitatem deitatis filii et spiritus sancti procedit etiam filius de patre ita sicut spiritus sanctus. Si vero unites dei in filio et spiritu sancto non illam habet vim consequentiae, ut uterque similiter sit genitus et procedens: videtur quod non ex hoc, quia unus deus est pater et spiritus sanctus, sequatur fillum esse de spiritu sancto /185/ aut spiritum sanctum esse de filio, quondam idem deus est pater et filius, ut dicis.

3. Ad quod ego: Habent utique a patre esse filius et spiritus sanctus, sed diverso modo; quia alter nascendo, alter procedendo, ut alii sint per hoc ab invicem, ut dictum est; et ideo cum nascitur unus, non potest cum eo nasci ille, qui per hoc est alius ab eo, quia non similiter nascitur sed procedit; et cum unus procedit, nequit ille simul procedere, qui per hoc est alius ab illo, quia non similiter procedit sed nascitur; et ideo non habet hic unites illam vim consequentiae, quia pluralitas obuiat, quae ex nativitate nascitur et processione. Nam et si per aliud non essent plures filius et spiritus sanctus, per hoc solum essent diversi. Cum autem dico ex eo, quia pater est unus deus cum filio aut cum spiritu sancto, sequi fillum esse de spiritu sancto aut spiritum sanctum de filio: nulla ibi pluralitas generatur, quae obuiet unitatis consequentiae, quia non utrumque dico esse, sed alterum tantum.

4. Non enim est deus de deo, nisi aut nascendo ut filius, aut procedendo ut spiritus sanctus. Filius autem non nascitur de spiritu sancto. Si enim nascitur de illo, est filius spiritus sancti, et spiritus sanctus pater eius. Sed alter alterius nec pater nec filius est. Non ergo nascitur filius de spiritu sancto. Nec minus apertum est quia non procedit de illo. Esset enim spiritus eiusdem spiritus sancti. Quod aperte negatur, cum spiritus sanctus dicitur et creditur spiritus filii. Non enim poses, esse spiritus sui spiritus. Quare non procedit filius de spiritu sancto. Nullo igitur modo est de spiritu sancto filius. Sequitur itaque inexpugnabili ratione spiritum sanctum esse de filio, sicuti est de patre.
\subsection{Chapter 2}
(against)
Nam si non est hoc ipsum [sc. Spiritus Sanctus] de patre quod est: cum sit unus idemque deus qui pater, inveniri nequit unde alius sit a patre. Non enim inde alius est, quia pater habet fillum et spiritus sanctus non habet fillum. Per hoc enim probari possum quia alii sunt ab invicem, non tamen haec est cause ut diversae sint personae. Quippe si duo sint homines, quorum alter habeas fillum alter non: quamvis per hoc ostendi possint esse diversi, non tamen ob hoc sunt ab invicem alit; quia quoquo modo sese habeant in habendo vel non habendo fillum, diversitatem tamen non amittunt. Ita in patre et spiritu sancto non quia alius habet, alius non habet fillum, idcirco sunt diversi; sed quondam diversi sunt, ideo nihil prohibet eos in habendo et non habendo fillum esse dissimiles.

Similiter responderi potest, si ideo dicitur alius, quia non procedit ab illo spiritus sanctus, sicut ipse procedit a patre. Quippe ut secundum illos loquar qui negant spiritum sanctum de filio procedere: sicut non haec est cause quia spiritum sanctum filius de se procedentem non habet sicut pater, ut alius sit a patre -- sequeretur enim ut, si spiritus sanctus procederet de filio, non esset filius alius a patre -- ita per hoc non est spiritus sanctus alius a patre, quia non habet fillum aut spiritum de se procedentem sicut pater. Et quemadmodum filius non ideo est alius a patre, quia patrem habet et pater non habet patrem -- si enim pater haberet patrem, /187/ alius tamen esset a filio -- ita spiritus sanctus, quia de aliquo procedit et pater a nullo, non propter hoc est alius a patre; quia si pater de aliquo procederet, non tamen minus esset alius a patre de quo procedit. Palam igitur est quia non ideo est spiritus sanctus alius a patre, quia non habet fillum aut spiritum de se procedentem sicut pater; nec quia de aliquo procedit et pater de nullo.

Sed neque per hoc intelligi potest esse a patre alius, quia est spiritus patris, si de illo non habet esse. Potest enim intelligi aliquis alius ab aliquo, priusquam sit illius, quamvis alicuius nequeat esse, nisi sit alius. Ut cum dicitur homo dominus alicuius aut homo alterius hominis, prius intelligitur alius ab illo cuius esse dicitur, quam sit eius dominus vel homo. Sic itaque, si spiritus sanctus non est de patre, nihil prohibet eum prius alium intelligi ab illo, quam sit illius. Quare non facit illum alium esse a patre hoc, quia spiritus eius est, si per hoc non habet ut sit alius ab illo, per quod est spiritus illius; sicut filius per hoc alius est a patre, per quod est filius eius, quod non est aliud nisi quia ex ipso existit nascendo.

Sed si postquam fuit contigit illi alium esse a patre: cum non sit alia persona, nisi quoniam alius est ab illo, non fuerunt semper tres illae personae, quia ista non semper fuit, sicut non semper fuit spiritus sanctus alius a patre. Quoniam itaque falsa sunt haec, patet quia in existendo habet unde alius est.

Esse autem nequit nisi aut ex aliquo, sicut filius, aut ex nullo, sicut pater. Quod si ex nullo, quemadmodum pater existit: aut ita existit unus quisque per se ut neuter ab altero quicquam habeat, et sunt duo dii pater /188/ et spiritus sanctus; aut quondam unus deus sunt, si uterque de nullo est, penitus nihil inveniri valet in fide Christiana unde sint ab invicem alit, sed unus idemque est pater et spiritus sanctus et una persona; quae vera fides abhorret. Non est ergo verum spiritum sanctum a nullo esse. Si autem est ex aliquo, non est nisi ex deo, qui est pater et filius et spiritus sanctus. Sed a se ipso nequit esse, quoniam nulla persona a se ipsa potest existere. Quare si quis negat eum esse a filio, negare nequit illum a patre esse.

Qui ergo spiritum sanctum dicere uult ex sola processione alium esse a patre, licet non sit ex illo, intelligit aut id ipsum esse procedere de patre solummodo quod est mitti vel dari a patre, ut cum mittit vel dat eum pater, tunc tantum procedat a patre; aut hoc esse procedere, quod est esse de patre. Sed si idem est procedere quod est dari vel mitti, procedit pariter a filio sicut a patre spiritus sanctus, quondam ab illo similiter mittitur et datur. Item si non est aliud spiritui sancto procedere quam mitti vel dari, non est alius a patre nec procedit a patre, nisi cum datur vel mittitur; quod nemo puto intelligit. Semper enim est alius a patre spiritus sanctus, etiam ante creaturam; non autem datur vel mittitur nisi creaturae. Nec tamen dicendum est quod accidat ei dari vel mitti. Nam cum ipse sit ubique et immutabilis: accipienti quidem accidit aliquid, quia circa illum fit quod prius non erat et abesse potest; circa spiritum sanctum vero nihil fit quod non erat. Cum enim caecus in luce consentit lucem, nec magis nec minus habet lux aliquid; et si depulsa caecitate sentiat caecus lucem, circa illum fit motus, non circa lucem. Patet itaque non esse spiritum sanctum alium a patre per sic intellectam processionem, ut non sit aliud illi procedere quam dari vel mitti. Est ergo patens eum per processionem habere esse de patre et per hoc alium esse a patre, sicut filius non per aliud est a patre alius quam per hoc, quia de illo existit. Est ergo deus de deo et procedit de deo, quia et ipse deus et pater deus, de quo est et procedit. /189/

Nam si esse spiritum sanctum de patre est cause ut sit de deo: cum dicitur esse de patre, non est intelligendum quod sit de hoc, quod pater deus est, id est de divina essentia; sed de hoc quod deus pater est, id est de hoc unde refertur ad filium. Erit ergo divine essentia in spiritu sancto non de deitate patris, sed de relatione; quod stultissimum est dicere.

Quamvis et si hoc aliquis velit accipere, non minus sequitur spiritum sanctum de filio quam de patre procedere. Nempe nulla est relatio patris sine relatione filii, sicut nihil est filii relatio sine patris relatione. Si ergo altera nihil est sine altera, non potest aliquid de relatione patris esse sine relatione filii. Quare sequetur spiritum sanctum esse de utraque, si est de una. Itaque si est de patre secundum relationem, erit similiter et de filio secundum eundem sensum. Verum quondam nemo tam insipiens est qui hoc opinetur, credendum et confitendum est ideo esse de patre spiritum sanctum, quia est de deo. Non autem magis est pater deus quam filius, sed unus solus uerus deus pater et filius. Quapropter si spiritus sanctus est de patre, quia est de deo qui pater est, negari nequit esse quoque de filio, cum sit de deo qui est filius.
