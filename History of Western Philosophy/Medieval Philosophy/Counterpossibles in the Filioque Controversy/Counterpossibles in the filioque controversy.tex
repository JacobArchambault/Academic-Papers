\documentclass[]{article}
\usepackage[backend=biber]{biblatex}
\usepackage{amssymb}
\bibliography{jacob}
%opening
\title{Counterpossibles and normal defaults in the \textit{filioque} controversy}
\author{}

\begin{document}

\maketitle

\section{Introduction}
A counterpossible conditional, or \textit{counterpossible} for short, is a conditional proposition whose antecedent is impossible. The best-known formal logical systems embrace the principle that from the impossible anything follows (\textit{ex impossibili quodlibet}). This principle follows straightforwardly from a common criterion for a valid consequence: a consequence is valid exactly when it is impossible for its antecedent to be true and its consequent not true. The antecedent of a counterpossible cannot be true; \textit{a fortiori}, it cannot be true with its consequent not true. The principle was first formulated by the 11th century logician William of Soissons,\autocite{Martin1986} and was logical orthodoxy by the turn of the 14th century.\footnote{\autocite[10, par. 32]{Green-Pedersen1980a}, \autocite[128-129, par. 70]{Green-Pedersen1980b}, \autocite[pp. 61, 248]{BurleyDPAL}, \autocite[III. 3. 38, pp. 727-731]{OckhamSL}, \autocite[I. 8, 1st conclusion]{Buridan2015}. For a recent defense, see \autocite{Williamson2016}. Williamson's position is critiqued in \autocite{French2017}.}

\textit{Filioque}, Latin for `and the Son', is a word gradually inserted into Western versions of the Niceno-Constantinopolitan Creed from the sixth to the eleventh century, whereby the Creed is made to state that the Holy Spirit proceeds from the Father \textit{and the Son}. The \textit{filioque doctrine} is that expressed in that clause; the \textit{filioque controversy}, that which erupted between Eastern and Western Christian churches over the theological status of the doctrine, with Western churches upholding and Eastern churches rejecting it. The filioque doctrine was the principal theological reason for the \textit{Great Schism}, the split between Eastern Orthodoxy and Western Christianity, which continues today.

As a claim of Trinitarian theology, the filioque doctrine is, whether true or false, necessarily so. Yet, partisans of both sides of the issue put forth non-trivial conditionals in which the opponent’s assumption was assumed in the antecedent. An examination of treatments of this doctrine thus provides an excellent case study in the logic of counterpossibles as it was applied in medieval theology.

In the paper, I review the earliest medieval defenses of the doctrine in Anselm of Canterbury. In Anselm's practice, the criterion for what follows from an impossible assumption is a causal one: what follows from an impossible antecedent is what is caused by what is posited in it. Anselm's own contribution to the controversy seems to have been not so much the introduction of any new premise, but rather of a new method of treating information mutually agreed upon by Eastern and Western disputants: specifically, Anselm's treatment of Trinitarian theology represents an early foray into default logic. Thus, the mutual estrangement of eastern and western positions on the matter may not lie fundamentally in a change in dogma, but rather in a change in logic.
\section{A brief history of the \textit{filioque} in western theology}
The teaching that the Holy Spirit proceeds from the Son has its roots in the Trinitarian theology of Saint Augustine and Boethius. In his \textit{De Trinitate}, the former developed the claim that the Holy Spirit was to be identified with the mutual love of the Father and Son for each other, thus facilitating an imagery on which the Father and the Son appear, as in a fruitful marital union, as two distinct sources of a third person, consequent upon the others; the latter, in his own \textit{De Trinitate}, developed a theology of divine unity on which whatever was predicated of one divine person was also predicated of the others, excepting those predicates connoting the distinctive relations of the divine persons to each other. Augustine's theology provided much of the impetus for the adoption of the \textit{filioque} clause, while Boethius' provided the framework for scholastic defenses of it.

The earliest creedal formulation of the \textit{filioque} is adopted at the Third Council of Toledo (c. 589), where the clause was inserted in the fight against \textit{Arianism}.\footnote{At the time of the Arian controversy, it was commonly accepted that the Christ had existed prior to being conceived and born of a Virgin, but rather had an existence as the word of God. Arianism represented an early attempt to make sense of the relation of the Father to the Son prior to the latter's being incarnated in the flesh. Arianism held that Christ was less than divine, and akin to a kind of Platonic demi-urge, a created being who nevertheless served as a ruling intermediary between God and lower beings. Its distinctive theses were that Christ was of a like essence (\textit{homoiousios}) to the Father; and that the Christ was a created being. Hence, though Christ existed prior to being incarnate, there was nevertheless a time when Christ did not exist.} In the seventh century, Maximus the Confessor explains the \textit{filioque} as a mere semantic departure from the eastern doctrine, while the doctrine is rejected by John of Damascus in the eighth. The first person to launch an extended polemic against the doctrine appears to have been Patriarch Photius the Great of Constantinople in the ninth century.

The doctrine of the Spirit's procession from the son was ultimately adopted at Rome in 1014 under Pope Benedict VII, and has been theological orthodoxy in western Christianity ever since. In 1054, the doctrine was the main reason for mutual excommunication of patriarch Michael Cerularius of Constantinople and the pope, resulting in the Great Schism, continued to this day, the western, Catholic (and later, Protestant) churches from the Eastern, Orthodox churches. The doctrine was defended by Anselm of Canterbury in the eleventh century, Peter Lombard in the twelfth, Thomas Aquinas in the thirteenth, and reaffirmed dogmatically at the council of Lyon in 1274, whose attendees included luminaries such as Albert the Great and Bonaventure. The last extended, major ecumenical discussion between representatives of the Eastern Orthodox and Roman Catholic communions occurred at the Council of Florence, begun in 1438. There, all but one of the Orthodox representatives, the exception being Mark of Ephesus, agreed to an understanding of the \textit{filioque} clause on which the Spirit was said to proceed from the son as if from one source (\textit{tamquam ab uno principio}). After the council, however, Mark proved to be more representative of the mindset of the churches at home: the clause was neither adopted nor accepted. Fifteen years later, Constantinople was sacked and captured by the Ottoman Turks, forestalling further ecumenical discussions.

\section{Anselm's \textit{de processione spiritus sancti}}
\subsection{Background}
While Eastern polemics against the \textit{filioque} doctrine go back to the ninth century, serious discussion of the eastern position in the west does not get underway until the latter half of the eleventh, in the wake of the mutual excommunication of the Pope in Rome and Patriarch of Constantinople in 1054. One of the earliest extended discussions of the doctrine occurs in Anselm of Canterbury's \textit{De Processione Spiritus Sancti}, completed around 1098. Anselm had defended the doctrine in his earlier \textit{Monologion}, where the doctrine is supported by the Augustinian assimilation of the Spirit to the love between the Father and the Son.\footnote{\autocite[ch. 57]{AnselmMonologion}. What is perhaps more notable is that there, Anselm appears to treat the doctrine as one derivable from reason alone.} Anselm wrote the \textit{Monologion} while prior of Bec abbey, and apart from perhaps some of his letters, this was the first of his works to be circulated.\autocite{Sharpe2009} By the time Anselm wrote the \textit{De Processione}, by contrast, he was Archbishop of Canterbury, a sought after person and the respected author of a large number of works.\autocite[Bk. 2, ch. 5, par. 46-48]{VA} The treatise is an edited version of a speech Anselm made at the Council of Bari, the earliest such council to address the doctrine from the western perspective, in October 1098.

Anselm's treatise is structured as follows: chapter one of the \textit{De Processione} explains the stances of each party to the dispute, then enumerates the theological doctrines  the two parties held in common. Anselm then puts forward some logical assumptions pertaining to the treatment of the Trinity, and argues for the western position, that the Holy Spirit proceeds from the Father and the Son, from the aforementioned logical and common assumptions. Chapter two considers potential replies on behalf of the Greek position. Chapters three through six bring forth arguments from scripture for Anselm's position, while chapters seven through eleven state and reply to objections to these arguments from the Greek side, with chapters seven and eleven discussing scriptural arguments, and eight through ten, philosophical arguments. Chapter twelve provides an argument from the Holy Spirit's being spirit of the Son to his proceeding through the Son; chapter thirteen considers the Greeks' objection to the \textit{filioque} clause's addition to the creed. Chapters fourteen through sixteen summarize Anselm's arguments and results.

\subsection{Anselm's default Trinitarian logic}
In what follows, we leave aside Anselm's scriptural arguments to concentrate on the various non-scriptural proofs he gives for his position. 

At the basis of Anselm's argument lie several assumptions concerning the logic of divine oneness. According to Anselm, divine unity requires that whatever is predicable of God as such must be predicable of each of the persons of the Holy Trinity, with two exceptions: 1) the proper names of each person, and 2) those predicates signifying the essential relations of each person of the Trinity to the others. As we shall see, Anselm assumes these relations are irreflexive, antisymmetric relations of procession between the persons themselves, i.e. those of begetting and spiration. Anselm explains his position as follows:

\begin{quote}
	Now, in accordance with the property of God's oneness, (which has no parts) it follows that whatever is said about the one God (who is as a whole whatever He is) is said about the whole of God the Father, God the Son, and God the Holy Spirit, because each of them considered by Himself is wholly and perfectly God. But the aforementioned opposition of relation which originates from the fact that God is from God in the aforesaid two ways (1) prevents the Father and the Son and the Holy Spirit from being called by one another's respective name and (2) prevents the distinguishing properties of any one of them from being attributed to either of the others [...]. Thus, the oneness never loses its own consequence in a case where no opposition of relation stands against it; and the relation does not lose what belongs to it except in the case where the inseparable oneness stands against it.\autocite[469-470]{AnselmDeProc}
\end{quote}

Anselm goes on to provide examples of how the unity of God entails consequences not prevented by the plurality of relations, e.g. that each person is eternal; prevents consequences that would otherwise follow from the plurality, e.g. that there are three gods; and of how the plurality of persons prevents consequences that would otherwise follow from Divine unity, e.g. that the Son is the Father. Anselm writes the following, for example, about the matter of God's eternity:

\begin{quote}
	We say (1) that the one God is Father and is Son and is Holy Spirit and (2) that they are one and the same God whether they are spoken of singly or two at a time or all three together. Therefore, if God is eternal, then because of the oneness of deity it follows of necessity that the Father is eternal, the Son is eternal, and the Holy Spirit is eternal. And since whether considered one at a time or more than one at a time they are one God, there is only one eternal God. The consequence is similar if God is called creator or just or any of the other names which do not signify any of the aforementioned relations.\autocite[470]{AnselmDeProc}
\end{quote}

What is remarkable about Anselm's arguments is that they clearly represent an early foray into \textit{default} reasoning, a species of non-monotonic reasoning.\footnote{For introductions to non-monotonic reasoning, see  \autocite{Reiter1980,Horty2001,Makinson2005,Strasser2014}, as well as the collections in \autocite{Ginsberg1987,SchurzLeitgeb2005}. For non-monotonic reasoning in Anselm, see \cite{Archambault2017e}.}  For Anselm, consequences following from divine unity are predicable of each person of the Trinity, provided nothing prevents it.

\subsection{A basic overview of default logic}
Before going any further, it will be useful to provide a basic overview of non-monotonic reasoning, and default logic in particular.

Informally, a consequence relation is monotonic if whatever can be proven with fewer premises remains proven when more premises are added. A non-monotonic consequence relation is one where this may fail to be the case, i.e. where adding more information may lead to something previously proven ceasing to be so. 

Formally, a consequence relation $\vdash$ is monotonic iff for premise sets $\Gamma, \Delta$ and conclusion $\phi$, if $\Gamma \vdash \phi, \Gamma \subseteq \Delta$, then $\Delta \vdash \phi$; and non-monotonic if for some $\Gamma, \Delta, \phi$, $\Gamma \vdash \phi$, and $\Gamma, \subseteq \Delta$, yet $\Delta \nvdash \phi$. More plainly, a consequence relation is monotonic if whatever is proven by a set is proven by any of its supersets, and non-monotonic otherwise.

Informally, a default logic is one which, in addition to rules governing logical constants, also makes use of \textit{default rules}, whereby certain conclusions may be assumed, provided that other, perhaps distinct, assumptions, are not contradicted by existing information. Default logics are a kind of non-monotonic logic first developed by Raymond Reiter.

Formally, a default theory $\Delta$ is a pair $(W, D)$, where $W$ is a set of premises, and $D$ a set of \textit{default rules}, of the form $\gamma : \theta / \tau$. Here, $\gamma, \theta, \tau$ are sentences of the default language: $\gamma$ is called the \textit{pre-requisite}; $\theta$, the \textit{justification}, and $\tau$ the \textit{conclusion}, or \textit{result}. The intuitive interpretation of a default rule is that given $\gamma$, one may infer $\tau$, provided $\theta$ can be assumed without contradiction. An important kind of default rule, of the form $\gamma : \theta / \theta$, is called a \textit{normal default}, and has the intuitive interpretation that given $\gamma$, $\theta$ may be assumed barring information to the contrary.

From here, an \textit{Extension} $E$ of a default theory $\Delta$ is defined as $E = E_{0} \cup E_{1} \cup ... \cup E_{n} \cup ...$, where $E_{0} = W$, and $E_{n+1} = Cn(E_{n}) \mid cup \{\tau (\gamma : \theta) / \tau \in D$, where $\neg\theta \notin E$ and $\gamma \in E_{n}\}$, and $Cn$ is the consequence relation to which the logical rules of the language conform, e.g. classical logic. Alternatively, an extension of a default theory $\Delta$ is a fixed point arrived at by the repeated application of logical, then default, rules to $W$ then to those conclusion sets generated by doing so. 

From here, a consequence relation $\vDash$ for a non-monotonic theory may be constructed in one of two standard ways. The most common way, pursued in \cite{Reiter1980}, is that on which $\Delta \vDash \phi$ iff $\phi$ is in each of $\Delta$'s extensions. A second way, explored in \cite{Horty1994}, is that on which $\Delta \vDash \phi$ iff $\phi$ is in \textit{some} extension of $\Delta$.

\subsection{Anselm's use of normal defaults in the \textit{De Processione Spiritus Sancti}}
To make Anselm's argument more conspicuous, let $G$ be a first-order unary predicate meaning `God', `$p$', `$f$', and `$s$' names respectively referring to the Father (\textit{Pater}), Son (\textit{Filius}), and Holy Spirit (\textit{Spiritus Sanctus}). We assume the domain of discourse is restricted to the three persons of the Trinity. The argument of Anselm's \textit{De Processione} can be expressed as the following default theory $\Delta = (W, D)$. $D$ consists of all defaults of the form $\langle\lambda X.A_{X}\rangle(G):  \langle \lambda x.A_{x}\rangle (y) / \langle \lambda x.A_{x}\rangle (y)$. $W$ consists of (1) those formulas of the form $\langle\lambda X.A_{X}\rangle(G)$ where $\lambda X.A_{X}$ predicates some predicate true of God by virtue of the divine nature; and (2) those formulae specifying the self-identity of each divine person, and his distinction from the others, and (4) the following formulae specifying the relations between the divine persons
\begin{displaymath}
pRs, \space pRf, \space \neg \exists x (xRx),\space \forall x, y (xRy \rightarrow \neg yRx), \space \forall x, y ((xRy \vee yRx) \vee x = y) 
\end{displaymath}

In the above, we assume $R$ represents the relationship of \textit{procession} between the divine persons: $pRs$ thus states that the Spirit proceeds from the Father; $pRf$, that the Son proceeds from the Father; the next three state, respectively, that no person proceeds from himself, that procession is assymetrical, and that for any two persons, one is from the other. These three assumptions correspond to the common mathematical assumptions of irreflexivity, assymmetry and linearity.

In this way, the assumptions that the Father, Son, and Spirit are each eternal, creator, etc. follow, as Anselm says they do, from the default assumption that whatever is predicable of God is predicable of each of the Father, Son, and Spirit. Likewise, the assumption that the Father is the Son is present as the conclusion of a default (since being the son is predicable of God), but not of a default rule that can be applied, since $p \ne f \in W$.\footnote{See \cite[470-471]{AnselmDeProc}.}

Anselm then considers the claim that God is from God. He writes: 

\begin{quote}
God exists from God. Once this point has been accepted, then since the Father and the Son and the Holy Spirit are the same God, it follows in accordance with this identity that God the Father is both God from God and God from whom God exists. Likewise, the Son is both God from God and God from whom God exists. And the same thing holds true for the Holy Spirit. 

Now because of the previously cited opposition, the Father cannot exist from God. For God does not exist except as Father or Son or Holy Spirit, or as two or three of these together. And so, God the Father cannot exist from God unless either from the Father (i.e., from Himself) or from the Son or from the Holy Spirit, or from two or three of them together. But He cannot exist from Himself, because the one existing from someone and the someone from whom he exist cannot be identical. Nor does God the Father exist from the Son; for the Son exists from Him, and thus He cannot exist from the Son. Nor does God the Father exist from the Holy Spirit; for the Holy Spirit exists from the Father, and the Father cannot be that spirit which exists from Himself.\autocite[471-472]{AnselmDeProc}
\end{quote}

In the above passage, Anselm gives those inferences that would be drawn in accordance with the default rule, namely that God the Father exists from God, hence that God the Father exists from Himself, and from the Son, and from the Holy Spirit. Anselm then makes use of the assumptions of the irreflexivity and assymetry of the relation 'existing from' to show that the Father is neither from the Father, nor from the Son, nor from the Holy Spirit, and hence not God from God. With the Son however, nothing prevents the Son's being from the Father, and hence the default rule $\langle \lambda X.pRX \rangle(G): \langle \lambda x.pRx \rangle(f) / \langle \lambda x.pRx \rangle(f) $ goes through to allow the conclusion that the Son is from the Father.

Having laid down his theological assumptions, Anselm begins his discussion of the \textit{filioque} with the claim that 
\begin{quote}
	either the Son exists from the Holy Spirit or the Holy Spirit exists from the Son. Anyone who denies this claim must also deny either that (1) there is only one God, or that (2) the Son is God, or that (3) the Holy Spirit is God, or that (4) God exists from God.\autocite[473]{AnselmDeProc}
\end{quote}

The first of the above enumerated claims is only present in the form of the default schemata, that what is predicable of the divine nature is predicable of each divine person, provided nothing opposes it (otherwise, if something were predicable of one but not the other divine persons, excepting their constitutive relations, then one would have to conclude that there is more than one God).\footnote{Note the use here of a limited version of the principle of the Indiscernibility of identicals - i.e. that (divine) persons identical (in nature) share the same predicates (excepting those connoting their constitutive relations to each other).} The second follows from substituting the `the son' and `the Holy Spirit' in the first place in the claim `God is God'. Likewise, the fourth premise will also be found as the justification in normal default rules where the prerequisite and result substitute the name of a person on either the first or second occurrence of `God'. 

Given the premises and defaults in $\Delta$ above, two distinct families of extensions may be generated: one, in which the Holy Spirit is from the Son; the other, on which the Son is from the Holy Spirit.\footnote{If one were to remove the premise of the Son and Spirit's non-identity from the premise set, a third extension, on which the Spirit is identical to the Son, would also be possible. Thomas Aquinas will use this possibility to advance the claim that if the Holy Spirit does not proceed from the Son, the Spirit is not distinct from the Son. Anselm, by contrast, assumes that the differences in mode of procession are sufficient to distinguish the persons from each other. Cf. \autocite[Bk. I, d. 11, q. 1, a. 2]{AquinasSent}. Anselm is later followed in his position by Henry of Ghent. See \cite[V, q. 9]{HenryQuod}; \cite{Martin2004}.} That one or the other must hold follows straightforwardly from the assumption of the linearity of proceeding/existing from, and from repeated applications of the default rules instructing one to substitute the name of a divine person for an occurrence of `God' in the formal equivalents of the premise `God is from God'. 

In order to rule out the claim that the Son is from the Holy Spirit, Anselm reasons as follows: Having shown that the Spirit must proceed from the Son or vice versa, he then uses the distinction between the modes of procession, i.e. begetting and spiration, to argue that the Son does not proceed from the Spirit, as follows. If the Son proceeded from the Spirit, he would either be spirated or be begotten. If he were spirated, he would be the Spirit of the Spirit, which he isn't. If he was begotten, the Spirit would be his Father, which he isn't. Therefore, the Son does not proceed from the Spirit. Therefore, since one must proceed from the other, the Spirit proceeds from the Son.

\section{Eastern replies}
Anselm assumes that the partisans of the Greeks can only reply to his position in one of four ways: (1) deny that God is one, (2) deny that the Son is God, (3) deny that the Holy Spirit is God, or (4) deny that God is from God. Denial of any of these would be regarded as heretical. But elsewhere, Anselm considers alternative ways of replying to his position, which reveal that the disagreement may not be one of agreed upon dogma, but rather one of the logical manner of treating the dogma. Two passages in particular are worth reading, both at the end of chapter one. 

In the first passage, Anselm considers the following objection: 

\begin{quote}
	Now suppose someone says: `Even if nothing is opposed hereto, it does not follow that the Son exists from the Father and the Holy Spirit simply because the Father and the Holy Spirit are one God; or the Holy Spirit to be from the Father and Son, because Father and Son are one God, provided the Son does not proceed from the Holy Spirit.'\autocite[473, alt.]{AnselmDeProc}
\end{quote}

Here, the interlocutor is not questioning Anselm's premises, but rather the need to default to one or the other of the claims that the Holy Spirit proceeds from the Son. Anselm replies: 

\begin{quote}
Let this person consider that when God exists from God, then either (1) the whole exists from the whole, or (2) a part exists from a part, or (3) the whole exists from a part, or (4) a part exists from the whole. But God has no parts. Therefore, it is impossible that God exist from God as a whole existing from a part, or as a part existing from a whole, or as a part existing from a part. Thus, it is necessary that if God exists from God, the whole exists from the whole. Hence, when the Son is said to exist from God, who is Father and Holy Spirit, either (1) the Father will be one whole and the Holy Spirit will be another whole, so that the Son exists from the whole of the Father and not from the whole of the Holy Spirit, or else (2) if the Father and the Holy Spirit are the same whole God, then of necessity when the Son exists from the whole of God, which one whole is both Father and Holy Spirit, then the Son exists both from the Father and from Holy Spirit - provided nothing opposes this. 

In the same manner, when the Holy Spirit is said to exist from the whole of God, who is both Father and Son, either (1) the Father will be one whole and the Son another whole, so that the Holy Spirit exists from the whole of the Father and not from the whole of the Son, or else (2) when the Holy Spirit exists from the Father He cannot fail to exist from the Son, if the Son does not exist from the Holy Spirit. For on no other basis can the Holy Spirit be denied to exist from the Son.\autocite[473-474]{AnselmDeProc}
\end{quote}

In his reply, Anselm doubles down on the need to default to one or the other of the claims that the Holy Spirit proceeds from the Son, or vice versa. But he grounds the requirement of doing so on the assumption that each person proceeding from another does so on account of the divine essence, shared by the divine persons, rather than it being that those persons proceed from God because they are from one or another divine person. Hence, persons proceed from (or, generate, as the case may have it) other persons \textit{because} they proceed from the divine nature, rather than vice versa.

It is this apparent reversal of priority that was objected to in one of the earliest attacks on the \textit{filioque} doctrine, that of Photius of Constantinople in the ninth century.\footnote{See \autocite[Part 1, par. 6]{PhotiusMyst}.} At the same time, Anselm's adoption of what is effectively a default logic of the Trinity allows him a means of obtaining the claim that the Spirit proceeds from the Father, while denying various other claims that Photius and others stated followed from this prioritizing of the divine essence, e.g. that by parity of reasoning each person proceeds from every other, or that each proceeds from himself.

The second passage to be considered puts forth the following objection to Anselm's view:

\begin{quote}
Suppose that when the Son exists from the Father, then since the Father and the Holy Spirit are one God it follows that the Son exists from the Holy Spirit. Or if the Holy Spirit exists from the Father, then because the Father and the Son are the same God, the Holy Spirit also exists from the Son. When the Father begets the Son He must also beget the Holy Spirit, because the Son and the Holy Spirit are one and the same God.\autocite[474]{AnselmDeProc}
\end{quote}

The argument presented here is a \textit{reductio}, the antecedent of which would be assumed impossible by the person putting it forth. It is not, however, trivial. Rather, without getting into the question of whether \textit{ex impossibili quodlibet} should be accepted formally, the argumentative practice here clearly rejects its use in materially impossible cases. Rather, according to both sides of the dispute, what follows from a posit taken up in a \textit{reductio} argument is what is \textit{caused} by the posit. Here, the impossible assumption taken up is that the Holy Spirit proceeds from the Father \textit{because} he proceeds from God (and hence also proceeds from the Son). This assumption is then said to also entail that the Father begets the Spirit and Spirates the son.\footnote{To avoid this consequence, Anselm posits a primitive distinction between these qualities of begetting and spiration: `The Son and the Holy Spirit exist from the Father - but in different ways. the one by being begotten, and the other by proceeding, so that for this reason they are distinct from each other'.\autocite[474]{AnselmDeProc} Given, however, the common assumption that divine operations are not distinct from the divine essence, and hence neither are they distinct from each other, this reply seems less than promising.} 

Anselm himself makes use of the same assumption, i.e. that what follows in a \textit{reductio} posited is not anything (or, nothing, for that matter), but merely that which is caused \textit{per se} by the posit, in his discussion of the opinion that the Holy Spirit is distinguished from the Father by his not having a son:

\begin{quote}
Through this fact [i.e. that the Father has a Son, but the Holy Spirit doesn't] They [i.e. the Father and Spirit] can be \textit{shown} to be different from each other; nevertheless, this fact is not the \textit{reason} they are different persons. Indeed, suppose that there are two men, one of whom has a son and the other of whom does not. Although through this fact they can be shown to be different from each other, nevertheless this fact is not the reason they are different from each other. For no matter what their state is with regard to whether they have or do not have a son, they do not lose their differentiation. Thus in the case of the Father and the Holy Spirit the fact that the one has a son whereas the other does not is not the reason they are different; rather, because they are different nothing prevents them from being unlike with respect to having and not having a son.\autocite[476]{AnselmDeProc}
\end{quote}

Here again, the assumed criterion for what follows from an impossible antecedent is the same: what is caused by it. Since having/not-having a Son is not the cause of the Spirit's distinction from the Father, neither does the latter follow from it. As the analogy to human persons makes clear, the same criterion is used for both contingent and impossible cases.\footnote{This criterion ultimately goes back to Aristotle. For discussion, see \autocite{Castagnoli2016}.}

\section{Change of logic, change of dogma}
From the above, the following two points may be gleaned. 

First, both partisans and opponents of \textit{filioque} clause, in the early stages of the dispute, held that what follows from an impossible antecedent is neither nothing nor everything, but what is \textit{caused} by what is posited in the antecedent. This shared assumption formed the backdrop against which various disagreements about what \textit{actually} caused what arose. 

Second, despite Anselm's protests to the contrary, what allowed Anselm to posit the \textit{filioque} was not merely a development in dogma, but a change in the logical treatment of it. The use of normal defaults in Trinitarian logic was an important and far-reaching innovation, and it seems to have been precisely on this point, rather than any particular theological opinion, that the Western position came to move away from the Eastern one. Thus, Anselm's argument in the \textit{de Processione} is informative not only for the theological opinions formed therein, but also for the \textit{way} that these conclusions are arrived at: here the change of dogma is, at least in part, attributable to a change in logic. 
\printbibliography
\end{document}
