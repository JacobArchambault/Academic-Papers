\documentclass[a4paper, 11pt]{article}
\usepackage{amssymb}
\usepackage[T1]{fontenc}
\usepackage[utf8]{inputenc}
\usepackage{lmodern}
\usepackage[backend=bibtex, style=verbose]{biblatex}
\bibliography{jacob}

\title{Grounding medieval consequence}
\author{Jacob Archambault}

\begin{document}

\maketitle


\begin{abstract}
Developed out of earlier work on Aristotelian topics, syllogistic, and fallacies, by the early fourteenth century the medieval theory of consequence came to provide the first unified framework for the treatment of inference as such. With this development came the task of unifying the various justifications for inferences treated in earlier frameworks. 

Prior to the appearance of theories of consequences, the task of providing a real foundation, or grounding, for good inferences is shared between theories of demonstration, such as those provided in commentaries on Aristotle’s Posterior Analytics, and theories of topical inference, passed on to the medievals via Boethius. But by the time of the earliest \emph{consequentiae}, most consequences were grounded in the theory of supposition, which began its own development in the twelfth century. 

Secondary literature on supposition has generally held that in the most common form of supposition, personal supposition, a term is taken to stand for individuals falling under it. In this paper, I show that for the earliest \emph{consequentiae} this is false: prior to William of Ockham’s work, personal supposition could also involve descent to concepts or types falling under a term, previously thought to be the exclusive provision of simple supposition. As such, a greater variety of ways of grounding consequence exists in the period than has hitherto been recognized.

\textbf{Keywords:} consequence, supposition, grounding, Boethius, Walter Burley, John Buridan
\end{abstract}
\section{Introduction}
Grounding is an irreflexive, assymetric, and transitive relation between different elements within an ontology, according to which one or several elements serve as grounds for another \autocite[364]{Schaffer2009}. The notion of grounding is meant to capture that expressed in non-causal uses of phrases like `because' or `in virtue of' \autocite[6-7]{Schnieder2019}, and commonly invoked examples of grounding include that of true statements in facts, of true non-atomic propositions in their atomic constituents, of qualities in substances, and of sets in their members. On one currently widespread understanding of metaphysics, the basic problem of metaphysics is to determine what beings are fundamental, and how other beings are grounded in them.\footnote{The contemporary resurgence of interest in grounding can be traced back to Jonathan Schaffer's 2009 `On What Grounds What', though Schaffer's article has important antecedents in Kit Fine's work on essence and Saul Kripke's on truth. See  \autocite{Fine1994,Fine2012,Kripke1975,Schaffer2009,Schaffer2016}. For an overview of the notion of grounding in contemporary literature, see Magali Roques' introduction to this volume.} 

In its broadest sense, a consequence is a relation obtaining between an antecedent and a consequent, signified by a sign of consequence \autocite{StrodeConsequentiis,Green-Pedersen1980a}; an antecedent, a premise or set of premises from which a consequent follows; a consequent, a conclusion following from an antecedent; a sign of consequence, a word, symbol, or phrase signifying a consequent's following from an antecedent, e.g. `if', `therefore', `because' and their analogues in other natural languages, the $\vDash$ of model theory, or the proof-theoretic $\vdash$. Though elements that would enter into the theory may be found in earlier work on Aristotelian syllogistic, fallacies, and topical argument, the theory of consequence proper first arises in the later medieval period. 

In what follows, I explain how consequence was grounded in the medieval period. I begin with a basic introduction to how consequence is grounded today, indicating its similarities to the best-known approach to formal consequence in later medieval philosophy, that of John Buridan. I then show how first medieval theories of topical argument, then of supposition served to ground earlier medieval accounts of inference, detailing two important shifts: the first from a non-unified, topics-based grounding of consequence to one based on the theory of supposition; the second, within the supposition-based theory, from an understanding of personal supposition licensing appeal to concepts to one based wholly on a sparse ontology of individuals. Over time, these changes in the foundations of the medieval theory of inference accelerate a shift in its focus, one away from a characteristically medieval concern with sound and demonstrative arguments towards a more modern focus on formal validity.

%Needs to explain grounding of modern consequence in domain models in order to provide a contrast case for medieval grounding.
\section{Grounding consequence in modern and later medieval logic}
\subsection{Grounding consequence today}
Since the early 20th century, research on consequence has centered on formal consequence, commonly identified with logical consequence. Such a consequence is usually defined either proof-theoretically, e.g. as one where the consequent follows from the antecedent via the strict application of the rules of a given proof-system,\footnote{Cf. \autocite{Prawitz1985,Schroeder-Heister2006,Hjortland2009,Franks2010}} or model-theoretically as one where every model of the antecedent is a model of the consequent \autocite{Tarski2002,Gomez-Torrente2000}. Though proof-theoretic definitions of consequence have become more popular in recent years, model-theoretic definitions remain the most common way of defining consequence today.

Modern model-theoretic definitions of consequence trace their lineage back to the groundbreaking 1930s work of Alfred Tarski. The most philosophically uncompromising attempts to ground consequence, e.g. that of Tarski himself, take the domain $D$ of all existents as their foundation and, given a formal language $L$ with a stock of non-logical constants as names for certain elements in the domain, determine whether a sentence $A$ follows from a class of sentences $\Gamma$ by obtaining  from $\Gamma \cup A$ a class $K$ of \emph{sentential functions}: functions which are like sentences, but replace like non-logical constants naming various objects with like variables. These in turn are said to be \emph{satisfied} (or not) by sequences of objects called \emph{models} which, when taken as arguments of said functions, evaluate to true (or false). On the Tarskian account, if every sequence of objects satisfying the sentential functions obtained from a collection of sentences $\Gamma$ in a language $L$ at the same time satisfies that obtained from a sentence $A$, then $\Gamma \vDash A$ is valid. 

More recent model-theoretic definitions of formal consequence have built on Tarski's definition in various ways. Canonical accounts of consequence have varied the size and members of the domain, as well as the interpretation of non-logical constants since at least the mid 1950's,\footnote{This innovation apparently goes back to John Kemeny. See \autocite{Schiemer2013,Kemeny1956,Kemeny1956b}.} and have added apparatus to handle phenomena including tense and possibility since the 1960's \autocite{Kripke1963a}. 

Across their various differences, modern model-theoretic definitions of formal consequence use the above described approach to ground every valid consequence of a formal language in a domain of actual existents as a whole, without depending for their validity on any existents in particular. And while there is widespread debate over the correct semantics for logical constants, over whether and how to include the referents of tense and modal sentences within domain models, and over the nature and existence of such referents, the broader framework within which these debates occur, with the priority it accords to formal consequence and domain models built out of existing entities as its ultimate foundation, remains widely accepted.

This centrality of formal consequence to modern thinking about the notion generally witnesses to that thinking's dominant aim: that of providing a theory of \emph{validity}, i.e. of which inference forms hold universally, regardless of their material content.\footnote{Cf. \autocite{DutilhNovaes2011,Etchemendy2008,MacFarlane2000}.} 
\subsection{Formal consequence in later medieval logic}
The Tarskian approach to formal consequence outlined above is a modification of an earlier candidate Tarski rejects, condition (F), presented as follows in his `On the concept of Following Logically':

\begin{quote}
	If in sentences of the class $K$ and in the sentence $X$ we replace the constant terms which are not general-logical terms correspondingly by arbitrary other constant terms (where we replace equiform constants everywhere by equiform constants ) and in this way we obtain a new class of sentences $K'$ and a new sentence $X'$ , then the sentence $X'$ must be true if only all sentences of the class $K'$ are true \autocite[183-184]{Tarski2002}.
\end{quote} 

(F), in turn, is nearly identical to a medieval account of consequence common at the University of Paris and other continental European universities from the fourteenth century. John Buridan, the most influential of the early fourteenth century Arts Masters at the University of Paris, gives his version as follows: 

\begin{quote}
A consequence is called formal if it is valid in all terms retaining a similar form. Or if you want to put it explicitly, a formal consequence is one where every proposition similar in form that might be formed would be a good consequence, e.g. `That which is A is B, so that which is B is A.'\footnote{\autocite[23-24]{BuridanTC}: `Consequentia `formalis' vocatur quae in omnibus terminis valet retenta forma consimili. Vel si vis expresse loqui de vi sermonis, consequentia formalis est cui omnis propositio similis in forma quae formaretur esset bona consequentia, ut `quod est A est B; ergo quod est B est A'. Translation taken from \autocite[68]{Buridan2015}. Cf. \autocite[105]{Pseudo-Scotus1891}.}
\end{quote}

Buridan goes on to hash out propositions `retaining a similar form' to another proposition as those only differing from it in their \emph{categorematic} terms - those terms whose primary function is to signify some being, in contrast with \emph{syncategorematic} terms like `and' and `not' which provide structure to other elements in a proposition. He writes: 
\begin{quote}
	I say that when we speak of matter and form, by the matter of a proposition or consequence we mean the purely categorematic terms, namely, the subject and predicate, setting aside the syncategoremes attached to them by which they are conjoined or denied or distributed or given a certain kind of supposition; we say all the rest pertains to the form. So we say that the copulas of both simple subject-predicate and compound propositions pertain to the form, as do negations, [other] signs, the number of propositions and terms and the mutual relation of all these, and relations of anaphoric terms and modes of signifying pertaining to the quantity of the proposition, for example, whether discrete or general, and many other things that the attentive reader can recognize if they occur.\footnote{\autocite[74]{Buridan2015} = \autocite[30.7-18]{BuridanTC}: `Et dico quod in proposito, prout de materia et forma hic loquimur, per `materiam' propositionis aut consequentiae intelligimus terminos pure categorematicos, scilicet subiecta et praedicata, circumscriptis syncategorematicis sibi appositis, per quae ipsa coniunguntur aut negantur aut distribuuntur vel ad certum modum suppositionis trahuntur; sed ad formam pertinere dicimus totum residuum. Unde copulas tam categoricarum quam hypotheticarum propositionum dicimus ad formam pertinere, et negationes, et signa, et numerum tam propositionum quam terminorum, et ordinem omnium praedictorum ad invicem, et relationes terminorum relativorum, et modos significandi pertinentes  ad quantitatem propositionis, ut est discretio et communitas, et multa quae diligentes possunt videre si occurrant.}
\end{quote}  

\subsection{Comparing Buridan and post-Tarskian groundings of consequence}
Buridanian and broadly post-Tarskian traditions define formal consequence by partitioning the syntactic elements of a consequence into formal and material, varying or leaving the interpretation of the material elements of a consequence indeterminate so as to focus on what is demanded by the interpretation of its formal elements \autocite[314-321]{DutilhNovaes2011}. While there are important differences between the two traditions - for Buridan, for instance, this partitioning takes place at the level of the individual inference, whereas for Tarski and his successors it takes place at the level of the language,\footnote{For an in-depth comparison of Buridan and Tarski's accounts, see \autocite[ch. 2]{Archambault2017d}.}  - both traditions share a common interest in grounding formally valid consequence in a (preferably sparse) domain of individuals. 

It may come as a surprise, then, that this focus on formal validity only arises fairly late in medieval logic - instead, from its infancy in Boethius, through its first mature formulation in the time of Peter Abelard, to the early fourteenth century work of Walter Burley, the theory of consequences focuses less on formal validity than on high-level categorization of the sources of materially correct arguments.\footnote{Cf.  \autocite{Martin2004,Martin2018,Archambault2018a}.} One can better understand the centrality of formally valid consequence to logic today by understanding the focus it replaced. And to do this, it'll be useful to further inspect its font, the logic of Boethius.

\section{The sources of medieval \emph{consequentiae} in hypothetical syllogistic and topical argument}
Among the earliest material which we might classify under the banner of consequence,\footnote{Early medieval logicians themselves did not: the earliest treaties dedicated to the concept of consequence as such don't appear until turn of the fourteenth century. See \autocite{Archambault2018a}.} one finds work on categorical syllogisms, hypothetical syllogisms, topical argument, and fallacies. Though neither `\emph{consequentia}' nor its variants occur in Boethius' \textit{On the Categorical Syllogism}, the terms occur frequently in Boethius' \emph{On Differential Topics} and especially his \emph{On Hypothetical Syllogisms}. In what follows, we'll first introduce the theory of topics later medievals inherited through Boethius, then we'll show how the theory was applied in Boethius hypothetical syllogistic.

\subsection{An overview of Boethian topical argument}
\subsubsection{The Boethian concept of an argument}
Following Cicero, Boethius defines a topic as `the seat of an argument', and an argument as `a reason granting credence to a doubtful matter'.\footnote{[1174C-D]\autocite{BDT}: `Argumentum est ratio rei dubiae faciens fidem \ldots Locus autem est sedes argumenti.'} In his discussion of the term `argument', Boethius distinguishes argument in the sense of \emph{what} is put forth (\emph{argumentum}) from argument in the sense of a speech putting forth an argument (\emph{argumentatio}) \autocite[1174C]{BDT}. Boethius means something different by the term than what we might mean by it today: he states, for instance, that a conclusion is a proposition established by arguments (note the plural) \autocite[1180C]{BDT}; that `if something is added to any thing, the whole is made greater' is an example of an argument that is both necessary and plausible (\emph{probabilis}), and argues, against the claim that all arguments must be plausible, that more advanced theorems of geometry may still serve as \emph{argumenta} for conclusions that follow from them even to one who has failed to grasp their necessity \autocite[1181A-C]{BDT}. 

\subsubsection{Maximal propositions as grounds of consequence}
Boethius clears up this initial opaqueness via his exposition the concept of a maximal proposition. For Boethius, the term \emph{argumentum} is ambiguous between two meanings: that of a maximal proposition, and that of the difference of such a proposition. By the first of these, he means a general proposition whose truth is grasped immediately, e.g. `equals subtracted from equals are equal', which is used to derive more specific claims less sure than it; or, in the absence of a proposition whose truth is immediately known, one whose plausibility is sufficiently vouched for, e.g. by an artisan within his proper domain. In its second sense, an argument is that feature, thing, or relation principally appealed to in the formulation of a maximal proposition, e.g. equality.\footnote{\autocite[1185A-B]{BDT}. Cf. \autocite{Holopainen2007,Archambault2017e}} 

Since, then, a topic is the seat of an argument, it is understood as the seat of a maximal proposition, and thereby of the difference of such a proposition. As such, the Boethian theory of topics aims to provide an account of the fundamental propositions to be appealed to in an argument, and the real features on account of which these hold in whatever way they do. It is in this way that the earlier medieval theory of topics serves as a theory for grounding consequence.

\subsubsection{The different kinds of topical argument}
The above will be better understood by examining Boethius' division of topics with some examples. Following the Aristotelian commentator Themistius, Boethius divides topics into intrinsic, extrinsic and middle topics, in accordance with the kinds of media, or \emph{middles}, they employ to reach a given conclusion. In an intrinsic topic, the argument proceeds by eliciting some property, description, or relation belonging to what is signified in the minor term to confirm or remove what is suggested of it by a major term. Types of intrinsic topics include arguments from the predication of a species to the predication of a genus (e.g. `Adam is a man, therefore he is an animal'), from a predication of a description to what follows from it (e.g. `God is that than which nothing greater can be thought, therefore God exists') or from the existence of a whole to that of its parts (e.g. `If there's a house, there's a roof, walls, and a foundation'). In an extrinsic topic, by contrast, one exploits a relation that the subject term of the desired conclusion bears to some other concept, constructs an argument about the latter, and thereafter leverages the original relation to infer something about the original subject one was inquiring about. Types of extrinsic topics include arguments from authority, from analogy, from various kinds of opposites, and \emph{a fortiori}/\emph{a minori} arguments.

\subsection{Boethian topical theory as ground of hypothetical syllogistic}
With this laid down, we can begin to inspect Boethius' main source of his remarks on consequence, his \emph{On Hypothetical Syllogisms}. 

Though Boethius does not identify consequences with conditionals, he does strongly associate these concepts. Boethius takes the Latin `\emph{conditionalis}' to mean the same as the Greek `\emph{hypothetica}'.\footnote{\autocite[1.3.2]{BHS}. Since Boethius counts not only conditionals in the modern sense, but also disjunctions among hypothetical propositions (on the grounds that `A or B', taking `or' to exclude the possibility that both disjuncts are true, is equivalent to `If A, then not B'), his understanding of `conditional' must differ from the syntactic definition employed today.} He distinguishes conditional from categorical propositions, stating that `in a conditional, the nature of a consequence' rather than that of a predication, `is assumed from the condition',\footnote{\autocite[1.1.6]{BHS}: `Primum igitur dicendum est quod praedicatiua propositio uim suam non in conditione sed in sola praedicatione constituit, in conditionali uero consequentiae ratio ex conditione suscipitur'.} 
and further distinguishes consequences that follow only accidentally 
- his example is `if fire is hot, the heavens are round' - 
from those, such as `if it's a man, it's an animal' that follow naturally \autocite[1.3.6-1.3.7]{BHS}. 

In both the \emph{On Hypothetical Syllogisms} and the \emph{On Differential Topics}, 
Boethius divides the hypothetical proposition into four types according to the quality of its antecedent and consequent: 
a positive proposition following from a positive, 
a negative from a negative, 
a negative from a positive, 
and a positive from a negative.\footnote{\autocite[1.3.5]{BHS} \autocite[1176B-C]{BDT}} Since it applies solely to hypothetical propositions whose immediate parts are atomic, the division is incomplete,\footnote{The division has also been taken to suggest that Boethius conflates the denial of a consequent with the denial of a consequence as a whole \autocite[157-158]{Martin2007}.} and even when restricted to these appears trivial at best. It's not. Rather, the intent of this fourfold division can be seen from examining Boethius' use of it in light of his theory of topical argument.

In discussing affirmations following from affirmations, Boethius writes the following:
%Example 11 suggests that the text I'm using is corrupt.
\begin{quote}
Now in order for one thing to precede and another to follow, in just these things it usually comes about as I recounted a little earlier. For a [1] genus, or [2] difference, or [3] definition, or [4] property, or [5] an inseparable accident follows from the species. Again, a species follows from the [6] property or [7] definition, the [8] difference and [9] definition follow from a property, and a [10] property or [11] difference follows from the definition in this way: for example, [1] if there's a man, there's an animal; and [2] if there's a man, it's rational; and [3] if there's a man, there's a mortal rational animal; and [4] if there's a man, it's risible; and [5] if there's an Ethiopian, he's black. [6] If [something] is risible, it's a man; [7] if there's a mortal rational animal, there's a man. [8] If [something] is risible, it's rational; [9] if [something] is risible, it's a mortal rational animal; if there's a mortal rational animal, it's [10] risible or [11] two-footed.\footnote{\autocite[1179A-B]{BDT}: Nam ut praecedat aliquid et aliud consequatur, in his fere rebus euenire solet quas paulo superius commemoraui. Speciem quippe sequitur genus, uel differentia, uel definitio, uel proprium, uel inseparabile accidens. Item proprium ac definitionem sequitur species, proprium uero sequitur differentia et definitio, et definitionem sequitur proprium uel differentia, hoc modo: nam si homo est, animal est; et si homo est, rationale est; et si homo est, animal rationale mortale est; et si homo est, risibile est; si Aethiops est, niger est. Si risibile est, homo est; si animal rationale mortale est, homo est. Si risible, rationale est; si risible est, animal rationale mortale est; si animal rationale mortale est, risibile uel bipes est.}
\end{quote}

We'll consider two further passages before expounding. Moving to the case of arguments with both an affirmative and a negative, Boethius writes `of those cases that consist of an affirmation and a denial, that division is usually that they are comprised either of diverse genera, or diverse species, or of contraries, or of habit and privation.'\footnote{\autocite[1179D]{BDT}: `Earum uero quaestionum quae ex affirmatione et negatione consistunt, illa fere diuisio est, quod uel in diuersis generibus, uel in diuersis speciebus, uel in contrariis, uel in priuatione atque habitu continentur'} And considering arguments from a negative to an affirmative, Boethius writes: 
\begin{quote}
It cannot happen that an affirmation follows a denial \ldots except in those contraries which lack a middle ground and where it is always necessary for one or the other of them to inhere, in this manner: if it is not day, it is night; if it is not dark, it is light.\footnote{\autocite[1180A]{BDT} Ut autem negationem affirmatio consequatur, quae erat quarta conditionalis propositionis differentia, fieri non potest, nisi in his contrariis quae medio carent, et quorum alterum semper inesse necesse est, hoc modo: si dies non est, nox est; si tenebrae non sunt, lux est.} 
\end{quote}

We can now begin to see what Boethius is attempting: the division of hypothetical propositions according to the quality of their parts lays the groundwork for an inquiry into what kind of real relations things signified by terms must stand in to each other in order for there to be relations of following among consequences they factor into. For instance, in an affirmative-to-affirmative consequent, the positing of a species, property, and difference follow reciprocally from each other, and one can move from the positing of a species to positing the corresponding genus, but not conversely. In arguments from an affirmation to a negation, different relations predominate, i.e.  diversity of genus or species, contrariety, or that of privation and habit, and the conditions under which an argument from a negative antecedent to an affirmative consequent hold are a subset of those which hold for the converse case.

\subsection{The limits of Boethius' grounding of consequence}
The examples Boethius gives above provide the theory's paradigmatic use cases, but also reveal its limitations. When Boethius discusses consequences in both the \emph{On Differential Topics} and the \emph{On Hypothetical Syllogisms}, the majority of cases he considers are arguments where the antecedent and consequent affirm or deny existence claims (e.g. `if it is day, it is not night', `if there's a man, there's an animal'), substitute a predicate term in the consequent for a different one in the antecedent (e.g. `if it's white, it's colored'), or provide a consequent where the implied subject for a predicate posited or denied in the consequent is an anaphoric referent back to what is posited or denied in the antecedent (e.g. `if there's an Ethiopian, he's black'). Now in Latin, these different kinds of inferences have the same syntactic form, i.e.  `if \emph{a} is (not), \emph{b} is (not)' (\emph{si a (non) est, b (non) est}). From the preponderance of his examples, it is clear that Boethius primarily conceives of the \emph{relata} of consequence not as propositions or their contents, but as the referents of terms.\footnote{See \autocite{Martin2007,Bosman2018}.} The advantage of such a theory is that it provides a grounding for why consequences with antecedents and consequents of a given quality hold: because the things signified by the terms posited or denied therein stand in relations like contrariety, that of a habit to its privation, that of a genus to a species, etc. The disadvantage is that the theory does not generalize to cover more complex cases, e.g. arguments with modal or non-atomic antecedents or consequents, those involving quantified or oblique terms, or those where the subject of the antecedent and consequent differ, e.g. `if the pilot of a ship should not be chosen by lot, neither should the governor of a city' - Boethius' own example of an argument by analogy \autocite[1191A-B]{BDT}. Crucially, though Boethius discusses a wide variety of middle and extrinsic topics with illuminating examples in his topical works, the less cohesive treatment he gives these does not fit naturally into the division of inferences he fleshes out in the \emph{On Hypothetical Syllogisms}. Furthermore, to the degree that the theory grounds consequences in the varied relations that it does, it would seem to presuppose a fairly robust metaphysical framework. It's perhaps for these reasons that we begin to see more streamlined ways of grounding inference arise in the later medieval period.

\section{Medieval consequence and the theory of supposition}
\subsection{Canonical supposition theory: Ockham and Buridan}
By the time the earliest treatises on consequences appear at the turn of the fourteenth century, the use of topics in discussions of consequence has been  altered by the development of \emph{supposition theory}. Supposition theory plays a role in medieval logic roughly analogous to that played by theories of reference today, but the medieval theory is better understood as a theory governing the interpretation of terms in propositional contexts \autocite{DutilhNovaes2007,DutilhNovaes2008b}. 

On one of the best known accounts, that of William of Ockham, supposition divides into three types: simple, material, and personal \autocite[193-197]{OckhamSL}. Personal supposition occurs when a term is taken to refer to what it signifies; simple,  when a term refers to its concept (understood as an intention of the soul); material, when one refers to itself as a spoken utterance or written word. Other accounts, e.g. that of John Buridan, view simple supposition as a variety of material supposition. But on both views, Personal supposition is taken to be the most typical, and is interpreted as the term's supposition for \emph{individuals}.\footnote{This nominalist reading is reflected in modern formalizations of the notion, e.g. those of Klima and Parsons. \autocite{Klima1988,Parsons2014}.} 

On these accounts, determinate supposition of a term able to stand for several individuals is interpreted in terms of the ability to descend \emph{salva veritate} from the original sentence the term is in to a disjunctive sentence where in each disjunction, the term in question is replaced by the name of a different individual member of the class denoted by it, with the number of disjuncts equal to the number of members. The case is similar for confused and distributed supposition, albeit employing conjunction. For example, both terms in the particular affirmative proposition `some logic articles are dry' have determinate supposition, while both terms of the universal negative proposition, `no knowledge is useless' have confused distributed supposition.\footnote{For fuller discussion of the notion of supposition in medieval philosophy, see \autocite{Klima1988,PriestRead1977,HodgesBurley,Parsons2014}.}

%Briefly and simplifying somewhat, if a common noun in a true sentence could be replaced with the name of a species or particular under it without loss of truth, then the term was said to be distributed; conversely, a common (or proper) noun signifying a species (or particular) which could be replaced \emph{salva veritate} with the name of the genus (or species) that the common noun's signifier was subsumed under was said to be undistributed or determinate.

%Figure out how to explain that while supposition theory was distinct from the theory of topics, it frequently borrowed from its vocabulary.
Where the classical Boethian theory of topics leveraged the categorical relations obtaining between the referents of terms to discover available inferences, the theory of supposition instead relied heavily on the notion of a term's distribution. Around the same time that the theory of supposition was rising to prominence,  the account of what a maximal proposition is was greatly simplified, becoming simply a rule in virtue of which an inference holds.\footnote{\autocite[76.5-7]{BurleyDPAL}: `Nam propositio maxima non est nisi regula, per quam consequentia tenet.' Cf. \autocite[31.3-6]{OckhamEL}.} Creating a semblance of continuity between the older and newer theory, justifications for inferences in terms of supposition borrowed heavily from the language of the topics, and the shift to grounding consequences in the supposition of their terms brought about a simplification in the number of `topics' actually appealed to in justifications of inference, with a vast number of inferences justified by rules like \emph{from an undistributed inferior to superior} and \emph{from a distributed superior to a distributed inferior}.\footnote{See \autocite[\textit{passim}]{Green-Pedersen1980a}} 

\subsection{Supposition theory in two early British \emph{De consequentiis}}
This standard view of how supposition works and the presumed priority it ascribes to personal supposition seem to be confirmed in an anonymous London consequence treatise - likely the earliest still extant - where the notion is only found in a part of the treatise asserting that convertible terms, e.g. `man' and `risible', share the same supposita \autocite[9, par. 27]{Green-Pedersen1980a}. The same pattern is found in Burley's \emph{de consequentiis}. Consider, for instance, the following passage on exceptive propositions:
\begin{quote}
With respect to the supposition of the predicate and subject in an exceptive, one should know that that from which the exception is taken, or the subject of the exceptive ... always stands confusedly and immovably with respect to the exception, and movably with respect to the predicate. With respect to the exception it stands immovably, because it is not possible to descend with respect to it; hence `every man besides Socrates runs, therefore Plato besides Socrates runs' does not follow. With respect to the predicate it is possible to descend, since every man besides Socrates runs, therefore Plato runs, and so on for singulars.'\footnote{\autocite[124, par. 58]{Green-Pedersen1980b}: `Quantum ad suppositionem praedicati et subiecti in exceptiva notandum quod illud a quo fit exceptio vel subiectum exceptivae quod eadem sunt, quod semper stat confuse et distributive immobiliter respectu exceptionis et mobiliter respectu praedicati. Respectu exceptionis stat immobiliter, quia respectu illius non contingit descendere; unde non sequitur `omnis homo praeter Socratem currit, ergo Plato praeter Socratem currit.' Tamen respectu praedicati contingit descendere, quia sequitur `omnis homo praeter Socratem currit, ergo Plato currit,' et sic de singulis.' Translations for \autocite{Green-Pedersen1980a,Green-Pedersen1980b} are taken from \autocite[171-273]{Archambault2017d}.}.
\end{quote}
This passage also makes clear that the supposition of a term is given relative to another term or terms which are held fixed. Thus, a term's supposition may differ based on what terms it is being considered with respect to.

\subsection{An alternative reading of personal supposition in the Parisian \emph{De consequentiis}}
Given the close connection between the rise of theories of supposition and those of consequence, it is surprising to find that the approaches to supposition in early discussions of consequence can diverge from the standard Ockham-Buridan model, sometimes significantly. In an anonymous work on consequences in Paris, BN lat. 16130, in the majority of cases where the term `supposition' is mentioned one finds that what a term is taken to supposit for is not individuals, but a concept under it. Consider, for instance, the following:

\begin{quote}
I show the consequence `a man is a white man, therefore a man is white' by this rule: positing \emph{per accidens}, also posits per se. But `white man' is a suppositum \emph{per accidens} of something. With respect to the same, `white' is also posited.\footnote{\autocite[18, par. 34]{Green-Pedersen1980a}: `Ostendo istam consequentiam `homo est homo albus, ergo homo est [homo] albus' per istam regulam: posito per accidens ponitur et per se. Sed `homo albus' est suppositum per accidens alicuius respectu eiusdem ponitur et `album'.}
\end{quote}

As we can see, what is taken to supposit here is not individual white men, but the term or concept `white man', which is `placed under' the concept `man'. 

Furthermore, the dominant division of supposition in the treatise is not that between material, simple, and personal, but that between \emph{per se} and accidental. The text explains the notion of \emph{per accidens} supposition as follows:

\begin{quote}
And [those] placed under \emph{per accidens} are those that are combined out of two inhering in each other contingently, as white man is combined \emph{per accidens} because it is put together out of man and white, which inhere in each other contingently. So `a man is a white man', etc. is contingent. This is contingent, and its equivalents (\emph{convertibilia}) are contingent, because in all of these a superior is assumed of a \emph{per accidens} inferior (since whatever is constituted by an addition with respect to another is inferior to it. Thus, `white man' is inferior to `man' and `white').\footnote{\autocite[19, par. 38]{Green-Pedersen1980a}: `Et sunt supposita per accidens ista quae aggregantur ex duobus contingenter sibi invicem inhaerentibus, ut illud `homo albus' est aggregatum per accidens, quia aggregatum ex `homine' et `albo', quae contingenter sibi invicem inhaerent. Ideo haec est contingens `homo est homo albus' etc. Illa est contingens, et eorum convertibilia sunt contingentia, quia in omnibus his praedicatur superius de inferiori [et] per accidens [album], quia unumquodque se habens per appositionem respectu alterius inferius est eo. Ideo `homo albus' inferius est `homine' et `albo''.}
\end{quote}

Here, the supposita themselves are not individuals, but forms or concepts.

\subsection{Supposition in Burley's \emph{On the Core of the Art of Logic}}
Walter Burley's \emph{On the Core of the Art of Logic} exists in two versions, a shorter, earlier version and a later, longer one more engaged with Ockham's work. Burley's \emph{Shorter Treatise} contains two different approaches to personal supposition.\footnote{For the translation of the title of Burley's work, see \autocite{SpadeMenn}.} Burley gives an example of the first where, in his fifth principal rule for consequences, he describes suppositional descent as licensing a descent from genus to species:

\begin{quote}
From the negation of the superior follows the negation of any inferior. And this rule must be understood [for] when the negated superior supposits personally. For it follows: Socrates is not an animal, therefore Socrates is not a man nor an ass, and so on of others.\footnote{\autocite[209.35-210.2]{BurleyDPAL}: `Quinta regula principalis est ista: Ad negationem superioris sequitur negatio cuiuslibet inferioris. Et haec regula est intelligenda, quando superius negatum supponit personaliter; sequitur enim: Sortes non est animal, ergo Sortes non est homo nec asinus et sic de aliis.'}
\end{quote} 
The key to this understanding of personal supposition is found in Burley's introduction of the notion in the \emph{Longer Treatise}, where he treats it as a division of \emph{suppositio formalis}.\footnote{According to Dutilh Novaes, Burley revives the notion of \emph{suppositio formalis} unmentioned by Peter of Spain, Roger Bacon, or the \emph{logica Lamberti}, from William of Sherwood \autocite[360]{DutilhNovaes2012c}.} Burley writes
\begin{quote}
Formal supposition is twofold, for sometimes a term supposits for its significate, sometimes for its suppositum or for some singular of which it is truly predicated. And thus formal supposition is divided into personal supposition and simple supposition.\footnote{\autocite[3.1-5]{BurleyDPAL}: `Suppositio formalis est duplex, quoniam terminus quandoque supponit pro suo significato, quandoque pro suo supposito vel pro aliquo singulari, de quo ipsum vere praedicatur Et ideo suppositio formalis dividitur in suppositionem personalem et in suppositionem simplicem'.}
\end{quote}

Burley continues:
\begin{quote}
Personal supposition is when a common term supposits for its inferiors, \emph{whether those inferiors be singular or common, whether they be things or words}, or when a concrete accidental term or a composite term supposits for that of which it is predicated accidentally.\footnote{\autocite[3.19-24]{BurleyDPAL}: `Suppositio personalis est, quando terminus communis supponit pro suis inferioribus, sive illa inferiora sint singularia sive communia, sive sint res sive voces, vel quando terminus concretus accidentalis vel terminus compositus supponit pro illo, de quo accidentaliter praedicatur'.}
\end{quote}
Burley's account here is more varied than the canonical account based on Buridan and Ockham, including both individuals and common natures as supposita in personal supposition. When Burley states that personal supposition also captures that of composite and concrete accidental terms, he appears to have in mind those cases we found discussed earlier in the Paris treatise, like `a man is white, therefore a man is a white man'. The account thus appears to better capture uses of personal supposition in ascents and descents in the earliest treatises on consequences. 

One might ask whether the account of supposition found in Burley's \emph{On the Core of the Art of Logic} and the anonymous Paris treatise demands a more extensive ontological foundation than that of Ockham or Buridan. Burley himself provides an emphatic `no'. Discussing the supposition of `man' in the sentence `Man is a species', Burley remarks, `\emph{I don't care at present} whether it [i.e. `man'] is a common thing outside of the soul or it is a concept; it suffices merely that that which the name primarily signifies is a species'.\footnote{\autocite[8.8-10]{BurleyDPAL}: `Sed sive illud commune sit res extra animam sive sit conceptus in anima, non curo quantum ad praesens; sed tantum sufficit, quod illud, quod hoc nomen primo significat, est species'. Cf. \autocite[196.33-37]{OckhamSL}} While committed to the existence of species in a non-foundational sense, Burley emphasizes that his account of supposition should hold regardless of what the ultimate foundations of the concepts a theory of inference employs may be. In this intention, his account contrasts with Ockham's contemporary theory, and perhaps also with Boethius' earlier grounding of consequence in the theory of topical argument.
\section{Conclusion}
In the preceding, we've detailed two changes in the foundations of logic that ultimately affect our understanding of the nature of the discipline as such. 

In earlier frameworks found, for instance, in Boethius, good arguments are discussed under a variety of types each with different grounds: some, for instance, argue from topics concerning language, others from real relations, others move from effects to cause, and still others relate premises to conclusion as cause to effect.

At the time of the earliest \emph{consequentiae}, the majority of consequences discussed are grounded in the theory of supposition. In this theory, personal supposition is the most frequently found, but consequences involving simple supposition are not uncommon. More importantly, terms in personal supposition may be taken to refer not only to individuals, but also to types. Both are permissible readings of the anonymous London \emph{De consequentiis}, both readings are explicitly found in Burley's logic, and the latter is the only variety explicitly invoked in an early anonymous Parisian treatise on consequences.

We can now see, then, that Ockham's restriction of personal supposition to exclusively refer to individuals, which today remains the dominant interpretation of \emph{all} theories of personal supposition in the secondary literature, was more novel than hitherto recognized. By the time Ockham's theory arises, the commonly accepted grounds of consequence have already shed much of the diversity found in the earlier theory of Boethius. This paves the way for a paradigm of thinking about consequence that remains to this day, one less focused on the myriad sources of demonstrative argument, and more focused on the universal application of argument forms to a common domain of individuals, whose concrete features are abstracted away. 

%Add something about the rise of formal consequence here.
Furthermore, in changing the account of supposition, Ockham also necessarily modified that of the theory of consequence which depended on it. While the earlier account of supposition, with its inclusion of forms or concepts, did better at explaining inferential practice, Ockham's sparse focus on individuals would grant the theory of consequence a surer foundation in reality at the price of leaving our grasp of it more opaque. In this way, Ockham's anti-metaphysical grounding for his logic may be regarded as \emph{more} metaphysical than that of Burley and his predecessors: where Burley's was consistent with, but did not require realism, Ockham's built a decision on the metaphysical foundations of logic directly into his work.\footnote{Cf. \autocite{Read2007}.} And it seems to be this decision, if any, that sets the stage not only for the defects of nominalism, but also for the rise of `extreme realism' to counter it.

%Not every consequence is grounded in a relation of supposition, e.g. conversions aren't. But every supposition grounds relations of consequence.

%What are the proximate and ultimate features in virtue of which a good consequence is good?

%\subsection{Groundings \emph{of} consequence}
%The number of dependency relations involved in discussions of consequence are myriad. Considering consequences as such, every consequence is constitutively dependent on the being of its antecedent and consequent.  Conversely, the being of antecedent and consequent \emph{as} antecedent and consequent is dependent on the being of the consequence which they are parts of. Every sound conclusion is jointly grounded in the truth of its premises and the validity of its logical form. Every proof-theoretic consequence is grounded in the availability of a procedure for reducing the proof to normal form. Every model theoretic consequence is grounded in every model of its antecedent being a model of its consequent (or, more controversially, in the non-existence of its counter-models). Soundness and completeness proofs can even be construed as attempts to ground the reliability of one consequence relation in that of another. But though these might loosely be called grounding relations, I don't think they fully merit the title - partially because not all these grounds suffice for what they ground, and partially because the grounding entities are not themselves fundamental. Because of this, it will be useful to look into more basic grounds for consequence, particularly those found in medieval work on topical argument and supposition.

%This does not mean that every consequence is itself grounded in the theory of supposition: consequences based on conversions appear not to be, %nor are formal consequences whose holding is independent of their matter.

%Grounding may be either partial or full \autocite{WygodaCohen2018}, is distinguished from supervenience by its hyperintensionality \autocite[364]{Schaffer2009} \autocite{Chilovi2018} and from modern accounts of causation by its synchronic and metaphysical, rather than diachronic and scientific, character \autocite{Schaffer2016}. 

%$A$ grounds $B$ fully when some state (e.g. existence, being true, etc.) of $A$ is sufficient for a corresponding state of $B$, while A grounding of $B$ in $A$ is partial when $A$ is factors into, but not does not suffice for, a full grounding of $B$, e.g. Socrates' existence fully grounds that of the set $\{Socrates\}$, and partially grounds that of the set $\{Socrates, Plato\}$. Likewise, A true conjunction is partially grounded in the truth of each conjunct taken individually, and fully grounded in the truth of both.

%\subsection{Grounding in medieval philosophy}
%One does not find an explicit literature on the notion of grounding \emph{as such} in the medieval period there was substantial discussion of issues that today would be subsumed under that term - in work on predicables and categories, on habit and privation, on the relation between matter and form, and elsewhere. Unlike modern discussion of the subject, medieval discussions do not sharply distinguish metaphysical from scientific contexts. Furthermore, medieval work on causation assumes a synchronic, rather than diachronic relation between cause and effect. As such, one also finds work pertinent to what we would call grounding in medieval texts related to causation.

%\subsubsection{Commentaries on the Posterior Analytics}
%Another important medieval locus for the justification of consequence occurs in commentaries on the \emph{Posterior Analytics}, in its discussion of the distinction between \emph{quia} and \emph{propter quid} demonstration. Briefly, a demonstration is a proof of a conclusion from premises that are both true and better known than the conclusion. Such demonstrations may occur in two ways. According to the aforementioned distinction, premises and conclusions are related to each other as cause and effect. In the highest form of demonstration, a \emph{propter quid} demonstration, the truths found in the premises are themselves causes of what is discoverered in the conclusion, and thereby serve as the intrinsic necessary, and immediate grounds for the truth of the conclusion. In a \emph{quia} demonstration, by contrast, also called a demonstration of the fact, the conclusion is itself a cause, metaphysically speaking, of what the premises assert to be the case, though from the perspective of understanding, the premises are still a cause of the knowledge of the conclusion. As I've discussed elsewhere, this distinction underlies the later distinction between \emph{a priori} and \emph{a posteriori} truths.


%Contemporary inquiries into the relation between grounding and consequence include the employment of consequence relations to define ground \autocite{Fine2012a}, appeals to a notion of grounding to define logical consequence \autocite{Schnieder2019}, and the use of consequence relations to determine the relation grounding holds to other important concepts.\footnote{\autocite{Chilovi2018,Fine2016a,Lubrano2019}. For an overview of the notion of grounding in contemporary literature, see Magali Roques' introduction to this volume.} 



%Chilovi2018  Kripke2018 Lubrano2018 Schnieder2019
%\printbibliography
\end{document}
