\section{Working appendix}
\subsection{Examples of Grounding}
1) being      : true statements, 

2) physical   : mental

3) true atomic propositions : non-atomic ones, 

4) substances : qualities, 

5) Socrates   : $\{Socrates\}$, 

6) natural features : normative features 

7) Linguistic use : meaning, 

8) Mathematical structure : numbers (given structuralism) 

9) Regularities in nature : causal relations (given Humeanism), 

10) Categorical features : dispositional ones (given quidditism), 

11) Determinate features of a thing : its determinable features, 

12) The being of a material host : that of a hole in it, 

13) An integrated whole : its fragmentary parts (given monism). 

14) A species : its genus.

\subsection{Extra thoughts}
%Given this critique along with skepticism about its divergence from related notions such as causation and supervenience, it remains necessary to find sources that might serve to further clarify the notion of grounding. To this point, many of the clearest examples of grounding have come from the field of logic, particularly those involving the grounding of complex propositions in their atomic constituents. But even these examples are incomplete: while there are clear analyses of grounding for conjunction, disjunction, and double negation, I've not found similarly clear discussion of conditionals or even negation. For this reason, I propose to examine the ways valid arguments are grounded in medieval logic.

%At this point, it's important to note that while the anonymous London treatise's use of supposition is consistent with the Ockhamist reading, it is also consistent with that found in Burley and the Paris treatise. Given this it seems likely that the intended reading there, too, may have been the latter. THIS IS PROBABLY NOT TRUE

\subsubsection{Material consequence as a relation of partial ground}
Unlike its modern counterparts, medieval logic did allow for a notion of consequence which was partial, namely \textit{material} consequence



Medieval approach to consequence is basically recursive: i.e. state a consequence, then show how other consequences follow from it (%how is this any different from an axiomatic approach?)

Grounding of whole in parts
Grounding of 

Questions of grounding show up in medieval discussions of consequence in three areas.
  grounding of (sound) conclusions in (true) premises - %(This idea is built into the original meaning of `validity' an argument is valid, i.e. strong, because its conclusion is supported by its premises. 
  %The problem of relevance is one of grounding: irrelevant conclusions, regardless of their truth value, fail to be grounded in their premises)
  grounding of the theory of consequence in the theory of supposition
  grounding of consequences in relations between beings 
  Grounding of derivative rules in primative ones.

Leaving aside those examples that amount more to slogans or aspirations than philosophical analyses, we can classify the remaining examples as follows: 

Intrinsic, extrinsic, necessary, sufficient
\begin{enumerate}
  \item Grounds whose being entails that of what they ground. (e.g. that of true statements in states of affairs, of non-atomic propositions in atomic ones, of generic qualities in specific ones)
  \item Grounds whose being is required for that of what they ground
\end{enumerate}

categories, predicables, privations, 



But happily, those examples in the field of logic are among the clearest. Specifically, it is widely assumed that the truth of propositions that introduce connectives is grounded in that of their more basic counterparts. For instance, the truth of $A \wedge B$ is grounded in that of $A$ and $B$, that of $A \vee B$ is grounded in either of its disjuncts, and that of $\neg\neg A$ is grounded in $A$.

Philosophy studies being as its subject matter. Being is not a genus. Every discipline studying a some materially distinct sphere has for its object some genus distinct from others. Therefore, philosophy lacks a materially distinct sphere.

component vs. net distinction is present in both causation and grounding.

Porphyrian predicables: Genus, species, difference, property, accident.

habit privation

matter to form

powers in 

Being --> per se, per accidens

a priori, a posteriori 



%Schaffer uses chemistry examples as examples of grounding. Hence, the idea that grounding is distinct from causation by being philosophical as opposed to scientific is false.  
\section{Quotes}
\subsection{Abelard, dialectica}

/459/ Videtur in supradicta diuisione omnis quaestio praedicatiua contineri, ut sunt illae in quibus de accidenti fundamentum uel de accidenti aliud accidens aut de se inuicem contingentia praedicantur, ueluti cum dicimus: 'utrum omne coloratum est corpus' uel 'omne grammaticum est homo' uel 'grammaticum musicum' uel 'rationale mortale'. Sed has omnes irregulares praedicationes quidam dicunt, cum neque sint secundum substantiam neque secundum accidens. Non itaque aliae quaestiones in supraposita diuisione cadunt nisi quae ex affirmatiua uera regulari ueniunt. Regulares uero eas dici uolunt quarum praedicationem iuxta Porphyrium regularem appellant, quae uidelicet uel secundum substantiam fit uel secundum accidens. %This is relevant to the origin of per se/per accidens consequence in per se/per accidens predication.
\subsection{Aquinas, PA commentary}
\subsubsection{Prologue}
Pars autem logicae, quae primo deservit processui, pars iudicativa dicitur, eo quod iudicium est cum certitudine scientiae. Et quia iudicium certum de effectibus haberi non potest nisi resolvendo in prima principia, ideo pars haec analytica vocatur, idest resolutoria. Certitudo autem iudicii, quae per resolutionem habetur, est, vel ex ipsa forma syllogismi tantum, et ad hoc ordinatur liber priorum analyticorum, qui est de syllogismo simpliciter; vel etiam cum hoc ex materia, quia sumuntur propositiones per se et necessariae, et ad hoc ordinatur liber posteriorum analyticorum, qui est de syllogismo demonstrativo.	Now the part of logic which is devoted to the first process is called the judicative part, because it leads to judgments possessed of the certitude of science. And because a certain and sure judgment touching effects cannot be obtained except by analyzing them into their first principles, this part is called analytical, i.e., resolvent. Furthermore, the certitude obtained by such an analysis of a judgment is derived either from the mere form of the syllogism—and to this is ordained the book of the Prior Ana1ytics which treats of the syllogism as such—or from the matter along with the form, because the propositions employed are per se and necessary [cf. infra, Lectures 10, 13]—and to this is ordained the book of the Posterior Analytics which is concerned with the demonstrative syllogism.
Secundo autem rationis processui deservit alia pars logicae, quae dicitur inventiva. Nam inventio non semper est cum certitudine. Unde de his, quae inventa sunt, iudicium requiritur, ad hoc quod certitudo habeatur. Sicut autem in rebus naturalibus, in his quae ut in pluribus agunt, gradus quidam attenditur (quia quanto virtus naturae est fortior, tanto rarius deficit a suo effectu), ita et in processu rationis, qui non est cum omnimoda certitudine, gradus aliquis invenitur, secundum quod magis et minus ad perfectam certitudinem acceditur. Per huiusmodi enim processum, quandoque quidem, etsi non fiat scientia, fit tamen fides vel opinio propter probabilitatem propositionum, ex quibus proceditur: quia ratio totaliter declinat in unam partem contradictionis, licet cum formidine alterius, et ad hoc ordinatur topica sive dialectica. Nam syllogismus dialecticus ex probabilibus est, de quo agit Aristoteles in libro topicorum.	To the second process of reason another part of logic called investigative is devoted. For investigation is not always accompanied by certitude. Hence in order to have certitude a judgment must be formed, bearing on that which has been investigated. But just as in the works of nature which succeed in the majority of cases certain levels are achieved—because the stronger the power of nature the more rarely does it fail to achieve its effect—so too in that process of reason which is not accompanied by complete certitude certain levels are found accordingly as one approaches more or less to complete certitude. For although science is not obtained by this process of reason, nevertheless belief or opinion is sometimes achieved (on account of the provability of the propositions one starts with), because reason leans completely to one side of a contradiction but with fear concerning the other side. The Topics or dialectics is devoted to this. For the dialectical syllogism which Aristotle treats in the book of Topics proceeds from premises which are provable.
\subsubsection{I, lecture 23, chapter 13}
Dicit ergo primo quod demonstratio quia per effectum est, si quis concludat quod planetae sunt prope propter hoc quod non scintillant. Non enim non scintillare est causa quod planetae sint prope, sed e converso. Propter hoc enim non scintillant planetae, quia sunt prope. Stellae enim fixae scintillant, quia visus in comprehensione earum caligat propter earum distantiam. Formetur ergo syllogismus sic: omne non scintillans est prope; planetae sunt non scintillantes; ergo sunt prope. Sit in quo c planetae, idest accipiatur planetae quasi minor extremitas. In quo autem b sit non scintillare, idest non scintillare accipiatur medius terminus. In quo autem a sit prope esse, idest prope esse accipiatur ut maior extremitas. Vera igitur est haec propositio: omne c est b, quia planetae non scintillant. Et iterum verum est quod omne b est a, quia omnis stella non scintillans prope est. Huiusmodi autem propositionis veritas oportet quod accipiatur per inductionem, aut per sensum, quia effectus hic est notior causa quantum ad sensum. Et sic sequitur conclusio quod omne c sit a. Et sic demonstratum est quod planetae sive stellae erraticae sunt prope. Hic igitur syllogismus non est propter quid, sed est quia. Non enim propter hoc quod non scintillant, planetae sunt prope, sed propter id quod prope sunt, non scintillant.	He says therefore first (78a30) that demonstration quia is through an effect if one concludes for example that the planets are near because they do not twinkle. For non-twinkling is not the cause why the planets are near, but vice versa: for the planets do not twinkle because they are near. For the fixed stars twinkle because in gazing at them the sight is beclouded on account of the distance. Therefore, the syllogism might be formed in the following way: “Whatever does not twinkle is near; but the planets do not twinkle: therefore, they are near.” Here we let C be the planets, i.e., let “planets” be the minor extreme, and let B consist in not twinkling, and A “to be near” be the major extreme. Then the proposition, “Every C is B,” is true, namely, the planets do not twinkle. Also it is true that “Every B is A,” i.e., every star that does not twinkle is near. Rowever, the truth of such a proposition must be obtained through induction or through sense perception, because the effect here is better known than the cause. And so, the conclusion, “Every C is A,” follows. In this way, then, it has been demonstrated that the planets, i.e., the wandering stars, are near. Consequently, this syllogism is not propter quid but quia. For it is not because they do not twinkle that planets are iiear but rather, because they are near, they do not twinkle.

Deinde cum dicit: contingit autem et per alterum etc., docet quomodo demonstratio quia convertatur in demonstrationem propter quid, dicens quod contingit et per alterum demonstrare alterum, idest per hoc quod est prope esse, demonstrare quod non scintillant; et sic erit demonstratio propter quid. Ut sit c erraticae, idest accipiatur stella erratica minor extremitas; in quo b sit prope esse, idest prope esse accipiatur ut medius terminus, quod supra erat maior extremitas; a sit non scintillare, idest accipiatur non scintillare maior extremitas, quod supra erat medius terminus. Est igitur et b in c, quia omnis planeta prope est; et a est in b, quia omnis planeta, qui prope est, non scintillat; quare sequitur quod et a sit in c, scilicet, quod omnis planeta non scintillet. Et sic erit syllogismus propter quid, cum accepta sit prima et immediata causa.
\subsubsection{London consequences}
Hence to know when terms convert and when they don't, one should see whether each term is predicated universally of the other or not; if so, then the terms convert. As is clear, this term `man' and this term `risible' convert, since each is predicated universally of the other; for `every man is risible' is universal and true, and likewise the converse `every risible /f.112ra/ is a man'. And consequently, `man' and `risible' convert with respect to their supposita, since all the things that are supposita of the one are supposita of the other; and if this were not so, they wouldn't convert. (par. 27).
\subsubsection{Paris consequences}
\subsubsection{Burley}
\begin{quote}
  Suppositio simplex est, quando terminus communis supponit pro suo significato primo vel pro omnibus contentis sub suo significato primo vel quando terminus singularis concretus vel terminus singularis compositus supponit pro suo significato totali, ut aliqualiter dictum est supra (DPAL 7.1-5).
\end{quote}

\subsection{Critiques}
It is not \textit{prima facie} clear that all the relations found in common examples of grounding  actually belong to common type. For this reason, one common critique of grounding, found, for instance, in the work of Kathrin Koslicki, denies the \textit{unity thesis} - i.e. the claim that there is a unified, non-equivocal notion of grounding.


one might consider the properties predicable of the consequence and grounding relations themselves. Both ground and consequence relations, for example, are transitive. But grounding is irreflexive and assymetric, while most consequence relations are reflexive and hence not assymetric. Furthermore, consequence is generally treated as a relation between either statements or propositions, while at least some notions of grounding aim to apply more broadly to things. Because of this it is clear that the two notions cannot be identified, nor are there obvious prospects for treating consequence as a species of the grounding relation or vice versa.

Next, 

per se/per accidens supposition as a ground for natural/accidental consequence
