%\documentclass[]{article}
%\usepackage[backend=bibtex]{biblatex}
%\bibliography{jacob}
%opening
%\title{Redefining occasionalism}
%\author{Jacob W. Archambault}

%\begin{document}

%\maketitle

%\begin{abstract}
%In secondary scholarship on occasionalism, it is common to distinguish partial occasionalists from full-blown occasionalists, with the former ascribing and the latter denying activity to the mind besides their common denial of activity to material bodies. I show that this distinction arises from a failure to distinguish action from production. Furthermore, if we define occasionalism extensionally to refer to the system of philosophy advocated by Louis de La Forge, Geraud de Cordemoy, and Nicolas Malebranche, full-blown occasionalism simply doesn't exist.
%\end{abstract}

%Nadler 1998, 221-222 argues that partial occasionalism is philosophically inconsistent.
%Nadler 1998, 223 has some good quotes from Malebranche contrasting God's causal power over minds and bodies.
%Substances are composed across instances (Occasionalism), or are instances composed across substances (Leibniz)
\chapter{Redefining Occasionalism}
\section{Introduction}
Occasionalism is an early modern philosophical system best known for its explanations of body-body interaction and the relation of rational minds to bodies, on which both the apparent causal activity of material bodies on each other and the correspondence of a rational creature's mental volitions with its bodily action are effected directly by God. 

In secondary literature on occasionalism, it has become commonplace to distinguish partial, or local occasionalism, from full-blown or global occasionalism, with the former affirming and the latter denying activity to minds besides bodies. This distinction is first hinted at in Daniel Garber's work on Descartes' account of motion, and first advanced shortly thereafter in Steven Nadler's work on Louis de La Forge \autocite{Garber1987,Nadler1993b}. The distinction has since become widespread, with its canonicity witnessed to by its acceptance in the Stanford Encyclopedia of Philosophy's entry on the subject.\footnote{See \autocite{Radner1993}, \autocite{Clatterbaugh1996}, \autocite{Nadler1998}, \autocite{Clarke2000}, \autocite{Bardout2002}, \autocite{Nadler2005}, \autocite{Kolesnik2006}, \autocite{Lee2007}, \autocite{Sangiacomo2014}, \autocite{sep-occasionalism}.}

In what follows, I show that the dominant division of occasionalism in secondary scholarship for the past thirty years, essentially since the beginning of scholarship examining differences among occasionalist thinkers, is without foundation: as a matter of the early modern historical record, full-blown occasionalism simply doesn't exist.

For the purposes of this essay, I take `occasionalism' to describe a historical movement in early modern thought begun among certain followers of Descartes, having as its main defenders Louis de La Forge, Geraud de Cordemoy, and Nicolas Malebranche \autocite[37]{Nadler2005}. Consequently, this article's thesis touches neither on those medieval Islamic thinkers whose thought shares real parallels with those of the occasionalists, nor on that one idiosyncratic graduate student in your own department who keeps requesting you do a seminar on \emph{The Search After Truth}. Nor do I directly address the coherence of occasionalism as such. Arguments therefore purporting to show that full-blown occasionalism is the only coherent form of the doctrine do not suffice to show that the various historical occasionalists were themselves bound to support it.\footnote{Cf. \autocite{Nadler1998,Winkler2011}.} 

I first show that the distinction between partial and full-blown occasionalism arises from a failure to distinguish action from production, one that the occasionalists themselves would have been familiar with. The main argument then proceeds by induction, showing successively that none of La Forge, Cordemoy, and Malebranche were full-blown occasionalists. Since none of the primary figures associated with occasionalism were actually full-blown occasionalists in the sense the secondary literature describes, it follows that the distinction itself is at best trivial, and the definition of occasionalism giving rise to it must be mistaken on pain of irrelevance.

%being half-Irish is not a different kind of thing from being Irish, it is being Irish only in part. 
%Consistency of occasionalism.
%source of the distinction: Garber's \textit{a priori}, uncited understanding of what occasionalism is. 

%restrict argument to post-Cartesians: leave Descartes, Islamic thinkers out of it.
%the proof is by induction: neither La Forge, nor Cordemoy, nor Malebranche are full-blown occasionalists. Therefore, it doesn't exist.
\section{The continuous creation argument for occasionalism}
Occasionalism is increasingly recognized as having its main impetus in a gradual expansion of the medieval doctrine of continuous creation.\footnote{\autocite[63]{Nadler1993b}, \autocite[219]{Nadler1998}. Cf. \autocite[287]{Winkler2011}}. According to this doctrine, creatures depend on God not merely in that God must have created them, but also in that God must conserve them at each instant.\footnote{See Thomas Aquinas, \emph{ST} Ia, q. 104, art. 1, res; cf. Descartes, AT VII. 369.} 

According to the earliest occasionalist accounts, God not only conserves material creatures in being, but also, since it is impossible to preserve a material object in existence without preserving it somewhere, must conserve them in a specific \emph{place} in doing so.\footnote{See La Forge, \emph{Traité de l'esprit de l'homme}, in \emph{Oevres philosophiques}, ed. Pierre Clair (Paris: Presses Universitaires de France, 1974), 240-41. Cf. Malebranche, \emph{Entretiens} VII, par. 10 = OC XII. 160; \autocite{Nadler1998}.} Given that motion is simply change of place on the Cartesian account occasionalism adopts, it follows that God, being immediately and solely responsible for the successive positioning of material objects, is likewise responsible for their motion. And given the only way material objects could act on each other would be by contact, itself a function of relative position, it follows, the occasionalist maintains, that material objects are causally inert.

\section{Full-blown occasionalism and the action-entails-production principle}
%Find Garber 1987's off-the-cuff definition. Show how it goes from Garber to Nadler, and then from Nadler to others. However, it does seem that others made the mistake independently, e.g. Baker 2005, as well.
In its clearest formulation, Nadler defines full occasionalism as the joint acceptance of the following theses:

\begin{enumerate}
	\item Natural objects, both minds and bodies, have no causal efficacy
	\item God alone is a true efficient cause.\footnote{\autocite[39]{Nadler2005}. Similar conceptions have been assumed by \autocite[174]{Battail1973}; \autocite[45-46]{Pyle2003}; \autocite[288]{Winkler2011}. A different conception, on which minds need not be wholly passive, is assumed in \autocite[358]{Radner1993}.}
\end{enumerate}

That the second of these is constitutive of occasionalism is uncontroversial. But whether the first should be so regarded depends on what is meant by `causal efficacy'. The occasionalist admits both minds and bodies are causes in a loose sense: both mental volition and bodily contact serve as occasions whereupon God must exercise his causal efficacy. On the other hand, the second thesis entails that minds other than God are not efficient causes. Since efficient causation is traditionally associated with production, a restricted reading of this entailment implies that creatures are not \emph{productive}. A stronger reading, drawn from a similarly traditional association of efficient causation with agency, holds neither minds nor bodies are \emph{active.}\footnote{Cf. Aquinas, \emph{in Metaph.} Bk. 5, lec. 2-3. In connection with Malebranche, see \autocite[255-256]{Baker2005}.} It is the stronger reading that is cemented in the terminological distinction between full and partial occasionalists. 

%AeP denies non-productive action
\section{The action-entails-production principle}
\subsection{In Leibniz}
The assumption that activity presupposes productivity plays an important role in Leibniz's polemic against occasionalism,\footnote{See GP IV. 586-87; \autocite{Rutherford1993}.} In \emph{On Nature Itself}, Leibniz describes occasionalism as follows:

\begin{quote}
	But now let us consider a little more closely the opinion of those \ldots who judge not things to act, but God at the presence of things and according to the aptness of things; and thus [judge] things to be occasions, not causes, and to receive, not to effect or elicit. When Cordemoy, La Forge, and other Cartesians set forth this doctrine, Malebranche especially adorned [it] with certain rhetorical lights on account of his acumen; but brought forth \ldots no solid reasons. Surely, if this doctrine leads to the point of even taking the \emph{immanent actions} of substances away \ldots, then it appears foreign to reason like nothing else.\footnote{GP IV. 509-510.}
\end{quote}
\section{The rejection of action-entails production in occasionalism}
\subsection{La Forge}
Here, Leibniz names Cordemoy, La Forge, and Malebranche as leading occasionalists, and describes occasionalists as being led to the conclusion that not even minds are active. But if they are so led, it will be against their own explicit commitments. La Forge, for instance, holds

\begin{quote}
	the essence of this faculty [i.e. the will] consists firstly in the fact that it is the active principle of all the mind's actions which chooses from itself and by itself, and determines itself to accept or reject what the understanding perceives or remain suspended when something is not yet perceived clearly enough.\footnote{\emph{Treatise} ch. XI, 97.}
\end{quote}

Elsewhere, La Forge assumes the mind is the cause of its own ideas:

\begin{quote}
	Although our thoughts follow one another and although it is the external objects or the first thoughts which provide an occasion for the will to determine itself and form the idea of subsequent thoughts, that does not imply that one should not say that it is the will which is the principal and proximate cause of the idea. Otherwise one would have to say that it is the external objects which produce the ideas that we have of them \emph{and not the mind} {[}..{]}.\footnote{\emph{Treatise} ch. X, 94. Emphasis mine.}
\end{quote}

In spite of this, La Forge is clear the mind produces nothing \emph{material}.

\begin{quote}
	{[}T{]}here is no creature, spiritual or corporeal, which can cause change in {[}matter{]} or in any of its parts, in the second moment of their creation, if the Creator does not do so himself.\footnote{\emph{Treatise}, 147. Cf. \autocite{Klima1993}.}
\end{quote}

\subsection{Cordemoy}
Like La Forge, Cordemoy describes willing as the mind's activity: `{[}God{]} made minds \ldots capable of action; they will.'\footnote{\emph{Traité de métaphysique,} CG 283, trans. from \autocite[52]{Nadler2005}. Cf. \emph{Discours physique,} CG 255:
	
	\begin{quote}
		Just as the body is a substance to which extension naturally belongs, so much so that, as for physical effects, it would cease to be a body if it ceased to be extended; in the same way the mind is a substance to which the power of determining itself belongs so naturally, that it would cease to be a mind if it ceased to will. (trans. from \autocite[47]{Nadler2005})
	\end{quote}
	
	Nadler argues Cordemoy is a `full-blown' occasionalist from his remarks that `it is just as impossible for souls to have new perceptions without God as it is for bodies to acquire new motions without him.' (\emph{Discours physique}, CG 255. Trans. from \autocite[50]{Nadler2005}). But these remarks and others to the same effect establish neither that the mind is inactive nor that it does not cause its ideas. Rather, such passages can be harmonized with those above on the assumptions i) that Cordemoy, like many of his contemporaries, was a compatibilist about divine and human willing, and ii) that because ideas are not real beings, but beings of reason, eidetic production does not provide an instance of real causation.} 

\subsection{Malebranche}
We also find this commitment to the active character of mind in Malebranche, who calls the mind's consent to the good an act, albeit `an immanent act that produces nothing material in our substance.'\footnote{\emph{Éclaircissement} 1 to \emph{De la recherché de la vérité} OC III, 25. Trans. from \autocite[52]{Nadler2005}.}

In his earlier works, Malebranche, too, does not rule out the mind being the productive cause of its own ideas: the mind's act `produces nothing \emph{material} in our substance';\footnote{\emph{Éclaircissement} 1 to \emph{De la recherché de la vérité} OC III, 25. Trans. from \autocite[52]{Nadler2005}. Emphasis mine.} minds `do not \ldots produce in themselves a reality, or a modification that \emph{physically} changes their substance.'\footnote{\emph{Réponse à la Dissertation} = OC 7, 568, Trans. from \autocite[52]{Nadler2005}. Emphasis mine. Cf. OC IX, 1129.} Here, both La Forge and Malebranche assume the production of ideas, though an activity, is not production in the proper sense, since the will's production of an idea is not the production of `a reality', i.e. a real substance or quality.

But while the \emph{Search after Truth} and its \emph{Elucidations} offer no rejection of the mind's productive power with respect to its own ideas, this sort of production \emph{is} ruled out in the later \emph{Dialogues on Metaphysics and on Religion}.\footnote{The \emph{Search} was first published in 1674-5; the \emph{Elucidations}, in 1678 as a supplement to the third edition of the \emph{Search}; the \emph{Dialogues}, in 1688.} In the person of Theodore, Malebranche states:

\begin{quote}
	If [ideas] are eternal, immutable, necessary, in a word, divine \ldots surely they will be more considerable than that matter which is inefficacious \ldots. Be careful. If it is you who give being to your ideas, it is by willing to think of them. Now, pray tell, how can you will to think of a circle, if you do not already have some idea of it, from which to form and complete it? Can something be willed without being known?\footnote{\emph{Dialogue} I. VII. = JS 12-13.}
\end{quote}

Here, though, Malebranche's argument does not proceed from anything about \emph{occasionalism}: the denial of the mind's productivity with respect to its ideas follows from i) the dependence of willing on knowledge; and ii) the eternity of the ideas themselves -- the human mind cannot produce them because \emph{nothing} produces them. Hence, leaving aside whether the passivity of minds might be an untoward consequence of occasionalist tenets, this assumption was not itself among those tenets.

\section{Conclusion}
The distinction between partial and full-blown occasionalism assumes that on full-blown occasionalism, neither minds nor bodies are active. But in order to prove that minds are passive, the distinction tacitly relies on the assumption that all action is productive action. All major occasionalist figures reject it. For the occasionalists generally, minds may be active without being causal; they may even be the producers of their own ideas without producing anything real. We should thus understand occasionalism more restrictedly in terms of the claims that i) God is the only efficient cause, and ii) consequently, neither human minds nor extended bodies are efficient causes.\footnote{Cf. \autocite[625-626]{Platt2011}; \autocite[101]{Gouhier1926}.}
\section{Abbreviations}

A = Gottfried Wilhelm Leibniz. \emph{Sämtliche Schriften und Briefe}.
Ed. Deutsche Akademier der Wissenschaften zu Berlin. Berlin: Akademie,
1923.

AG = Gottfried Wilhelm Leibniz. \emph{Philosophical Writings}. Ed. R.
Ariew and D. Garber. Indianapolis: Hackett, 1989.

AT = René Descartes, \emph{Oeuvres de Descartes}, vols. 1-12, ed. Adam
and Tannery, revised edition. Paris: Vrin/CNRS, 1964-76.

C = \emph{Opuscules et fragments inédits de Leibniz}. Ed. by L.
Couturat. Paris: Alcan 1903. Reprinted Hildesheim: Georg Olms 1961.

CSMK = \emph{The Philosophical Writings of Descartes}, vols. 1-3, trans.
J. Cottingham,, R. Stoothoff, D. Murdoch, and A. Kenny. Cambridge:
Cambridge University Press, 1985-1991. Vols. 1 and 2 cited as CSM.

E = Benedictus de Spinoza, \emph{Ethica}. In Opera quotquot reperta
sunt. Ed. J. Van Vloten and J. P. N. Land. The Hague: Martinus Nijhoff.
Tomus Primus. pp. 35-273.

GP = Die Philosophischen Schriften von Gottfried Wilhelm Leibniz. Ed. C.
I. Gerhardt. 7 vols. Berlin: Weidmann, 1875-90. Reprinted Hildesheim:
Georg Olms, 1960.

\emph{In Metaph.} = Thomas Aquinas, \emph{Commentary on Aristotle's
	Metaphysics}. Trans. John P. Rowan. Html-edited by Joseph Kenny, O. P.
with addition of Aquinas's Latin and and Aristotle's Greek text.
\url{http://dhspriory.org/thomas/metaphysics5.htm}

JS = \emph{Nicolas Malebranche: Dialogues on Metaphysics and on
	Religion}. Ed. Nicholas Jolley and David Scott. Cambridge: Cambridge
University Press, 1997.

L = Gottfried Wilhelm Leibniz. \emph{Philosophical Papers and Letters}.
Ed. and trans. Leroy E. Loemker. Dordrecht: Kluwer, 1989.

LDV = Gottfried Wilhelm Leibniz. \emph{The Leibniz-De Volder
	Correspondence}. Ed. and trans. Paul Lodge. New Haven: Yale University
Press, 2013. Reference is to original language page.

GLW = \emph{Briefwechsel zwischen Leibniz und Christian Wolff}. Ed. C.
I. Gerhardt. Hildesheim: Georg Olms, 1963.

\emph{Med.} = René Descartes. \emph{Meditationes de Prima Philosophia}.
Cited by book and paragraph.

\emph{Med.} \emph{chr.} = Nicolas Malebranche. \emph{Méditations
	Chrétiennes}.

NE = Gottfried Wilhelm Leibniz \emph{New Essays on Human Understanding}.
Ed. P. Remnant and J. Bennett. Cambridge: Cambridge University Press,
1982.

NG = Nicolas Malebranche. \emph{Treatise on Nature and Grace}.
Translated with an introduction and notes by Patrick Riley. Oxford:
Clarendon Press.

\emph{OC} = \emph{Oeuvres Complètes de Malebranche}. Directeur A.
Robinet. 20 volumes. Paris: J. Vrin, 1958-1967.

\emph{Réponse} = Abraham Gaultier. \emph{Réponse en forme de
	dissertation à un théologien, Qui demande ce que veulent dire les
	sceptiques, qui cherchent la verité par tout dans la Nature, comme dans
	les écrits des philosophes; lors qu'ils pensent que la Vie et la Mort
	sont la même chose}. Ed. Olivier Bloch. Paris: Les Belles Lettres, coll.
Encre Marine, 2004.

\emph{ST} = Thomas Aquinas. \emph{Summa Theologiae}. Fathers of the
English Dominican Province, trans. Allen, TX: Christian Classics,
1948/1981.

T = Gottfried Wilhelm Leibniz, \emph{Theodicy}. Trans. E. M. Huggard. La
Salle, IL: Open Court, 1985.

\emph{Treatise} = Louis de La Forge, \emph{Treatise on the Human Mind
	(1664)}. Translation with an introduction and notes by D. M. Clarke.
Dordrecht: Kluwer, 1997.


%\printbibliography

%\end{document}
