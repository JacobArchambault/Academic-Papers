\documentclass{article}
\usepackage[backend=bibtex]{biblatex}
\usepackage[T1]{fontenc}
\usepackage[utf8]{inputenc}
\usepackage{lmodern}
\bibliography{jacob}

\title{Distinguishing Leibniz's system of pre-established harmony from the system of occasional causes}
\author{Jacob Archambault}

\begin{document}

\maketitle

  
\begin{abstract}
One of the more persistent interpretations of
Leibniz's system of pre-established harmony is as a temporal dislocation
of occasionalism: whatever God is always doing on the occasionalist
account he need only have done once -- at creation -- on Leibniz's. In
accordance with this interpretation, the difference between the systems
of La Forge, Cordemoy, and Malebranche, on the one hand, and of Leibniz,
on the other, is one of how involved God is in the world.

I show that Leibniz himself never intended to distinguish the systems in
this way. Rather, the basic difference between the systems should be
found in their accounts of the nature of substance. Recognizing this
allows us to better appreciate the character of Leibnizian and
occasionalist systems \emph{as} systems; to turn the page on discussions
of occasionalism centered on Leibniz's interpretation; and to lay the
groundwork for an improved taxonomy of the place of these systems in
early modernity.
\end{abstract}

Keywords: Gottfried Wilhelm Leibniz; Nicolas Malebranche; Louis de La
Forge; Geraud de Cordemoy; occasionalism; pre-established harmony; early
modern theories of substance; causation in early modern physics;
mind-body problem.

\section{Introduction}

One of the more persistent interpretations of Leibniz's system of
pre-established harmony is as a temporal dislocation of occasionalism:
whatever God is always doing on the occasionalist account he need only
have done once -- at creation -- on Leibniz's.\footnote{The problem of
  distinguishing Leibniz's system from occasionalism was first raised by
  Arnauld; see \autocite[vol. II, 84-90]{GP}} In accordance with this, the difference
between occasionalism and Leibniz's pre-established harmony is one of
how involved God is in the world. Call this the \emph{traditional
interpretation} of the difference between the systems.\footnote{The
  interpretation of Leibnizian and occasionalist systems is closely
  connected to questions surrounding \emph{divine concurrence and
  conservation}, i.e. whether and how God preserves created beings in
  existence, and concurs with their actions. Whipple shows that Leibniz rejects
  an account of moment-to-moment conservation of creatures, not because
  he rejects the absolute dependence of creatures on God, but because he
  rejects the division of time into instants suggested by the
  traditional understanding of the doctrine: 
  \begin{quote}
  	`a literal commitment to an
  	instantaneous account of continual creation would severely compromise
  	the coherence of Leibniz's theory of finite substance \ldots. If
  	Leibniz's considered view on the ontological status of instantaneous
  	states is the one suggested earlier, then it is not surprising to see
  	him claiming that the continual creation doctrine seems to imply that
  	creatures \emph{never exist}' \autocite[870]{Whipple2010}.
  \end{quote}}

The claim that occasionalism and the pre-established harmony differ
foremost in degree of divine activity has been assumed both as an
account of the genuine difference between occasionalism and the
pre-established harmony, and as one of what Leibniz himself took the
difference to be. Russell, Nadler, and Detlefsen attribute this
interpretation to Leibniz and broadly agree with it. Clarke, Jolley,
Scott, and Lee take the pre-established harmony to be closer to
occasionalism than Leibniz himself realized or cared to admit.\footnote{\autocite{Russell1951}; \autocite[31-32]{Nadler1993}; \autocite[449]{Detlefsen2003}; \autocite[121]{Clarke1989};
  \autocite{Clarke1995}; \autocite{Scott1997}; \autocite[246]{Jolley2002}; \autocite[106-107]{Jolley1990}; \autocite[39]{Black1997}; \autocite[230]{Lee2004}.} Though there is lively debate on what the
occasionalist position was and how/whether it differed from Leibniz's,
it is generally not contested that Leibniz characterized the difference
between his and occasionalist systems in the above way. The effects of
this assumption have been striking.

In scholarship on occasionalism, the assumption has abetted a dispute
between `generalist' and `particularist' interpretations of
Malebranche's claim that God only acts by general volitions.\footnote{For
  two superb treatments of the general will in Malebranche, see \autocite{Schmaltz2008} and \autocite{Wahl2011}.} The \emph{generalist} interpretation holds that,
with the exception of miracles, God only acts by willing general laws of
nature; the \emph{particularist} interpretation, by contrast, takes the
immediate object of God's volitions to be particular events, albeit
events in accord with general laws. Nadler, whose work is most
prominently associated with the particularist interpretation, first
introduced it as a defense of a `traditional' reading he attributes to
Leibniz,\footnote{`Leibniz accuses Malebranche of introducing
  `continuous miracles' into the course of nature at every moment, and
  of having God `intervene' with the laws of bodies and of thought.
  \ldots I shall argue that we ought to keep to this traditional
  reading of Malebranche's occasionalism.' \autocite[32]{Nadler1993}.} and
maintains the alternative description `more closely resembles Leibniz's
preestablished harmony.'\footnote{\autocite[31]{Nadler1993}.} Many of his
opponents have agreed.\footnote{See esp. \autocite{Clarke1995} and \autocite{Scott1997}.}

In scholarship on Leibniz, the assumption has encouraged assimilating
Leibniz's disagreements with occasionalism to those over methodological
naturalism in the history and philosophy of science, culminating in the
charge that Leibniz was secretly a \emph{metaphysical} naturalist
besides being a methodological one. This reading of Leibniz's critique
has been used to distinguish between an `esoteric' and `exoteric'
Leibniz;\footnote{\autocite{Russell1951}; cf. \autocite{Shields1986}; \autocite[277-281]{Wilson1989}.
  More recent work has largely rejected Russell's distinction. See
  \autocite[4-9]{Mercer2001}; \autocite[420-426]{Rateau2008}; \autocite{Antognazza2009}. For a more
  measured account of Leibniz's use of exoteric writing, see \autocite{Whipple2015}.} and more mundanely, to promote a more disjoint view of
Leibniz's development than plausible, one on which Leibniz makes a
series of concessions to occasionalism before dropping the contrast
altogether.\footnote{\autocite{Scott1997}. This has been part of a broader trend
  away from holistic accounts of Leibniz's philosophy and towards views
  emphasizing its development. See \autocite[373]{Wilson1999}; \autocite[78]{Garber2008};
  \autocite[388]{Garber2009}; \autocite[223-232]{Garber2014}. Though Leibniz's views on various topics
  certainly developed, and though the exoteric/esoteric distinction of
  Russell and others is cruder than more recent developmentalist work,
  the \emph{function} of developmentalist hypotheses in Leibniz
  scholarship has been much the same as the earlier distinction:
  fragmentation at the service of manageability. While paying verbal
  respects to the complexity of Leibniz's thought, such readings, in
  failing to respect the \emph{integrity} of that thought, reduce that
  complexity to confusion. For more positive assessments of Leibniz's
  systematicity, see \autocite{Rescher1981}, \autocite{Goldenbaum2002}.}

In the following, I show the basic difference between occasionalism and
the pre-established harmony should be found not in their degree of
divine activity, but in their accounts of substance.\footnote{That
  Leibniz critiques Malebranche's view of substance, albeit as one
  critique among others, has been recognized by \autocite[40]{Rutherford1993};
  \autocite[299-301]{Garber1995}; \autocite[193-194]{Garber2009}; \autocite[280-281, fn. 16]{Brown2000}. The
  clearest statement of the centrality of this difference for Leibniz
  comes from \autocite{Wahl2011}:

  \begin{quote}
  Leibniz's real problem with Malebranche's view should not have been
  expressed in terms of miracles or the role of secondary causes at all.
  His real problem is with Malebranche's concept of substance. (237; see
  also \autocite{Whipple2010})
  \end{quote}

  Wahl, however, does not purse the issue in its own right. Wahl's
  phrasing also has the unfortunate effect of contrasting disagreements
  over substance with those over miracles and secondary causes, rather
  than recognizing that the disagreements over the first form the basis
  for Leibniz's disagreements over the second and third.} Furthermore,
\emph{Leibniz himself never intended to distinguish his own system from
occasionalism in the traditional way}. In untying this small knot, three
larger goals may be attained:

\begin{enumerate}
\item We undercut the historical roots of the current paradigm of
  scholarship on occasionalism, thereby resetting the terms of
  discussion around which that scholarship has revolved.
\item We arrive at an integral account of occasionalist and Leibnizian
  systems \emph{as systems}, thereby reinstating the status of both
  Malebranche and Leibniz as systematic philosophers. In the latter
  case, this prepares the ground for a deeper and more unified
  appreciation of Leibniz's development.
\item We achieve a deeper understanding of the \emph{differences} between
  Leibniz and Malebranche's systems, and thereby lay the groundwork for
  an improved taxonomy of the places of these systems within early
  modernity.
\end{enumerate}

The paper has two tasks: the first, negative, to provide a diagnosis of
why the traditional interpretation is tempting and show where it is
flawed; the second, to show that Leibniz fundamentally distinguishes his
system from occasionalism by contrasting their accounts of substance,
and that the various other distinctions between the systems are, in
Leibniz's view, consequent upon this.

Since the task is that of distinguishing two \emph{systems,} it is
necessary that such an account be general in nature; since the number of
disagreements over Leibniz's corpus in particular are enormous, I shall
not address those disagreements directly, but indicate those relevant in
passing; and since interpretive assumptions like those of duplicity
aren't directly refutable, it becomes necessary to simply offer a better
account. The proof of my interpretation of both systems shall be, so to
speak, in the pudding: in the resulting coherence of the systems
themselves, and the exegetical utility of the interpretation for
situating these accounts in the dialectic of early modernity.

I begin by reviewing current thinking on the defining marks of
occasionalism, and advance a more restricted account better supported by
canonical occasionalist texts. I then locate Leibniz's two-clocks
analogy, used to distinguish the occasionalist system from his own,
among a broader array of clock metaphors in early modernity. I show that
the traditional interpretation of Leibniz's pre-established harmony is
closely bound up with a failure to identify the salient features of
Leibniz's use of the analogy, albeit an understandable one given
alternative uses of the image from the period. I then show how Leibniz's
different uses of the analogy clarify both the systematic character of
Leibniz's pre-established harmony, as well as its differences from
occasionalism, directly addressing the issues Leibniz sought to contrast
the two systems on: the nature of substance, the communication of
substances, and mind-body union.

\section{The Defining Marks of the System of Occasional
Causes}

While occasionalism is often associated with the problems of
individuation and the mind-body relation, these provide it with neither
its most characteristic doctrines nor its essential motivation. Rather,
one better understands occasionalism as a gradual expansion of the
doctrine of continuous creation. According to this doctrine, creatures
depend on God not merely in that God must have created them, but also in
that God must conserve them at each instant.\footnote{See Thomas
  Aquinas, \emph{ST} Ia, q. 104, art. 1, res; cf. Descartes, AT VII.
  369.} The earliest occasionalist accounts insist that God not only
conserves material creatures in being, but also, since it is impossible
to preserve a material object in existence without preserving it
somewhere, must conserve them in a specific \emph{place} in doing
so.\footnote{See La Forge, \emph{Traité de l'esprit de l'homme}, in
  \emph{Oevres philosophiques}, ed. Pierre Clair (Paris: Presses
  Universitaires de France, 1974), 240-41. Cf. Malebranche,
  \emph{Entretiens} VII, par. 10 = OC XII. 160; \autocite{Nadler1998}.} Given that
motion is simply change of place on the Cartesian account, it follows
that God, being immediately and solely responsible for the successive
positioning of material objects, is likewise responsible for their
motion. Material objects therefore do not move themselves. And given the
only way material objects could act on each other would be by contact,
itself a function of relative position, it follows, the occasionalist
maintains, that material objects are causally inert.

Though the above argument establishes nothing about whether \emph{minds}
are active or causal, recent scholarship holds occasionalist minds, like
occasionalist bodies, are purely passive. Nadler defines occasionalism
as the joint acceptance of the following theses:

\begin{enumerate}
\item
  Natural objects, both minds and bodies, have no causal efficacy
\item
  God alone is a true efficient cause.\footnote{\autocite[39]{Nadler2005}. Similar
    conceptions have been assumed by \autocite[45-46]{Pyle2003}; Battail 1973,
    174; \autocite[288]{Winkler2011}. A different conception, on which minds need
    not be wholly passive, is assumed in \autocite[358]{Radner1993}.}
\end{enumerate}

That the second of these is constitutive of occasionalism is
uncontroversial. Whether the first should be so regarded depends on what
is meant by `causal efficacy'. The occasionalist admits both minds and
bodies are causes in a loose sense: both mental volition and bodily
contact serve as occasions whereupon God must exercise his causal
efficacy. On the other hand, the second above thesis entails that minds
other than God are not efficient causes. Since efficient causation is
traditionally associated with production, a restricted reading of this
entailment implies that creatures are not \emph{productive}. A stronger
reading, drawn from a similarly traditional association of efficient
causation with agency, holds neither minds nor bodies are
\emph{active.}\footnote{Cf. Aquinas, \emph{in Metaph.} Bk. 5, lec. 2-3.
  In connection with Malebranche, see \autocite[255-256]{Baker2005}.} The stronger
reading is cemented in a terminological distinction between full and
partial occasionalists, the latter being those who ascribe activity to
the mind.\footnote{This broader reading is accepted by \autocite{Nadler2005}; cf.
  \autocite[288]{Winkler2011}. Nadler accepts that La Forge views the mind as
  active, but for this reason regards him as only a partial
  occasionalist \autocite[227]{Nadler1998}. The distinction between partial and
  full occasionalism is also found in \autocite{Radner1993}, \autocite{Garber1987}, \autocite{Clarke2000}, \autocite{Bardout2002}, and \autocite{Kolesnik2006}.} But while the
assumption that activity presupposes productivity plays an important
role in Leibniz's polemic against occasionalism,\footnote{See GP IV.
  586-87; \autocite{Rutherford1993}.} all major occasionalist figures reject the
assumption. For the occasionalists generally, minds may be active
without being causal; they may even be the producers of their own ideas
without producing anything real. We should thus understand occasionalism
more restrictedly in terms of the claims that i) God is the only
efficient cause, and ii) consequently, neither human minds nor extended
bodies are efficient causes.\footnote{Cf. \autocite[625-626]{Platt2011}; \autocite[101]{Gouhier1926}.}

In \emph{On Nature Itself}, Leibniz describes occasionalism as follows:

\begin{quote}
But now let us consider a little more closely the opinion of those
\ldots who judge not things to act, but God at the presence of
things and according to the aptness of things; and thus {[}judge{]}
things to be occasions, not causes, and to receive, not to effect or
elicit. When Cordemoy, La Forge, and other Cartesians set forth this
doctrine, Malebranche especially adorned {[}it{]} with certain
rhetorical lights on account of his acumen; but brought forth
\ldots no solid reasons. Surely, if this doctrine leads to the
point of even taking the \emph{immanent actions} of substances away
\ldots, then it appears foreign to reason like nothing
else.\footnote{GP IV. 509-510.}
\end{quote}

Here, Leibniz names Cordemoy, La Forge, and Malebranche as leading
occasionalists, and describes occasionalists as being led to the
conclusion that not even minds are active. But if they are so led, it
will be against their own explicit commitments. La Forge, for instance,
holds

\begin{quote}
the essence of this faculty {[}i.e. the will{]} consists firstly in the
fact that it is the active principle of all the mind's actions which
chooses from itself and by itself, and determines itself to accept or
reject what the understanding perceives or remain suspended when
something is not yet perceived clearly enough.\footnote{\emph{Treatise}
  ch. XI, 97.}
\end{quote}

Like La Forge, Cordemoy describes willing as the mind's activity:
`{[}God{]} made minds \ldots capable of action; they
will.'\footnote{\emph{Traité de métaphysique,} CG 283, trans. from
  \autocite[52]{Nadler2005}. Cf. \emph{Discours physique,} CG 255:

  \begin{quote}
  Just as the body is a substance to which extension naturally belongs,
  so much so that, as for physical effects, it would cease to be a body
  if it ceased to be extended; in the same way the mind is a substance
  to which the power of determining itself belongs so naturally, that it
  would cease to be a mind if it ceased to will. (trans. from \autocite[47]{Nadler2005})
  \end{quote}

  Nadler argues Cordemoy is a `full-blown' occasionalist from his
  remarks that `it is just as impossible for souls to have new
  perceptions without God as it is for bodies to acquire new motions
  without him.' (\emph{Discours physique}, CG 255. Trans. from
  \autocite[50]{Nadler2005}). But these remarks and others to the same effect establish
  neither that the mind is inactive nor that it does not cause its
  ideas. Rather, such passages can be harmonized with those above on the
  assumptions i) that Cordemoy, like many of his contemporaries, was a
  compatibilist about divine and human willing, and ii) that because
  ideas are not real beings, but beings of reason, eidetic production
  does not provide an instance of real causation.} We also find this
commitment to the active character of mind in Malebranche, who calls the
mind's consent to the good an act, albeit `an immanent act that produces
nothing material in our substance.'\footnote{\emph{Éclaircissement} 1 to
  \emph{De la recherché de la vérité} OC III, 25. Trans. from \autocite[52]{Nadler2005}.}

Elsewhere, La Forge assumes the mind is the cause of its own ideas:

\begin{quote}
Although our thoughts follow one another and although it is the external
objects or the first thoughts which provide an occasion for the will to
determine itself and form the idea of subsequent thoughts, that does not
imply that one should not say that it is the will which is the principal
and proximate cause of the idea. Otherwise one would have to say that it
is the external objects which produce the ideas that we have of them
\emph{and not the mind} {[}..{]}.\footnote{\emph{Treatise} ch. X, 94.
  Emphasis mine.}
\end{quote}

In spite of this, La Forge is clear the mind produces nothing
\emph{material}.

\begin{quote}
{[}T{]}here is no creature, spiritual or corporeal, which can cause
change in {[}matter{]} or in any of its parts, in the second moment of
their creation, if the Creator does not do so himself.\footnote{\emph{Treatise},
  147. Cf. \autocite{Klima1993}.}
\end{quote}

In his earlier works, Malebranche, too, does not rule out the mind being
the productive cause of its own ideas: the mind's act `produces nothing
\emph{material} in our substance';\footnote{\emph{Éclaircissement} 1 to
  \emph{De la recherché de la vérité} OC III, 25. Trans. from \autocite[52]{Nadler2005}. Emphasis mine.} minds `do not \ldots produce in
themselves a reality, or a modification that \emph{physically} changes
their substance.'\footnote{\emph{Réponse à la Dissertation} = OC 7, 568,
  Trans. from \autocite[52]{Nadler2005}. Emphasis mine. Cf. OC IX, 1129.} Here,
both La Forge and Malebranche assume the production of ideas, though an
activity, is not production in the proper sense, since the will's
production of an idea is not the production of `a reality', i.e. a real
substance or quality.

But while the \emph{Search after Truth} and its \emph{Elucidations}
offer no rejection of the mind's productive power with respect to its
own ideas, this sort of production \emph{is} ruled out in the later
\emph{Dialogues on Metaphysics and on Religion}.\footnote{The
  \emph{Search} was first published in 1674-5; the \emph{Elucidations},
  in 1678 as a supplement to the third edition of the \emph{Search}; the
  \emph{Dialogues}, in 1688.} In the person of Theodore, Malebranche
states:

\begin{quote}
If {[}ideas{]} are eternal, immutable, necessary, in a word, divine
\ldots surely they will be more considerable than that matter
which is inefficacious \ldots. Be careful. If it is you who give
being to your ideas, it is by willing to think of them. Now, pray tell,
how can you will to think of a circle, if you do not already have some
idea of it, from which to form and complete it? Can something be willed
without being known?\footnote{\emph{Dialogue} I. VII. = JS 12-13.}
\end{quote}

Here, though, Malebranche's argument does not proceed from anything
about \emph{occasionalism}: the denial of the mind's productivity with
respect to its ideas follows from i) the dependence of willing on
knowledge; and ii) the eternity of the ideas themselves -- the human
mind cannot produce them because \emph{nothing} produces them. Hence,
leaving aside whether the passivity of minds might be an untoward
consequence of occasionalist tenets, this assumption was not itself
among those tenets.

\section{The Uses of Clock Metaphors in the Early Modern Philosophical
Imaginary}

Clocks hold an important place in the early modern philosophical
imaginary. Like computers today, clocks were put to a wide variety of
uses by early modern authors with often disparate philosophical views.
Descartes compares a well-made and badly-made clock in that both equally
follow the laws of nature.\footnote{\emph{Med.} VI. 17 = AT VII. 84 =
  CSM II. 58.} Geulincx, prefiguring Leibniz, uses clock imagery to
express the harmony of creation.\footnote{\autocite[168]{Cooney1978}.} Malebranche
compares a God who wills by particular volitions rather than by general
laws to `a watchmaker with a watch which would stop at each moment
without his aid.'\footnote{\emph{Med. chr.}, OC X, 78. Trans. from \autocite[235]{Wahl2011}.} Fontenelle uses the analogy of a watchmaker in a design
argument for the existence of God.\footnote{\autocite[618]{Durant1963}.}
Employed to the same purpose, the analogy is later contested in Hume's
\emph{Dialogues}. Berkeley mentions the analogy in connection with the
Dominican scholastic Durandus of St. Pourçain, `who held the world to be
a machine like a clock, put in motion by God, but afterwards continuing
to go of itself.'\footnote{Letter to Johnson, Nov. 25, 1729. In \autocite[291]{Winkler2011}. Cf. Vailati 2002, 230: `In the end, one can hardly escape
  the impression that Leibniz's position was closer to Durandus's than
  he cared to admit.'} Criticizing Descartes' association of truth with
clarity and distinctness, Abraham Gaultier held human matter
spontaneously takes on higher functions in a broader cultural context,
as the parts of a clock give rise to an instrument used to tell time.
This same theme was later picked up by La Mettrie and Diderot.\footnote{Gaultier,
  \emph{Reponse}, 85-86; see \autocite{Kaitaro2016}.}


\subsection{The Two-clocks Analogy in the Traditional Interpretation
}

Leibniz, too, makes prominent use of clock imagery to describe his
system, famously in the two-clocks analogy distinguishing his system
from both occasionalism, and scholasticism. The earliest use of the
analogy comes not from Leibniz himself, but from Simon Foucher, in a
letter recording objections to Leibniz's \emph{New System of
Nature}.\footnote{GP IV. 488-489.} Leibniz's first use occurs in a
letter to Basnage de Beauval:

\begin{quote}
Consider two clocks or watches in perfect agreement. Now this can happen
in \emph{three ways}: the \emph{first} is that of a natural influence.
This is what Huygens experienced, to his great surprise. He had
suspended two pendula from the same piece of wood, and the constant
swinging of the pendula transmitted similar vibrations to the particles
of wood. But these vibrations could not continue in an orderly way
without interfering with each other, at least while the two pendula were
not in accord with one another, it happened in a marvelous way that even
when the swings of the pendula had been intentionally disturbed, they
came to swing together again, almost as if they were two strings in
unison. \emph{The second way} to make two faulty clocks always agree
would be to have them watched over by a competent workman, who would
adjust them and get them to agree at every moment. \emph{The third way}
is to construct these two clocks from the start with so much skill and
accuracy that one can be certain of their subsequent
agreement.\footnote{GP IV. 498 = AG 147-148.}
\end{quote}

At the center of the traditional interpretation is a particular
understanding of the two-clocks analogy. On this interpretation, the
salient features of the analogy are two: 1) the image of the first
clockmaker, i.e. the God of the occasionalist system, as a busybody; and
2) the image of the second clockmaker, i.e. the Leibnizian God, as
ensuring the agreement `in advance', i.e. at a chronologically prior
point in time.

For Russell, Malebranche `held that \ldots the changes in matter
corresponding to those in mind must be effected by the direct operation
of God in each case. In Leibniz, on the contrary, only one original
miracle was required to start all the clocks \ldots---the rest
was all effected naturally.'\footnote{\autocite[137]{Russell1951}.} Nadler
contrasts his reading of Malebranche, where `Malebranche's God is
personally, directly, and immediately responsible'\footnote{\autocite[32]{Nadler1993}.} for changes in nature, with a Leibnizian one where `God
originally {[}establishes{]} such a correspondence once and for all by
means of a few general volitions.'\footnote{Ibid., 31} For Clarke, no
difference is to be found between the systems concerning God's activity,
and so Malebranche and Leibniz's disagreements reduce to a verbal
dispute.\footnote{\autocite[121]{Clarke1989}; \autocite[passim]{Clarke1995}; cf. \autocite[39]{Black1997}. Likewise, for Sukjae Lee, `the key difference {[}between Leibniz
  and Malebranche's systems{]} is that Leibniz accepts the force of
  reasons as a genuine type of causation while Malebranche does not.'
  \autocite[230]{Lee2004}. Lee goes on to claim the difference `is not merely
  terminological,' but his argument is unconvincing.} Scott, contrasting
the `force of independence' given to Leibnizian monads at creation with
a Malebranchian view on which `creatures are entirely dependent on God
from moment to moment,'\footnote{\autocite[452]{Scott1997}. In Leibniz's time, Des
  Bosses argued that Leibniz's account of active force undermined the
  need for divine concurrence. See GP II. 293.} construes Leibniz's
variant applications of the example\footnote{GP IV. 498; 520, 522; VI.
  540-41.} as a series of implicit concessions ending with Leibniz
abandoning the metaphor altogether.\footnote{\autocite[462]{Scott1997}. For
  criticism, see \autocite[321-322]{Bobro2008}.} Additional variations on this
reading have been advanced by Jolley,\footnote{Jolley 2002 takes Leibniz
  to describe Malebranche's system as one on which `the occasionalist
  God is a busybody God.' \autocite[246]{Jolley2002}; cf. \autocite[106-107]{Jolley1990}.}
Detlefsen,\footnote{`Leibniz \ldots makes clear that
  occasionalism is miraculous because it posits God's constant activity
  in the so-called natural world.' \autocite[443]{Detlefsen2003}.} Stuart
Brown,\footnote{`Leibniz took {[}Malebranche's thought{]} a step closer
  to what is commonly referred to as `deism,' in which miracles and
  particular providence is denied outright.' \autocite[273]{Brown2000}.} and
others.\footnote{McCracken 1983, 101; \autocite{Brown2007} suggests Leibniz could
  have accounted for Newtonian gravitation---described by Leibniz as a
  `perpetual miracle'---by rolling it `into that initial, mega-miracle
  of creation.' (147).}

Given the widespread association of clock metaphors with limitations on
God's activity -- with Malebranche, to hold God doesn't act by
particular volitions; with Fontenelle, Berkeley, and Hume, in
association with Deism; and in Gaultier, La Mettrie, and Diderot, with
materialism -- it is perhaps unavoidable that Leibniz's employment of
the metaphor would be located along this trajectory. But taken on their
own terms, Leibniz's uses don't suggest this.

To see this, we should reflect on what this traditional interpretation
of the clock analogy actually requires. Since the Leibnizian God is
portrayed as fixing the clocks `from the start', it must be possible to
contrast acts of creation temporally. Since the Leibnizian clock working
on its own is contrasted with the occasionalist clockmaker's tampering,
there should be a roughly inverse correlation between creaturely and
divine activity. And since Leibniz admits the occasionalist hypothesis
as a metaphysical possible one, it must be possible for God to be more
or less active.

Leibniz rejects these presuppositions. With Malebranche and Spinoza --
but against Descartes, Hobbes, Locke and Newton -- Leibniz accepts that
God is in eternity, outside time.\footnote{See the references in \autocite[414]{Gorham2008}. Cf. \autocite[250]{Jolley2002}. A growing scholarly literature reads
  Leibniz as refusing to characterize even the \emph{objects} of
  creation as temporal, strictly speaking. See \autocite{Whipple2010}, \autocite{Whipple2011}; \autocite{Uchii2015}. Cf. Arthur 1985; \autocite{Lloyd2008}.} Though Malebranche held that
God's continuous creation crowds out creaturely activity,\footnote{JS
  VII, 104-126.} Leibniz rejected this inverse relationship, even
linking its acceptance to Spinozism.\footnote{GP IV, 567-68; cf. GP IV,
  509, 515; \autocite[318-320]{Bobro2008}.} Lastly, Leibniz had rejected the
quantitative understanding of the contrast prior to his earliest
explication of the clock analogy. Arnauld, himself no friend of
occasionalism, provides perhaps the earliest ascription of the
traditional interpretation to Leibniz, accusing Leibniz of
misrepresenting the occasionalists:

\begin{quote}
Those who maintain that my will is the occasional cause of the movement
of my arm and that God is its real cause do not claim that God does this
in time by a new act of will each time that I wish to raise my arm, but
by that single act of the eternal will by which he has willed to do
everything which he has foreseen it will be necessary to do, in order
that the universe might be such as he has decided it ought to
be.\footnote{GP II. 84. Translation from \autocite[246]{Jolley2002}.}
\end{quote}

Leibniz replies:

\begin{quote}
The authors of occasional causes {[}...{]} introduce a miracle that is
not less so for being continual. For it seems to me that the notion of a
miracle does not consist in rarity {[}...{]}. It seems to me that
following {[}common{]} usage, a miracle differs internally and by the
substance of the act of a common action, and not by an external accident
of frequent repetition; and that properly speaking, God performs a
miracle when he does a thing that surpasses the forces he has given to
creatures \emph{and that he conserves} in them.\footnote{GP II. 92-93.}
\end{quote}

In his reply, Leibniz's complaint against occasionalism is not that God
would be supposed to act frequently rather than in the beginning, or
irregularly rather than according to a general law, but in place of
creaturely power rather than in accord with it.\footnote{Leibniz both
  ascribes to Malebranche and endorses the view that God wills evils
  only by general acts of the will at GP VI. 238. Cf. \autocite{Schmaltz2010}.}
Without prompting, Leibniz explicitly confirms God's action on the
creature at all times via divine concurrence, and does not seem to
regard his system and occasionalism as in conflict on this
point.\footnote{See also GP II. 91-92; IV. 588; VI. 295-296; VII. 564.}

\subsection{Leibniz's Clocks and the Unity of the Pre-established
harmony as a
System}

The difficulties with the traditional interpretation are not only
theoretical. The mutual compatibility of Leibniz's different uses of the
analogy is already implied in the title of the \emph{New System} --- the
\emph{Système nouveau de la nature et de la communication des
substances, aussi bien que de l'union qu'il y a entre l'âme et le
corps}.\footnote{GP IV. 677.} Leibniz is proposing a new system, the
pre-established harmony, to replace an old system, occasionalism, which
Leibniz calls `the Cartesian system,'\footnote{GP IV. 520.} or more
frequently, the `system of occasional causes.'\footnote{GP IV. 483, 520.}
Each, while being one system, attempts a unified solution to three
different problems: the nature of substance, their communication, and
the union of soul and body. Leibniz's use of the two-clocks example on
these topics confirms the essential unity of his thinking about them.

In Foucher's use of the clock image, Foucher sees Leibnizian forces as
principles of unity and activity, but then asks what the point of
according these to creatures would be, if not to affirm that creatures
act \emph{on each other}? `After all, what is this whole grand artifice
in substances to the service of, if not to establish the belief that the
ones act on the others?'\footnote{GP I. 425.}

Leibniz's earliest use of the analogy is written with this charge in
mind. This is why he goes into the detail he does in describing Huygens'
experiments, which have not been given the attention they deserve for
understanding Leibniz's analogy. Huygens found that the pendula of the
two clocks, representing the realist system, disturbed each other. But
ultimately, they came into harmony with each other, reaching equilibrium
while retaining their own proper movement. The realist system is thus
represented as disharmonious, disorderly. In contrast to this, Leibniz's
system establishes the harmony eventually reached in Huygen's experiment
`from the start'. Thus, this language contrasts Leibniz's system not
with occasionalism, but with the scholastic approach. This phrase is
notably absent in later uses of the analogy, e.g. the remarks on Bayle's
dictionary, where the point of contrast shifts from scholasticism to
occasionalism.

Unlike in the occasionalist system -- and, for that matter, the
Newtonian one -- the motion Leibniz attributes to an object is not
merely its passive movement along a straight line, which would have to
be constantly adjusted to account for non-linear motion. Like Huygens'
swinging pendula, the harmonious motions Leibniz's beings follow may be
complex.

\begin{quote}
When it is said that a simple being will always act uniformly, a
distinction needs to be made. If to act uniformly is to follow
perpetually the same law of order or of succession {[}...{]} I agree
that in this sense every simple being and even every composite being
acts uniformly. But if uniformly means similarly, I do not
agree.\footnote{GP IV. 522 = L 495.}
\end{quote}

While the analogy does not receive sustained treatment after 1705,
Leibniz continues to allude to it up to his death.\footnote{See GP VII.
  352 = AG 320-21; GP VII. 417-418 = AG. 345-46.} In the Leibniz-Clarke
correspondence in particular, Leibniz pursues a strategy of assimilating
the errors of the Newtonians to occasionalism, a sensible one given the
popularity of his \emph{Theodicy} with Princess Caroline of Wales, who
initiated and mediated the correspondence.\footnote{See {[}reference
  omitted{]}. On Caroline's role in the correspondence, see
  \autocite{Bertoloni-Meli1999}.} The polemical point of the analogy would thus
not have been lost on Leibniz's contemporary readers:
\emph{Occasionalist substances don't work.} Leibniz's bodies and minds,
by contrast, intrinsically follow the laws of efficient and final
causes.\footnote{Cf. GP VII. 417-418 = AG. 345-46.}

\section{Reintegrating Leibniz's Pre-established
Harmony}

\subsection{The Nature of Leibnizian
Substances}

In explicating the difference between occasionalism and the
pre-established harmony, we follow the order Leibniz himself marks out
for us. According to the full title of the \emph{New System}, the
problem that receives top billing is the nature of substance.

Malebranche and other occasionalists respond to this problem by adopting
the Cartesian division of substance into \emph{res extensa} and
\emph{res cogitans}.\footnote{See JS I. ii, 6; III. x-xi, 39-40. Cf.
  \autocite{Bardout2000}; \autocite[247-249]{Pessin2004}.} It is on account of this
ontological thesis that the problems of the communication of mind and
body and the interaction of substances arise. Occasionalism accepts the
ontological thesis and appeals to God to solve the other two problems.
Leibniz calls for a modification of the ontological thesis itself.

Leibniz begins the \emph{New System} critiquing the Cartesian concept of
body, \emph{res extensa}, as incapable of designating \emph{what} is
extended. This concept, says Leibniz, can neither distinguish one body
from another, nor individuate any body as \emph{one}. `It is impossible
to find the principles of a true unity in matter alone or in what is
only passive; since the whole here is nothing but a collection or mass
of parts to infinity.'\footnote{GP IV. 478; A 2.1, 511; L 274.} For
Leibniz, \emph{res extensa} cannot be the ultimate building block for
everything else, but must itself be subordinated to a principle of
unity.'\footnote{GP IV. 478-79.}

But Leibniz does not critique the second half of the Cartesian dichotomy
-- \emph{res cogitans}, which Descartes explicitly identifies with
soul.\footnote{Cf. \emph{Med.} II. 6} Rather, Leibniz follows Descartes
in this identification: `Moreover, by means of the soul or form, there
is a true unity which answers to what we call \emph{I} in us.'\footnote{GP
  IV. 482.} Drawing on Spinoza,\footnote{Cf. E III. \emph{propositiones}
  6-9.} Leibniz holds `their nature consists in \emph{force} and
{[}...{]} from this follows something analogous to sentiment or
appetite; and also that it would be necessary to conceive of them on the
pattern of the notion we have of souls.'\footnote{GP IV. 479.}
Elsewhere, he writes metaphysical points `have something of the vital
and a kind of perception.'\footnote{Ibid. 483. Cf. GP II. 282 = L 539:
  It is `essential to substance that its present state involves its
  future states and vice versa. And there is nowhere else that force is
  to be found or a basis for the transition to new perceptions.'} In
spite of his far-ranging differences with Descartes, Leibniz thus
follows him in granting a privileged status to the \emph{cogito} as the
\emph{terminus a quo} of his metaphysics.\footnote{Cf. GP VI. 502 = AG
  188: `Since I conceive that other beings can also have the right to
  say `I', or that it can be said for them, it is through this that I
  conceive what is called substance.' GP II. 270: `It can be further
  suggested that this principle of activity {[}force{]} is intelligible
  to us in the highest degree because it forms to some extent an
  analogue to what is intrinsic to ourselves, namely, representing and
  striving.' Quoted from \autocite[1177]{Lodge2015}. See also \autocite[329-331]{Huffman1993}; \autocite[327-328, 338, 347-349]{Adams1993}; \autocite{Stevenson1997}; \autocite[63-81]{Heidegger1998}; \autocite{Lodge2015}.}

For Leibniz, the soul is thus a kind of substantial form, the most
prominent characteristic of which is indivisibility.\footnote{See GP IV.
  479} It is a `true unity,' a `real unity,' a `substantial unity'; a
`real and living point,' a `metaphysical point'; an `atom of
substance.'\footnote{GP IV. 478, 482.} These descriptions contrast
Leibnizian forms with two different kinds of points: a) mathematical
points, which are exact but not real, and b) physical points, which are
real but not strictly points at all, since they are always in principle
divisible.\footnote{GP IV. 478, 483.}

Capitalizing on the occasionalist depiction of the soul's perceptual
activity as God's pushing the soul towards the highest good, \footnote{Cf.
  Malebranche, \emph{NG} III. VII, 172; Cordemoy, \emph{Traité de
  Metaphysique}, CG 284.} Leibniz portrays perception as an act of
striving forward, a desire tending towards its fulfillment.\footnote{Cf.
  LDV 318. \autocite{Jorgensen2015}; This insight has been put to especially good
  use in McDonough's studies of the role of teleology in Leibniz's
  physics. See \autocite{McDonough2009,McDonough2016}.} Leibniz thus highlights the
temporal structure of consciousness by seeing each new perception as
unfolding from the last. Each impression leads to the next, in the same
way a masterful painting -- one might, for instance, think of Van Gogh's
\emph{Starry Night} -- leads the eye across the canvass, so that it
might be taken in fully, and yet is taken in as a unity in spite of the
mind's inability to encompass it in a single perception.\footnote{The
  analogy of the temporal unity of a beautiful piece of music --
  suggested by Leibniz's very decision to call his system a harmony,
  serves the same point. See esp. GP II. 95.} The unity of Leibnizian
substances is a unity of duration: `We give to our forms only duration,
which the Gassendists grant to their atoms'\footnote{GP IV. 479; cf. GP
  IV. 508 = AG 159-160; GP II. 251 = AG 176; \autocite[870]{Whipple2010}: `the
  notion of a \emph{substance} that does \emph{not endure} is a
  contradiction in Leibnizian terms.'} -- a unity through time,
experienced in the case of the soul as the unity of
consciousness.\footnote{Cf. T. 384. By `consciousness', I mean the
  habitual state of an entity capable of perception, within which more
  or less confused and distinct perceptions may occur: what the
  phenomenological tradition calls a `stream of \emph{cogitationes}'
  (\autocite[31-33]{Husserl1960}). Consciousness in this sense is not to be
  identified with conscious perception in Leibniz's sense, which Leibniz
  identifies with distinct perception, or apperception (\autocite[36]{McRae1976}, 36;
  \autocite[53]{Simmons2001}; NE 162; GLW 32).}

The exact characteristics of Leibnizian force may change from creature
to creature , but the analogy of force or activity remains constant even
in the lowest beings. Thus, if Cartesian occasionalism constitutes being
as the disjunction of thinking or extension, Leibniz constitutes
thinking as appetition, and appetition as force.

\subsection{Their Communication}

By using the term `atom', Leibniz indicates not only that these forms
are unities in themselves, but also that they serve as building blocks
in a larger structure: without true unities, there would be no
multitude; the continuum is unable to be composed of mathematical
points; metaphysical points express the universe.\footnote{GP IV. 478,
  483. Cf. \autocite{Arthur1998}:

  \begin{quote}
  Of course, if the composition of the continuum is understood as a
  purely mathematical problem, one may well wonder what bearing physical
  considerations could have on it. But for Leibniz and his
  contemporaries, the problem was not restricted to the composition of
  purely mathematical entities-\/-such as whether a line is composed out
  of points or infinitesimals or neither-\/-but was understood as
  applying to all existing quantities and their composition.
  \end{quote}} These unities, as unities, constitute the
unity-in-multiplicity that is the universe. Leibnizian souls must be the
kinds of things that can belong together, not like atoms in a heap, but
intrinsically; thus, the communication of substances becomes a problem
for Leibniz akin with that of their nature.

In Leibniz's thinking, the problem of the interaction of substances
cannot be separated from that of the constitution of the world: first,
because the latter problem does not admit a more materialistic solution,
since material unity is derivative; second, because partitioning off the
objective world as something separate and distinct from conscious
subjects only raises anew all of the problems of mind-matter interaction
where it should solve them.\footnote{GP IV. 478.}

Leibniz believes the problem of the interaction of substances can be
solved by the idea of substance as force, conceived as analogous to
perception. But there is another aspect of perception as force, apart
from its already-mentioned `forward-looking' character, to which we have
yet to give due attention: namely, its character as
`expression.'\footnote{GP IV. 483, 485.} Leibniz means this literally:
the world, as stream of \emph{cogitationes}, is the expression of
thought. Leibniz makes the same point by referring to perception as a
kind of production.\footnote{GP IV 485.} What is expressed or produced
by the substance is the universe: the phenomenal world shared by the
different substances, each expressing it from their own `point of
view.'\footnote{GP IV. 483} In expressing this same universe from its
own standpoint, each creature thus also expresses every other. Leibniz
denies that creatures strictly act on each other not mainly to dismiss
the entry of an occult `influx' into the realm of physics, but because
the preceding point ensures substances are already radically
interconnected, and no influx could make them more so. This is the
source of Leibniz's famous designation of substances as `mirrors of the
universe.'\footnote{Cf. GP IV. 434. The account here offered of
  perception as force is thus not in conflict with that, in \autocite{Simmons2001}
  and \autocite{Puryear2006}, of perception as isomorphism, but provides a basis
  for it.}

\subsection{The Union of Mind and
Body}

\subsubsection{The Relation of Mind to Body
Generally}
Each Leibnizian nature, as force, expresses every other nature from its
own viewpoint. It does this by producing phenomena, which through the
harmonious laws by which they follow each other, are perceived as a
universe, a continuum, a manifold. From this standpoint, the problem of
mind-body interaction cannot be one of finding a medium by which
ontologically distinct kinds interact, since the dualist ontology giving
rise to the problem has been undercut. But the problem \emph{can}
manifest itself in a somewhat surprising way, directly related to that
of the communication of substances, and posed thus: the entire universe
is expressed through my perceptual activity; given this, the whole
perceived world is, properly speaking, mine, no more and no less than
that portion referred to as \emph{my} body. The purported
interconnectivity of substances may thus transform itself into
Spinozism.\footnote{From Leibniz's time to ours, the persistence of the
  charge of covert Spinozism evinces this concern. On whether Leibniz
  ever embraced a form of Spinozism, cf. \autocite{Mercer1999}, \autocite{Kulstad2002}.}

As Russell already recognized, Leibniz's solution to the mind-body
problem will have more in common with Spinoza than with
Malebranche.\footnote{Cf. \autocite[139]{Russell1951}.} In this connection, we
return to Leibniz's remark that `it is impossible to find the principles
of a true unity in matter, or in that which is only passive.'\footnote{GP
  IV. 478; cf. GP III. 552; IV. 395; VI. 237; \autocite[30-31]{Howard2017}.}
Here, Leibniz identifies matter with passivity in contrast with form,
which has the character of activity, specifically of an activity
analogous to appetition or perception, conceived as the production of
the stream of \emph{cogitationes}. It is as these \emph{cogitationes} --
i.e. as the product of the activity of perception -- that the passivity
of matter must be conceived; as the complement of the activity of form,
which results in a complete substance.\footnote{Cf. \autocite{Furth1967};
  \autocite{Rutherford1990}, \autocite{Rutherford1995}; \autocite{Adams1994}.} Hence Leibniz's classifies body as
\emph{phenomenon}.\footnote{The identification of matter and phenomenon
  is already in Geulincx. See \autocite[179]{Cooney1978}.} Immediately after the
\emph{New System}'s rejection of the occasionalist hypothesis on
mind-body interaction, Leibniz states:

\begin{quote}
And our interior sentiments \ldots being nothing besides
phenomena consequent on external things, or better, true appearances,
like well-ordered dreams -- it is necessary that these internal
perceptions in the soul itself come to it by its own original
constitution -- that is, by its representative nature \ldots
which was given to it from its creation, and which give it its
individual character.\footnote{GP IV. 484.}
\end{quote}

Here, the antecedent remarks clarify the communication of substances,
from which are derived (beginning at `it is necessary') insights into
the solution to the mind-body problem. Other beings are encountered in
and through `phenomena,' `appearances,' or more provocatively, `dreams.'
What distinguishes these from mere dreams is twofold: first, they are
consequent on external beings; second, they are `well-regulated.' The
first restates Leibniz's claim that substances express each other; the
second emphasizes their ontological separateness: substances are
solitary dreamers, and only by the contingent master plan of God do they
express the same dream. Against Spinozistic monism, the contingency of
beings is secured simultaneously with their individuation via the
morally certain claim\footnote{See GP VII. 320.} that these phenomena
are, by analogy with one's own \emph{cogito}, the expressions of other
vital beings from their own respective viewpoints.\footnote{GP IV. 483.}

From the separateness and intersubjectivity of substances, Leibniz
concludes the perceptions of the soul come to it via its own
constitution, and `give it its individual character.'\footnote{GP IV.
  484.} This is obscure at first sight. But given Leibniz's
identification of matter with passivity, appearances, phenomena, it is
nothing other than an old scholastic adage drastically reworked from a
Cartesian viewpoint: that matter is the principle of
individuation.\footnote{Cf. GP II. 118-120.}

\subsubsection{The Union of Particular Minds and
Bodies}

The above considerations show how the concrete substance, as a union of
the active and passive, form and matter, \emph{cogito} and
\emph{cogitatum}, is formed; they do not explain the connection of the
substance to what it recognizes as its \emph{own} body within
perception. But given the impossibility of an influx of the soul into
the body, one cannot hope for some \emph{tertium quid} securing their
union. The union of soul and body cannot be something other than the
harmony of their mutual laws.\footnote{Cf. Leibniz's remarks on the
  problem of skepticism at GP VII. 319-320.} Leibniz will agree with the
occasionalists that, when considered closely, the experience of moving
one's hand is nothing other than the conjunction of one's volition with
the datum of the moving hand, which, for Leibniz, is nothing other than
the harmony of the laws of my mind with those of the infinitely many
substances that make up my body. Leibniz remarks:

\begin{quote}
It would have been very wrong of me to object to the Cartesians that the
agreement God immediately maintains, between soul and body, according to
them, does not bring about a true union, since, to be sure, my
pre-established harmony would do no better than it does. But since the
metaphysical union one adds is not a phenomenon, and since no one has
ever given an intelligible notion of it, I did not take it upon myself
to seek a reason for it.\footnote{GP VI. 595-96 = AG 197.}
\end{quote}

For Leibniz, the \emph{phenomenal} union of mind and body cannot be
something added to the aforementioned harmony of laws; Leibniz's system
was never meant to explain anything more than this.

\section{Conclusion}

Having unfolded the pre-established harmony in its proper order, we can
now summarily distinguish it from the system of occasional causes.

The primary difference between the two systems lies in their accounts of
the nature of substance. While Malebranche accepts Cartesian ontology,
Leibniz reforms this ontology from within through his designation of
perception as force. Leibniz's substances have natures able to account
for their action; Malebranche's, like clocks in need of a perpetual
supervisor, do not. Leibniz presses the point of occasionalism entailing
a perpetual miracle in connection with this complaint: `It does not
suffice to say God has enacted a general law, for besides the decree,
there also must be a natural means of executing it.'\footnote{GP IV 520.}

On the interaction of substances, occasionalists have direct recourse to
God when they should have instead ascribed to Him the creation of and
concurrence with natures able to account for this interaction.\footnote{GP
  I. 391.} Leibnizian substances do this by being `mirrors of the
universe,' harmoniously expressing the universe and each other from
their own points of view.

On the mind-body problem, occasionalists reject mind-body interaction
and require God to move bodies in accord with our volitions. Leibniz
instead reads the mind-body relation as one between activity and
passivity, à la Spinoza;\footnote{Cf. E I. Prop. 14, Scholium; II. Prop.
  5, demonstratio.} and rejects the terms of the debate on which the
communication of mind and body could be some \emph{tertium quid} other
than the harmonious concordance of their laws.

In brief, Leibniz avoids the harrowed natures of the occasionalists via
a Cartesian adaptation of Spinoza's distinction between mind and body as
one between activity and passivity, adding his own designation of
perception as force. Leibniz avoids Spinozistic monism by agreeing with
the occasionalists that the causal interaction of substances is only
inferred from conjunction, and not proven in metaphysical rigor. This
effectively secures Leibniz a space in which he can postulate a
multitude of substances acting in harmonious accord with each other by
following their proper laws -- that is, it secures a space in which to
reach toward one of the main goals of his philosophy: the reconciliation
of knowledge with piety.

\section{Abbreviations}

A = Gottfried Wilhelm Leibniz. \emph{Sämtliche Schriften und Briefe}.
Ed. Deutsche Akademier der Wissenschaften zu Berlin. Berlin: Akademie,
1923.

AG = Gottfried Wilhelm Leibniz. \emph{Philosophical Writings}. Ed. R.
Ariew and D. Garber. Indianapolis: Hackett, 1989.

AT = René Descartes, \emph{Oeuvres de Descartes}, vols. 1-12, ed. Adam
and Tannery, revised edition. Paris: Vrin/CNRS, 1964-76.

C = \emph{Opuscules et fragments inédits de Leibniz}. Ed. by L.
Couturat. Paris: Alcan 1903. Reprinted Hildesheim: Georg Olms 1961.

CSMK = \emph{The Philosophical Writings of Descartes}, vols. 1-3, trans.
J. Cottingham,, R. Stoothoff, D. Murdoch, and A. Kenny. Cambridge:
Cambridge University Press, 1985-1991. Vols. 1 and 2 cited as CSM.

E = Benedictus de Spinoza, \emph{Ethica}. In Opera quotquot reperta
sunt. Ed. J. Van Vloten and J. P. N. Land. The Hague: Martinus Nijhoff.
Tomus Primus. pp. 35-273.

GP = Die Philosophischen Schriften von Gottfried Wilhelm Leibniz. Ed. C.
I. Gerhardt. 7 vols. Berlin: Weidmann, 1875-90. Reprinted Hildesheim:
Georg Olms, 1960.

\emph{In Metaph.} = Thomas Aquinas, \emph{Commentary on Aristotle's
Metaphysics}. Trans. John P. Rowan. Html-edited by Joseph Kenny, O. P.
with addition of Aquinas's Latin and and Aristotle's Greek text.
\url{http://dhspriory.org/thomas/metaphysics5.htm}

JS = \emph{Nicolas Malebranche: Dialogues on Metaphysics and on
Religion}. Ed. Nicholas Jolley and David Scott. Cambridge: Cambridge
University Press, 1997.

L = Gottfried Wilhelm Leibniz. \emph{Philosophical Papers and Letters}.
Ed. and trans. Leroy E. Loemker. Dordrecht: Kluwer, 1989.

LDV = Gottfried Wilhelm Leibniz. \emph{The Leibniz-De Volder
Correspondence}. Ed. and trans. Paul Lodge. New Haven: Yale University
Press, 2013. Reference is to original language page.

GLW = \emph{Briefwechsel zwischen Leibniz und Christian Wolff}. Ed. C.
I. Gerhardt. Hildesheim: Georg Olms, 1963.

\emph{Med.} = René Descartes. \emph{Meditationes de Prima Philosophia}.
Cited by book and paragraph.

\emph{Med.} \emph{chr.} = Nicolas Malebranche. \emph{Méditations
Chrétiennes}.

NE = Gottfried Wilhelm Leibniz \emph{New Essays on Human Understanding}.
Ed. P. Remnant and J. Bennett. Cambridge: Cambridge University Press,
1982.

NG = Nicolas Malebranche. \emph{Treatise on Nature and Grace}.
Translated with an introduction and notes by Patrick Riley. Oxford:
Clarendon Press.

\emph{OC} = \emph{Oeuvres Complètes de Malebranche}. Directeur A.
Robinet. 20 volumes. Paris: J. Vrin, 1958-1967.

\emph{Réponse} = Abraham Gaultier. \emph{Réponse en forme de
dissertation à un théologien, Qui demande ce que veulent dire les
sceptiques, qui cherchent la verité par tout dans la Nature, comme dans
les écrits des philosophes; lors qu'ils pensent que la Vie et la Mort
sont la même chose}. Ed. Olivier Bloch. Paris: Les Belles Lettres, coll.
Encre Marine, 2004.

\emph{ST} = Thomas Aquinas. \emph{Summa Theologiae}. Fathers of the
English Dominican Province, trans. Allen, TX: Christian Classics,
1948/1981.

T = Gottfried Wilhelm Leibniz, \emph{Theodicy}. Trans. E. M. Huggard. La
Salle, IL: Open Court, 1985.

\emph{Treatise} = Louis de La Forge, \emph{Treatise on the Human Mind
(1664)}. Translation with an introduction and notes by D. M. Clarke.
Dordrecht: Kluwer, 1997.

\section{Bibliography}
\printbibliography
\end{document}
