\documentclass[]{article}

%opening
\title{Counterpart Cognition}
\author{}

\begin{document}

\maketitle

\begin{abstract}
A theory of cognition qualifies as \textit{representational} iff a) it takes the immediate objects of a cognitive act to be something possibly distinct from the ultimate objects of that act; and b) these immediate objects, whatever they may be, somehow stand for or otherwise represent the ultimate objects of cognition. An early version of the theory is clearly present in Descartes, while the most representative account of the theory is likely that of Locke. More controversially, versions of representationalism have been ascribed to Aquinas\footnote{Claude Pannaccio, ``Aquinas on Intellectual Representation," in \textit{Ancient and Medieval Theories of Intentionality} (Leiden: Brill, 2001), 185-201} and Aristotle.\footnote{Victor Caston, ``Aristotle and the Problem of Intentionality," \textit{Philosophy and Phenomenological Research} 58:2 (June 1998), 249-298} More recently, versions of representational, or, more commonly, \textit{indirect}, realism have been advocated by Fodor, Lycan, and Dretske, among others.

\textit{Counterpart theory} is a metaphysical theory according to which the truthmakers for attributions of modal properties to individuals in this world are instantiations of those properties in counterparts of those individuals at other possible worlds, i.e. at other universes wholly distinct from and inaccessible to our own world, but in some sense just as real as ours. 

One of the major difficulties of counterpart theory has been the problem of \textit{epistemic access}: if possible worlds are really as inaccessible as counterpart theorists say, then it is hard to see how they could play any role in \textit{our} modalizing judgments.\footnote{Note that the Humphrey objection of Kripke (1971) is itself parasitic on this more basic objection: if Humphrey has no epistemic access to a world in which his counterpart wins, then \textit{a fortiori} neither can he have any more complex comportment - say, joy or envy - towards his counterpart's more fortunate situation.}

This paper shows that the conditions for veridical representation on any representationalist theory of cognition are strictly analogous to the truth conditions of the modalized judgments of counterpart theory. First, it provides a translation from the primitive predicates of counterpart theory to those of any representationalist theory, according to which the members of $W$ distinct from the actual world are reinterpreted as \textit{minds}, and $C$ is reinterpreted as a representation relation. Next, I show that a representationalist theory thus interpreted satisfies (with some minor modifications) the postulates of counterpart theory, and the representation relation fails to have just those properties that Lewis (1968) denies to the counterpart relation. Third, I show that under this interpretation, two traditional problems of epistemology - the problem of \textit{external world skepticism} and the problem of \textit{other minds} - turn out to be variants of the aforementioned difficulty of counterpart theory. I conclude that barring a successful attempt to break the analogy - and in spite of both the apparent plausibility of representationalism and the \textit{prima facie} implausibility of counterpart theory - the theories ought to stand or fall together.
\end{abstract}

\end{document}
