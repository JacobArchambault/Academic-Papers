\documentclass[]{article}

%opening
\title{}
\author{}

\begin{document}

\maketitle

\begin{abstract}

\end{abstract}

\section{A note on aims and methodology}
My reader is welcome to regard this dissertation simply as a contribution to the history of logic. But in doing so, she would be missing the impetus behind my taking up the project in the first place. This dissertation clarifies a concept - that of formal consequence - operant in logic today. The means chosen for attaining this has been a historical exposition of the genesis of this concept in medieval logic. While the study of the history of philosophy is in a better place now than it was a few generations ago, it has become so by clearing a space for itself, as it were, wherein it is regarded as valuable for its own sake. And so, to continue with the metaphor, the history of philosophy has attained its current respectability as a subfield precisely by remaining a \textit{subfield} - a marginal area of interest that, while it may offer accidental helps to theoretical philosophy, is not strictly necessary for its operation. And so even for those who have no particular antipathy to historical study, explicating a concept \textit{through} its history may seem to be a case of adopting a means intrinsically inadequate to the desired end. This problem is compounded by the fact that the topic under discussion is one of \textit{logic}, which, on one conception, is (or should be) ontologically `neutral', i.e. completely free from assumptions about the way the world actually is, and \textit{a fortiori} uninformed by that world's history.

\section{Objections}
Hence, we have two fundamental objections to the shape of our inquiry as it stands. I call the first \textit{the logician's objection to metaphysics}; the second, \textit{the theorist's objection to history}. Their content is as follows: 
\begin{quote}
\textbf{Objection 1: the logician's objection to metaphysics} - project $p$ aims to show how a concept lifted from a metaphysical framework bears on a logical one. But since logic is ontologically neutral, no concept belonging to metaphysics has any bearing on another belonging to logic. And so no inquiry can show that a metaphysical concept bears on a logical one.
\end{quote}
\begin{quote}
\textbf{Objection 2: the theorist's objection to history} - project $p$ aims to clarify the content of a concept $C$ with domain $D$ via the history of $C$. But the content of no concept is clarified by its history. Proof: the content of any concept necessarily belongs to it. But the history of a concept belongs to it contingently, and nothing belonging to another in a contingent manner can clarify what belongs to the same necessarily. Hence, history cannot illuminate conceptual content.
\end{quote}
The first objection raises a problem for the relation between the concepts of one kind of inquiry and those of another, The second concerns the relation of history to any properly theoretical endeavor. I begin by answering the first.

\section{Ontological neutrality and the ontology of logic}
In order to determine what is involved in the ontological neutrality of logic, one must first have unified accounts of what ontology and metaphysics respectively consist in. In the case of metaphysics, the current state of the discipline suggests no such account; rather, metaphysics comes across as a grab-bag discipline encompassing traditionally philosophical topics that don't seem to fit anywhere else.\footnote{For instance, the most recent edition of Blackwell's \textit{Metaphysics: An Anthology} includes the following headings: causation; identity; modality; objects; ontology; persistence; persons; properties. Some of these (identity, modality) have connections at least as strong with other disciplines (logic, in this case) as with metaphysics. Others (ontology, objects) are not clearly differentiated from each other. Still others (persistence, persons) seem to have domains too restricted to qualify as metaphysical, and only bear faint relations to the discipline as traditionally conceived. This is not so much a fault of this particular volume (or any of the many others encoding the same \textit{prima facie} arbitrariness) as a testimony to the state of the discipline itself, i.e. one wherein most of its practitioners are content to work on smaller problems without ever addressing what it means to say that this work is work in \textit{metaphysics}.} Ontology is slightly better off, inasmuch as its object is contained in the slogan `ontology studies what exists'. But This statement is itself not sufficiently clear. Accordingly, different accounts of the meaning of this slogan provide different notions of ontological neutrality. We will examine each of these in turn.

\subsection{Ontology as the determination of types of existents through first-order quantification}
The slogan `ontology studies what exists' is most often understood to mean ontology provides an exhaustive classification of kinds of beings. If ontology asks the question `what exists?', the sort of answer it envisions could be given in the form of a universal quantifier ranging over a disjunction, with each disjunct attributing a monadic predicate to the value of a variable bound by the universal quantifier, such that the empty set is the value of no predicate, the intersection of the range of any two predicates is the empty set, and the union of all of them yields the domain of quantification. More succinctly, ontology aims to provide a disjoint and exhaustive classification of the most basic kinds of beings. If the requirement that the intersection of different predicates be empty is dropped, the result is a mediate aim of ontology, i.e. determining what kinds of beings exist. If the ontological enterprise aimed at is eliminativist in character, then the beings in the latter group would be regarded as not `really' existing, and a translation schema for reducing these sentences to others quantifying over only primitive existents would likely have to be given; if the enterprise is merely reductionist, then the beings of this latter group would probably be regarded as existing in some diminished sense.

On such an account, for logic to be ontologically neutral would be for it to be in principle unfit for the use the ontologist makes of it, i.e. determining kinds of existing entities through the use of first-order quantification.

\subsection{The object of ontology as ways of being}

A second notion of ontology can take as our aim not the enumeration of types of entities, but the determination of their properties.\footnote{Here, the term `property' should be understood in the broad sense of whatever can be predicated of an individual, rather than as, e.g. Lewisian classes. See Lewis (1983), p. 344} Ontology in this sense would not be concerned so much with beings as the \textit{being} of beings; and, especially if considered as the determination of \textit{most basic} ways of being, would seem to have its predecessor, if not its first instantiation, in Aristotle's \textit{Categories}.

On this account, the ontological neutrality of logic would consist in this: that logic does not determine or otherwise reveal the way things are, and so one cannot `read off' the structures or properties of things from logic.

Typically, this concept of ontological neutrality is invoked against those who would take the syntax of a \textit{language} - whether natural, formal, mental, or otherwise - to reveal the structure of reality. Even so, the claim can be understood in several ways. 

Most innocently, the claim can be taken to mean that the syntactical structure of the logical language one uses cannot as such be relied on to reveal the structure of reality. In this sense, its force is merely that one cannot assume that the logic one is using for the inquiry and domain at hand is the right one. This claim is primarily methodological in character, and seems unobjectionable.

In another sense, the phrase could be taken to mean that a logic is not as such in the business of determining the way things are; and so\textit{qua} logic, the question of a correct application is one that simply does not come up. Some logic might, for all we know, hold the key to the hitherto hidden structure of reality. But as logicians, the question of a correct or incorrect application simply doesn't concern us. By analogy, one might call Euclidean geometry `false' and Riemannian geometry `true' inasmuch as the latter seems to provide a correct account of the nature of space, where the former does not. But strictly speaking, neither the one nor the other has as its aim any possible application, but both have as their immediate object certain eidetic structures to which each is perfectly adequate. In this sense, too, the ontological neutrality of logic seems unobjectionable: actually doing logic \textit{does} require prescinding from ontological questions and concerns.

Lastly, the claim can be taken in a stronger sense to mean that \textit{even if} one has the correct logical language, this does not of itself determine or otherwise presuppose a specification of the (necessary, actual, possible) structure of reality.

The most important thing to note about the above modal parsings is that \textit{all} of them have corollaries that are metaphysical in character; and so as defenses of ontological neutrality, they are self-undermining. For instance, if a language doesn't determine or otherwise specify the ways things actually are, then \textit{things are such that} that language doesn't specify their actual structure. So rather than enforcing ontological neutrality, these kinds of defenses presuppose an answer to the metaphysical question of how a logical language relates to the world. To state the broader principle, the negation of a metaphysical claim remains a metaphysical claim: if, e.g. the claim that modal logic presupposes modal realism is metaphysical in character, then so, too, is the claim that it \textit{doesn't}.

I suspect the appeal of ontological neutrality in the sense described derives from a broader picture of the relation between logic and metaphysics. Roughly, the relation is analogous to that present in Aquinas' understanding of the relation between philosophy and theology.\footnote{See \textit{ST} Ia, q. 1.} For Aquinas, the truths of philosophy do not provide a decision for or against the truths of sacred doctrine. Rather, philosophy takes one to a certain point, and then if one wishes to go farther, the gaps remaining must be filled in by theology. In this way, philosophy provides a solid foundation for theological study without replacing it. Similarly, on the account envisioned by partisans of ontological neutrality, the study of logic (whether conceived strictly as the study of the formal properties of formal systems, or broadly as more or less synonymous with philosophy of language) is taken to provide a sure and unbiased foundation for the study of higher philosophical sciences, such as metaphysics. It is given the first place pedagogically because it provides a method universally applicable in later inquiries without in any way predetermining the results of these inquiries. As such, appeals to ontological neutrality in the described sense accord with and contribute to a vision of philosophy that is, broadly speaking, both foundationalist and progressivist in character: it is foundationalist inasmuch as it accepts aim of finding a space free of presuppositions from which philosophical inquiry should begin, whether this space is found in epistemology, phenomenology, philosophy of language, semantics, or elsewhere; it is progressivist inasmuch as it sees this space as one ultimately to be filled. It accepts a \textit{terminus a quo} distinct from its \textit{terminus ad quem}, and views the philosophical endeavor as one of moving from the one to the other.\footnote{I take it this is all that is essentially involved in progressivism. And so properly speaking, progressivism is at best derivatively associated with the view that things `keep getting better'.}

The basic enterprise encapsulated in works like Aristotle's \textit{Metaphysics} is not determining which kinds of beings there are, but more basically that of determining \textit{what it is} for a being \textit{to be} a being. If one takes `being' to be said univocally of all beings, then one way of answering this question is to give the kind of universally quantified disjunction mentioned in the previous subsection.\footnote{Hence, the modern understanding of ontology relates to the more traditional one as a genuine attenuation of it.} But it is the multiplicity of ways of answering this original question that undergirds the unity and multiplicity of traditionally metaphysical disciplines: metaphysics discusses causality because causes are beings in a higher sense than their effects; identity, because being a being as such implies being self-identical; modality, because the extension of being to being-potential coincides with the discovery of a notion of being extending even to what are, strictly speaking, non-beings, etc.

In keeping with this theme, while it is unlikely that a) the presence of a metaphysical notion in our logician's toolkit would lead to the failure of ontological neutrality in the most common sense, or even that b) the presence of such a notion would provide us with a clue to the way logic ideally should be, c) it \textit{does} uncover a structural analogy conditioning what it presently means for something to be logical, thereby providing a clue to the being of logicality.

My project is interested in ontology in the aforementioned way, albeit in a limited sense. The project is not an inquiry into the being of beings exactly insofar as they are beings, and so it does not belong to ontology as such. But it is an inquiry into the being of formality as it appears in logic, and so, in this sense, forms part of an inquiry into the ontology of logic\footnote{Note that by this I do \textit{not} mean a determination of types belonging to the domain of a limited region of ontological space. Rather, I mean an inquiry into what it means for something to be logical, i.e. what it is to be logical}. The inquiry maintains a relation to metaphysics inasmuch as the sense of formality considered is at least analogously related to that found in Aristotelian metaphysics, the latter concept clearly providing the source material for the former. If I thought the domain of logic were uniquely circumscribable, I would describe this project as a regional ontology, in Husserl's sense. Since, however, I think the universality of logic precludes its specificity, it would be better to regard the project as contributing to a critique of formal logic - not in the sense of a criticism, but rather in the sense of `critique' invoked by Kant and Husserl.

\subsection{A response to the question}

At this point, we can return to the objection as phrased. The crux of the objection is that to distinct sciences belong distinct concepts. While this might not at all seem self-evident, I think it is true. Most basically, every science will be what it is as directed towards a more or less restricted domain of beings, understood in a certain respect. The concepts proper to a science will be those that must be added to one's common stock of concepts in order to understand the specified domain in the sense required, all of which will typically belong to that science inasmuch as they relate to some primary sense. When these senses are encapsulated in a language, the resulting body of words and phrases will constitute the \textit{jargon} of that science. 

Now, a condition for a term' $t$'s belonging to a language $L$ is that it have some use in $L$, and so must meaningfully relate to at least one other term in $L$.\footnote{The following argument is a variation on a point first made by Sommers, 1962} For parallel reasons, the senses indicated by these terms themselves must similarly relate.\footnote{It should be evident to my reader that I am using `concept', `sense', `idea', and the like in a fairly interchangeable way. This is possibile on account of a duality inherent in the concept of a concept as such. On the one hand, `concept' can refer to that by which or in which a given sense is captured. On the other, it can mean the sense itself as understood. I think both uses are legitimate - much as, for instance, the word `judgment' is used to indicate both the act of judging and what is judged. But I also think that too heavy a reliance on the first meaning of `concept' has left many of us thinking that concepts are a kind of third entity mediating between the mind and ordinary external objects. This is unfortunate in two senses: the first, in that it invariably treats concepts as reified; the second, in that it treats concepts as primarily suppositive rather than signifying. 

For both of these reasons, I also avoid calling a nexus of concepts a `conceptual scheme', laden as that phrase is with the connotation that such a scheme is just an object or collection of objects applicable to others.} In some cases, the concepts proper to one science will be a proper subset of those belonging to another one. In others, the concepts proper to different sciences will be disparate, etc. Lastly, because these senses are at least mediately relatable, so, too, will be the sciences studying them.\footnote{It is this point which grounds the unity of the sciences, and thereby also provides the occasion for the actualization of this unity in the structure of the university.}

Roughly, the number of concepts\footnote{I. e. both the common and proper concepts of a science} necessary for understanding a given domain of objects will be inversely proportional to the size of that domain, the reason for this being that stricter concepts implicitly contain in their sense more general ones. Inasmuch as these more general senses are implicitly presupposed by sciences focusing on the more specific senses, it is possible for there to be a science that explicitly thematizes these more general concepts. This dependence of senses grounds a corresponding dependence between the sciences aiming at the clarification of those senses, and so the science explicitly thematizing the more general sense will accordingly be called \textit{higher} than the other.

Now, to say that one science does not presuppose another is to say that the senses explicitly thematized by the former do not themselves implicitly contain those explicated by the latter. And so, it is also to say that the latter is not higher than the former. Therefore, to say logic does not presuppose metaphysics is to say both 1) that the concepts studied in logic do not in the aforementioned way presuppose those explicated in metaphysics, and 2) that metaphysics is not a higher science than logic. But the primary sense studied in metaphysics is that of \textit{being}. If, then, logic does not presuppose metaphysics, then the concepts it studies do not presuppose or otherwise include that of being. This does not mean that logic does not study being \textit{qua} being: it means that logic does not study being in \textit{any} sense. This, in turn, can only occur in one of two ways. If logic is not a higher science, then either a) logic is higher than metaphysics, or b) no subordinating relation obtains between them - they are disparate.\footnote{My reader will have noticed that these are, logically, not the only ways for sets of concepts to relate to each other: identity and partial overlap are also options. I don't mention identity because I think that two purportedly distinct sciences sharing the \textit{exact} same concepts are not two sciences, but one, just as two objects sharing the exact same properties are the same object (Call this principle of the identity of indiscernibles for sciences, if you like). I don't mention the case of partial overlap because I think whatever can be said of that case also applies to sciences with completely distinct conceptual apparatus.} If the former,then the primary sense studied in logic - call it $N$ - will itself be contained in that of being, and so being will be a derivative mode of $N$-ing. If the latter, then the notions of the one and the other will be in principle relatable by some higher concept, itself in principle thematizable by a higher science.\footnote{In this way, the former approach actually undergirds the Quine-Goodman program of `reading off' a metaphysics from one's logic. Contrary to what one might assume, this approach is not arrogant because it reads metaphysical considerations into logic, but precisely because it \textit{doesn't}: the logic dictates the metaphysics because the logic itself is treated in practice as untouchable by it. 

The latter approach, by contrast, is what has motivates the rise of semantics from the 20th century to today, which exactly parallels the rise of epistemology in the 17th: in the 17th century, one began by separating out the mental realm from that of being as if they were members of a common genus, only to take epistemology as a \textit{tertium quid} required to connect the two; now, instead of cutting our heads off, it is \textit{language} that is amputated from reality, and we bring in semantics to bridge the gap.} But neither of these can occur: to be logical, after all, is \textit{to be} logical. And so, far from needing to be free from metaphysics, logic must presuppose it for its very sense. The appearance to the contrary only occurs in the presence of a more attenuated understanding of metaphysics or ontology.

\section{The place of history in theoretical philosophy}
'History' is said in two senses: in one sense, it refers to the course of events narrated; in another, it names the manner in which these events are narrated. I take the second sense to be grounded in the first, and I do not wish to concern myself with it directly. Rather, in what follows, I investigate the relation of a concept to its history in the first sense.

The theorist's objection to history contends that a concept's contingent history is not relevant to the determination of its necessary internal content, relying on the basic assumption that contingent matters do not shed light on necessary ones.

The force of this basic assumption rests on an equivocation. The imagery of shedding light on, illuminating, clarifying, straightforwardly implies a hierarchical ordering, wherein one matter is benefited by another (i.e. it is illuminated), but not conversely. To assume that the necessary clarifies the contingent is to assume that the necessary is that on account of which the contingent is able to be made manifest: necessity grounds contingency. Obviously, these sorts of judgments are taking the kind of clarification at hand in thoroughly ontological terms, wherein one kind of object is made capable of appearing as it does on account of its subordinate relation to another. Understood thus, the judgment is true, but not relevant. On the other hand, the verb `to clarify' and its synonyms can be taken in the above judgment to express not an ontological, but an evidentiary grounding. As such, the judgment would mean: the contingent does not disclose the necessary. Understood in this way, the judgment is relevant, but false. On the contrary, far from being impossible, evidence of necessity is hardly grounded in any other way. Even if, as Aristotle put it, all knowledge is taken from the senses, the immediate objects of which are contingent, this would not imply that all knowledge is \textit{about} contingents. 

Thus, most basically, history serves theoretical philosophy in an `exemplary' manner: it provides it with examples and data from which to adduce deeper claims. 

The sort of dependence involved here, however, is relatively weak. It does not belong to history as such, but is simply part of the manner in which theorizing presupposes experience generally. However, this sense secures for theorizing at least a possible relation to history.

But one might wonder whether theoretical philosophy presupposes history in a stronger way. And the answer to this can be given simply by considering the conditions for the possibility of the objection itself. 

\subsection{The multiplicity of shapes capable of being sustained by an ontological inquiry}
\subsubsection{In general}
It should be clear even to the most a-historical philosopher that the theorist's objection to history would not arise in every epoch with equal ease; and that it would not, even if raised, carry equal force in every age. It would carry little weight, for instance, in an epoch where deference to ancient authorities was regarded as the surest pathway to truth.

As far as this point goes, the presence of an objection or viewpoint is no different than the coming to presence of anything else. At the extremes of this phenomenon are cases where the structural conditions in being either necessitate certain developments, or make them impossible. In the case at hand, the objection would not arise if the distinction between the necessary and contingent did not appear to be involve two distinct realms, an understanding  historically indebted to the separation of the contingent from the eidetic first carried out in the ontology of Plato. To give a more mundane example, equity in the workplace, either as an ideal or as a reality, presupposes the workplace - that is, it presupposes a world in which the locus of work is distinct from that of the home, and thereby presupposes much of the development that historically came to pass in the industrial revolution. Similarly, the push for marriage equality begun in the late 2000s presupposes the presence of marriage as a public good,\footnote{Otherwise, it would not be a matter for governmental regulation}, as well as the assumed normality of lifelong monogamous relationships.\footnote{Without this, the push for legal recognition of same sex relationships would be unlikely to take the shape of a push for \textit{marriage} rights} The development of Euclidean plane geometry in ancient Greece presuppposed: the prior development of writing, as a way of keeping track of long chains of reasoning; the existence of relatively flat surfaces, as objects from which the concept of two-dimensional Euclidean space might be constructed; and even the development of skills like carpentry, the problems of which provide the occasion for the concern with purely theoretical structures to arise in the first place.\footnote{This example is discussed at length in Husserl's \textit{The Origin of Geometry.}} Lastly, the distinction between objectual and substitutional quantification only appears significant in the presence of the notion that there are more things in being than can be named even in principle, a notion that only came to fruition in the real analysis of Dedekind, Cantor, and others. 

In short, the coming into existence of any concept presupposes certain conditions without which it would not be enabled, and others without which it would be unlikely to flourish. These conditions always take a certain historical shape, not all of the parts of which are strictly necessary for the existence or flourishing of the notion itself. When the enabling conditions for a structure are replaced with others, or drastically changed, this often brings about a major shift in the structure itself: think for instance, of the change in the British Monarchy as it has moved from having its main support in the notion of the king as appointed by God, to its current state, where it is in great part propped up by tabloid culture.

The appearance of any theoretical structure is always an appearance within a world. This is the precondition for the perdurance of recognizably the same concepts, structures, institutions, etc. through different variations. A concept can undergo variation while remaining the selfsame concept only because it is capable of undergoing certain changes in relations that we deem external to it: it appears in the light of its surroundings, taking on different contours and meanings in accordance with the place it occupies in a world. Such relations are never, in the first place, relations of utterances to context: they are relations of beings to the senses they are capable of sustaining within a manifold.\footnote{Today, for instance, the notion of the formal bears a particularly close relation to that of the technical. But this was not always the case: the opposite relation would even seem to have obtained in, e.g. Aristotle's denying substantial forms to artifacts. For Aristotle, the formal was paradigmatically present in the presence of \textit{life}. For us, by contrast, life itself is understood on the model of artifacts - specifically, computers. On the one hand, the \textit{contingent} obtaining of the identity between the formal and the technical is why it makes so little sense to describe Aristotle as a functionalist. And yet on the other, the contingent \textit{obtaining} of the relation, made possible by the inversion of the traditional estimation of the relation of the artificial to the natural, is the reason why it makes as much sense as it does.}

In this way, a structure can bear certain relations to its surroundings that nevertheless do not belong to it as such. Within its worldly habitat, an eidetic structure can intimate, portend, presuppose, prevent, oppose, etc. This is why, for instance, the question `what is the meaning of this?' can carry the sense of `what will come of this?'

What this means is that things really \textit{do} mean all sorts of things in all sorts of ways. A police officer, for instance, may signify both justice and brutality, even if the latter signification (or the former, for that matter) is in no way connected to the being of policehood as such. Hence, the judgment `the content of a concept belongs to it necessarily', expressed in the objection, is false. The \textit{essential} content of a concept belongs to it necessarily. But this itself is quite a different thing.

Perhaps more importantly, the judgment `the content of a concept belongs to it necessarily' reflects a philosophical comportment the aim of which is the purification of a content of its historically variable characteristics, i.e. it belongs to the project of the determination of essence. The role taken by the historical in such an endeavor is, by design, purely negative. But all importantly, an investigation of the being of a given content need not take the form of essence explication, for the simple reason that the being of any concept at any time goes beyond what belongs to it essentially.

\subsubsection{With reference to the being of history}

Given, then, that an explication of the being of a being need not be restricted to essence explication, it remains ask how the structure of a content relates to its history as such, and thereby to determine why and how the history of a concept pertains to it.

The answer lies in a reevaluation of another assumption in the objection, viz., that the history of a content belongs to it contingently. Let us grant that the same essential content of a given concept could have been arrived at by a vastly different historical process. Even so, the presently actuated notion of formality as it appears in the concept of formal consequence, just like any other actuated structure, retains an ineliminable relation to its own past. In ordinary contexts, when someone says that something is necessary, she means to say that it is fixed, unchangeable, perhaps even fated. She need not mean that it belongs to the matter of which it is spoken essentially. Not all necessity is essential.\footnote{See Fine, 1994.} And there is a clear sense in which, even if no concept bears an essential relation to its origins, every historically actuated eidetic structure bears a relation to its origins that is fixed, unchangeable, accomplished once and for all. This point is, I believe, what Aristotle and his followers alluded to in the doctrine of the necessity of the past; the same point is present in a different way in Kripke's notion of the necessity of origins.

For any concept, its history has the sense of something completed, an accomplishment.\footnote{This is why the past tense of a verb is called the perfect tense.} As such, \textit{every historically enacted past sense of a concept remains available for future actuation}. Sometimes, this reactivation can be realized materially, as occurred, for instance, in the late 19th century revival of the Olympic games. At other times, it can be returned to actuality in a merely intellective mode, i.e. it can be recollected. 

Because they bear an ineliminable relation to their prior moments, concepts never completely `shake off' their material origins. This form of non-essential necessitation is why, for instance, the etymology of a word can be so illuminating, even when the oldest senses of the word are no longer heard in its everyday tokens.\footnote{For a beautiful example of this point in action, see Heidegger (1951).}

That being said, it remains possible, within the hustle and bustle of the practical work to which a concept (especially a higher one) contributes, for its innermost sense to withdraw, and thereby to lie dormant and unactuated. This mode of withdrawal, as it occurs in understanding, is what we call `forgetting'.

In the advance whereby a concept acquires new senses, it is possible for these senses to occlude others, or to conflict with a sense already acquired. The internal tension between the distinct elements of the concept as developed is called `confusion'. A latent confusion brought to full presence is called a contradiction. The process whereby a concept i) fails to manifest what is most essential to it and ii) acquires multiply unsustainable senses, may be called its corruption, a process that often ends with the concept receding for a time into obscurity. 

The work of explication of the history of a concept - and more broadly as I see it, the work of philosophy itself - is a work of restoration: it recovers what has been lost. Certainly, the sort of recovery envisioned is not the rapid accumulation of information no longer current. Rather, it is the restoration of an idea to its original integrity, to a wholeness wherein the matter itself may be safeguarded. And inasmuch as the \textit{loss} of integrity occurs in time, the work of returning it must constantly look back to this contingent passing of history to determine exactly what it is that has been lost, and thereby to know what can be done. Such work never looks to the mere recovery of the past, but rather takes this past as an inheritance to be transformed by being set forth into the future. Simultaneously, in recollecting the idea's first coming into actuality, the philosopher sets her eye on what has likely since withdrawn; and knowing it, is better able to shelter it. 

In short, a great part of the work of philosophy consists in i) peeling back layers of senses hastily accrued, and ii) thereby bringing forth the essential senses grounding these added layers. These essential senses are, in the purest sense, presuppositions: they are what is laid under a manifestation as what enables it to spring forth; they are its groundwork, its origin, its \textit{Ursprung}. Historical philosophical investigation is the investigation of these origins.

Taking these words in a quite broader sense than that in which they are usually understood, we might think of the philosophical endeavor as such as a kind of topology. It is concerned not merely with the essential \textit{rather than} the changing and contingent, but instead investigates being in its fullest sense as a locus in the very midst of which the act of philosophizing itself stands. A central part of this is our thoughtful attunement towards the being of what has come before us, i.e. to the past. Another legitimate, albeit limited way in which such an investigation can be carried forth is as the determination of what it is in a being that is internal and perduring, i.e. as essence explication.

If this appears unfamiliar, it is in part because even in historical investigation, the most common way in which the object of investigation is understood is as what is set before the researcher, i.e. as what is present. Guided by this visual metaphor, the investigation then usually proceeds in one of two ways. In the first way, the object is taken as an object of analysis;\footnote{The facility in promoting this sort of investigation is part of the reason why thinkers themselves are often represented as mere collections of claims.} in the second way, the object is taken as something to overcome, or to get past. The way of proceeding that I have sketched, at least, shows that there are more ways for being to appear than the objectual mode; and it thereby opens up less worn paths for philosophical investigation.

\section{Conclusion}
The preceding began by opening up an aporia within the current state of logic: logic generally, and logical consequence in particular, is stamped with the language of formality. But though the language of formality is frequent and central to current logical research, the meaning and role played by this language remain unclarified. Especially surprising is that this language has clear roots in the hylomorphism of Aristotle's \textit{Physics}, a framework that has for the most part been explicitly abandoned by logicians and philosophers alike. 

The notion of formal consequence present in the model theoretic tradition from Tarski to today has its main predecessor in the account of formal consequence given by the 14th century Parisian arts master John Buridan. In contrast with our own time, where the adoption of the language of formality in logic has taken place rather unconciously, the medieval appropriation of hylomorphism in logic took place at the height of medieval engagement with Aristotle, and so could not but be a thoughtful appropriation. Therefore, the investigation of this more original appropriation, at the dawn of the development of accounts of consequences, is capable of illuminating not merely the past meaning of formal consequence, but also its present meaning. 

The shape taken by the investigation is that of an excavation of meanings accrued and occluded at the time of the concept's genesis. In accordance with this shape, the investigation will move chronologically backwards: beginning from Buridan, then moving back to the earlier accounts of Ockham, Burley, and the members of the Paris Arts faculty at the turn of the 14th century.

Though an investigation of the place of a metaphysical concept in logic, the project is not interested in engaging in metaphysics or ontology as commonly understood; though an investigation primarily belonging to the history of logic, the project is not free from ontological implications. Indeed, as an inquiry into the being of formality in logic, the project can and should be construed as a contribution to the ontology of logic, albeit in a different sense than that in which `ontology' is typically understood.

Though the mode of investigation is historical, the end is theoretical: the project illuminates the present concept of formality by uncovering its original meaning. As a past sense of the notion currently in use, this sense remains available for actuation and recovery. The diligent execution of this recovery is a work of restoration, wherein accrued meanings are peeled back, hidden and more basic senses are set forth, and the notion itself is restored to integrity.

\end{document}