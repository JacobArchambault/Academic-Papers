\documentclass[]{article}

%opening
\title{On the role of history in theoretical philosophical inquiry}
\author{Jacob Archambault}

\begin{document}
	
	\maketitle
	
	\begin{abstract}
		
	\end{abstract}

\section{The place of history in theoretical philosophy}
'History' is said in two senses: in one sense, it refers to the course of events narrated; in another, it names the manner in which these events are narrated. I take the second sense to be grounded in the first, and I do not wish to concern myself with it directly. Rather, in what follows, I investigate the relation of a concept to its history in the first sense.

Let us consider the following common objection to the relevance of historical inquiry to theoretical work. Call it \textit{the theorist's objection to history}. Its content is as follows:
\begin{quote}
	Project $p$ aims to clarify the content of a concept $C$ with domain $D$ via the history of $C$. But the content of no concept is clarified by its history. Proof: the content of any concept necessarily belongs to it. But the history of a concept belongs to it contingently, and nothing belonging to another in a contingent manner can clarify what belongs to the same necessarily. Hence, history cannot illuminate conceptual content.
\end{quote}
This objection concerns the relation of history to any properly theoretical endeavor. The theorist's objection to history contends that a concept's contingent history is not relevant to the determination of its necessary internal content, relying on the basic assumption that contingent matters do not shed light on necessary ones.

The force of this basic assumption rests on an equivocation. The imagery of shedding light on, illuminating, clarifying, straightforwardly implies a hierarchical ordering, wherein one matter is benefited by another (i.e. it is illuminated), but not conversely. To assume that the necessary clarifies the contingent is to assume that the necessary is that on account of which the contingent is able to be made manifest: necessity grounds contingency. Obviously, these sorts of judgments are taking the kind of clarification at hand in thoroughly ontological terms, wherein one kind of object is made capable of appearing as it does on account of its subordinate relation to another. Understood thus, the judgment is true, but not relevant. On the other hand, the verb `to clarify' and its synonyms can be taken in the above judgment to express not an ontological, but an evidentiary grounding. As such, the judgment would mean: the contingent does not disclose the necessary. Understood in this way, the judgment is relevant, but false. On the contrary, far from being impossible, evidence of necessity is hardly grounded in any other way. Even if, as Aristotle put it, all knowledge is taken from the senses, the immediate objects of which are contingent, this would not imply that all knowledge is \textit{about} contingents. 

Thus, most basically, history serves theoretical philosophy in an `exemplary' manner: it provides it with examples and data from which to adduce deeper claims. 

The sort of dependence involved here, however, is relatively weak. It does not belong to history as such, but is simply part of the manner in which theorizing presupposes experience generally. However, this sense secures for theorizing at least a possible relation to history.

But one might wonder whether theoretical philosophy presupposes history in a stronger way. And the answer to this can be given simply by considering the conditions for the possibility of the objection itself. 

\subsection{The multiplicity of shapes capable of being sustained by an ontological inquiry}
\subsubsection{In general}
It should be clear even to the most a-historical philosopher that the theorist's objection to history would not arise in every epoch with equal ease; and that it would not, even if raised, carry equal force in every age. It would carry little weight, for instance, in an epoch where deference to ancient authorities was regarded as the surest pathway to truth.

Here, the presence of an objection or viewpoint is no different than the coming to presence of anything else. At the extremes of this phenomenon are cases where the structural conditions in being either necessitate certain developments, or make them impossible. Consider the following examples. 
\begin{itemize}
	\item Equity in the workplace, either as an ideal or as a reality, presupposes the workplace - that is, it presupposes a world in which the locus of work is distinct from that of the home, and thereby presupposes much of the development that historically came to pass in the industrial revolution. 
	\item The push for marriage equality begun in the late 2000s presupposes the presence of marriage as a public good,\footnote{Otherwise, it would not be a matter for governmental regulation}, as well as the assumed normality of lifelong monogamous relationships.\footnote{Without this, the push for legal recognition of same sex relationships would be unlikely to take the shape of a push for \textit{marriage} rights} 
	\item The development of Euclidean plane geometry in ancient Greece presuppposed: the prior development of writing, as a way of keeping track of long chains of reasoning; the existence of relatively flat surfaces, as objects from which the concept of two-dimensional Euclidean space might be constructed; and even the development of skills like carpentry, the problems of which provide the occasion for the concern with purely theoretical structures to arise in the first place.\footnote{This example is discussed at length in Husserl's \textit{The Origin of Geometry.}}
	\item In philosophical logic, the distinction between objectual and substitutional quantification only appears significant in the presence of the notion that there are more things in being than can be named even in principle, a notion that only came to fruition in the real analysis of Dedekind, Cantor, and others.  
	\item In the case of the objection itself, it would not arise if the distinction between the necessary and contingent did not appear to be one between two distinct realms, an understanding historically indebted to the separation of the contingent from the eidetic first carried out in the ontology of Plato. 
\end{itemize}

In short, the coming into existence of any concept presupposes certain conditions without which it would not be enabled, and others without which it would be unlikely to flourish. These conditions always take a certain historical shape, not all of the parts of which are strictly necessary for the existence or flourishing of the notion itself. When the enabling conditions for a structure are replaced with others, or drastically changed, this often brings about a major shift in the structure itself: think for instance, of the change in the British Monarchy as it has moved from having its main support in the notion of the king as appointed by God, to its current state, where it is in great part propped up by tabloid culture.

The appearance of any theoretical structure is always an appearance within a world. This is the precondition for the perdurance of recognizably the same concepts, structures, institutions, etc. through different variations. A concept can undergo variation while remaining the selfsame concept only because it is capable of undergoing certain changes in relations that we deem external to it: it appears in the light of its surroundings, taking on different contours and meanings in accordance with the place it occupies in a world. 

Today, for instance, the notion of the formal bears a particularly close relation to that of the technical. But this was not always the case: the opposite relation would even seem to have obtained in, e.g. Aristotle's denying substantial forms to artifacts. For Aristotle, the formal was paradigmatically present in the presence of \textit{life}. For us, by contrast, life itself is understood on the model of artifacts - specifically, computers. On the one hand, the \textit{contingent} obtaining of the identity between the formal and the technical is why it makes little sense to describe Aristotle as a functionalist. And yet on the other, the contingent \textit{obtaining} of the relation, made possible by the inversion of the traditional estimation of the relation of the artificial to the natural, is the reason why it makes as much sense as it does.

Such relations are never, in the first place, relations of utterances to context: they are relations of beings to the senses they are capable of sustaining within a manifold. In this way, a structure can bear certain relations to its surroundings that nevertheless do not belong to it as such. Within its worldly habitat, an eidetic structure can intimate, portend, presuppose, prevent, oppose, etc. This is why, for instance, the question `what is the meaning of this?' can carry the sense of `what will come of this?'

What this means is that things really \textit{do} mean all sorts of things in all sorts of ways. A police officer, for instance, may signify both justice and brutality, even if the latter signification (or the former, for that matter) is in no way connected to the being of policehood as such. Hence, the judgment `the content of a concept belongs to it necessarily', expressed in the objection, is false. The \textit{essential} content of a concept belongs to it necessarily. But this itself is quite a different thing.

Perhaps more importantly, the judgment `the content of a concept belongs to it necessarily' reflects a philosophical comportment the aim of which is the purification of a content of its historically variable characteristics, i.e. it belongs to the project of the determination of essence. The role taken by the historical in such an endeavor is, by design, purely negative. But all importantly, an investigation of the being of a given content need not take the form of essence explication, for the simple reason that the being of any concept at any time goes beyond what belongs to it essentially.

\subsubsection{With reference to the being of history}
Given, then, that an explication of the being of a being need not be restricted to essence explication, it remains ask how the structure of a content relates to its history as such, and thereby to determine why and how the history of a concept pertains to it.

The answer lies in a reevaluation of another assumption in the objection, viz., that the history of a content belongs to it contingently. Let us grant that the same essential content of a given concept could have been arrived at by a vastly different historical process. Even so, the presently actuated notion of formality as it appears in the concept of formal consequence, just like any other actuated structure, retains an ineliminable relation to its own past. In ordinary contexts, when someone says that something is necessary, she means to say that it is fixed, unchangeable, perhaps even fated. She need not mean that it belongs to the matter of which it is spoken essentially. Not all necessity is essential.\footnote{See Fine, 1994.} And there is a clear sense in which, even if no concept bears an essential relation to its origins, every historically actuated eidetic structure bears a relation to its origins that is fixed, unchangeable, accomplished once and for all. This point is, I believe, what Aristotle and his followers alluded to in the doctrine of the necessity of the past; the same point is present in a different way in Kripke's notion of the necessity of origins.

For any concept, its history has the sense of something completed, an accomplishment.\footnote{This is why the past tense of a verb is called the perfect tense.} As such, \textit{every historically enacted past sense of a concept remains available for future actuation}. Sometimes, this reactivation can be realized materially, as occurred, for instance, in the late 19th century revival of the Olympic games. At other times, it can be returned to actuality in a merely intellective mode, i.e. it can be recollected. 

Because they bear an ineliminable relation to their prior moments, concepts never completely `shake off' their material origins. This form of non-essential necessitation is why, for instance, the etymology of a word can be illuminating, even when the oldest senses of the word are no longer heard in its everyday tokens.\footnote{For a beautiful example of this point in action, see Heidegger (1951).}

That being said, it remains possible, within the hustle and bustle of the practical work to which a concept (especially a higher one) contributes, for its innermost sense to withdraw, and thereby to lie dormant and unactuated. This mode of withdrawal, as it occurs in understanding, is what we call `forgetting'.

In the advance whereby a concept acquires new senses, it is possible for these senses to occlude others, or to conflict with a sense already acquired. The internal tension between the distinct elements of the concept as developed is called `confusion'. A latent confusion brought to full presence is called a contradiction. The process whereby a concept i) fails to manifest what is most essential to it and ii) acquires multiply unsustainable senses, may be called its corruption, a process that often ends with the concept receding for a time into obscurity. 

The work of explication of the history of a concept - and more broadly as I see it, the work of philosophy itself - is a work of restoration: it recovers what has been lost. Certainly, the sort of recovery envisioned is not the rapid accumulation of information no longer current. Rather, it is the restoration of an idea to its original integrity, to a wholeness wherein the matter itself may be safeguarded. And inasmuch as the \textit{loss} of integrity occurs in time, the work of returning it must constantly look back to this contingent passing of history to determine exactly what it is that has been lost, and thereby to know what can be done. Such work never looks to the mere recovery of the past, but rather takes this past as an inheritance to be transformed by being set forth into the future. Simultaneously, in recollecting the idea's first coming into actuality, the philosopher sets her eye on what has likely since withdrawn; and knowing it, is better able to shelter it. 

In short, a great part of the work of philosophy consists in i) peeling back layers of senses hastily accrued, and ii) thereby bringing forth the essential senses grounding these added layers. These essential senses are, in the purest sense, presuppositions: they are what is laid under a manifestation as what enables it to spring forth; they are its groundwork, its origin, its \textit{Ursprung}. Historical philosophical investigation is the investigation of these origins.

Taking these words in a quite broader sense than that in which they are usually understood, we might think of the philosophical endeavor as such as a kind of topology. It is concerned not merely with the essential \textit{rather than} the changing and contingent, but instead investigates being in its fullest sense as a locus in the very midst of which the act of philosophizing itself stands. A central part of this is our thoughtful attunement towards the being of what has come before us, i.e. to the past. Another legitimate, albeit limited way in which such an investigation can be carried forth is as the determination of what it is in a being that is internal and perduring, i.e. as essence explication.

If this appears unfamiliar, it is in part because even in historical investigation, the most common way in which the object of investigation is understood is as what is set before the researcher, i.e. as what is present. Guided by this visual metaphor, the investigation then usually proceeds in one of two ways. In the first way, the object is taken as an object of analysis;\footnote{The facility in promoting this sort of investigation is part of the reason why philosophers past are themselves often represented as mere collections of claims.} in the second way, the object is taken as something to overcome, or to get past. The way of proceeding that I have sketched, at least, shows that there are more ways for being to appear than the objectual mode; and it thereby opens up less worn paths for philosophical investigation.

\section{Conclusion}
Logic generally, and logical consequence in particular, is stamped with the language of formality. But though the language of formality is frequent and central to current logical research, the meaning and role played by this language remain unclarified. Especially surprising is that this language has clear roots in the hylomorphism of Aristotle's \textit{Physics}, a framework that has for the most part been explicitly abandoned by logicians and philosophers alike. 

The notion of formal consequence present in the model theoretic tradition from Tarski to today has its main predecessor in the account of formal consequence given by the 14th century Parisian arts master John Buridan. In contrast with our own time, where the adoption of the language of formality in logic has taken place rather unconciously, the medieval appropriation of hylomorphism in logic took place at the height of medieval engagement with Aristotle, and so could not but be a thoughtful appropriation. Therefore, the investigation of this more original appropriation, at the dawn of the development of accounts of consequences, is capable of illuminating not merely the past meaning of formal consequence, but also its present meaning. 

The shape taken by the investigation is that of an excavation of meanings accrued and occluded at the time of the concept's genesis. In accordance with this shape, the investigation will move chronologically backwards: beginning from Buridan, then moving back to the earlier accounts of Ockham, Burley, and the members of the Paris Arts faculty at the turn of the 14th century.

Though an investigation primarily belonging to the history of logic, the project is not free from ontological implications. Indeed, as an inquiry into the being of formality in logic, the project can and should be construed as a contribution to the ontology of logic, albeit in a different sense than that in which `ontology' is typically understood.

Though the mode of investigation is historical, the end is theoretical: the project illuminates the present concept of formality by uncovering its original meaning. As a past sense of the notion currently in use, this sense remains available for actuation and recovery. The diligent execution of this recovery is a work of restoration, wherein accrued meanings are peeled back, hidden and more basic senses are set forth, and the notion itself is restored to integrity.
\end{document}