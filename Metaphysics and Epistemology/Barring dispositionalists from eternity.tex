\documentclass[]{article}
\usepackage{amssymb}
%opening
\title{Barring dispositionalists from eternity}
\author{}

\begin{document}

\maketitle

\begin{abstract}
Dispositionalism grounds modality in the powers of existing things. Eternalism assumes `exists' ranges equally over past, present, and future entities. In this article, I show that eternalism is incompatible with the most plausible understanding of what it is that dispositionalist modality explains: generation and corruption. As such, the dispositionalist is forced, against dispositionalism's traditional categorization as a form of actualism, to deny that everything exists.
\end{abstract}

\section{Preliminaries}
Dispositionalism about modality grounds metaphysical possibility and necessity in the powers of actual things. Though dispositionalist accounts have arguably been around since Aristotle, they've received renewed attention in recent literature on the topic. Recent accounts include \cite{Simchen2006}, \cite{Borghini2008}, \cite{Jacobs2010}, \cite{Vetter2013}, and \cite{Yates2015}.

Actualism is the view that all and only actual entities exist \cite{Plantinga1974} \cite{Plantinga1976} \cite{Adams1981}. This statement is often meant in two closely related, albeit distinct senses: first, as a metaphysical statement about the existence of objects; second as a statement about the appropriate range of existential and universal quantifiers \cite[p. 436]{LinskyZalta1994}. According to the first sense, to be is to be actual; according to the second, the existence of objects is appropriately captured by existential quantification - to be is to be the value of a bound variable \cite{Quine1948}. Actualism is contrasted with possibilism, according to which possible objects exist strictly speaking \cite{Lewis1986} \cite{Tomberlin1996}; and with the adoption of possibilist quantification, whether used to express ontological commitment or not. One may be an actualist in one sense without being one in the other, though the acceptance of actualist quantification is often motivated by the adoption of metaphysical actualism.

The distinction between presentism and eternalism in discussions of time parallels that between actualism and possibilism about modality \cite{Noonan2013}. According to presentism, only present objects exist - to be is to be present; whereas on the eternalist view, past and future objects exist no less than present objects. As in its modal analogue, these metaphysical views may be reflected in the adoption of presentist (tensed) or eternalist (tenseless) quantification \cite{Prior2003a} \cite{Baron2015}.

Modal actualism holds that only actual entities may serve as truthmakers for modal claims. Hence, though modal actualists accept that there are true modal claims, they deny that non-actual objects - mere possibilia - can factor into these claims. While some varieties of modal actualism restrict the range of objects available, and hence the possibilities genuinely available in one's first-order modal theory,  others dodge these restrictions in one of two ways: by stuffing proxies for possibilia into the actual world, or by adopting a nonstandard reading of the existential quantifier in modal contexts - in short, either diluting the primary sense of `actualism' by reading `exists' in a nonstandard way, or its secondary sense by reading the existential quantifier in a nonstandard way.\footnote{The approach of \cite{LinskyZalta1994} and \cite{LinskyZalta1996}, critiqued in \cite{Bennett2006}, could be seen as exemplifying the first; that of \cite{White1985}, as an instance of the second.} Dispositionalism is typically classified among the more restricting varieties of actualism \cite{Contessa2010} \cite{Vetter2011}.

The dispositionalist differs from most actualists in two major respects. First, because dispositionalist modality tracks powers to bring about, the dispositionalist requires certain states, and thus some of the beings and properties involved in those states, to move from being non-actual to being actual, and vice versa. Second, the dispositionalist completely rejects the appeal to possible worlds, even as a heuristic. Possibilities are intra-worldly possibilities to become actual within the order of things as they stand.

Dispositionalists typically presuppose some form of realism about the referents of first order predicates, and are nearly unanimous in taking dispositions to be properties of some sort.\footnote{Cf. \cite{Pruss2002}, \cite{Martin2008}, \cite{Contessa2015}.} Given this, dispositionalist accounts of modality standardly divide into two kinds: one Platonic, the other Aristotelian \cite{Fitch1996} \cite[pp. 234-238]{Jacobs2010}. A Platonic theory takes properties to be abstract objects, and equates an object $o$'s having a property $P$ with $o$'s bearing the appropriate relation - instantiation, participation, etc. - to $P$. In a Platonic account, properties, being abstract, do not depend for their existence on concrete particulars, though they may depend on them for their being instantiated. An Aristotelian approach, in contrast, might take properties to be tropes, as in \cite{Martin2008}; or universals, as in \cite{Armstrong1997}. Aristotelian theories take only instantiated properties to exist.
%can this paragraph be eliminated?
\section{The material inadequacy of eternalist dispositionalism}
That actualist quantification should also be eternalist has been assumed since the earliest and best known accounts of actualism.\footnote{See, for instance, \cite{Plantinga1976}, \cite{Adams1981}.} At the level of quantification over powers, the approach is favoured by \cite{Yates2015} and \cite{Borghini2008}. But eternalist quantification has begun to loosen its grip on dispositionalism. The account of \cite{Vetter2013}, for instance, hints at a diachronic approach to dispositionalist modality with presentist quantification, while the monotonicity condition on valuations of formulas over chains in \cite{Jacobs2010}'s semantics assumes a kind of `growing block' dispositionalism, where the set of powers available grows over time.

The following, however, should pry dispositionalism from the hands of eternalism for good. The adoption of eternalist quantification has clear and decisively negative consequences for the dispositionalist. 

Consider what different theories of modality are taken to be theories \textit{of}. For a Lewisian modal realist, modal claims track brute similarity relations between ourselves and our various trans-world counterparts. For many non-dispositionalist actualists, modal claims capture our basic intuitions about what modal truths there are. But for the dispositionalist, modality tracks dispositions or powers, and powers are fundamentally powers to bring about a certain state. Otherwise put, dispositionalists posit powers to explain coming to be, along with its corollary passing away. \textit{Dispositions explain generation and corruption}.

Now, generation and corruption come in two varieties: substantial and accidental. Substantial generation [corruption] results in the coming into [passing out of] existence of new objects; accidental, in the coming into [passing out of] existence of new properties. When becoming or passing away occurs, something moves from not existing to existing, i.e. passes from non-being into being, or vice versa. If, however, our first-order quantifiers range over the world as a completed totality, then there is \textit{never} an instance of an object moving from within the scope of our quantifiers to outside of it, or vice versa: no new objects come into, or out of, existence. For the same reason, if the domain of our second-order quantifiers ranges over all past, present, and future entities, then no new properties --- and \textit{a fortiori}, no new powers --- ever come into or out of existence, either. 

Dispositionalism grounds modality in powers, and powers are powers to bring things about. But adopting a powers-based account of modality is only sensible if some of the things that can be brought about \textit{are not}. Assume $<p>$ is some state of affairs. If $<p>$ holds in the actual world at any point, then there is no need to talk about it coming to be: the objects and properties involved in $<p>$ are already captured by the semantics for the non-modal portion of our language.  If, on the other hand, $<p>$ does not hold at any point throughout the history of the actual world, then the only way in which it can be possible is as a possibility that is never realized - which is just to say that it can never be brought about. Proof: assume the contrary. Then there is an extension of the actual world with $<p>$ as an existing state of affairs. But if this is so, then the state of affairs constituting $<p>$ is already in the scope of our non-modal quantifiers and predicates, which we assumed it wasn't. Contradiction.

In short, an eternalist dispositionalism is unable to account for the very things dispositions exist to account for in the first place.

\section{Conclusion}
The paramount concern of an account of possibility couched in terms of powers to bring about is to account for generation and corruption, whether of things or of properties, within the order of the universe as it stands; and quantification ranging over all times renders any account of this inadequate. To be brought about just means to be brought into connection with the order of things as an extension of it; eternalist quantification, however, presupposes that the order of things is already completely captured – closed, therefore unextendable.

The dispositionalist has ways out of this problem: for instance, the situation can be remedied by the adoption of a presentist, growing-block, or otherwise restricted approach to quantifier scope. It is uncertain whether the eternalist has any way out of the problem. But what is certain is that against its traditional association with actualism, the dispositionalist has to deny one of the theses that helped make actualism of any stripe appealing in the first place: that everything exists.
\bibliography{jacob}
\bibliographystyle{plain}
\end{document}
